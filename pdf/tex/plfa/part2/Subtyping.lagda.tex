\hypertarget{Subtyping}{%
\chapter{Subtyping: Records}\label{Subtyping}}

\begin{fence}
\begin{code}
module plfa.part2.Subtyping where
\end{code}
\end{fence}

This chapter introduces \emph{subtyping}, a concept that plays an
important role in object-oriented languages. Subtyping enables code to
be more reusable by allowing it to work on objects of many different
types. Thus, subtyping provides a kind of polymorphism. In general, a
type \texttt{A} can be a subtype of another type \texttt{B}, written
\texttt{A\ \textless{}:\ B}, when an object of type \texttt{A} has all
the capabilities expected of something of type \texttt{B}. Or put
another way, a type \texttt{A} can be a subtype of type \texttt{B} when
it is safe to substitute something of type \texttt{A} into code that
expects something of type \texttt{B}. This is know as the \emph{Liskov
Substitution Principle}. Given this semantic meaning of subtyping, a
subtype relation should always be reflexive and transitive. When
\texttt{A} is a subtype of \texttt{B}, that is,
\texttt{A\ \textless{}:\ B}, we may also refer to \texttt{B} as a
supertype of \texttt{A}.

To formulate a type system for a language with subtyping, one simply
adds the \emph{subsumption rule}, which states that a term of type
\texttt{A} can also have type \texttt{B} if \texttt{A} is a subtype of
\texttt{B}.

\begin{myDisplay}
⊢<: : ∀{Γ M A B}
  → Γ ⊢ M ⦂ A
→ A <: B
    -----------
  → Γ ⊢ M ⦂ B
\end{myDisplay}

In this chapter we study subtyping in the relatively simple context of
records and record types. A \emph{record} is a grouping of named values,
called \emph{fields}. For example, one could represent a point on the
Cartesian plane with the following record.

\begin{myDisplay}
{ x = 4, y = 1 }
\end{myDisplay}

A \emph{record type} gives a type for each field. In the following, we
specify that the fields \texttt{x} and \texttt{y} both have type
\texttt{ℕ}.

\begin{myDisplay}
{ x : `ℕ,  y : `ℕ }
\end{myDisplay}

One record type is a subtype of another if it has all of the fields of
the supertype and if the types of those fields are subtypes of the
corresponding fields in the supertype. So, for example, a point in three
dimensions is a subtype of a point in two dimensions.

\begin{myDisplay}
{ x : `ℕ,  y : `ℕ, z : `ℕ } <: { x : `ℕ,  y : `ℕ }
\end{myDisplay}

The elimination form for records is field access (aka. projection),
written \texttt{M\ \#\ l}, and whose dynamic semantics is defined by the
following reduction rule, which says that accessing the field
\texttt{lⱼ} returns the value stored at that field.

\begin{myDisplay}
{l₁=v₁, ..., lⱼ=vⱼ, ..., lᵢ=vᵢ} # lⱼ —→  vⱼ
\end{myDisplay}

In this chapter we add records and record types to the simply typed
lambda calculus (STLC) and prove type safety. It is instructive to see
how the proof of type safety changes to handle subtyping. Also, the
presence of subtyping makes the choice between extrinsic and intrinsic
typing more interesting by. If we wish to include the subsumption rule,
then we cannot use intrinsically typed terms, as intrinsic terms only
allow for syntax-directed rules, and subsumption is not syntax directed.
A standard alternative to the subsumption rule is to instead use
subtyping in the typing rules for the elimination forms, an approach
called algorithmic typing. Here we choose to include the subsumption
rule and extrinsic typing, but we give an exercise at the end of the
chapter so the reader can explore algorithmic typing with intrinsic
terms.

\hypertarget{imports}{%
\section{Imports}\label{imports}}

\begin{fence}
\begin{code}
open import Data.Empty using (⊥; ⊥-elim)
open import Data.Empty.Irrelevant renaming (⊥-elim to ⊥-elimi)
open import Data.Fin using (Fin; zero; suc)
open import Data.Nat using (ℕ; zero; suc; _≤_; z≤n; s≤s; _<_; _+_)
open import Data.Nat.Properties
    using (m+n≤o⇒m≤o; m+n≤o⇒n≤o; n≤0⇒n≡0; ≤-pred; ≤-refl; ≤-trans; m≤m+n; n≤m+n)
open import Data.Product using (_×_; proj₁; proj₂; Σ-syntax) renaming (_,_ to ⟨_,_⟩)
open import Data.String using (String; _≟_)
open import Data.Unit using (⊤; tt)
open import Data.Vec using (Vec; []; _∷_; lookup)
open import Data.Vec.Membership.Propositional using (_∈_)
open import Data.Vec.Membership.DecPropositional _≟_ using (_∈?_)
open import Data.Vec.Membership.Propositional.Properties using (∈-lookup)
open import Data.Vec.Relation.Unary.Any using (here; there)
import Relation.Binary.PropositionalEquality as Eq
open Eq using (_≡_; refl; sym; trans; cong)
open import Relation.Nullary using (Dec; yes; no; ¬_)
\end{code}
\end{fence}

\hypertarget{syntax}{%
\section{Syntax}\label{syntax}}

The syntax includes that of the STLC with a few additions regarding
records that we explain in the following sections.

\begin{fence}
\begin{code}
infixl 5 _,_
infix 4 _⊆_
infix 5 _<:_
infix  4 _⊢_⦂_
infix 4 _⊢*_⦂_
infix  4 _∋_⦂_
infix  4 Canonical_⦂_

infixr 7 _⇒_

infix  5 ƛ_
infix  5 μ_
infixl 7 _·_
infix  8 `suc_
infix  9 `_
infixl 7 _#_
infix 5 {_⦂_}
infix 5 {_:=_}

infix 5 _[_]
infix 2 _—→_
\end{code}
\end{fence}

\hypertarget{record-fields-and-their-properties}{%
\section{Record Fields and their
Properties}\label{record-fields-and-their-properties}}

We represent field names as strings.

\begin{fence}
\begin{code}
Name : Set
Name = String
\end{code}
\end{fence}

A record is traditionally written as follows

\begin{myDisplay}
{ l₁ = M₁, ..., lᵢ = Mᵢ }
\end{myDisplay}

so a natural representation is a list of label-term pairs. However, we
find it more convenient to represent records as a pair of vectors
(Agda's \texttt{Vec} type), one vector of fields and one vector of
terms:

\begin{myDisplay}
l₁, ..., lᵢ
M₁, ..., Mᵢ
\end{myDisplay}

This representation has the advantage that the traditional subscript
notation \texttt{lᵢ} corresponds to indexing into a vector.

Likewise, a record type, traditionally written as

\begin{myDisplay}
{ l₁ : A₁, ..., lᵢ : Aᵢ }
\end{myDisplay}

will be represented as a pair of vectors, one vector of fields and one
vector of types.

\begin{myDisplay}
l₁, ..., lᵢ
A₁, ..., Aᵢ
\end{myDisplay}

The field names of a record type must be distinct, which we define as
follows.

\begin{fence}
\begin{code}
distinct : ∀{A : Set}{n} → Vec A n → Set
distinct [] = ⊤
distinct (x ∷ xs) = ¬ (x ∈ xs) × distinct xs
\end{code}
\end{fence}

For vectors of distinct elements, lookup is injective.

\begin{fence}
\begin{code}
distinct-lookup-inj : ∀ {n}{ls : Vec Name n}{i j : Fin n}
   → distinct ls  →  lookup ls i ≡ lookup ls j
   → i ≡ j
distinct-lookup-inj {ls = x ∷ ls} {zero} {zero} ⟨ x∉ls , dls ⟩ lsij = refl
distinct-lookup-inj {ls = x ∷ ls} {zero} {suc j} ⟨ x∉ls , dls ⟩ refl =
    ⊥-elim (x∉ls (∈-lookup j ls))
distinct-lookup-inj {ls = x ∷ ls} {suc i} {zero} ⟨ x∉ls , dls ⟩ refl =
    ⊥-elim (x∉ls (∈-lookup i ls))
distinct-lookup-inj {ls = x ∷ ls} {suc i} {suc j} ⟨ x∉ls , dls ⟩ lsij =
    cong suc (distinct-lookup-inj dls lsij)
\end{code}
\end{fence}

We shall need to convert from an irrelevant proof of distinctness to a
relevant one. In general, the laundering of irrelevant proofs into
relevant ones is easy to do when the predicate in question is decidable.
The following is a decision procedure for whether a vector is distinct.

\begin{fence}
\begin{code}
distinct? : ∀{n} → (xs : Vec Name n) → Dec (distinct xs)
distinct? [] = yes tt
distinct? (x ∷ xs)
    with x ∈? xs
... | yes x∈xs = no λ z → proj₁ z x∈xs
... | no x∉xs
    with distinct? xs
... | yes dxs = yes ⟨ x∉xs , dxs ⟩
... | no ¬dxs = no λ x₁ → ¬dxs (proj₂ x₁)
\end{code}
\end{fence}

With this decision procedure in hand, we define the following function
for laundering irrelevant proofs of distinctness into relevant ones.

\begin{fence}
\begin{code}
distinct-relevant : ∀ {n}{fs : Vec Name n} .(d : distinct fs) → distinct fs
distinct-relevant {n}{fs} d
    with distinct? fs
... | yes dfs = dfs
... | no ¬dfs = ⊥-elimi (¬dfs d)
\end{code}
\end{fence}

The fields of one record are a \emph{subset} of the fields of another
record if every field of the first is also a field of the second.

\begin{fence}
\begin{code}
_⊆_ : ∀{n m} → Vec Name n → Vec Name m → Set
xs ⊆ ys = (x : Name) → x ∈ xs → x ∈ ys
\end{code}
\end{fence}

This subset relation is reflexive and transitive.

\begin{fence}
\begin{code}
⊆-refl : ∀{n}{ls : Vec Name n} → ls ⊆ ls
⊆-refl {n}{ls} = λ x x∈ls → x∈ls

⊆-trans : ∀{l n m}{ns  : Vec Name n}{ms  : Vec Name m}{ls  : Vec Name l}
   → ns ⊆ ms  →  ms ⊆ ls  →  ns ⊆ ls
⊆-trans {l}{n}{m}{ns}{ms}{ls} ns⊆ms ms⊆ls = λ x z → ms⊆ls x (ns⊆ms x z)
\end{code}
\end{fence}

If \texttt{y} is an element of vector \texttt{xs}, then \texttt{y} is at
some index \texttt{i} of the vector.

\begin{fence}
\begin{code}
lookup-∈ : ∀{ℓ}{A : Set ℓ}{n} {xs : Vec A n}{y}
   → y ∈ xs
   → Σ[ i ∈ Fin n ] lookup xs i ≡ y
lookup-∈ {xs = x ∷ xs} (here refl) = ⟨ zero , refl ⟩
lookup-∈ {xs = x ∷ xs} (there y∈xs)
    with lookup-∈ y∈xs
... | ⟨ i , xs[i]=y ⟩ = ⟨ (suc i) , xs[i]=y ⟩
\end{code}
\end{fence}

If one vector \texttt{ns} is a subset of another \texttt{ms}, then for
any element \texttt{lookup\ ns\ i}, there is an equal element in
\texttt{ms} at some index.

\begin{fence}
\begin{code}
lookup-⊆ : ∀{n m : ℕ}{ns : Vec Name n}{ms : Vec Name m}{i : Fin n}
   → ns ⊆ ms
   → Σ[ k ∈ Fin m ] lookup ns i ≡ lookup ms k
lookup-⊆ {suc n} {m} {x ∷ ns} {ms} {zero} ns⊆ms
    with lookup-∈ (ns⊆ms x (here refl))
... | ⟨ k , ms[k]=x ⟩ =
      ⟨ k , (sym ms[k]=x) ⟩
lookup-⊆ {suc n} {m} {x ∷ ns} {ms} {suc i} x∷ns⊆ms =
    lookup-⊆ {n} {m} {ns} {ms} {i} (λ x z → x∷ns⊆ms x (there z))
\end{code}
\end{fence}

\hypertarget{types}{%
\section{Types}\label{types}}

\begin{fence}
\begin{code}
data Type : Set where
  _⇒_   : Type → Type → Type
  `ℕ    : Type
  {_⦂_} : {n : ℕ} (ls : Vec Name n) (As : Vec Type n) → .{d : distinct ls} → Type
\end{code}
\end{fence}

In addition to function types \texttt{A\ ⇒\ B} and natural numbers
\texttt{ℕ}, we have the record type \texttt{{\ ls\ ⦂\ As\ }}, where
\texttt{ls} is a vector of field names and \texttt{As} is a vector of
types, as discussed above. We require that the field names be distinct,
but we do not want the details of the proof of distinctness to affect
whether two record types are equal, so we declare that parameter to be
irrelevant by placing a \texttt{.} in front of it.

\hypertarget{subtyping}{%
\section{Subtyping}\label{subtyping}}

The subtyping relation, written \texttt{A\ \textless{}:\ B}, defines
when an implicit cast is allowed via the subsumption rule. The following
data type definition specifies the subtyping rules for natural numbers,
functions, and record types. We discuss each rule below.

\begin{fence}
\begin{code}
data _<:_ : Type → Type → Set where
  <:-nat : `ℕ <: `ℕ

  <:-fun : ∀{A B C D : Type}
    → C <: A  → B <: D
      ----------------
    → A ⇒ B <: C ⇒ D

  <:-rcd :  ∀{m}{ks : Vec Name m}{Ss : Vec Type m}.{d1 : distinct ks}
             {n}{ls : Vec Name n}{Ts : Vec Type n}.{d2 : distinct ls}
    → ls ⊆ ks
    → (∀{i : Fin n}{j : Fin m} → lookup ks j ≡ lookup ls i
                               → lookup Ss j <: lookup Ts i)
      ------------------------------------------------------
    → { ks ⦂ Ss } {d1} <: { ls ⦂ Ts } {d2}
\end{code}
\end{fence}

The rule \texttt{\textless{}:-nat} says that \texttt{ℕ} is a subtype of
itself.

The rule \texttt{\textless{}:-fun} is the traditional rule for function
types, which is best understood with the below diagram. It answers the
question, when can a function of type \texttt{A\ ⇒\ B} be used in place
of a function of type \texttt{C\ ⇒\ D}. It will be called with an
argument of type \texttt{C}, so we need to convert from \texttt{C} to
\texttt{A}. We then call the function to get the result of type
\texttt{B}. Finally, we need to convert from \texttt{B} to \texttt{D}.
Note that the direction of subtyping for the parameters is swapped
(\texttt{C\ \textless{}:\ A}), a phenomenon named contra-variance.

\begin{myDisplay}
C <: A
⇓    ⇓
D :> B
\end{myDisplay}

The record subtyping rule (\texttt{\textless{}:-rcd}) characterizes when
a record of one type can safely be used in place of another record type.
The first premise of the rule expresses \emph{width subtyping} by
requiring that all the field names in \texttt{ls} are also in
\texttt{ks}. So it allows the hiding of fields when going from a subtype
to a supertype. The second premise of the record subtyping rule
(\texttt{\textless{}:-rcd}) expresses \emph{depth subtyping}, that is,
it allows the types of the fields to change according to subtyping. The
following is an abbreviation for this premise.

\begin{fence}
\begin{code}
_⦂_<:_⦂_ : ∀ {m n} → Vec Name m → Vec Type m → Vec Name n → Vec Type n → Set
_⦂_<:_⦂_ {m}{n} ks Ss ls Ts = (∀{i : Fin n}{j : Fin m}
    → lookup ks j ≡ lookup ls i  →  lookup Ss j <: lookup Ts i)
\end{code}
\end{fence}

\hypertarget{subtyping-is-reflexive}{%
\section{Subtyping is Reflexive}\label{subtyping-is-reflexive}}

In this section we prove that subtyping is reflexive, that is,
\texttt{A\ \textless{}:\ A} for any type \texttt{A}. The proof does not
go by induction on the type \texttt{A} because of the
\texttt{\textless{}:-rcd} rule. We instead use induction on the size of
the type. So we first define size of a type, and the size of a vector of
types, as follows.

\begin{fence}
\begin{code}
ty-size : (A : Type) → ℕ
vec-ty-size : ∀ {n : ℕ} → (As : Vec Type n) → ℕ

ty-size (A ⇒ B) = suc (ty-size A + ty-size B)
ty-size `ℕ = 1
ty-size { ls ⦂ As } = suc (vec-ty-size As)
vec-ty-size {n} [] = 0
vec-ty-size {n} (x ∷ xs) = ty-size x + vec-ty-size xs
\end{code}
\end{fence}

The size of a type is always positive.

\begin{fence}
\begin{code}
ty-size-pos : ∀ {A} → 0 < ty-size A
ty-size-pos {A ⇒ B} = s≤s z≤n
ty-size-pos {`ℕ} = s≤s z≤n
ty-size-pos {{ fs ⦂ As } } = s≤s z≤n
\end{code}
\end{fence}

The size of a type in a vector is less-or-equal in size to the entire
vector.

\begin{fence}
\begin{code}
lookup-vec-ty-size : ∀{k} {As : Vec Type k} {j}
   → ty-size (lookup As j) ≤ vec-ty-size As
lookup-vec-ty-size {suc k} {A ∷ As} {zero} = m≤m+n _ _
lookup-vec-ty-size {suc k} {A ∷ As} {suc j} = ≤-trans (lookup-vec-ty-size {k} {As}) (n≤m+n _ _)
\end{code}
\end{fence}

Here is the proof of reflexivity, by induction on the size of the type.

\begin{fence}
\begin{code}
<:-refl-aux : ∀{n}{A}{m : ty-size A ≤ n} → A <: A
<:-refl-aux {0}{A}{m}
    with ty-size-pos {A}
... | pos rewrite n≤0⇒n≡0 m
    with pos
... | ()
<:-refl-aux {suc n}{`ℕ}{m} = <:-nat
<:-refl-aux {suc n}{A ⇒ B}{m} =
  let A<:A = <:-refl-aux {n}{A}{m+n≤o⇒m≤o _ (≤-pred m) } in
  let B<:B = <:-refl-aux {n}{B}{m+n≤o⇒n≤o _ (≤-pred m) } in
  <:-fun A<:A B<:B
<:-refl-aux {suc n}{{ ls ⦂ As } {d} }{m} = <:-rcd {d1 = d}{d2 = d} ⊆-refl G
    where
    G : ls ⦂ As <: ls ⦂ As
    G {i}{j} lij rewrite distinct-lookup-inj (distinct-relevant d) lij =
        let As[i]≤n = ≤-trans (lookup-vec-ty-size {As = As}{i}) (≤-pred m) in
        <:-refl-aux {n}{lookup As i}{As[i]≤n}
\end{code}
\end{fence}

The theorem statement uses \texttt{n} as an upper bound on the size of
the type \texttt{A} and proceeds by induction on \texttt{n}.

\begin{itemize}
\item
  If it is \texttt{0}, then we have a contradiction because the size of
  a type is always positive.
\item
  If it is \texttt{suc\ n}, we proceed by cases on the type \texttt{A}.

  \begin{itemize}
  \tightlist
  \item
    If it is \texttt{ℕ}, then we have \texttt{ℕ\ \textless{}:\ ℕ} by
    rule \texttt{\textless{}:-nat}.
  \item
    If it is \texttt{A\ ⇒\ B}, then by induction we have
    \texttt{A\ \textless{}:\ A} and \texttt{B\ \textless{}:\ B}. We
    conclude that \texttt{A\ ⇒\ B\ \textless{}:\ A\ ⇒\ B} by rule
    \texttt{\textless{}:-fun}.
  \item
    If it is \texttt{{\ ls\ ⦂\ As\ }}, then it suffices to prove that
    \texttt{ls\ ⊆\ ls} and \texttt{ls\ ⦂\ As\ \textless{}:\ ls\ ⦂\ As}.
    The former is proved by \texttt{⊆-refl}. Regarding the latter, we
    need to show that for any \texttt{i} and \texttt{j},
    \texttt{lookup\ ls\ j\ ≡\ lookup\ ls\ i} implies
    \texttt{lookup\ As\ j\ \textless{}:\ lookup\ As\ i}. Because
    \texttt{lookup} is injective on distinct vectors, we have
    \texttt{i\ ≡\ j}. We then use the induction hypothesis to show that
    \texttt{lookup\ As\ i\ \textless{}:\ lookup\ As\ i}, noting that the
    size of \texttt{lookup\ As\ i} is less-than-or-equal to the size of
    \texttt{As}, which is one smaller than the size of
    \texttt{{\ ls\ ⦂\ As\ }}.
  \end{itemize}
\end{itemize}

The following corollary packages up reflexivity for ease of use.

\begin{fence}
\begin{code}
<:-refl : ∀{A} → A <: A
<:-refl {A} = <:-refl-aux {ty-size A}{A}{≤-refl}
\end{code}
\end{fence}

\hypertarget{subtyping-is-transitive}{%
\section{Subtyping is transitive}\label{subtyping-is-transitive}}

\begin{fence}
\begin{code}
<:-trans : ∀{A B C}
    → A <: B   →   B <: C
      -------------------
    → A <: C
<:-trans {A₁ ⇒ A₂} {B₁ ⇒ B₂} {C₁ ⇒ C₂} (<:-fun A₁<:B₁ A₂<:B₂) (<:-fun B₁<:C₁ B₂<:C₂) =
    <:-fun (<:-trans B₁<:C₁ A₁<:B₁) (<:-trans A₂<:B₂ B₂<:C₂)
<:-trans {.`ℕ} {`ℕ} {.`ℕ} <:-nat <:-nat = <:-nat
<:-trans {{ ls ⦂ As }{d1} } {{ ms ⦂ Bs } {d2} } {{ ns ⦂ Cs } {d3} }
    (<:-rcd ms⊆ls As<:Bs) (<:-rcd ns⊆ms Bs<:Cs) =
    <:-rcd (⊆-trans ns⊆ms ms⊆ls) As<:Cs
    where
    As<:Cs : ls ⦂ As <: ns ⦂ Cs
    As<:Cs {i}{j} ls[j]=ns[i]
        with lookup-⊆ {i = i} ns⊆ms
    ... | ⟨ k , ns[i]=ms[k] ⟩ =
        let As[j]<:Bs[k] = As<:Bs {k}{j} (trans ls[j]=ns[i] ns[i]=ms[k]) in
        let Bs[k]<:Cs[i] = Bs<:Cs {i}{k} (sym ns[i]=ms[k]) in
        <:-trans As[j]<:Bs[k] Bs[k]<:Cs[i]
\end{code}
\end{fence}

The proof is by induction on the derivations of
\texttt{A\ \textless{}:\ B} and \texttt{B\ \textless{}:\ C}.

\begin{itemize}
\item
  If both derivations end with \texttt{\textless{}:-nat}: then we
  immediately conclude that \texttt{ℕ\ \textless{}:\ ℕ}.
\item
  If both derivations end with \texttt{\textless{}:-fun}: we have
  \texttt{A₁\ ⇒\ A₂\ \textless{}:\ B₁\ ⇒\ B₂} and
  \texttt{B₁\ ⇒\ B₂\ \textless{}:\ C₁\ ⇒\ C₂}. So
  \texttt{A₁\ \textless{}:\ B₁} and \texttt{B₁\ \textless{}:\ C₁}, thus
  \texttt{A₁\ \textless{}:\ C₁} by the induction hypothesis. We also
  have \texttt{A₂\ \textless{}:\ B₂} and \texttt{B₂\ \textless{}:\ C₂},
  so by the induction hypothesis we have \texttt{A₂\ \textless{}:\ C₂}.
  We conclude that \texttt{A₁\ ⇒\ A₂\ \textless{}:\ C₁\ ⇒\ C₂} by rule
  \texttt{\textless{}:-fun}.
\item
  If both derivations end with \texttt{\textless{}:-rcd}, so we have
  \texttt{{\ ls\ ⦂\ As\ }\ \textless{}:\ {\ ms\ ⦂\ Bs\ }} and
  \texttt{{\ ms\ ⦂\ Bs\ }\ \textless{}:\ {\ ns\ ⦂\ Cs\ }}. From
  \texttt{ls\ ⊆\ ms} and \texttt{ms\ ⊆\ ns} we have \texttt{ls\ ⊆\ ns}
  because \texttt{⊆} is transitive. Next we need to prove that
  \texttt{ls\ ⦂\ As\ \textless{}:\ ns\ ⦂\ Cs}. Suppose
  \texttt{lookup\ ls\ j\ ≡\ lookup\ ns\ i} for an arbitrary \texttt{i}
  and \texttt{j}. We need to prove that
  \texttt{lookup\ As\ j\ \textless{}:\ lookup\ Cs\ i}. By the induction
  hypothesis, it suffices to show that
  \texttt{lookup\ As\ j\ \textless{}:\ lookup\ Bs\ k} and
  \texttt{lookup\ Bs\ k\ \textless{}:\ lookup\ Cs\ i} for some
  \texttt{k}. We can obtain the former from
  \texttt{{\ ls\ ⦂\ As\ }\ \textless{}:\ {\ ms\ ⦂\ Bs\ }} if we can
  prove that \texttt{lookup\ ls\ j\ ≡\ lookup\ ms\ k}. We already have
  \texttt{lookup\ ls\ j\ ≡\ lookup\ ns\ i} and we obtain
  \texttt{lookup\ ns\ i\ ≡\ lookup\ ms\ k} by use of the lemma
  \texttt{lookup-⊆}, noting that \texttt{ns\ ⊆\ ms}. We can obtain the
  later, that \texttt{lookup\ Bs\ k\ \textless{}:\ lookup\ Cs\ i}, from
  \texttt{{\ ms\ ⦂\ Bs\ }\ \textless{}:\ {\ ns\ ⦂\ Cs\ }}. It
  suffices to show that \texttt{lookup\ ms\ k\ ≡\ lookup\ ns\ i}, which
  we have by symmetry.
\end{itemize}

\hypertarget{contexts}{%
\section{Contexts}\label{contexts}}

We choose to represent variables with de Bruijn indices, so contexts are
sequences of types.

\begin{fence}
\begin{code}
data Context : Set where
  ∅   : Context
  _,_ : Context → Type → Context
\end{code}
\end{fence}

\hypertarget{variables-and-the-lookup-judgment}{%
\section{Variables and the lookup
judgment}\label{variables-and-the-lookup-judgment}}

The lookup judgment is a three-place relation, with a context, a de
Bruijn index, and a type.

\begin{fence}
\begin{code}
data _∋_⦂_ : Context → ℕ → Type → Set where

  Z : ∀ {Γ A}
      ------------------
    → Γ , A ∋ 0 ⦂ A

  S : ∀ {Γ x A B}
    → Γ ∋ x ⦂ A
      ------------------
    → Γ , B ∋ (suc x) ⦂ A
\end{code}
\end{fence}

\begin{itemize}
\item
  The index \texttt{0} has the type at the front of the context.
\item
  For the index \texttt{suc\ x}, we recursively look up its type in the
  remaining context \texttt{Γ}.
\end{itemize}

\hypertarget{terms-and-the-typing-judgment}{%
\section{Terms and the typing
judgment}\label{terms-and-the-typing-judgment}}

As mentioned above, variables are de Bruijn indices, which we represent
with natural numbers.

\begin{fence}
\begin{code}
Id : Set
Id = ℕ
\end{code}
\end{fence}

Our terms are extrinsic, so we define a \texttt{Term} data type similar
to the one in the \protect\hyperlink{Lambda}{Lambda} chapter, but
adapted for de Bruijn indices. The two new term constructors are for
record creation and field access.

\begin{fence}
\begin{code}
data Term : Set where
  `_                      : Id → Term
  ƛ_                      : Term → Term
  _·_                     : Term → Term → Term
  `zero                   : Term
  `suc_                   : Term → Term
  case_[zero⇒_|suc⇒_]     : Term → Term → Term → Term
  μ_                      : Term → Term
  {_:=_}                 : {n : ℕ} (ls : Vec Name n) (Ms : Vec Term n) → Term
  _#_                     : Term → Name → Term
\end{code}
\end{fence}

In a record \texttt{{\ ls\ :=\ Ms\ }}, we refer to the vector of terms
\texttt{Ms} as the \emph{field initializers}.

The typing judgment takes the form \texttt{Γ\ ⊢\ M\ ⦂\ A} and relies on
an auxiliary judgment \texttt{Γ\ ⊢*\ Ms\ ⦂\ As} for typing a vector of
terms.

\begin{fence}
\begin{code}
data _⊢*_⦂_ : Context → ∀ {n} → Vec Term n → Vec Type n → Set

data _⊢_⦂_ : Context → Term → Type → Set where

  ⊢` : ∀ {Γ x A}
    → Γ ∋ x ⦂ A
      -----------
    → Γ ⊢ ` x ⦂ A

  ⊢ƛ  : ∀ {Γ A B N}
    → (Γ , A) ⊢ N ⦂ B
      ---------------
    → Γ ⊢ (ƛ N) ⦂ A ⇒ B

  ⊢· : ∀ {Γ A B L M}
    → Γ ⊢ L ⦂ A ⇒ B
    → Γ ⊢ M ⦂ A
      -------------
    → Γ ⊢ L · M ⦂ B

  ⊢zero : ∀ {Γ}
      --------------
    → Γ ⊢ `zero ⦂ `ℕ

  ⊢suc : ∀ {Γ M}
    → Γ ⊢ M ⦂ `ℕ
      ---------------
    → Γ ⊢ `suc M ⦂ `ℕ

  ⊢case : ∀ {Γ A L M N}
    → Γ ⊢ L ⦂ `ℕ
    → Γ ⊢ M ⦂ A
    → Γ , `ℕ ⊢ N ⦂ A
      ---------------------------------
    → Γ ⊢ case L [zero⇒ M |suc⇒ N ] ⦂ A

  ⊢μ : ∀ {Γ A M}
    → Γ , A ⊢ M ⦂ A
      -------------
    → Γ ⊢ μ M ⦂ A

  ⊢rcd : ∀ {Γ n}{Ms : Vec Term n}{As : Vec Type n}{ls : Vec Name n}
     → Γ ⊢* Ms ⦂ As
     → (d : distinct ls)
     → Γ ⊢ { ls := Ms } ⦂ { ls ⦂ As } {d}


  ⊢# : ∀ {n : ℕ}{Γ A M l}{ls : Vec Name n}{As : Vec Type n}{i}{d}
     → Γ ⊢ M ⦂ { ls ⦂ As }{d}
     → lookup ls i ≡ l
     → lookup As i ≡ A
     → Γ ⊢ M # l ⦂ A

  ⊢<: : ∀{Γ M A B}
    → Γ ⊢ M ⦂ A   → A <: B
      --------------------
    → Γ ⊢ M ⦂ B

data _⊢*_⦂_ where
  ⊢*-[] : ∀{Γ} → Γ ⊢* [] ⦂ []

  ⊢*-∷ : ∀ {n}{Γ M}{Ms : Vec Term n}{A}{As : Vec Type n}
     → Γ ⊢ M ⦂ A
     → Γ ⊢* Ms ⦂ As
     → Γ ⊢* (M ∷ Ms) ⦂ (A ∷ As)
\end{code}
\end{fence}

Most of the typing rules are adapted from those in the
\protect\hyperlink{Lambda}{Lambda} chapter. Here we discuss the three
new rules.

\begin{itemize}
\item
  Rule \texttt{⊢rcd}: A record is well-typed if the field initializers
  \texttt{Ms} have types \texttt{As}, to match the record type. Also,
  the vector of field names is required to be distinct.
\item
  Rule \texttt{⊢\#}: A field access is well-typed if the term \texttt{M}
  has record type, the field \texttt{l} is at some index \texttt{i} in
  the record type's vector of field names, and the result type
  \texttt{A} is at index \texttt{i} in the vector of field types.
\item
  Rule \texttt{⊢\textless{}:}: (Subsumption) If a term \texttt{M} has
  type \texttt{A}, and \texttt{A\ \textless{}:\ B}, then term \texttt{M}
  also has type \texttt{B}.
\end{itemize}

\hypertarget{renaming-and-substitution}{%
\section{Renaming and Substitution}\label{renaming-and-substitution}}

In preparation of defining the reduction rules for this language, we
define simultaneous substitution using the same recipe as in the
\protect\hyperlink{DeBruijn}{DeBruijn} chapter, but adapted to extrinsic
terms. Thus, the \texttt{subst} function is split into two parts: a raw
\texttt{subst} function that operators on terms and a
\texttt{subst-pres} lemma that proves that substitution preserves types.
We define \texttt{subst} in this section and postpone
\texttt{subst-pres} to the
\protect\hyperlink{subtyping-preservation}{Preservation} section.
Likewise for \texttt{rename}.

We begin by defining the \texttt{ext} function on renamings.

\begin{fence}
\begin{code}
ext : (Id → Id) → (Id → Id)
ext ρ 0      =  0
ext ρ (suc x)  =  suc (ρ x)
\end{code}
\end{fence}

The \texttt{rename} function is defined mutually with the auxiliary
\texttt{rename-vec} function, which is needed in the case for records.

\begin{fence}
\begin{code}
rename-vec : (Id → Id) → ∀{n} → Vec Term n → Vec Term n

rename : (Id → Id) → (Term → Term)
rename ρ (` x)          =  ` (ρ x)
rename ρ (ƛ N)          =  ƛ (rename (ext ρ) N)
rename ρ (L · M)        =  (rename ρ L) · (rename ρ M)
rename ρ (`zero)        =  `zero
rename ρ (`suc M)       =  `suc (rename ρ M)
rename ρ (case L [zero⇒ M |suc⇒ N ]) =
    case (rename ρ L) [zero⇒ rename ρ M |suc⇒ rename (ext ρ) N ]
rename ρ (μ N)          =  μ (rename (ext ρ) N)
rename ρ { ls := Ms } = { ls := rename-vec ρ Ms }
rename ρ (M # l)       = (rename ρ M) # l

rename-vec ρ [] = []
rename-vec ρ (M ∷ Ms) = rename ρ M ∷ rename-vec ρ Ms
\end{code}
\end{fence}

With the \texttt{rename} function in hand, we can define the
\texttt{exts} function on substitutions.

\begin{fence}
\begin{code}
exts : (Id → Term) → (Id → Term)
exts σ 0      =  ` 0
exts σ (suc x)  =  rename suc (σ x)
\end{code}
\end{fence}

We define \texttt{subst} mutually with the auxiliary \texttt{subst-vec}
function, which is needed in the case for records.

\begin{fence}
\begin{code}
subst-vec : (Id → Term) → ∀{n} → Vec Term n → Vec Term n

subst : (Id → Term) → (Term → Term)
subst σ (` k)          =  σ k
subst σ (ƛ N)          =  ƛ (subst (exts σ) N)
subst σ (L · M)        =  (subst σ L) · (subst σ M)
subst σ (`zero)        =  `zero
subst σ (`suc M)       =  `suc (subst σ M)
subst σ (case L [zero⇒ M |suc⇒ N ])
                       =  case (subst σ L) [zero⇒ subst σ M |suc⇒ subst (exts σ) N ]
subst σ (μ N)          =  μ (subst (exts σ) N)
subst σ { ls := Ms }  = { ls := subst-vec σ Ms }
subst σ (M # l)        = (subst σ M) # l

subst-vec σ [] = []
subst-vec σ (M ∷ Ms) = (subst σ M) ∷ (subst-vec σ Ms)
\end{code}
\end{fence}

As usual, we implement single substitution using simultaneous
substitution.

\begin{fence}
\begin{code}
subst-zero : Term → Id → Term
subst-zero M 0       =  M
subst-zero M (suc x) =  ` x

_[_] : Term → Term → Term
_[_] N M =  subst (subst-zero M) N
\end{code}
\end{fence}

\hypertarget{values}{%
\section{Values}\label{values}}

We extend the definition of \texttt{Value} to include a clause for
records. In a call-by-value language, a record is usually only
considered a value if all its field initializers are values. Here we
instead treat records in a lazy fashion, declaring any record to be a
value, to save on some extra bookkeeping.

\begin{fence}
\begin{code}
data Value : Term → Set where

  V-ƛ : ∀ {N}
      -----------
    → Value (ƛ N)

  V-zero :
      -------------
      Value (`zero)

  V-suc : ∀ {V}
    → Value V
      --------------
    → Value (`suc V)

  V-rcd : ∀{n}{ls : Vec Name n}{Ms : Vec Term n}
    → Value { ls := Ms }
\end{code}
\end{fence}

\hypertarget{reduction}{%
\section{Reduction}\label{reduction}}

The following datatype \texttt{\_—→\_} defines the reduction relation
for the STLC with records. We discuss the two new rules for records in
the following paragraph.

\begin{fence}
\begin{code}
data _—→_ : Term → Term → Set where

  ξ-·₁ : ∀ {L L′ M : Term}
    → L —→ L′
      ---------------
    → L · M —→ L′ · M

  ξ-·₂ : ∀ {V  M M′ : Term}
    → Value V
    → M —→ M′
      ---------------
    → V · M —→ V · M′

  β-ƛ : ∀ {N W : Term}
    → Value W
      --------------------
    → (ƛ N) · W —→ N [ W ]

  ξ-suc : ∀ {M M′ : Term}
    → M —→ M′
      -----------------
    → `suc M —→ `suc M′

  ξ-case : ∀ {L L′ M N : Term}
    → L —→ L′
      -------------------------------------------------------
    → case L [zero⇒ M |suc⇒ N ] —→ case L′ [zero⇒ M |suc⇒ N ]

  β-zero :  ∀ {M N : Term}
      ----------------------------------
    → case `zero [zero⇒ M |suc⇒ N ] —→ M

  β-suc : ∀ {V M N : Term}
    → Value V
      -------------------------------------------
    → case (`suc V) [zero⇒ M |suc⇒ N ] —→ N [ V ]

  β-μ : ∀ {N : Term}
      ----------------
    → μ N —→ N [ μ N ]

  ξ-# : ∀ {M M′ : Term}{l}
    → M —→ M′
    → M # l —→ M′ # l

  β-# : ∀ {n}{ls : Vec Name n}{Ms : Vec Term n} {l}{j : Fin n}
    → lookup ls j ≡ l
      ---------------------------------
    → { ls := Ms } # l —→  lookup Ms j
\end{code}
\end{fence}

We have just two new reduction rules: * Rule \texttt{ξ-\#}: A field
access expression \texttt{M\ \#\ l} reduces to \texttt{M′\ \#\ l}
provided that \texttt{M} reduces to \texttt{M′}.

\begin{itemize}
\tightlist
\item
  Rule \texttt{β-\#}: When field access is applied to a record, and if
  the label \texttt{l} is at position \texttt{j} in the vector of field
  names, then result is the term at position \texttt{j} in the field
  initializers.
\end{itemize}

\hypertarget{canonical-forms}{%
\section{Canonical Forms}\label{canonical-forms}}

As in the \protect\hyperlink{Properties}{Properties} chapter, we define
a \texttt{Canonical\ V\ ⦂\ A} relation that characterizes the well-typed
values. The presence of the subsumption rule impacts its definition
because we must allow the type of the value \texttt{V} to be a subtype
of \texttt{A}.

\begin{fence}
\begin{code}
data Canonical_⦂_ : Term → Type → Set where

  C-ƛ : ∀ {N A B C D}
    →  ∅ , A ⊢ N ⦂ B
    → A ⇒ B <: C ⇒ D
      -------------------------
    → Canonical (ƛ N) ⦂ (C ⇒ D)

  C-zero :
      --------------------
      Canonical `zero ⦂ `ℕ

  C-suc : ∀ {V}
    → Canonical V ⦂ `ℕ
      ---------------------
    → Canonical `suc V ⦂ `ℕ

  C-rcd : ∀{n m}{ls : Vec Name n}{ks : Vec Name m}{Ms As Bs}{dls}
    → ∅ ⊢* Ms ⦂ As
    → (dks : distinct ks)
    → { ks ⦂ As }{dks}  <: { ls ⦂ Bs }{dls}
    → Canonical { ks := Ms } ⦂ { ls ⦂ Bs } {dls}
\end{code}
\end{fence}

Every closed, well-typed value is canonical:

\begin{fence}
\begin{code}
canonical : ∀ {V A}
  → ∅ ⊢ V ⦂ A
  → Value V
    -----------
  → Canonical V ⦂ A
canonical (⊢` ())          ()
canonical (⊢ƛ ⊢N)          V-ƛ         =  C-ƛ ⊢N <:-refl
canonical (⊢· ⊢L ⊢M)       ()
canonical ⊢zero            V-zero      =  C-zero
canonical (⊢suc ⊢V)        (V-suc VV)  =  C-suc (canonical ⊢V VV)
canonical (⊢case ⊢L ⊢M ⊢N) ()
canonical (⊢μ ⊢M)          ()
canonical (⊢rcd ⊢Ms d) VV = C-rcd {dls = d} ⊢Ms d <:-refl
canonical (⊢<: ⊢V <:-nat) VV = canonical ⊢V VV
canonical (⊢<: ⊢V (<:-fun {A}{B}{C}{D} C<:A B<:D)) VV
    with canonical ⊢V VV
... | C-ƛ {N}{A′}{B′}{A}{B} ⊢N  AB′<:AB = C-ƛ ⊢N (<:-trans AB′<:AB (<:-fun C<:A B<:D))
canonical (⊢<: ⊢V (<:-rcd {ks = ks}{ls = ls}{d2 = dls} ls⊆ks ls⦂Ss<:ks⦂Ts)) VV
    with canonical ⊢V VV
... | C-rcd {ks = ks′} ⊢Ms dks′ As<:Ss =
      C-rcd {dls = distinct-relevant dls} ⊢Ms dks′ (<:-trans As<:Ss (<:-rcd ls⊆ks ls⦂Ss<:ks⦂Ts))
\end{code}
\end{fence}

The case for subsumption (\texttt{⊢\textless{}:}) is interesting. We
proceed by cases on the derivation of subtyping.

\begin{itemize}
\item
  If the last rule is \texttt{\textless{}:-nat}, then we have
  \texttt{∅\ ⊢\ V\ ⦂\ ℕ} and the induction hypothesis gives us
  \texttt{Canonical\ V\ ⦂\ ℕ}.
\item
  If the last rule is \texttt{\textless{}:-fun}, then we have
  \texttt{A\ ⇒\ B\ \textless{}:\ C\ ⇒\ D} and
  \texttt{∅\ ⊢\ ƛ\ N\ ⦂\ A\ ⇒\ B}. By the induction hypothesis, we have
  \texttt{∅\ ,\ A′\ ⊢\ N\ ⦂\ B′} and
  \texttt{A′\ ⇒\ B′\ \textless{}:\ A\ ⇒\ B} for some \texttt{A′} and
  \texttt{B′}. We conclude that \texttt{Canonical\ (ƛ\ N)\ ⦂\ C\ ⇒\ D}
  by rule \texttt{C-ƛ} and the transitivity of subtyping.
\item
  If the last rule is \texttt{\textless{}:-rcd}, then we have
  \texttt{{\ ls\ ⦂\ Ss\ }\ \textless{}:\ {\ ks\ ⦂\ Ts\ }} and
  \texttt{∅\ ⊢\ {\ ks′\ :=\ Ms\ }\ ⦂\ {\ ks\ ⦂\ Ss\ }}. By the
  induction hypothesis, we have \texttt{∅\ ⊢*\ Ms\ ⦂\ As},
  \texttt{distinct\ ks′}, and
  \texttt{{\ ks′\ ⦂\ As\ }\ \textless{}:\ {\ ks\ ⦂\ Ss\ }}. We
  conclude that
  \texttt{Canonical\ {\ ks′\ :=\ Ms\ }\ ⦂\ {\ ks\ ⦂\ Ts\ }} by rule
  \texttt{C-rcd} and the transitivity of subtyping.
\end{itemize}

If a term is canonical, then it is also a value.

\begin{fence}
\begin{code}
value : ∀ {M A}
  → Canonical M ⦂ A
    ----------------
  → Value M
value (C-ƛ _ _)     = V-ƛ
value C-zero        = V-zero
value (C-suc CM)    = V-suc (value CM)
value (C-rcd _ _ _) = V-rcd
\end{code}
\end{fence}

A canonical value is a well-typed value.

\begin{fence}
\begin{code}
typed : ∀ {V A}
  → Canonical V ⦂ A
    ---------------
  → ∅ ⊢ V ⦂ A
typed (C-ƛ ⊢N AB<:CD) = ⊢<: (⊢ƛ ⊢N) AB<:CD
typed C-zero = ⊢zero
typed (C-suc c) = ⊢suc (typed c)
typed (C-rcd ⊢Ms dks As<:Bs) = ⊢<: (⊢rcd ⊢Ms dks) As<:Bs
\end{code}
\end{fence}

\hypertarget{progress}{%
\section{\texorpdfstring{Progress }{Progress }}\label{progress}}

The Progress theorem states that a well-typed term may either take a
reduction step or it is already a value. The proof of Progress is like
the one in the \protect\hyperlink{Properties}{Properties}; it proceeds
by induction on the typing derivation and most of the cases remain the
same. Below we discuss the new cases for records and subsumption.

\begin{fence}
\begin{code}
data Progress (M : Term) : Set where

  step : ∀ {N}
    → M —→ N
      ----------
    → Progress M

  done :
      Value M
      ----------
    → Progress M
\end{code}
\end{fence}

\begin{fence}
\begin{code}
progress : ∀ {M A}
  → ∅ ⊢ M ⦂ A
    ----------
  → Progress M
progress (⊢` ())
progress (⊢ƛ ⊢N)                            = done V-ƛ
progress (⊢· ⊢L ⊢M)
    with progress ⊢L
... | step L—→L′                            = step (ξ-·₁ L—→L′)
... | done VL
        with progress ⊢M
...     | step M—→M′                        = step (ξ-·₂ VL M—→M′)
...     | done VM
        with canonical ⊢L VL
...     | C-ƛ ⊢N _                          = step (β-ƛ VM)
progress ⊢zero                              =  done V-zero
progress (⊢suc ⊢M) with progress ⊢M
...  | step M—→M′                           =  step (ξ-suc M—→M′)
...  | done VM                              =  done (V-suc VM)
progress (⊢case ⊢L ⊢M ⊢N) with progress ⊢L
... | step L—→L′                            =  step (ξ-case L—→L′)
... | done VL with canonical ⊢L VL
...   | C-zero                              =  step β-zero
...   | C-suc CL                            =  step (β-suc (value CL))
progress (⊢μ ⊢M)                            =  step β-μ
progress (⊢# {n}{Γ}{A}{M}{l}{ls}{As}{i}{d} ⊢M ls[i]=l As[i]=A)
    with progress ⊢M
... | step M—→M′                            =  step (ξ-# M—→M′)
... | done VM
    with canonical ⊢M VM
... | C-rcd {ks = ms}{As = Bs} ⊢Ms _ (<:-rcd ls⊆ms _)
    with lookup-⊆ {i = i} ls⊆ms
... | ⟨ k , ls[i]=ms[k] ⟩                   =  step (β-# {j = k}(trans (sym ls[i]=ms[k]) ls[i]=l))
progress (⊢rcd x d)                         =  done V-rcd
progress (⊢<: {A = A}{B} ⊢M A<:B)           =  progress ⊢M
\end{code}
\end{fence}

\begin{itemize}
\item
  Case \texttt{⊢\#}: We have \texttt{Γ\ ⊢\ M\ ⦂\ {\ ls\ ⦂\ As\ }},
  \texttt{lookup\ ls\ i\ ≡\ l}, and \texttt{lookup\ As\ i\ ≡\ A}. By the
  induction hypothesis, either \texttt{M\ —→\ M′} or \texttt{M} is a
  value. In the later case we conclude that
  \texttt{M\ \#\ l\ —→\ M′\ \#\ l} by rule \texttt{ξ-\#}. On the other
  hand, if \texttt{M} is a value, we invoke the canonical forms lemma to
  show that \texttt{M} has the form \texttt{{\ ms\ :=\ Ms\ }} where
  \texttt{∅\ ⊢*\ Ms\ ⦂\ Bs} and \texttt{ls\ ⊆\ ms}. By lemma
  \texttt{lookup-⊆}, we have \texttt{lookup\ ls\ i\ ≡\ lookup\ ms\ k}
  for some \texttt{k}. Thus, we have \texttt{lookup\ ms\ k\ ≡\ l} and we
  conclude \texttt{{\ ms\ :=\ Ms\ }\ \#\ l\ —→\ lookup\ Ms\ k} by rule
  \texttt{β-\#}.
\item
  Case \texttt{⊢rcd}: we immediately characterize the record as a value.
\item
  Case \texttt{⊢\textless{}:}: we invoke the induction hypothesis on
  sub-derivation of \texttt{Γ\ ⊢\ M\ ⦂\ A}.
\end{itemize}

\hypertarget{preservation}{%
\section{\texorpdfstring{Preservation
}{Preservation }}\label{preservation}}

In this section we prove that when a well-typed term takes a reduction
step, the result is another well-typed term with the same type.

As mentioned earlier, we need to prove that substitution preserve types,
which in turn requires that renaming preserves types. The proofs of
these lemmas are adapted from the intrinsic versions of the
\texttt{ext}, \texttt{rename}, \texttt{exts}, and \texttt{subst}
functions in the \protect\hyperlink{DeBruijn}{DeBruijn} chapter.

We define the following abbreviation for a \emph{well-typed renaming}
from Γ to Δ, that is, a renaming that sends variables in Γ to variables
in Δ with the same type.

\begin{fence}
\begin{code}
_⦂ᵣ_⇒_ : (Id → Id) → Context → Context → Set
ρ ⦂ᵣ Γ ⇒ Δ = ∀ {x A} → Γ ∋ x ⦂ A → Δ ∋ ρ x ⦂ A
\end{code}
\end{fence}

The \texttt{ext} function takes a well-typed renaming from Γ to Δ and
extends it to become a renaming from (Γ , B) to (Δ , B).

\begin{fence}
\begin{code}
ext-pres : ∀ {Γ Δ ρ B}
  → ρ ⦂ᵣ Γ ⇒ Δ
    --------------------------------
  → ext ρ ⦂ᵣ (Γ , B) ⇒ (Δ , B)
ext-pres {ρ = ρ } ρ⦂ Z = Z
ext-pres {ρ = ρ } ρ⦂ (S {x = x} ∋x) =  S (ρ⦂ ∋x)
\end{code}
\end{fence}

Next we prove that both \texttt{rename} and \texttt{rename-vec} preserve
types. We use the \texttt{ext-pres} lemma in each of the cases with a
variable binding: \texttt{⊢ƛ}, \texttt{⊢μ}, and \texttt{⊢case}.

\begin{fence}
\begin{code}
ren-vec-pres : ∀ {Γ Δ ρ}{n}{Ms : Vec Term n}{As : Vec Type n}
  → ρ ⦂ᵣ Γ ⇒ Δ  →  Γ ⊢* Ms ⦂ As  →  Δ ⊢* rename-vec ρ Ms ⦂ As

rename-pres : ∀ {Γ Δ ρ M A}
  → ρ ⦂ᵣ Γ ⇒ Δ
  → Γ ⊢ M ⦂ A
    ------------------
  → Δ ⊢ rename ρ M ⦂ A
rename-pres ρ⦂ (⊢` ∋w)           =  ⊢` (ρ⦂ ∋w)
rename-pres {ρ = ρ} ρ⦂ (⊢ƛ ⊢N)   =  ⊢ƛ (rename-pres {ρ = ext ρ} (ext-pres {ρ = ρ} ρ⦂) ⊢N)
rename-pres {ρ = ρ} ρ⦂ (⊢· ⊢L ⊢M)=  ⊢· (rename-pres {ρ = ρ} ρ⦂ ⊢L) (rename-pres {ρ = ρ} ρ⦂ ⊢M)
rename-pres {ρ = ρ} ρ⦂ (⊢μ ⊢M)   =  ⊢μ (rename-pres {ρ = ext ρ} (ext-pres {ρ = ρ} ρ⦂) ⊢M)
rename-pres ρ⦂ (⊢rcd ⊢Ms dls)    = ⊢rcd (ren-vec-pres ρ⦂ ⊢Ms ) dls
rename-pres {ρ = ρ} ρ⦂ (⊢# {d = d} ⊢M lif liA) = ⊢# {d = d}(rename-pres {ρ = ρ} ρ⦂ ⊢M) lif liA
rename-pres {ρ = ρ} ρ⦂ (⊢<: ⊢M lt) = ⊢<: (rename-pres {ρ = ρ} ρ⦂ ⊢M) lt
rename-pres ρ⦂ ⊢zero               = ⊢zero
rename-pres ρ⦂ (⊢suc ⊢M)           = ⊢suc (rename-pres ρ⦂ ⊢M)
rename-pres ρ⦂ (⊢case ⊢L ⊢M ⊢N)    =
    ⊢case (rename-pres ρ⦂ ⊢L) (rename-pres ρ⦂ ⊢M) (rename-pres (ext-pres ρ⦂) ⊢N)

ren-vec-pres ρ⦂ ⊢*-[] = ⊢*-[]
ren-vec-pres {ρ = ρ} ρ⦂ (⊢*-∷ ⊢M ⊢Ms) =
  let IH = ren-vec-pres {ρ = ρ} ρ⦂ ⊢Ms in
  ⊢*-∷ (rename-pres {ρ = ρ} ρ⦂ ⊢M) IH
\end{code}
\end{fence}

A \emph{well-typed substitution} from Γ to Δ sends variables in Γ to
terms of the same type in the context Δ.

\begin{fence}
\begin{code}
_⦂_⇒_ : (Id → Term) → Context → Context → Set
σ ⦂ Γ ⇒ Δ = ∀ {A x} → Γ ∋ x ⦂ A → Δ ⊢ subst σ (` x) ⦂ A
\end{code}
\end{fence}

The \texttt{exts} function sends well-typed substitutions from Γ to Δ to
well-typed substitutions from (Γ , B) to (Δ , B). In the case for
\texttt{S}, we note that
\texttt{exts\ σ\ (suc\ x)\ ≡\ rename\ sub\ (σ\ x)}, so we need to prove
\texttt{Δ\ ,\ B\ ⊢\ rename\ suc\ (σ\ x)\ ⦂\ A}, which we obtain by the
\texttt{rename-pres} lemma.

\begin{fence}
\begin{code}
exts-pres : ∀ {Γ Δ σ B}
  → σ ⦂ Γ ⇒ Δ
    --------------------------------
  → exts σ ⦂ (Γ , B) ⇒ (Δ , B)
exts-pres {σ = σ} σ⦂ Z = ⊢` Z
exts-pres {σ = σ} σ⦂ (S {x = x} ∋x) = rename-pres {ρ = suc} S (σ⦂ ∋x)
\end{code}
\end{fence}

Now we prove that both \texttt{subst} and \texttt{subst-vec} preserve
types. We use the \texttt{exts-pres} lemma in each of the cases with a
variable binding: \texttt{⊢ƛ}, \texttt{⊢μ}, and \texttt{⊢case}.

\begin{fence}
\begin{code}
subst-vec-pres : ∀ {Γ Δ σ}{n}{Ms : Vec Term n}{A}
  → σ ⦂ Γ ⇒ Δ  →  Γ ⊢* Ms ⦂ A  →  Δ ⊢* subst-vec σ Ms ⦂ A

subst-pres : ∀ {Γ Δ σ N A}
  → σ ⦂ Γ ⇒ Δ
  → Γ ⊢ N ⦂ A
    -----------------
  → Δ ⊢ subst σ N ⦂ A
subst-pres σ⦂ (⊢` eq)            = σ⦂ eq
subst-pres {σ = σ} σ⦂ (⊢ƛ ⊢N)    = ⊢ƛ (subst-pres{σ = exts σ}(exts-pres {σ = σ} σ⦂) ⊢N)
subst-pres {σ = σ} σ⦂ (⊢· ⊢L ⊢M) = ⊢· (subst-pres{σ = σ} σ⦂ ⊢L) (subst-pres{σ = σ} σ⦂ ⊢M)
subst-pres {σ = σ} σ⦂ (⊢μ ⊢M)    = ⊢μ (subst-pres{σ = exts σ} (exts-pres{σ = σ} σ⦂) ⊢M)
subst-pres σ⦂ (⊢rcd ⊢Ms dls) = ⊢rcd (subst-vec-pres σ⦂ ⊢Ms ) dls
subst-pres {σ = σ} σ⦂ (⊢# {d = d} ⊢M lif liA) =
    ⊢# {d = d} (subst-pres {σ = σ} σ⦂ ⊢M) lif liA
subst-pres {σ = σ} σ⦂ (⊢<: ⊢N lt) = ⊢<: (subst-pres {σ = σ} σ⦂ ⊢N) lt
subst-pres σ⦂ ⊢zero = ⊢zero
subst-pres σ⦂ (⊢suc ⊢M) = ⊢suc (subst-pres σ⦂ ⊢M)
subst-pres σ⦂ (⊢case ⊢L ⊢M ⊢N) =
    ⊢case (subst-pres σ⦂ ⊢L) (subst-pres σ⦂ ⊢M) (subst-pres (exts-pres σ⦂) ⊢N)

subst-vec-pres σ⦂ ⊢*-[] = ⊢*-[]
subst-vec-pres {σ = σ} σ⦂ (⊢*-∷ ⊢M ⊢Ms) =
    ⊢*-∷ (subst-pres {σ = σ} σ⦂ ⊢M) (subst-vec-pres σ⦂ ⊢Ms)
\end{code}
\end{fence}

The fact that single substitution preserves types is a corollary of
\texttt{subst-pres}.

\begin{fence}
\begin{code}
substitution : ∀{Γ A B M N}
   → Γ ⊢ M ⦂ A
   → (Γ , A) ⊢ N ⦂ B
     ---------------
   → Γ ⊢ N [ M ] ⦂ B
substitution {Γ}{A}{B}{M}{N} ⊢M ⊢N = subst-pres {σ = subst-zero M} G ⊢N
    where
    G : ∀ {C : Type} {x : ℕ}
      → (Γ , A) ∋ x ⦂ C → Γ ⊢ subst (subst-zero M) (` x) ⦂ C
    G {C} {zero} Z = ⊢M
    G {C} {suc x} (S ∋x) = ⊢` ∋x
\end{code}
\end{fence}

We require just one last lemma before we get to the proof of
preservation. The following lemma establishes that field access
preserves types.

\begin{fence}
\begin{code}
field-pres : ∀{n}{As : Vec Type n}{A}{Ms : Vec Term n}{i : Fin n}
         → ∅ ⊢* Ms ⦂ As
         → lookup As i ≡ A
         → ∅ ⊢ lookup Ms i ⦂ A
field-pres {i = zero} (⊢*-∷ ⊢M ⊢Ms) refl = ⊢M
field-pres {i = suc i} (⊢*-∷ ⊢M ⊢Ms) As[i]=A = field-pres ⊢Ms As[i]=A
\end{code}
\end{fence}

The proof is by induction on the typing derivation.

\begin{itemize}
\item
  Case \texttt{⊢-*-{[}{]}}: This case yields a contradiction because
  \texttt{Fin\ 0} is uninhabitable.
\item
  Case \texttt{⊢-*-∷}: So we have \texttt{∅\ ⊢\ M\ ⦂\ B} and
  \texttt{∅\ ⊢*\ Ms\ ⦂\ As}. We proceed by cases on \texttt{i}.

  \begin{itemize}
  \item
    If it is \texttt{0}, then lookup yields term \texttt{M} and
    \texttt{A\ ≡\ B}, so we conclude that \texttt{∅\ ⊢\ M\ ⦂\ A}.
  \item
    If it is \texttt{suc\ i}, then we conclude by the induction
    hypothesis.
  \end{itemize}
\end{itemize}

We conclude this chapter with the proof of preservation. We discuss the
cases particular to records and subtyping in the paragraph following the
Agda proof.

\begin{fence}
\begin{code}
preserve : ∀ {M N A}
  → ∅ ⊢ M ⦂ A
  → M —→ N
    ----------
  → ∅ ⊢ N ⦂ A
preserve (⊢` ())
preserve (⊢ƛ ⊢N)                 ()
preserve (⊢· ⊢L ⊢M)              (ξ-·₁ L—→L′)     =  ⊢· (preserve ⊢L L—→L′) ⊢M
preserve (⊢· ⊢L ⊢M)              (ξ-·₂ VL M—→M′)  =  ⊢· ⊢L (preserve ⊢M M—→M′)
preserve (⊢· ⊢L ⊢M)              (β-ƛ VL)
    with canonical ⊢L V-ƛ
... | C-ƛ ⊢N (<:-fun CA BC)                       =  ⊢<: (substitution (⊢<: ⊢M CA) ⊢N) BC
preserve ⊢zero                   ()
preserve (⊢suc ⊢M)               (ξ-suc M—→M′)    =  ⊢suc (preserve ⊢M M—→M′)
preserve (⊢case ⊢L ⊢M ⊢N)        (ξ-case L—→L′)   =  ⊢case (preserve ⊢L L—→L′) ⊢M ⊢N
preserve (⊢case ⊢L ⊢M ⊢N)        (β-zero)         =  ⊢M
preserve (⊢case ⊢L ⊢M ⊢N)        (β-suc VV)
    with canonical ⊢L (V-suc VV)
... | C-suc CV                                    =  substitution (typed CV) ⊢N
preserve (⊢μ ⊢M)                 (β-μ)            =  substitution (⊢μ ⊢M) ⊢M
preserve (⊢# {d = d} ⊢M lsi Asi) (ξ-# M—→M′)      =  ⊢# {d = d} (preserve ⊢M M—→M′) lsi Asi
preserve (⊢# {ls = ls}{i = i} ⊢M refl refl) (β-# {ls = ks}{Ms}{j = j} ks[j]=l)
    with canonical ⊢M V-rcd
... | C-rcd {As = Bs} ⊢Ms dks (<:-rcd {ks = ks} ls⊆ks Bs<:As)
    with lookup-⊆ {i = i} ls⊆ks
... | ⟨ k , ls[i]=ks[k] ⟩
    with field-pres {i = k} ⊢Ms refl
... | ⊢Ms[k]⦂Bs[k]
    rewrite distinct-lookup-inj dks (trans ks[j]=l ls[i]=ks[k]) =
    let Ms[k]⦂As[i] = ⊢<: ⊢Ms[k]⦂Bs[k] (Bs<:As (sym ls[i]=ks[k])) in
    Ms[k]⦂As[i]
preserve (⊢<: ⊢M B<:A) M—→N                       =  ⊢<: (preserve ⊢M M—→N) B<:A
\end{code}
\end{fence}

Recall that the proof is by induction on the derivation of
\texttt{∅\ ⊢\ M\ ⦂\ A} with cases on \texttt{M\ —→\ N}.

\begin{itemize}
\item
  Case \texttt{⊢\#} and \texttt{ξ-\#}: So
  \texttt{∅\ ⊢\ M\ ⦂\ {\ ls\ ⦂\ As\ }}, \texttt{lookup\ ls\ i\ ≡\ l},
  \texttt{lookup\ As\ i\ ≡\ A}, and \texttt{M\ —→\ M′}. We apply the
  induction hypothesis to obtain \texttt{∅\ ⊢\ M′\ ⦂\ {\ ls\ ⦂\ As\ }}
  and then conclude by rule \texttt{⊢\#}.
\item
  Case \texttt{⊢\#} and \texttt{β-\#}. We have
  \texttt{∅\ ⊢\ {\ ks\ :=\ Ms\ }\ ⦂\ {\ ls\ ⦂\ As\ }},
  \texttt{lookup\ ls\ i\ ≡\ l}, \texttt{lookup\ As\ i\ ≡\ A},
  \texttt{lookup\ ks\ j\ ≡\ l}, and
  \texttt{{\ ks\ :=\ Ms\ }\ \#\ l\ —→\ lookup\ Ms\ j}. By the
  canonical forms lemma, we have \texttt{∅\ ⊢*\ Ms\ ⦂\ Bs},
  \texttt{ls\ ⊆\ ks} and \texttt{ks\ ⦂\ Bs\ \textless{}:\ ls\ ⦂\ As}. By
  lemma \texttt{lookup-⊆} we have
  \texttt{lookup\ ls\ i\ ≡\ lookup\ ks\ k} for some \texttt{k}. Also, we
  have \texttt{∅\ ⊢\ lookup\ Ms\ k\ ⦂\ lookup\ Bs\ k} by lemma
  \texttt{field-pres}. Then because
  \texttt{ks\ ⦂\ Bs\ \textless{}:\ ls\ ⦂\ As} and
  \texttt{lookup\ ks\ k\ ≡\ lookup\ ls\ i}, we have
  \texttt{lookup\ Bs\ k\ \textless{}:\ lookup\ As\ i}. So by rule
  \texttt{⊢\textless{}:} we have
  \texttt{∅\ ⊢\ lookup\ Ms\ k\ ⦂\ lookup\ As\ i}. Finally, because
  \texttt{lookup} is injective and
  \texttt{lookup\ ks\ j\ ≡\ lookup\ ks\ k}, we have \texttt{j\ ≡\ k} and
  conclude that \texttt{∅\ ⊢\ lookup\ Ms\ j\ ⦂\ lookup\ As\ i}.
\item
  Case \texttt{⊢\textless{}:}. We have \texttt{∅\ ⊢\ M\ ⦂\ B},
  \texttt{B\ \textless{}:\ A}, and \texttt{M\ —→\ N}. We apply the
  induction hypothesis to obtain \texttt{∅\ ⊢\ N\ ⦂\ B} and then
  subsumption to conclude that \texttt{∅\ ⊢\ N\ ⦂\ A}.
\end{itemize}

\hypertarget{exercise-intrinsic-records-stretch}{%
\subsubsection{\texorpdfstring{Exercise \texttt{intrinsic-records}
(stretch)}{Exercise intrinsic-records (stretch)}}\label{exercise-intrinsic-records-stretch}}

Formulate the language of this chapter, STLC with records, using
intrinsically typed terms. This requires taking an algorithmic approach
to the type system, which means that there is no subsumption rule and
instead subtyping is used in the elimination rules. For example, the
rule for function application would be:

\begin{myDisplay}
_·_ : ∀ {Γ A B C}
  → Γ ⊢ A ⇒ B
  → Γ ⊢ C
  → C <: A
    -------
  → Γ ⊢ B
\end{myDisplay}

\hypertarget{exercise-type-check-records-practice}{%
\subsubsection{\texorpdfstring{Exercise \texttt{type-check-records}
(practice)}{Exercise type-check-records (practice)}}\label{exercise-type-check-records-practice}}

Write a recursive function whose input is a \texttt{Term} and whose
output is a typing derivation for that term, if one exists.

\begin{myDisplay}
type-check : (M : Term) → (Γ : Context) → Maybe (Σ[ A ∈ Type ] Γ ⊢ M ⦂ A)
\end{myDisplay}

\hypertarget{exercise-variants-recommended}{%
\subsubsection{\texorpdfstring{Exercise \texttt{variants}
(recommended)}{Exercise variants (recommended)}}\label{exercise-variants-recommended}}

Add variants to the language of this chapter and update the proofs of
progress and preservation. The variant type is a generalization of a sum
type, similar to the way the record type is a generalization of product.
The following summarizes the treatment of variants in the book Types and
Programming Languages. A variant type is traditionally written:

\begin{myDisplay}
〈l₁:A₁, ..., lᵤ:Aᵤ〉
\end{myDisplay}

The term for introducing a variant is

\begin{myDisplay}
〈l=t〉
\end{myDisplay}

and the term for eliminating a variant is

\begin{myDisplay}
case L of 〈l₁=x₁〉 ⇒ M₁ | ... | 〈lᵤ=xᵤ〉 ⇒ Mᵤ
\end{myDisplay}

The typing rules for these terms are

\begin{myDisplay}
(T-Variant)
Γ ⊢ Mⱼ : Aⱼ
---------------------------------
Γ ⊢ 〈lⱼ=Mⱼ〉 : 〈l₁=A₁, ... , lᵤ=Aᵤ〉


(T-Case)
Γ ⊢ L : 〈l₁=A₁, ... , lᵤ=Aᵤ〉
∀ i ∈ 1..u,   Γ,xᵢ:Aᵢ ⊢ Mᵢ : B
---------------------------------------------------
Γ ⊢ case L of 〈l₁=x₁〉 ⇒ M₁ | ... | 〈lᵤ=xᵤ〉 ⇒ Mᵤ  : B
\end{myDisplay}

The non-algorithmic subtyping rules for variants are

\begin{myDisplay}
(S-VariantWidth)
------------------------------------------------------------
〈l₁=A₁, ..., lᵤ=Aᵤ〉   <:   〈l₁=A₁, ..., lᵤ=Aᵤ, ..., lᵤ₊ₓ=Aᵤ₊ₓ〉

(S-VariantDepth)
∀ i ∈ 1..u,    Aᵢ <: Bᵢ
---------------------------------------------
〈l₁=A₁, ..., lᵤ=Aᵤ〉   <:   〈l₁=B₁, ..., lᵤ=Bᵤ〉

(S-VariantPerm)
∀i∈1..u, ∃j∈1..u, kⱼ = lᵢ and Aⱼ = Bᵢ
----------------------------------------------
〈k₁=A₁, ..., kᵤ=Aᵤ〉   <:   〈l₁=B₁, ..., lᵤ=Bᵤ〉
\end{myDisplay}

Come up with an algorithmic subtyping rule for variant types.

\hypertarget{exercise--alternative-stretch}{%
\subsubsection{\texorpdfstring{Exercise
\texttt{\textless{}:-alternative}
(stretch)}{Exercise \textless:-alternative (stretch)}}\label{exercise--alternative-stretch}}

Revise this formalisation of records with subtyping (including proofs of
progress and preservation) to use the non-algorithmic subtyping rules
for Chapter 15 of Types and Programming Languages, which we list here:

\begin{myDisplay}
(S-RcdWidth)
--------------------------------------------------------------
{ l₁:A₁, ..., lᵤ:Aᵤ, ..., lᵤ₊ₓ:Aᵤ₊ₓ } <: { l₁:A₁, ..., lᵤ:Aᵤ }

(S-RcdDepth)
    ∀i∈1..u, Aᵢ <: Bᵢ
----------------------------------------------
{ l₁:A₁, ..., lᵤ:Aᵤ } <: { l₁:B₁, ..., lᵤ:Bᵤ }

(S-RcdPerm)
∀i∈1..u, ∃j∈1..u, kⱼ = lᵢ and Aⱼ = Bᵢ
----------------------------------------------
{ k₁:A₁, ..., kᵤ:Aᵤ } <: { l₁:B₁, ..., lᵤ:Bᵤ }
\end{myDisplay}

You will most likely need to prove inversion lemmas for the subtype
relation of the form:

\begin{myDisplay}
If S <: T₁ ⇒ T₂, then S ≡ S₁ ⇒ S₂, T₁ <: S₁, and S₂ <: T₂, for some S₁, S₂.

If S <: { lᵢ : Tᵢ | i ∈ 1..n }, then S ≡ { kⱼ : Sⱼ | j ∈ 1..m }
and { lᵢ | i ∈ 1..n } ⊆ { kⱼ | j ∈ 1..m }
and Sⱼ <: Tᵢ for every i and j such that lᵢ = kⱼ.
\end{myDisplay}

\hypertarget{references}{%
\section{References}\label{references}}

\begin{itemize}
\item
  John C. Reynolds. Using Category Theory to Design Implicit Conversions
  and Generic Operators. In Semantics-Directed Compiler Generation,
  1980. LNCS Volume 94.
\item
  Luca Cardelli. A semantics of multiple inheritance. In Semantics of
  Data Types, 1984. Springer.
\item
  Barbara H. Liskov and Jeannette M. Wing. A Behavioral Notion of
  Subtyping. In ACM Trans. Program. Lang. Syst. Volume 16, 1994.
\item
  Types and Programming Languages. Benjamin C. Pierce. The MIT Press.
  2002.
\end{itemize}

