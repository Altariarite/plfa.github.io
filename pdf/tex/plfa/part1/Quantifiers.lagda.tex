\hypertarget{Quantifiers}{%
\chapter{Quantifiers: Universals and existentials}\label{Quantifiers}}

\begin{fence}
\begin{code}
module plfa.part1.Quantifiers where
\end{code}
\end{fence}

This chapter introduces universal and existential quantification.

\hypertarget{imports}{%
\section{Imports}\label{imports}}

\begin{fence}
\begin{code}
import Relation.Binary.PropositionalEquality as Eq
open Eq using (_≡_; refl)
open import Data.Nat using (ℕ; zero; suc; _+_; _*_)
open import Relation.Nullary using (¬_)
open import Data.Product using (_×_; proj₁; proj₂) renaming (_,_ to ⟨_,_⟩)
open import Data.Sum using (_⊎_; inj₁; inj₂)
open import plfa.part1.Isomorphism using (_≃_; extensionality)
\end{code}
\end{fence}

\hypertarget{universals}{%
\section{Universals}\label{universals}}

We formalise universal quantification using the dependent function type,
which has appeared throughout this book. For instance, in Chapter
Induction we showed addition is associative:

\begin{myDisplay}
+-assoc : ∀ (m n p : ℕ) → (m + n) + p ≡ m + (n + p)
\end{myDisplay}

which asserts for all natural numbers \texttt{m}, \texttt{n}, and
\texttt{p} that \texttt{(m\ +\ n)\ +\ p\ ≡\ m\ +\ (n\ +\ p)} holds. It
is a dependent function, which given values for \texttt{m}, \texttt{n},
and \texttt{p} returns evidence for the corresponding equation.

In general, given a variable \texttt{x} of type \texttt{A} and a
proposition \texttt{B\ x} which contains \texttt{x} as a free variable,
the universally quantified proposition \texttt{∀\ (x\ :\ A)\ →\ B\ x}
holds if for every term \texttt{M} of type \texttt{A} the proposition
\texttt{B\ M} holds. Here \texttt{B\ M} stands for the proposition
\texttt{B\ x} with each free occurrence of \texttt{x} replaced by
\texttt{M}. Variable \texttt{x} appears free in \texttt{B\ x} but bound
in \texttt{∀\ (x\ :\ A)\ →\ B\ x}.

Evidence that \texttt{∀\ (x\ :\ A)\ →\ B\ x} holds is of the form

\begin{myDisplay}
λ (x : A) → N x
\end{myDisplay}

where \texttt{N\ x} is a term of type \texttt{B\ x}, and \texttt{N\ x}
and \texttt{B\ x} both contain a free variable \texttt{x} of type
\texttt{A}. Given a term \texttt{L} providing evidence that
\texttt{∀\ (x\ :\ A)\ →\ B\ x} holds, and a term \texttt{M} of type
\texttt{A}, the term \texttt{L\ M} provides evidence that \texttt{B\ M}
holds. In other words, evidence that \texttt{∀\ (x\ :\ A)\ →\ B\ x}
holds is a function that converts a term \texttt{M} of type \texttt{A}
into evidence that \texttt{B\ M} holds.

Put another way, if we know that \texttt{∀\ (x\ :\ A)\ →\ B\ x} holds
and that \texttt{M} is a term of type \texttt{A} then we may conclude
that \texttt{B\ M} holds:

\begin{fence}
\begin{code}
∀-elim : ∀ {A : Set} {B : A → Set}
  → (L : ∀ (x : A) → B x)
  → (M : A)
    -----------------
  → B M
∀-elim L M = L M
\end{code}
\end{fence}

As with \texttt{→-elim}, the rule corresponds to function application.

Functions arise as a special case of dependent functions, where the
range does not depend on a variable drawn from the domain. When a
function is viewed as evidence of implication, both its argument and
result are viewed as evidence, whereas when a dependent function is
viewed as evidence of a universal, its argument is viewed as an element
of a data type and its result is viewed as evidence of a proposition
that depends on the argument. This difference is largely a matter of
interpretation, since in Agda a value of a type and evidence of a
proposition are indistinguishable.

Dependent function types are sometimes referred to as dependent
products, because if \texttt{A} is a finite type with values
\texttt{x₁\ ,\ ⋯\ ,\ xₙ}, and if each of the types
\texttt{B\ x₁\ ,\ ⋯\ ,\ B\ xₙ} has \texttt{m₁\ ,\ ⋯\ ,\ mₙ} distinct
members, then \texttt{∀\ (x\ :\ A)\ →\ B\ x} has
\texttt{m₁\ *\ ⋯\ *\ mₙ} members. Indeed, sometimes the notation
\texttt{∀\ (x\ :\ A)\ →\ B\ x} is replaced by a notation such as
\texttt{Π{[}\ x\ ∈\ A\ {]}\ (B\ x)}, where \texttt{Π} stands for
product. However, we will stick with the name dependent function,
because (as we will see) dependent product is ambiguous.

\hypertarget{exercise--distrib--recommended}{%
\subsubsection{\texorpdfstring{Exercise \texttt{∀-distrib-×}
(recommended)}{Exercise ∀-distrib-× (recommended)}}\label{exercise--distrib--recommended}}

Show that universals distribute over conjunction:

\begin{fence}
\begin{code}
postulate
  ∀-distrib-× : ∀ {A : Set} {B C : A → Set} →
    (∀ (x : A) → B x × C x) ≃ (∀ (x : A) → B x) × (∀ (x : A) → C x)
\end{code}
\end{fence}

Compare this with the result (\texttt{→-distrib-×}) in Chapter
\protect\hyperlink{Connectives}{Connectives}.

\hypertarget{exercise--implies--practice}{%
\subsubsection{\texorpdfstring{Exercise \texttt{⊎∀-implies-∀⊎}
(practice)}{Exercise ⊎∀-implies-∀⊎ (practice)}}\label{exercise--implies--practice}}

Show that a disjunction of universals implies a universal of
disjunctions:

\begin{fence}
\begin{code}
postulate
  ⊎∀-implies-∀⊎ : ∀ {A : Set} {B C : A → Set} →
    (∀ (x : A) → B x) ⊎ (∀ (x : A) → C x)  →  ∀ (x : A) → B x ⊎ C x
\end{code}
\end{fence}

Does the converse hold? If so, prove; if not, explain why.

\hypertarget{exercise---practice}{%
\subsubsection{\texorpdfstring{Exercise \texttt{∀-×}
(practice)}{Exercise ∀-× (practice)}}\label{exercise---practice}}

Consider the following type.

\begin{fence}
\begin{code}
data Tri : Set where
  aa : Tri
  bb : Tri
  cc : Tri
\end{code}
\end{fence}

Let \texttt{B} be a type indexed by \texttt{Tri}, that is
\texttt{B\ :\ Tri\ →\ Set}. Show that \texttt{∀\ (x\ :\ Tri)\ →\ B\ x}
is isomorphic to \texttt{B\ aa\ ×\ B\ bb\ ×\ B\ cc}. Hint: you will need
to postulate a version of extensionality that works for dependent
functions.

\hypertarget{existentials}{%
\section{Existentials}\label{existentials}}

Given a variable \texttt{x} of type \texttt{A} and a proposition
\texttt{B\ x} which contains \texttt{x} as a free variable, the
existentially quantified proposition \texttt{Σ{[}\ x\ ∈\ A\ {]}\ B\ x}
holds if for some term \texttt{M} of type \texttt{A} the proposition
\texttt{B\ M} holds. Here \texttt{B\ M} stands for the proposition
\texttt{B\ x} with each free occurrence of \texttt{x} replaced by
\texttt{M}. Variable \texttt{x} appears free in \texttt{B\ x} but bound
in \texttt{Σ{[}\ x\ ∈\ A\ {]}\ B\ x}.

We formalise existential quantification by declaring a suitable
inductive type:

\begin{fence}
\begin{code}
data Σ (A : Set) (B : A → Set) : Set where
  ⟨_,_⟩ : (x : A) → B x → Σ A B
\end{code}
\end{fence}

We define a convenient syntax for existentials as follows:

\begin{fence}
\begin{code}
Σ-syntax = Σ
infix 2 Σ-syntax
syntax Σ-syntax A (λ x → B) = Σ[ x ∈ A ] B
\end{code}
\end{fence}

This is our first use of a syntax declaration, which specifies that the
term on the left may be written with the syntax on the right. The
special syntax is available only when the identifier \texttt{Σ-syntax}
is imported.

Evidence that \texttt{Σ{[}\ x\ ∈\ A\ {]}\ B\ x} holds is of the form
\texttt{⟨\ M\ ,\ N\ ⟩} where \texttt{M} is a term of type \texttt{A},
and \texttt{N} is evidence that \texttt{B\ M} holds.

Equivalently, we could also declare existentials as a record type:

\begin{fence}
\begin{code}
record Σ′ (A : Set) (B : A → Set) : Set where
  field
    proj₁′ : A
    proj₂′ : B proj₁′
\end{code}
\end{fence}

Here record construction

\begin{myDisplay}
record
  { proj₁′ = M
  ; proj₂′ = N
  }
\end{myDisplay}

corresponds to the term

\begin{myDisplay}
⟨ M , N ⟩
\end{myDisplay}

where \texttt{M} is a term of type \texttt{A} and \texttt{N} is a term
of type \texttt{B\ M}.

Products arise as a special case of existentials, where the second
component does not depend on a variable drawn from the first component.
When a product is viewed as evidence of a conjunction, both of its
components are viewed as evidence, whereas when it is viewed as evidence
of an existential, the first component is viewed as an element of a
datatype and the second component is viewed as evidence of a proposition
that depends on the first component. This difference is largely a matter
of interpretation, since in Agda a value of a type and evidence of a
proposition are indistinguishable.

Existentials are sometimes referred to as dependent sums, because if
\texttt{A} is a finite type with values \texttt{x₁\ ,\ ⋯\ ,\ xₙ}, and if
each of the types \texttt{B\ x₁\ ,\ ⋯\ B\ xₙ} has
\texttt{m₁\ ,\ ⋯\ ,\ mₙ} distinct members, then
\texttt{Σ{[}\ x\ ∈\ A\ {]}\ B\ x} has \texttt{m₁\ +\ ⋯\ +\ mₙ} members,
which explains the choice of notation for existentials, since \texttt{Σ}
stands for sum.

Existentials are sometimes referred to as dependent products, since
products arise as a special case. However, that choice of names is
doubly confusing, since universals also have a claim to the name
dependent product and since existentials also have a claim to the name
dependent sum.

A common notation for existentials is \texttt{∃} (analogous to
\texttt{∀} for universals). We follow the convention of the Agda
standard library, and reserve this notation for the case where the
domain of the bound variable is left implicit:

\begin{fence}
\begin{code}
∃ : ∀ {A : Set} (B : A → Set) → Set
∃ {A} B = Σ A B

∃-syntax = ∃
syntax ∃-syntax (λ x → B) = ∃[ x ] B
\end{code}
\end{fence}

The special syntax is available only when the identifier
\texttt{∃-syntax} is imported. We will tend to use this syntax, since it
is shorter and more familiar.

Given evidence that \texttt{∀\ x\ →\ B\ x\ →\ C} holds, where \texttt{C}
does not contain \texttt{x} as a free variable, and given evidence that
\texttt{∃{[}\ x\ {]}\ B\ x} holds, we may conclude that \texttt{C}
holds:

\begin{fence}
\begin{code}
∃-elim : ∀ {A : Set} {B : A → Set} {C : Set}
  → (∀ x → B x → C)
  → ∃[ x ] B x
    ---------------
  → C
∃-elim f ⟨ x , y ⟩ = f x y
\end{code}
\end{fence}

In other words, if we know for every \texttt{x} of type \texttt{A} that
\texttt{B\ x} implies \texttt{C}, and we know for some \texttt{x} of
type \texttt{A} that \texttt{B\ x} holds, then we may conclude that
\texttt{C} holds. This is because we may instantiate that proof that
\texttt{∀\ x\ →\ B\ x\ →\ C} to any value \texttt{x} of type \texttt{A}
and any \texttt{y} of type \texttt{B\ x}, and exactly such values are
provided by the evidence for \texttt{∃{[}\ x\ {]}\ B\ x}.

Indeed, the converse also holds, and the two together form an
isomorphism:

\begin{fence}
\begin{code}
∀∃-currying : ∀ {A : Set} {B : A → Set} {C : Set}
  → (∀ x → B x → C) ≃ (∃[ x ] B x → C)
∀∃-currying =
  record
    { to      =  λ{ f → λ{ ⟨ x , y ⟩ → f x y }}
    ; from    =  λ{ g → λ{ x → λ{ y → g ⟨ x , y ⟩ }}}
    ; from∘to =  λ{ f → refl }
    ; to∘from =  λ{ g → extensionality λ{ ⟨ x , y ⟩ → refl }}
    }
\end{code}
\end{fence}

The result can be viewed as a generalisation of currying. Indeed, the
code to establish the isomorphism is identical to what we wrote when
discussing \protect\hyperlink{Connectives-implication}{implication}.

\hypertarget{exercise--distrib--recommended-1}{%
\subsubsection{\texorpdfstring{Exercise \texttt{∃-distrib-⊎}
(recommended)}{Exercise ∃-distrib-⊎ (recommended)}}\label{exercise--distrib--recommended-1}}

Show that existentials distribute over disjunction:

\begin{fence}
\begin{code}
postulate
  ∃-distrib-⊎ : ∀ {A : Set} {B C : A → Set} →
    ∃[ x ] (B x ⊎ C x) ≃ (∃[ x ] B x) ⊎ (∃[ x ] C x)
\end{code}
\end{fence}

\hypertarget{exercise--implies--practice-1}{%
\subsubsection{\texorpdfstring{Exercise \texttt{∃×-implies-×∃}
(practice)}{Exercise ∃×-implies-×∃ (practice)}}\label{exercise--implies--practice-1}}

Show that an existential of conjunctions implies a conjunction of
existentials:

\begin{fence}
\begin{code}
postulate
  ∃×-implies-×∃ : ∀ {A : Set} {B C : A → Set} →
    ∃[ x ] (B x × C x) → (∃[ x ] B x) × (∃[ x ] C x)
\end{code}
\end{fence}

Does the converse hold? If so, prove; if not, explain why.

\hypertarget{exercise---practice-1}{%
\subsubsection{\texorpdfstring{Exercise \texttt{∃-⊎}
(practice)}{Exercise ∃-⊎ (practice)}}\label{exercise---practice-1}}

Let \texttt{Tri} and \texttt{B} be as in Exercise \texttt{∀-×}. Show
that \texttt{∃{[}\ x\ {]}\ B\ x} is isomorphic to
\texttt{B\ aa\ ⊎\ B\ bb\ ⊎\ B\ cc}.

\hypertarget{an-existential-example}{%
\section{An existential example}\label{an-existential-example}}

Recall the definitions of \texttt{even} and \texttt{odd} from Chapter
\protect\hyperlink{Relations}{Relations}:

\begin{fence}
\begin{code}
data even : ℕ → Set
data odd  : ℕ → Set

data even where

  even-zero : even zero

  even-suc : ∀ {n : ℕ}
    → odd n
      ------------
    → even (suc n)

data odd where
  odd-suc : ∀ {n : ℕ}
    → even n
      -----------
    → odd (suc n)
\end{code}
\end{fence}

A number is even if it is zero or the successor of an odd number, and
odd if it is the successor of an even number.

We will show that a number is even if and only if it is twice some other
number, and odd if and only if it is one more than twice some other
number. In other words, we will show:

\texttt{even\ n} iff \texttt{∃{[}\ m\ {]}\ (\ \ \ \ m\ *\ 2\ ≡\ n)}

\texttt{odd\ \ n} iff \texttt{∃{[}\ m\ {]}\ (1\ +\ m\ *\ 2\ ≡\ n)}

By convention, one tends to write constant factors first and to put the
constant term in a sum last. Here we've reversed each of those
conventions, because doing so eases the proof.

Here is the proof in the forward direction:

\begin{fence}
\begin{code}
even-∃ : ∀ {n : ℕ} → even n → ∃[ m ] (    m * 2 ≡ n)
odd-∃  : ∀ {n : ℕ} →  odd n → ∃[ m ] (1 + m * 2 ≡ n)

even-∃ even-zero                       =  ⟨ zero , refl ⟩
even-∃ (even-suc o) with odd-∃ o
...                    | ⟨ m , refl ⟩  =  ⟨ suc m , refl ⟩

odd-∃  (odd-suc e)  with even-∃ e
...                    | ⟨ m , refl ⟩  =  ⟨ m , refl ⟩
\end{code}
\end{fence}

We define two mutually recursive functions. Given evidence that
\texttt{n} is even or odd, we return a number \texttt{m} and evidence
that \texttt{m\ *\ 2\ ≡\ n} or \texttt{1\ +\ m\ *\ 2\ ≡\ n}. We induct
over the evidence that \texttt{n} is even or odd:

\begin{itemize}
\item
  If the number is even because it is zero, then we return a pair
  consisting of zero and the evidence that twice zero is zero.
\item
  If the number is even because it is one more than an odd number, then
  we apply the induction hypothesis to give a number \texttt{m} and
  evidence that \texttt{1\ +\ m\ *\ 2\ ≡\ n}. We return a pair
  consisting of \texttt{suc\ m} and evidence that
  \texttt{suc\ m\ *\ 2\ ≡\ suc\ n}, which is immediate after
  substituting for \texttt{n}.
\item
  If the number is odd because it is the successor of an even number,
  then we apply the induction hypothesis to give a number \texttt{m} and
  evidence that \texttt{m\ *\ 2\ ≡\ n}. We return a pair consisting of
  \texttt{suc\ m} and evidence that \texttt{1\ +\ m\ *\ 2\ ≡\ suc\ n},
  which is immediate after substituting for \texttt{n}.
\end{itemize}

This completes the proof in the forward direction.

Here is the proof in the reverse direction:

\begin{fence}
\begin{code}
∃-even : ∀ {n : ℕ} → ∃[ m ] (    m * 2 ≡ n) → even n
∃-odd  : ∀ {n : ℕ} → ∃[ m ] (1 + m * 2 ≡ n) →  odd n

∃-even ⟨  zero , refl ⟩  =  even-zero
∃-even ⟨ suc m , refl ⟩  =  even-suc (∃-odd ⟨ m , refl ⟩)

∃-odd  ⟨     m , refl ⟩  =  odd-suc (∃-even ⟨ m , refl ⟩)
\end{code}
\end{fence}

Given a number that is twice some other number we must show it is even,
and a number that is one more than twice some other number we must show
it is odd. We induct over the evidence of the existential, and in the
even case consider the two possibilities for the number that is doubled:

\begin{itemize}
\item
  In the even case for \texttt{zero}, we must show \texttt{zero\ *\ 2}
  is even, which follows by \texttt{even-zero}.
\item
  In the even case for \texttt{suc\ n}, we must show
  \texttt{suc\ m\ *\ 2} is even. The inductive hypothesis tells us that
  \texttt{1\ +\ m\ *\ 2} is odd, from which the desired result follows
  by \texttt{even-suc}.
\item
  In the odd case, we must show \texttt{1\ +\ m\ *\ 2} is odd. The
  inductive hypothesis tell us that \texttt{m\ *\ 2} is even, from which
  the desired result follows by \texttt{odd-suc}.
\end{itemize}

This completes the proof in the backward direction.

\hypertarget{exercise--even-odd-practice}{%
\subsubsection{\texorpdfstring{Exercise \texttt{∃-even-odd}
(practice)}{Exercise ∃-even-odd (practice)}}\label{exercise--even-odd-practice}}

How do the proofs become more difficult if we replace \texttt{m\ *\ 2}
and \texttt{1\ +\ m\ *\ 2} by \texttt{2\ *\ m} and
\texttt{2\ *\ m\ +\ 1}? Rewrite the proofs of \texttt{∃-even} and
\texttt{∃-odd} when restated in this way.

\begin{fence}
\begin{code}
-- Your code goes here
\end{code}
\end{fence}

\hypertarget{exercise----practice}{%
\subsubsection{\texorpdfstring{Exercise \texttt{∃-\textbar{}-≤}
(practice)}{Exercise ∃-\textbar-≤ (practice)}}\label{exercise----practice}}

Show that \texttt{y\ ≤\ z} holds if and only if there exists a
\texttt{x} such that \texttt{x\ +\ y\ ≡\ z}.

\begin{fence}
\begin{code}
-- Your code goes here
\end{code}
\end{fence}

\hypertarget{existentials-universals-and-negation}{%
\section{Existentials, Universals, and
Negation}\label{existentials-universals-and-negation}}

Negation of an existential is isomorphic to the universal of a negation.
Considering that existentials are generalised disjunction and universals
are generalised conjunction, this result is analogous to the one which
tells us that negation of a disjunction is isomorphic to a conjunction
of negations:

\begin{fence}
\begin{code}
¬∃≃∀¬ : ∀ {A : Set} {B : A → Set}
  → (¬ ∃[ x ] B x) ≃ ∀ x → ¬ B x
¬∃≃∀¬ =
  record
    { to      =  λ{ ¬∃xy x y → ¬∃xy ⟨ x , y ⟩ }
    ; from    =  λ{ ∀¬xy ⟨ x , y ⟩ → ∀¬xy x y }
    ; from∘to =  λ{ ¬∃xy → extensionality λ{ ⟨ x , y ⟩ → refl } }
    ; to∘from =  λ{ ∀¬xy → refl }
    }
\end{code}
\end{fence}

In the \texttt{to} direction, we are given a value \texttt{¬∃xy} of type
\texttt{¬\ ∃{[}\ x\ {]}\ B\ x}, and need to show that given a value
\texttt{x} that \texttt{¬\ B\ x} follows, in other words, from a value
\texttt{y} of type \texttt{B\ x} we can derive false. Combining
\texttt{x} and \texttt{y} gives us a value \texttt{⟨\ x\ ,\ y\ ⟩} of
type \texttt{∃{[}\ x\ {]}\ B\ x}, and applying \texttt{¬∃xy} to that
yields a contradiction.

In the \texttt{from} direction, we are given a value \texttt{∀¬xy} of
type \texttt{∀\ x\ →\ ¬\ B\ x}, and need to show that from a value
\texttt{⟨\ x\ ,\ y\ ⟩} of type \texttt{∃{[}\ x\ {]}\ B\ x} we can derive
false. Applying \texttt{∀¬xy} to \texttt{x} gives a value of type
\texttt{¬\ B\ x}, and applying that to \texttt{y} yields a
contradiction.

The two inverse proofs are straightforward, where one direction requires
extensionality.

\hypertarget{exercise--implies--recommended}{%
\subsubsection{\texorpdfstring{Exercise \texttt{∃¬-implies-¬∀}
(recommended)}{Exercise ∃¬-implies-¬∀ (recommended)}}\label{exercise--implies--recommended}}

Show that existential of a negation implies negation of a universal:

\begin{fence}
\begin{code}
postulate
  ∃¬-implies-¬∀ : ∀ {A : Set} {B : A → Set}
    → ∃[ x ] (¬ B x)
      --------------
    → ¬ (∀ x → B x)
\end{code}
\end{fence}

Does the converse hold? If so, prove; if not, explain why.

\hypertarget{Quantifiers-Bin-isomorphism}{%
\subsubsection{\texorpdfstring{Exercise \texttt{Bin-isomorphism}
(stretch)}{Exercise Bin-isomorphism (stretch)}}\label{Quantifiers-Bin-isomorphism}}

Recall that Exercises \protect\hyperlink{Naturals-Bin}{Bin},
\protect\hyperlink{Induction-Bin-laws}{Bin-laws}, and
\protect\hyperlink{Relations-Bin-predicates}{Bin-predicates} define a
datatype \texttt{Bin} of bitstrings representing natural numbers, and
asks you to define the following functions and predicates:

\begin{myDisplay}
to   : ℕ → Bin
from : Bin → ℕ
Can  : Bin → Set
\end{myDisplay}

And to establish the following properties:

\begin{myDisplay}
from (to n) ≡ n

----------
Can (to n)

Can b
---------------
to (from b) ≡ b
\end{myDisplay}

Using the above, establish that there is an isomorphism between
\texttt{ℕ} and \texttt{∃{[}\ b\ {]}(Can\ b)}.

We recommend proving the following lemmas which show that, for a given
binary number \texttt{b}, there is only one proof of \texttt{One\ b} and
similarly for \texttt{Can\ b}.

\begin{myDisplay}
≡One : ∀{b : Bin} (o o' : One b) → o ≡ o'

≡Can : ∀{b : Bin} (cb : Can b) (cb' : Can b) → cb ≡ cb'
\end{myDisplay}

Many of the alternatives for proving \texttt{to∘from} turn out to be
tricky. However, the proof can be straightforward if you use the
following lemma, which is a corollary of \texttt{≡Can}.

\begin{myDisplay}
proj₁≡→Can≡ : {cb cb′ : ∃[ b ](Can b)} → proj₁ cb ≡ proj₁ cb′ → cb ≡ cb′
\end{myDisplay}

\begin{fence}
\begin{code}
-- Your code goes here
\end{code}
\end{fence}

\hypertarget{standard-library}{%
\section{Standard library}\label{standard-library}}

Definitions similar to those in this chapter can be found in the
standard library:

\begin{fence}
\begin{code}
import Data.Product using (Σ; _,_; ∃; Σ-syntax; ∃-syntax)
\end{code}
\end{fence}

\hypertarget{unicode}{%
\section{Unicode}\label{unicode}}

This chapter uses the following unicode:

\begin{myDisplay}
Π  U+03A0  GREEK CAPITAL LETTER PI (\Pi)
Σ  U+03A3  GREEK CAPITAL LETTER SIGMA (\Sigma)
∃  U+2203  THERE EXISTS (\ex, \exists)
\end{myDisplay}

