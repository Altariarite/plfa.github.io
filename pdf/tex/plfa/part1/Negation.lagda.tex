\hypertarget{Negation}{%
\chapter{Negation: Negation, with intuitionistic and classical
logic}\label{Negation}}

\begin{fence}
\begin{code}
module plfa.part1.Negation where
\end{code}
\end{fence}

This chapter introduces negation, and discusses intuitionistic and
classical logic.

\hypertarget{imports}{%
\section{Imports}\label{imports}}

\begin{fence}
\begin{code}
open import Relation.Binary.PropositionalEquality using (_≡_; refl)
open import Data.Nat using (ℕ; zero; suc)
open import Data.Empty using (⊥; ⊥-elim)
open import Data.Sum using (_⊎_; inj₁; inj₂)
open import Data.Product using (_×_)
open import plfa.part1.Isomorphism using (_≃_; extensionality)
\end{code}
\end{fence}

\hypertarget{negation}{%
\section{Negation}\label{negation}}

Given a proposition \texttt{A}, the negation \texttt{¬\ A} holds if
\texttt{A} cannot hold. We formalise this idea by declaring negation to
be the same as implication of false:

\begin{fence}
\begin{code}
¬_ : Set → Set
¬ A = A → ⊥
\end{code}
\end{fence}

This is a form of \emph{reductio ad absurdum}: if assuming \texttt{A}
leads to the conclusion \texttt{⊥} (an absurdity), then we must have
\texttt{¬\ A}.

Evidence that \texttt{¬\ A} holds is of the form

\begin{myDisplay}
λ{ x → N }
\end{myDisplay}

where \texttt{N} is a term of type \texttt{⊥} containing as a free
variable \texttt{x} of type \texttt{A}. In other words, evidence that
\texttt{¬\ A} holds is a function that converts evidence that \texttt{A}
holds into evidence that \texttt{⊥} holds.

Given evidence that both \texttt{¬\ A} and \texttt{A} hold, we can
conclude that \texttt{⊥} holds. In other words, if both \texttt{¬\ A}
and \texttt{A} hold, then we have a contradiction:

\begin{fence}
\begin{code}
¬-elim : ∀ {A : Set}
  → ¬ A
  → A
    ---
  → ⊥
¬-elim ¬x x = ¬x x
\end{code}
\end{fence}

Here we write \texttt{¬x} for evidence of \texttt{¬\ A} and \texttt{x}
for evidence of \texttt{A}. This means that \texttt{¬x} must be a
function of type \texttt{A\ →\ ⊥}, and hence the application
\texttt{¬x\ x} must be of type \texttt{⊥}. Note that this rule is just a
special case of \texttt{→-elim}.

We set the precedence of negation so that it binds more tightly than
disjunction and conjunction, but less tightly than anything else:

\begin{fence}
\begin{code}
infix 3 ¬_
\end{code}
\end{fence}

Thus, \texttt{¬\ A\ ×\ ¬\ B} parses as \texttt{(¬\ A)\ ×\ (¬\ B)} and
\texttt{¬\ m\ ≡\ n} as \texttt{¬\ (m\ ≡\ n)}.

In \emph{classical} logic, we have that \texttt{A} is equivalent to
\texttt{¬\ ¬\ A}. As we discuss below, in Agda we use
\emph{intuitionistic} logic, where we have only half of this
equivalence, namely that \texttt{A} implies \texttt{¬\ ¬\ A}:

\begin{fence}
\begin{code}
¬¬-intro : ∀ {A : Set}
  → A
    -----
  → ¬ ¬ A
¬¬-intro x  =  λ{¬x → ¬x x}
\end{code}
\end{fence}

Let \texttt{x} be evidence of \texttt{A}. We show that assuming
\texttt{¬\ A} leads to a contradiction, and hence \texttt{¬\ ¬\ A} must
hold. Let \texttt{¬x} be evidence of \texttt{¬\ A}. Then from \texttt{A}
and \texttt{¬\ A} we have a contradiction, evidenced by \texttt{¬x\ x}.
Hence, we have shown \texttt{¬\ ¬\ A}.

An equivalent way to write the above is as follows:

\begin{fence}
\begin{code}
¬¬-intro′ : ∀ {A : Set}
  → A
    -----
  → ¬ ¬ A
¬¬-intro′ x ¬x = ¬x x
\end{code}
\end{fence}

Here we have simply converted the argument of the lambda term to an
additional argument of the function. We will usually use this latter
style, as it is more compact.

We cannot show that \texttt{¬\ ¬\ A} implies \texttt{A}, but we can show
that \texttt{¬\ ¬\ ¬\ A} implies \texttt{¬\ A}:

\begin{fence}
\begin{code}
¬¬¬-elim : ∀ {A : Set}
  → ¬ ¬ ¬ A
    -------
  → ¬ A
¬¬¬-elim ¬¬¬x  =  λ x → ¬¬¬x (¬¬-intro x)
\end{code}
\end{fence}

Let \texttt{¬¬¬x} be evidence of \texttt{¬\ ¬\ ¬\ A}. We will show that
assuming \texttt{A} leads to a contradiction, and hence \texttt{¬\ A}
must hold. Let \texttt{x} be evidence of \texttt{A}. Then by the
previous result, we can conclude \texttt{¬\ ¬\ A}, evidenced by
\texttt{¬¬-intro\ x}. Then from \texttt{¬\ ¬\ ¬\ A} and \texttt{¬\ ¬\ A}
we have a contradiction, evidenced by \texttt{¬¬¬x\ (¬¬-intro\ x)}.
Hence we have shown \texttt{¬\ A}.

Another law of logic is \emph{contraposition}, stating that if
\texttt{A} implies \texttt{B}, then \texttt{¬\ B} implies \texttt{¬\ A}:

\begin{fence}
\begin{code}
contraposition : ∀ {A B : Set}
  → (A → B)
    -----------
  → (¬ B → ¬ A)
contraposition f ¬y x = ¬y (f x)
\end{code}
\end{fence}

Let \texttt{f} be evidence of \texttt{A\ →\ B} and let \texttt{¬y} be
evidence of \texttt{¬\ B}. We will show that assuming \texttt{A} leads
to a contradiction, and hence \texttt{¬\ A} must hold. Let \texttt{x} be
evidence of \texttt{A}. Then from \texttt{A\ →\ B} and \texttt{A} we may
conclude \texttt{B}, evidenced by \texttt{f\ x}, and from \texttt{B} and
\texttt{¬\ B} we may conclude \texttt{⊥}, evidenced by
\texttt{¬y\ (f\ x)}. Hence, we have shown \texttt{¬\ A}.

Using negation, it is straightforward to define inequality:

\begin{fence}
\begin{code}
_≢_ : ∀ {A : Set} → A → A → Set
x ≢ y  =  ¬ (x ≡ y)
\end{code}
\end{fence}

It is trivial to show distinct numbers are not equal:

\begin{fence}
\begin{code}
_ : 1 ≢ 2
_ = λ()
\end{code}
\end{fence}

This is our first use of an absurd pattern in a lambda expression. The
type \texttt{M\ ≡\ N} is occupied exactly when \texttt{M} and \texttt{N}
simplify to identical terms. Since \texttt{1} and \texttt{2} simplify to
distinct normal forms, Agda determines that there is no possible
evidence that \texttt{1\ ≡\ 2}. As a second example, it is also easy to
validate Peano's postulate that zero is not the successor of any number:

\begin{fence}
\begin{code}
peano : ∀ {m : ℕ} → zero ≢ suc m
peano = λ()
\end{code}
\end{fence}

The evidence is essentially the same, as the absurd pattern matches all
possible evidence of type \texttt{zero\ ≡\ suc\ m}.

Given the correspondence of implication to exponentiation and false to
the type with no members, we can view negation as raising to the zero
power. This indeed corresponds to what we know for arithmetic, where

\begin{myDisplay}
0 ^ n  ≡  1,  if n ≡ 0
       ≡  0,  if n ≢ 0
\end{myDisplay}

Indeed, there is exactly one proof of \texttt{⊥\ →\ ⊥}. We can write
this proof two different ways:

\begin{fence}
\begin{code}
id : ⊥ → ⊥
id x = x

id′ : ⊥ → ⊥
id′ ()
\end{code}
\end{fence}

But, using extensionality, we can prove these equal:

\begin{fence}
\begin{code}
id≡id′ : id ≡ id′
id≡id′ = extensionality (λ())
\end{code}
\end{fence}

By extensionality, \texttt{id\ ≡\ id′} holds if for every \texttt{x} in
their domain we have \texttt{id\ x\ ≡\ id′\ x}. But there is no
\texttt{x} in their domain, so the equality holds trivially.

Indeed, we can show any two proofs of a negation are equal:

\begin{fence}
\begin{code}
assimilation : ∀ {A : Set} (¬x ¬x′ : ¬ A) → ¬x ≡ ¬x′
assimilation ¬x ¬x′ = extensionality (λ x → ⊥-elim (¬x x))
\end{code}
\end{fence}

Evidence for \texttt{¬\ A} implies that any evidence of \texttt{A}
immediately leads to a contradiction. But extensionality quantifies over
all \texttt{x} such that \texttt{A} holds, hence any such \texttt{x}
immediately leads to a contradiction, again causing the equality to hold
trivially.

\hypertarget{exercise--irreflexive-recommended}{%
\subsubsection{\texorpdfstring{Exercise \texttt{\textless{}-irreflexive}
(recommended)}{Exercise \textless-irreflexive (recommended)}}\label{exercise--irreflexive-recommended}}

Using negation, show that
\protect\hyperlink{Relations-strict-inequality}{strict inequality} is
irreflexive, that is, \texttt{n\ \textless{}\ n} holds for no
\texttt{n}.

\begin{fence}
\begin{code}
-- Your code goes here
\end{code}
\end{fence}

\hypertarget{exercise-trichotomy-practice}{%
\subsubsection{\texorpdfstring{Exercise \texttt{trichotomy}
(practice)}{Exercise trichotomy (practice)}}\label{exercise-trichotomy-practice}}

Show that strict inequality satisfies
\protect\hyperlink{Relations-trichotomy}{trichotomy}, that is, for any
naturals \texttt{m} and \texttt{n} exactly one of the following holds:

\begin{itemize}
\tightlist
\item
  \texttt{m\ \textless{}\ n}
\item
  \texttt{m\ ≡\ n}
\item
  \texttt{m\ \textgreater{}\ n}
\end{itemize}

Here ``exactly one'' means that not only one of the three must hold, but
that when one holds the negation of the other two must also hold.

\begin{fence}
\begin{code}
-- Your code goes here
\end{code}
\end{fence}

\hypertarget{exercise--dual--recommended}{%
\subsubsection{\texorpdfstring{Exercise \texttt{⊎-dual-×}
(recommended)}{Exercise ⊎-dual-× (recommended)}}\label{exercise--dual--recommended}}

Show that conjunction, disjunction, and negation are related by a
version of De Morgan's Law.

\begin{myDisplay}
¬ (A ⊎ B) ≃ (¬ A) × (¬ B)
\end{myDisplay}

This result is an easy consequence of something we've proved previously.

\begin{fence}
\begin{code}
-- Your code goes here
\end{code}
\end{fence}

Do we also have the following?

\begin{myDisplay}
¬ (A × B) ≃ (¬ A) ⊎ (¬ B)
\end{myDisplay}

If so, prove; if not, can you give a relation weaker than isomorphism
that relates the two sides?

\hypertarget{intuitive-and-classical-logic}{%
\section{Intuitive and Classical
logic}\label{intuitive-and-classical-logic}}

In Gilbert and Sullivan's \emph{The Gondoliers}, Casilda is told that as
an infant she was married to the heir of the King of Batavia, but that
due to a mix-up no one knows which of two individuals, Marco or
Giuseppe, is the heir. Alarmed, she wails ``Then do you mean to say that
I am married to one of two gondoliers, but it is impossible to say
which?'' To which the response is ``Without any doubt of any kind
whatever.''

Logic comes in many varieties, and one distinction is between
\emph{classical} and \emph{intuitionistic}. Intuitionists, concerned by
assumptions made by some logicians about the nature of infinity, insist
upon a constructionist notion of truth. In particular, they insist that
a proof of \texttt{A\ ⊎\ B} must show \emph{which} of \texttt{A} or
\texttt{B} holds, and hence they would reject the claim that Casilda is
married to Marco or Giuseppe until one of the two was identified as her
husband. Perhaps Gilbert and Sullivan anticipated intuitionism, for
their story's outcome is that the heir turns out to be a third
individual, Luiz, with whom Casilda is, conveniently, already in love.

Intuitionists also reject the law of the excluded middle, which asserts
\texttt{A\ ⊎\ ¬\ A} for every \texttt{A}, since the law gives no clue as
to \emph{which} of \texttt{A} or \texttt{¬\ A} holds. Heyting formalised
a variant of Hilbert's classical logic that captures the intuitionistic
notion of provability. In particular, the law of the excluded middle is
provable in Hilbert's logic, but not in Heyting's. Further, if the law
of the excluded middle is added as an axiom to Heyting's logic, then it
becomes equivalent to Hilbert's. Kolmogorov showed the two logics were
closely related: he gave a double-negation translation, such that a
formula is provable in classical logic if and only if its translation is
provable in intuitionistic logic.

Propositions as Types was first formulated for intuitionistic logic. It
is a perfect fit, because in the intuitionist interpretation the formula
\texttt{A\ ⊎\ B} is provable exactly when one exhibits either a proof of
\texttt{A} or a proof of \texttt{B}, so the type corresponding to
disjunction is a disjoint sum.

(Parts of the above are adopted from ``Propositions as Types'', Philip
Wadler, \emph{Communications of the ACM}, December 2015.)

\hypertarget{excluded-middle-is-irrefutable}{%
\section{Excluded middle is
irrefutable}\label{excluded-middle-is-irrefutable}}

The law of the excluded middle can be formulated as follows:

\begin{fence}
\begin{code}
postulate
  em : ∀ {A : Set} → A ⊎ ¬ A
\end{code}
\end{fence}

As we noted, the law of the excluded middle does not hold in
intuitionistic logic. However, we can show that it is
\emph{irrefutable}, meaning that the negation of its negation is
provable (and hence that its negation is never provable):

\begin{fence}
\begin{code}
em-irrefutable : ∀ {A : Set} → ¬ ¬ (A ⊎ ¬ A)
em-irrefutable = λ k → k (inj₂ (λ x → k (inj₁ x)))
\end{code}
\end{fence}

The best way to explain this code is to develop it interactively:

\begin{myDisplay}
em-irrefutable k = ?
\end{myDisplay}

Given evidence \texttt{k} that \texttt{¬\ (A\ ⊎\ ¬\ A)}, that is, a
function that given a value of type \texttt{A\ ⊎\ ¬\ A} returns a value
of the empty type, we must fill in \texttt{?} with a term that returns a
value of the empty type. The only way we can get a value of the empty
type is by applying \texttt{k} itself, so let's expand the hole
accordingly:

\begin{myDisplay}
em-irrefutable k = k ?
\end{myDisplay}

We need to fill the new hole with a value of type \texttt{A\ ⊎\ ¬\ A}.
We don't have a value of type \texttt{A} to hand, so let's pick the
second disjunct:

\begin{myDisplay}
em-irrefutable k = k (inj₂ λ{ x → ? })
\end{myDisplay}

The second disjunct accepts evidence of \texttt{¬\ A}, that is, a
function that given a value of type \texttt{A} returns a value of the
empty type. We bind \texttt{x} to the value of type \texttt{A}, and now
we need to fill in the hole with a value of the empty type. Once again,
the only way we can get a value of the empty type is by applying
\texttt{k} itself, so let's expand the hole accordingly:

\begin{myDisplay}
em-irrefutable k = k (inj₂ λ{ x → k ? })
\end{myDisplay}

This time we do have a value of type \texttt{A} to hand, namely
\texttt{x}, so we can pick the first disjunct:

\begin{myDisplay}
em-irrefutable k = k (inj₂ λ{ x → k (inj₁ x) })
\end{myDisplay}

There are no holes left! This completes the proof.

The following story illustrates the behaviour of the term we have
created. (With apologies to Peter Selinger, who tells a similar story
about a king, a wizard, and the Philosopher's stone.)

Once upon a time, the devil approached a man and made an offer: ``Either
(a) I will give you one billion dollars, or (b) I will grant you any
wish if you pay me one billion dollars. Of course, I get to choose
whether I offer (a) or (b).''

The man was wary. Did he need to sign over his soul? No, said the devil,
all the man need do is accept the offer.

The man pondered. If he was offered (b) it was unlikely that he would
ever be able to buy the wish, but what was the harm in having the
opportunity available?

``I accept,'' said the man at last. ``Do I get (a) or (b)?''

The devil paused. ``I choose (b).''

The man was disappointed but not surprised. That was that, he thought.
But the offer gnawed at him. Imagine what he could do with his wish!
Many years passed, and the man began to accumulate money. To get the
money he sometimes did bad things, and dimly he realised that this must
be what the devil had in mind. Eventually he had his billion dollars,
and the devil appeared again.

``Here is a billion dollars,'' said the man, handing over a valise
containing the money. ``Grant me my wish!''

The devil took possession of the valise. Then he said, ``Oh, did I say
(b) before? I'm so sorry. I meant (a). It is my great pleasure to give
you one billion dollars.''

And the devil handed back to the man the same valise that the man had
just handed to him.

(Parts of the above are adopted from ``Call-by-Value is Dual to
Call-by-Name'', Philip Wadler, \emph{International Conference on
Functional Programming}, 2003.)

\hypertarget{exercise-classical-stretch}{%
\subsubsection{\texorpdfstring{Exercise \texttt{Classical}
(stretch)}{Exercise Classical (stretch)}}\label{exercise-classical-stretch}}

Consider the following principles:

\begin{itemize}
\tightlist
\item
  Excluded Middle: \texttt{A\ ⊎\ ¬\ A}, for all \texttt{A}
\item
  Double Negation Elimination: \texttt{¬\ ¬\ A\ →\ A}, for all
  \texttt{A}
\item
  Peirce's Law: \texttt{((A\ →\ B)\ →\ A)\ →\ A}, for all \texttt{A} and
  \texttt{B}.
\item
  Implication as disjunction: \texttt{(A\ →\ B)\ →\ ¬\ A\ ⊎\ B}, for all
  \texttt{A} and \texttt{B}.
\item
  De Morgan: \texttt{¬\ (¬\ A\ ×\ ¬\ B)\ →\ A\ ⊎\ B}, for all \texttt{A}
  and \texttt{B}.
\end{itemize}

Show that each of these implies all the others.

\begin{fence}
\begin{code}
-- Your code goes here
\end{code}
\end{fence}

\hypertarget{exercise-stable-stretch}{%
\subsubsection{\texorpdfstring{Exercise \texttt{Stable}
(stretch)}{Exercise Stable (stretch)}}\label{exercise-stable-stretch}}

Say that a formula is \emph{stable} if double negation elimination holds
for it:

\begin{fence}
\begin{code}
Stable : Set → Set
Stable A = ¬ ¬ A → A
\end{code}
\end{fence}

Show that any negated formula is stable, and that the conjunction of two
stable formulas is stable.

\begin{fence}
\begin{code}
-- Your code goes here
\end{code}
\end{fence}

\hypertarget{standard-prelude}{%
\section{Standard Prelude}\label{standard-prelude}}

Definitions similar to those in this chapter can be found in the
standard library:

\begin{fence}
\begin{code}
import Relation.Nullary using (¬_)
import Relation.Nullary.Negation using (contraposition)
\end{code}
\end{fence}

\hypertarget{unicode}{%
\section{Unicode}\label{unicode}}

This chapter uses the following unicode:

\begin{myDisplay}
¬  U+00AC  NOT SIGN (\neg)
≢  U+2262  NOT IDENTICAL TO (\==n)
\end{myDisplay}

