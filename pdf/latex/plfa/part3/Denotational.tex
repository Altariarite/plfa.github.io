\hypertarget{Denotational}{%
\chapter{Denotational: Denotational semantics of untyped lambda
calculus}\label{Denotational}}

\begin{fence}
\begin{code}%
\>[0]\AgdaKeyword{module}\AgdaSpace{}%
\AgdaModule{plfa.part3.Denotational}\AgdaSpace{}%
\AgdaKeyword{where}\<%
\end{code}
\end{fence}

The lambda calculus is a language about \emph{functions}, that is,
mappings from input to output. In computing we often think of such
mappings as being carried out by a sequence of operations that transform
an input into an output. But functions can also be represented as data.
For example, one can tabulate a function, that is, create a table where
each row has two entries, an input and the corresponding output for the
function. Function application is then the process of looking up the row
for a given input and reading off the output.

We shall create a semantics for the untyped lambda calculus based on
this idea of functions-as-tables. However, there are two difficulties
that arise. First, functions often have an infinite domain, so it would
seem that we would need infinitely long tables to represent functions.
Second, in the lambda calculus, functions can be applied to functions.
They can even be applied to themselves! So it would seem that the tables
would contain cycles. One might start to worry that advanced techniques
are necessary to address these issues, but fortunately this is not the
case!

The first problem, of functions with infinite domains, is solved by
observing that in the execution of a terminating program, each lambda
abstraction will only be applied to a finite number of distinct
arguments. (We come back later to discuss diverging programs.) This
observation is another way of looking at Dana Scott's insight that only
continuous functions are needed to model the lambda calculus.

The second problem, that of self-application, is solved by relaxing the
way in which we lookup an argument in a function's table. Naively, one
would look in the table for a row in which the input entry exactly
matches the argument. In the case of self-application, this would
require the table to contain a copy of itself. Impossible! (At least, it
is impossible if we want to build tables using inductive data type
definitions, which indeed we do.) Instead it is sufficient to find an
input such that every row of the input appears as a row of the argument
(that is, the input is a subset of the argument). In the case of
self-application, the table only needs to contain a smaller copy of
itself, which is fine.

With these two observations in hand, it is straightforward to write down
a denotational semantics of the lambda calculus.

\hypertarget{imports}{%
\section{Imports}\label{imports}}

\begin{fence}
\begin{code}%
\>[0]\AgdaKeyword{open}\AgdaSpace{}%
\AgdaKeyword{import}\AgdaSpace{}%
\AgdaModule{Agda.Primitive}\AgdaSpace{}%
\AgdaKeyword{using}\AgdaSpace{}%
\AgdaSymbol{(}\AgdaPrimitive{lzero}\AgdaSymbol{;}\AgdaSpace{}%
\AgdaPrimitive{lsuc}\AgdaSymbol{)}\<%
\\
\>[0]\AgdaKeyword{open}\AgdaSpace{}%
\AgdaKeyword{import}\AgdaSpace{}%
\AgdaModule{Data.Empty}\AgdaSpace{}%
\AgdaKeyword{using}\AgdaSpace{}%
\AgdaSymbol{(}\AgdaFunction{⊥-elim}\AgdaSymbol{)}\<%
\\
\>[0]\AgdaKeyword{open}\AgdaSpace{}%
\AgdaKeyword{import}\AgdaSpace{}%
\AgdaModule{Data.Nat}\AgdaSpace{}%
\AgdaKeyword{using}\AgdaSpace{}%
\AgdaSymbol{(}\AgdaDatatype{ℕ}\AgdaSymbol{;}\AgdaSpace{}%
\AgdaInductiveConstructor{zero}\AgdaSymbol{;}\AgdaSpace{}%
\AgdaInductiveConstructor{suc}\AgdaSymbol{)}\<%
\\
\>[0]\AgdaKeyword{open}\AgdaSpace{}%
\AgdaKeyword{import}\AgdaSpace{}%
\AgdaModule{Data.Product}\AgdaSpace{}%
\AgdaKeyword{using}\AgdaSpace{}%
\AgdaSymbol{(}\AgdaOperator{\AgdaFunction{\AgdaUnderscore{}×\AgdaUnderscore{}}}\AgdaSymbol{;}\AgdaSpace{}%
\AgdaRecord{Σ}\AgdaSymbol{;}\AgdaSpace{}%
\AgdaFunction{Σ-syntax}\AgdaSymbol{;}\AgdaSpace{}%
\AgdaFunction{∃}\AgdaSymbol{;}\AgdaSpace{}%
\AgdaFunction{∃-syntax}\AgdaSymbol{;}\AgdaSpace{}%
\AgdaField{proj₁}\AgdaSymbol{;}\AgdaSpace{}%
\AgdaField{proj₂}\AgdaSymbol{)}\<%
\\
\>[0][@{}l@{\AgdaIndent{0}}]%
\>[2]\AgdaKeyword{renaming}\AgdaSpace{}%
\AgdaSymbol{(}\AgdaOperator{\AgdaInductiveConstructor{\AgdaUnderscore{},\AgdaUnderscore{}}}\AgdaSpace{}%
\AgdaSymbol{to}\AgdaSpace{}%
\AgdaOperator{\AgdaInductiveConstructor{⟨\AgdaUnderscore{},\AgdaUnderscore{}⟩}}\AgdaSymbol{)}\<%
\\
\>[0]\AgdaKeyword{open}\AgdaSpace{}%
\AgdaKeyword{import}\AgdaSpace{}%
\AgdaModule{Data.Sum}\<%
\\
\>[0]\AgdaKeyword{open}\AgdaSpace{}%
\AgdaKeyword{import}\AgdaSpace{}%
\AgdaModule{Data.Vec}\AgdaSpace{}%
\AgdaKeyword{using}\AgdaSpace{}%
\AgdaSymbol{(}\AgdaDatatype{Vec}\AgdaSymbol{;}\AgdaSpace{}%
\AgdaInductiveConstructor{[]}\AgdaSymbol{;}\AgdaSpace{}%
\AgdaOperator{\AgdaInductiveConstructor{\AgdaUnderscore{}∷\AgdaUnderscore{}}}\AgdaSymbol{)}\<%
\\
\>[0]\AgdaKeyword{open}\AgdaSpace{}%
\AgdaKeyword{import}\AgdaSpace{}%
\AgdaModule{Relation.Binary.PropositionalEquality}\<%
\\
\>[0][@{}l@{\AgdaIndent{0}}]%
\>[2]\AgdaKeyword{using}\AgdaSpace{}%
\AgdaSymbol{(}\AgdaOperator{\AgdaDatatype{\AgdaUnderscore{}≡\AgdaUnderscore{}}}\AgdaSymbol{;}\AgdaSpace{}%
\AgdaOperator{\AgdaFunction{\AgdaUnderscore{}≢\AgdaUnderscore{}}}\AgdaSymbol{;}\AgdaSpace{}%
\AgdaInductiveConstructor{refl}\AgdaSymbol{;}\AgdaSpace{}%
\AgdaFunction{sym}\AgdaSymbol{;}\AgdaSpace{}%
\AgdaFunction{cong}\AgdaSymbol{;}\AgdaSpace{}%
\AgdaFunction{cong₂}\AgdaSymbol{;}\AgdaSpace{}%
\AgdaFunction{cong-app}\AgdaSymbol{)}\<%
\\
\>[0]\AgdaKeyword{open}\AgdaSpace{}%
\AgdaKeyword{import}\AgdaSpace{}%
\AgdaModule{Relation.Nullary}\AgdaSpace{}%
\AgdaKeyword{using}\AgdaSpace{}%
\AgdaSymbol{(}\AgdaOperator{\AgdaFunction{¬\AgdaUnderscore{}}}\AgdaSymbol{)}\<%
\\
\>[0]\AgdaKeyword{open}\AgdaSpace{}%
\AgdaKeyword{import}\AgdaSpace{}%
\AgdaModule{Relation.Nullary.Negation}\AgdaSpace{}%
\AgdaKeyword{using}\AgdaSpace{}%
\AgdaSymbol{(}\AgdaFunction{contradiction}\AgdaSymbol{)}\<%
\\
\>[0]\AgdaKeyword{open}\AgdaSpace{}%
\AgdaKeyword{import}\AgdaSpace{}%
\AgdaModule{Function}\AgdaSpace{}%
\AgdaKeyword{using}\AgdaSpace{}%
\AgdaSymbol{(}\AgdaOperator{\AgdaFunction{\AgdaUnderscore{}∘\AgdaUnderscore{}}}\AgdaSymbol{)}\<%
\\
\>[0]\AgdaKeyword{open}\AgdaSpace{}%
\AgdaKeyword{import}\AgdaSpace{}%
\AgdaModule{plfa.part2.Untyped}\<%
\\
\>[0][@{}l@{\AgdaIndent{0}}]%
\>[2]\AgdaKeyword{using}%
\>[61I]\AgdaSymbol{(}\AgdaDatatype{Context}\AgdaSymbol{;}\AgdaSpace{}%
\AgdaInductiveConstructor{★}\AgdaSymbol{;}\AgdaSpace{}%
\AgdaOperator{\AgdaDatatype{\AgdaUnderscore{}∋\AgdaUnderscore{}}}\AgdaSymbol{;}\AgdaSpace{}%
\AgdaInductiveConstructor{∅}\AgdaSymbol{;}\AgdaSpace{}%
\AgdaOperator{\AgdaInductiveConstructor{\AgdaUnderscore{},\AgdaUnderscore{}}}\AgdaSymbol{;}\AgdaSpace{}%
\AgdaInductiveConstructor{Z}\AgdaSymbol{;}\AgdaSpace{}%
\AgdaOperator{\AgdaInductiveConstructor{S\AgdaUnderscore{}}}\AgdaSymbol{;}\AgdaSpace{}%
\AgdaOperator{\AgdaDatatype{\AgdaUnderscore{}⊢\AgdaUnderscore{}}}\AgdaSymbol{;}\AgdaSpace{}%
\AgdaOperator{\AgdaInductiveConstructor{`\AgdaUnderscore{}}}\AgdaSymbol{;}\AgdaSpace{}%
\AgdaOperator{\AgdaInductiveConstructor{\AgdaUnderscore{}·\AgdaUnderscore{}}}\AgdaSymbol{;}\AgdaSpace{}%
\AgdaOperator{\AgdaInductiveConstructor{ƛ\AgdaUnderscore{}}}\AgdaSymbol{;}\<%
\\
\>[61I][@{}l@{\AgdaIndent{0}}]%
\>[9]\AgdaOperator{\AgdaFunction{\#\AgdaUnderscore{}}}\AgdaSymbol{;}\AgdaSpace{}%
\AgdaFunction{twoᶜ}\AgdaSymbol{;}\AgdaSpace{}%
\AgdaFunction{ext}\AgdaSymbol{;}\AgdaSpace{}%
\AgdaFunction{rename}\AgdaSymbol{;}\AgdaSpace{}%
\AgdaFunction{exts}\AgdaSymbol{;}\AgdaSpace{}%
\AgdaFunction{subst}\AgdaSymbol{;}\AgdaSpace{}%
\AgdaFunction{subst-zero}\AgdaSymbol{;}\AgdaSpace{}%
\AgdaOperator{\AgdaFunction{\AgdaUnderscore{}[\AgdaUnderscore{}]}}\AgdaSymbol{)}\<%
\\
\>[0]\AgdaKeyword{open}\AgdaSpace{}%
\AgdaKeyword{import}\AgdaSpace{}%
\AgdaModule{plfa.part2.Substitution}\AgdaSpace{}%
\AgdaKeyword{using}\AgdaSpace{}%
\AgdaSymbol{(}\AgdaFunction{Rename}\AgdaSymbol{;}\AgdaSpace{}%
\AgdaPostulate{extensionality}\AgdaSymbol{;}\AgdaSpace{}%
\AgdaFunction{rename-id}\AgdaSymbol{)}\<%
\end{code}
\end{fence}

\hypertarget{values}{%
\section{Values}\label{values}}

The \texttt{Value} data type represents a finite portion of a function.
We think of a value as a finite set of pairs that represent input-output
mappings. The \texttt{Value} data type represents the set as a binary
tree whose internal nodes are the union operator and whose leaves
represent either a single mapping or the empty set.

\begin{itemize}
\item
  The ⊥ value provides no information about the computation.
\item
  A value of the form \texttt{v\ ↦\ w} is a single input-output mapping,
  from input \texttt{v} to output \texttt{w}.
\item
  A value of the form \texttt{v\ ⊔\ w} is a function that maps inputs to
  outputs according to both \texttt{v} and \texttt{w}. Think of it as
  taking the union of the two sets.
\end{itemize}

\begin{fence}
\begin{code}%
\>[0]\AgdaKeyword{infixr}\AgdaSpace{}%
\AgdaNumber{7}\AgdaSpace{}%
\AgdaOperator{\AgdaInductiveConstructor{\AgdaUnderscore{}↦\AgdaUnderscore{}}}\<%
\\
\>[0]\AgdaKeyword{infixl}\AgdaSpace{}%
\AgdaNumber{5}\AgdaSpace{}%
\AgdaOperator{\AgdaInductiveConstructor{\AgdaUnderscore{}⊔\AgdaUnderscore{}}}\<%
\\
%
\\[\AgdaEmptyExtraSkip]%
\>[0]\AgdaKeyword{data}\AgdaSpace{}%
\AgdaDatatype{Value}\AgdaSpace{}%
\AgdaSymbol{:}\AgdaSpace{}%
\AgdaPrimitiveType{Set}\AgdaSpace{}%
\AgdaKeyword{where}\<%
\\
\>[0][@{}l@{\AgdaIndent{0}}]%
\>[2]\AgdaInductiveConstructor{⊥}\AgdaSpace{}%
\AgdaSymbol{:}\AgdaSpace{}%
\AgdaDatatype{Value}\<%
\\
%
\>[2]\AgdaOperator{\AgdaInductiveConstructor{\AgdaUnderscore{}↦\AgdaUnderscore{}}}\AgdaSpace{}%
\AgdaSymbol{:}\AgdaSpace{}%
\AgdaDatatype{Value}\AgdaSpace{}%
\AgdaSymbol{→}\AgdaSpace{}%
\AgdaDatatype{Value}\AgdaSpace{}%
\AgdaSymbol{→}\AgdaSpace{}%
\AgdaDatatype{Value}\<%
\\
%
\>[2]\AgdaOperator{\AgdaInductiveConstructor{\AgdaUnderscore{}⊔\AgdaUnderscore{}}}\AgdaSpace{}%
\AgdaSymbol{:}\AgdaSpace{}%
\AgdaDatatype{Value}\AgdaSpace{}%
\AgdaSymbol{→}\AgdaSpace{}%
\AgdaDatatype{Value}\AgdaSpace{}%
\AgdaSymbol{→}\AgdaSpace{}%
\AgdaDatatype{Value}\<%
\end{code}
\end{fence}

The \texttt{⊑} relation adapts the familiar notion of subset to the
Value data type. This relation plays the key role in enabling
self-application. There are two rules that are specific to functions,
\texttt{⊑-fun} and \texttt{⊑-dist}, which we discuss below.

\begin{fence}
\begin{code}%
\>[0]\AgdaKeyword{infix}\AgdaSpace{}%
\AgdaNumber{4}\AgdaSpace{}%
\AgdaOperator{\AgdaDatatype{\AgdaUnderscore{}⊑\AgdaUnderscore{}}}\<%
\\
%
\\[\AgdaEmptyExtraSkip]%
\>[0]\AgdaKeyword{data}\AgdaSpace{}%
\AgdaOperator{\AgdaDatatype{\AgdaUnderscore{}⊑\AgdaUnderscore{}}}\AgdaSpace{}%
\AgdaSymbol{:}\AgdaSpace{}%
\AgdaDatatype{Value}\AgdaSpace{}%
\AgdaSymbol{→}\AgdaSpace{}%
\AgdaDatatype{Value}\AgdaSpace{}%
\AgdaSymbol{→}\AgdaSpace{}%
\AgdaPrimitiveType{Set}\AgdaSpace{}%
\AgdaKeyword{where}\<%
\\
%
\\[\AgdaEmptyExtraSkip]%
\>[0][@{}l@{\AgdaIndent{0}}]%
\>[2]\AgdaInductiveConstructor{⊑-bot}\AgdaSpace{}%
\AgdaSymbol{:}\AgdaSpace{}%
\AgdaSymbol{∀}\AgdaSpace{}%
\AgdaSymbol{\{}\AgdaBound{v}\AgdaSymbol{\}}\AgdaSpace{}%
\AgdaSymbol{→}\AgdaSpace{}%
\AgdaInductiveConstructor{⊥}\AgdaSpace{}%
\AgdaOperator{\AgdaDatatype{⊑}}\AgdaSpace{}%
\AgdaBound{v}\<%
\\
%
\\[\AgdaEmptyExtraSkip]%
%
\>[2]\AgdaInductiveConstructor{⊑-conj-L}\AgdaSpace{}%
\AgdaSymbol{:}\AgdaSpace{}%
\AgdaSymbol{∀}\AgdaSpace{}%
\AgdaSymbol{\{}\AgdaBound{u}\AgdaSpace{}%
\AgdaBound{v}\AgdaSpace{}%
\AgdaBound{w}\AgdaSymbol{\}}\<%
\\
\>[2][@{}l@{\AgdaIndent{0}}]%
\>[4]\AgdaSymbol{→}\AgdaSpace{}%
\AgdaBound{v}\AgdaSpace{}%
\AgdaOperator{\AgdaDatatype{⊑}}\AgdaSpace{}%
\AgdaBound{u}\<%
\\
%
\>[4]\AgdaSymbol{→}%
\>[132I]\AgdaBound{w}\AgdaSpace{}%
\AgdaOperator{\AgdaDatatype{⊑}}\AgdaSpace{}%
\AgdaBound{u}\<%
\\
\>[.][@{}l@{}]\<[132I]%
\>[6]\AgdaComment{-----------}\<%
\\
%
\>[4]\AgdaSymbol{→}\AgdaSpace{}%
\AgdaSymbol{(}\AgdaBound{v}\AgdaSpace{}%
\AgdaOperator{\AgdaInductiveConstructor{⊔}}\AgdaSpace{}%
\AgdaBound{w}\AgdaSymbol{)}\AgdaSpace{}%
\AgdaOperator{\AgdaDatatype{⊑}}\AgdaSpace{}%
\AgdaBound{u}\<%
\\
%
\\[\AgdaEmptyExtraSkip]%
%
\>[2]\AgdaInductiveConstructor{⊑-conj-R1}\AgdaSpace{}%
\AgdaSymbol{:}\AgdaSpace{}%
\AgdaSymbol{∀}\AgdaSpace{}%
\AgdaSymbol{\{}\AgdaBound{u}\AgdaSpace{}%
\AgdaBound{v}\AgdaSpace{}%
\AgdaBound{w}\AgdaSymbol{\}}\<%
\\
\>[2][@{}l@{\AgdaIndent{0}}]%
\>[4]\AgdaSymbol{→}%
\>[145I]\AgdaBound{u}\AgdaSpace{}%
\AgdaOperator{\AgdaDatatype{⊑}}\AgdaSpace{}%
\AgdaBound{v}\<%
\\
\>[.][@{}l@{}]\<[145I]%
\>[6]\AgdaComment{-----------}\<%
\\
%
\>[4]\AgdaSymbol{→}\AgdaSpace{}%
\AgdaBound{u}\AgdaSpace{}%
\AgdaOperator{\AgdaDatatype{⊑}}\AgdaSpace{}%
\AgdaSymbol{(}\AgdaBound{v}\AgdaSpace{}%
\AgdaOperator{\AgdaInductiveConstructor{⊔}}\AgdaSpace{}%
\AgdaBound{w}\AgdaSymbol{)}\<%
\\
%
\\[\AgdaEmptyExtraSkip]%
%
\>[2]\AgdaInductiveConstructor{⊑-conj-R2}\AgdaSpace{}%
\AgdaSymbol{:}\AgdaSpace{}%
\AgdaSymbol{∀}\AgdaSpace{}%
\AgdaSymbol{\{}\AgdaBound{u}\AgdaSpace{}%
\AgdaBound{v}\AgdaSpace{}%
\AgdaBound{w}\AgdaSymbol{\}}\<%
\\
\>[2][@{}l@{\AgdaIndent{0}}]%
\>[4]\AgdaSymbol{→}%
\>[158I]\AgdaBound{u}\AgdaSpace{}%
\AgdaOperator{\AgdaDatatype{⊑}}\AgdaSpace{}%
\AgdaBound{w}\<%
\\
\>[.][@{}l@{}]\<[158I]%
\>[6]\AgdaComment{-----------}\<%
\\
%
\>[4]\AgdaSymbol{→}\AgdaSpace{}%
\AgdaBound{u}\AgdaSpace{}%
\AgdaOperator{\AgdaDatatype{⊑}}\AgdaSpace{}%
\AgdaSymbol{(}\AgdaBound{v}\AgdaSpace{}%
\AgdaOperator{\AgdaInductiveConstructor{⊔}}\AgdaSpace{}%
\AgdaBound{w}\AgdaSymbol{)}\<%
\\
%
\\[\AgdaEmptyExtraSkip]%
%
\>[2]\AgdaInductiveConstructor{⊑-trans}\AgdaSpace{}%
\AgdaSymbol{:}\AgdaSpace{}%
\AgdaSymbol{∀}\AgdaSpace{}%
\AgdaSymbol{\{}\AgdaBound{u}\AgdaSpace{}%
\AgdaBound{v}\AgdaSpace{}%
\AgdaBound{w}\AgdaSymbol{\}}\<%
\\
\>[2][@{}l@{\AgdaIndent{0}}]%
\>[4]\AgdaSymbol{→}\AgdaSpace{}%
\AgdaBound{u}\AgdaSpace{}%
\AgdaOperator{\AgdaDatatype{⊑}}\AgdaSpace{}%
\AgdaBound{v}\<%
\\
%
\>[4]\AgdaSymbol{→}%
\>[174I]\AgdaBound{v}\AgdaSpace{}%
\AgdaOperator{\AgdaDatatype{⊑}}\AgdaSpace{}%
\AgdaBound{w}\<%
\\
\>[.][@{}l@{}]\<[174I]%
\>[6]\AgdaComment{-----}\<%
\\
%
\>[4]\AgdaSymbol{→}\AgdaSpace{}%
\AgdaBound{u}\AgdaSpace{}%
\AgdaOperator{\AgdaDatatype{⊑}}\AgdaSpace{}%
\AgdaBound{w}\<%
\\
%
\\[\AgdaEmptyExtraSkip]%
%
\>[2]\AgdaInductiveConstructor{⊑-fun}\AgdaSpace{}%
\AgdaSymbol{:}\AgdaSpace{}%
\AgdaSymbol{∀}\AgdaSpace{}%
\AgdaSymbol{\{}\AgdaBound{v}\AgdaSpace{}%
\AgdaBound{w}\AgdaSpace{}%
\AgdaBound{v′}\AgdaSpace{}%
\AgdaBound{w′}\AgdaSymbol{\}}\<%
\\
\>[2][@{}l@{\AgdaIndent{0}}]%
\>[4]\AgdaSymbol{→}\AgdaSpace{}%
\AgdaBound{v′}\AgdaSpace{}%
\AgdaOperator{\AgdaDatatype{⊑}}\AgdaSpace{}%
\AgdaBound{v}\<%
\\
%
\>[4]\AgdaSymbol{→}%
\>[189I]\AgdaBound{w}\AgdaSpace{}%
\AgdaOperator{\AgdaDatatype{⊑}}\AgdaSpace{}%
\AgdaBound{w′}\<%
\\
\>[.][@{}l@{}]\<[189I]%
\>[6]\AgdaComment{-------------------}\<%
\\
%
\>[4]\AgdaSymbol{→}\AgdaSpace{}%
\AgdaSymbol{(}\AgdaBound{v}\AgdaSpace{}%
\AgdaOperator{\AgdaInductiveConstructor{↦}}\AgdaSpace{}%
\AgdaBound{w}\AgdaSymbol{)}\AgdaSpace{}%
\AgdaOperator{\AgdaDatatype{⊑}}\AgdaSpace{}%
\AgdaSymbol{(}\AgdaBound{v′}\AgdaSpace{}%
\AgdaOperator{\AgdaInductiveConstructor{↦}}\AgdaSpace{}%
\AgdaBound{w′}\AgdaSymbol{)}\<%
\\
%
\\[\AgdaEmptyExtraSkip]%
%
\>[2]\AgdaInductiveConstructor{⊑-dist}\AgdaSpace{}%
\AgdaSymbol{:}\AgdaSpace{}%
\AgdaSymbol{∀\{}\AgdaBound{v}\AgdaSpace{}%
\AgdaBound{w}\AgdaSpace{}%
\AgdaBound{w′}\AgdaSymbol{\}}\<%
\\
\>[2][@{}l@{\AgdaIndent{0}}]%
\>[6]\AgdaComment{---------------------------------}\<%
\\
\>[2][@{}l@{\AgdaIndent{0}}]%
\>[4]\AgdaSymbol{→}\AgdaSpace{}%
\AgdaBound{v}\AgdaSpace{}%
\AgdaOperator{\AgdaInductiveConstructor{↦}}\AgdaSpace{}%
\AgdaSymbol{(}\AgdaBound{w}\AgdaSpace{}%
\AgdaOperator{\AgdaInductiveConstructor{⊔}}\AgdaSpace{}%
\AgdaBound{w′}\AgdaSymbol{)}\AgdaSpace{}%
\AgdaOperator{\AgdaDatatype{⊑}}\AgdaSpace{}%
\AgdaSymbol{(}\AgdaBound{v}\AgdaSpace{}%
\AgdaOperator{\AgdaInductiveConstructor{↦}}\AgdaSpace{}%
\AgdaBound{w}\AgdaSymbol{)}\AgdaSpace{}%
\AgdaOperator{\AgdaInductiveConstructor{⊔}}\AgdaSpace{}%
\AgdaSymbol{(}\AgdaBound{v}\AgdaSpace{}%
\AgdaOperator{\AgdaInductiveConstructor{↦}}\AgdaSpace{}%
\AgdaBound{w′}\AgdaSymbol{)}\<%
\end{code}
\end{fence}

The first five rules are straightforward. The rule \texttt{⊑-fun}
captures when it is OK to match a higher-order argument
\texttt{v′\ ↦\ w′} to a table entry whose input is \texttt{v\ ↦\ w}.
Considering a call to the higher-order argument. It is OK to pass a
larger argument than expected, so \texttt{v} can be larger than
\texttt{v′}. Also, it is OK to disregard some of the output, so
\texttt{w} can be smaller than \texttt{w′}. The rule \texttt{⊑-dist}
says that if you have two entries for the same input, then you can
combine them into a single entry and joins the two outputs.

The \texttt{⊑} relation is reflexive.

\begin{fence}
\begin{code}%
\>[0]\AgdaFunction{⊑-refl}\AgdaSpace{}%
\AgdaSymbol{:}\AgdaSpace{}%
\AgdaSymbol{∀}\AgdaSpace{}%
\AgdaSymbol{\{}\AgdaBound{v}\AgdaSymbol{\}}\AgdaSpace{}%
\AgdaSymbol{→}\AgdaSpace{}%
\AgdaBound{v}\AgdaSpace{}%
\AgdaOperator{\AgdaDatatype{⊑}}\AgdaSpace{}%
\AgdaBound{v}\<%
\\
\>[0]\AgdaFunction{⊑-refl}\AgdaSpace{}%
\AgdaSymbol{\{}\AgdaInductiveConstructor{⊥}\AgdaSymbol{\}}\AgdaSpace{}%
\AgdaSymbol{=}\AgdaSpace{}%
\AgdaInductiveConstructor{⊑-bot}\<%
\\
\>[0]\AgdaFunction{⊑-refl}\AgdaSpace{}%
\AgdaSymbol{\{}\AgdaBound{v}\AgdaSpace{}%
\AgdaOperator{\AgdaInductiveConstructor{↦}}\AgdaSpace{}%
\AgdaBound{v′}\AgdaSymbol{\}}\AgdaSpace{}%
\AgdaSymbol{=}\AgdaSpace{}%
\AgdaInductiveConstructor{⊑-fun}\AgdaSpace{}%
\AgdaFunction{⊑-refl}\AgdaSpace{}%
\AgdaFunction{⊑-refl}\<%
\\
\>[0]\AgdaFunction{⊑-refl}\AgdaSpace{}%
\AgdaSymbol{\{}\AgdaBound{v₁}\AgdaSpace{}%
\AgdaOperator{\AgdaInductiveConstructor{⊔}}\AgdaSpace{}%
\AgdaBound{v₂}\AgdaSymbol{\}}\AgdaSpace{}%
\AgdaSymbol{=}\AgdaSpace{}%
\AgdaInductiveConstructor{⊑-conj-L}\AgdaSpace{}%
\AgdaSymbol{(}\AgdaInductiveConstructor{⊑-conj-R1}\AgdaSpace{}%
\AgdaFunction{⊑-refl}\AgdaSymbol{)}\AgdaSpace{}%
\AgdaSymbol{(}\AgdaInductiveConstructor{⊑-conj-R2}\AgdaSpace{}%
\AgdaFunction{⊑-refl}\AgdaSymbol{)}\<%
\end{code}
\end{fence}

The \texttt{⊔} operation is monotonic with respect to \texttt{⊑}, that
is, given two larger values it produces a larger value.

\begin{fence}
\begin{code}%
\>[0]\AgdaFunction{⊔⊑⊔}\AgdaSpace{}%
\AgdaSymbol{:}\AgdaSpace{}%
\AgdaSymbol{∀}\AgdaSpace{}%
\AgdaSymbol{\{}\AgdaBound{v}\AgdaSpace{}%
\AgdaBound{w}\AgdaSpace{}%
\AgdaBound{v′}\AgdaSpace{}%
\AgdaBound{w′}\AgdaSymbol{\}}\<%
\\
\>[0][@{}l@{\AgdaIndent{0}}]%
\>[2]\AgdaSymbol{→}%
\>[248I]\AgdaBound{v}\AgdaSpace{}%
\AgdaOperator{\AgdaDatatype{⊑}}\AgdaSpace{}%
\AgdaBound{v′}%
\>[12]\AgdaSymbol{→}%
\>[15]\AgdaBound{w}\AgdaSpace{}%
\AgdaOperator{\AgdaDatatype{⊑}}\AgdaSpace{}%
\AgdaBound{w′}\<%
\\
\>[.][@{}l@{}]\<[248I]%
\>[4]\AgdaComment{-----------------------}\<%
\\
%
\>[2]\AgdaSymbol{→}\AgdaSpace{}%
\AgdaSymbol{(}\AgdaBound{v}\AgdaSpace{}%
\AgdaOperator{\AgdaInductiveConstructor{⊔}}\AgdaSpace{}%
\AgdaBound{w}\AgdaSymbol{)}\AgdaSpace{}%
\AgdaOperator{\AgdaDatatype{⊑}}\AgdaSpace{}%
\AgdaSymbol{(}\AgdaBound{v′}\AgdaSpace{}%
\AgdaOperator{\AgdaInductiveConstructor{⊔}}\AgdaSpace{}%
\AgdaBound{w′}\AgdaSymbol{)}\<%
\\
\>[0]\AgdaFunction{⊔⊑⊔}\AgdaSpace{}%
\AgdaBound{d₁}\AgdaSpace{}%
\AgdaBound{d₂}\AgdaSpace{}%
\AgdaSymbol{=}\AgdaSpace{}%
\AgdaInductiveConstructor{⊑-conj-L}\AgdaSpace{}%
\AgdaSymbol{(}\AgdaInductiveConstructor{⊑-conj-R1}\AgdaSpace{}%
\AgdaBound{d₁}\AgdaSymbol{)}\AgdaSpace{}%
\AgdaSymbol{(}\AgdaInductiveConstructor{⊑-conj-R2}\AgdaSpace{}%
\AgdaBound{d₂}\AgdaSymbol{)}\<%
\end{code}
\end{fence}

The \texttt{⊑-dist} rule can be used to combine two entries even when
the input values are not identical. One can first combine the two inputs
using ⊔ and then apply the \texttt{⊑-dist} rule to obtain the following
property.

\begin{fence}
\begin{code}%
\>[0]\AgdaFunction{⊔↦⊔-dist}\AgdaSpace{}%
\AgdaSymbol{:}\AgdaSpace{}%
\AgdaSymbol{∀\{}\AgdaBound{v}\AgdaSpace{}%
\AgdaBound{v′}\AgdaSpace{}%
\AgdaBound{w}\AgdaSpace{}%
\AgdaBound{w′}\AgdaSpace{}%
\AgdaSymbol{:}\AgdaSpace{}%
\AgdaDatatype{Value}\AgdaSymbol{\}}\<%
\\
\>[0][@{}l@{\AgdaIndent{0}}]%
\>[2]\AgdaSymbol{→}\AgdaSpace{}%
\AgdaSymbol{(}\AgdaBound{v}\AgdaSpace{}%
\AgdaOperator{\AgdaInductiveConstructor{⊔}}\AgdaSpace{}%
\AgdaBound{v′}\AgdaSymbol{)}\AgdaSpace{}%
\AgdaOperator{\AgdaInductiveConstructor{↦}}\AgdaSpace{}%
\AgdaSymbol{(}\AgdaBound{w}\AgdaSpace{}%
\AgdaOperator{\AgdaInductiveConstructor{⊔}}\AgdaSpace{}%
\AgdaBound{w′}\AgdaSymbol{)}\AgdaSpace{}%
\AgdaOperator{\AgdaDatatype{⊑}}\AgdaSpace{}%
\AgdaSymbol{(}\AgdaBound{v}\AgdaSpace{}%
\AgdaOperator{\AgdaInductiveConstructor{↦}}\AgdaSpace{}%
\AgdaBound{w}\AgdaSymbol{)}\AgdaSpace{}%
\AgdaOperator{\AgdaInductiveConstructor{⊔}}\AgdaSpace{}%
\AgdaSymbol{(}\AgdaBound{v′}\AgdaSpace{}%
\AgdaOperator{\AgdaInductiveConstructor{↦}}\AgdaSpace{}%
\AgdaBound{w′}\AgdaSymbol{)}\<%
\\
\>[0]\AgdaFunction{⊔↦⊔-dist}\AgdaSpace{}%
\AgdaSymbol{=}\AgdaSpace{}%
\AgdaInductiveConstructor{⊑-trans}\AgdaSpace{}%
\AgdaInductiveConstructor{⊑-dist}%
\>[293I]\AgdaSymbol{(}\AgdaFunction{⊔⊑⊔}\AgdaSpace{}%
\AgdaSymbol{(}\AgdaInductiveConstructor{⊑-fun}\AgdaSpace{}%
\AgdaSymbol{(}\AgdaInductiveConstructor{⊑-conj-R1}\AgdaSpace{}%
\AgdaFunction{⊑-refl}\AgdaSymbol{)}\AgdaSpace{}%
\AgdaFunction{⊑-refl}\AgdaSymbol{)}\<%
\\
\>[293I][@{}l@{\AgdaIndent{0}}]%
\>[28]\AgdaSymbol{(}\AgdaInductiveConstructor{⊑-fun}\AgdaSpace{}%
\AgdaSymbol{(}\AgdaInductiveConstructor{⊑-conj-R2}\AgdaSpace{}%
\AgdaFunction{⊑-refl}\AgdaSymbol{)}\AgdaSpace{}%
\AgdaFunction{⊑-refl}\AgdaSymbol{))}\<%
\end{code}
\end{fence}

If the join \texttt{u\ ⊔\ v} is less than another value \texttt{w}, then
both \texttt{u} and \texttt{v} are less than \texttt{w}.

\begin{fence}
\begin{code}%
\>[0]\AgdaFunction{⊔⊑-invL}\AgdaSpace{}%
\AgdaSymbol{:}\AgdaSpace{}%
\AgdaSymbol{∀\{}\AgdaBound{u}\AgdaSpace{}%
\AgdaBound{v}\AgdaSpace{}%
\AgdaBound{w}\AgdaSpace{}%
\AgdaSymbol{:}\AgdaSpace{}%
\AgdaDatatype{Value}\AgdaSymbol{\}}\<%
\\
\>[0][@{}l@{\AgdaIndent{0}}]%
\>[2]\AgdaSymbol{→}%
\>[307I]\AgdaBound{u}\AgdaSpace{}%
\AgdaOperator{\AgdaInductiveConstructor{⊔}}\AgdaSpace{}%
\AgdaBound{v}\AgdaSpace{}%
\AgdaOperator{\AgdaDatatype{⊑}}\AgdaSpace{}%
\AgdaBound{w}\<%
\\
\>[.][@{}l@{}]\<[307I]%
\>[4]\AgdaComment{---------}\<%
\\
%
\>[2]\AgdaSymbol{→}\AgdaSpace{}%
\AgdaBound{u}\AgdaSpace{}%
\AgdaOperator{\AgdaDatatype{⊑}}\AgdaSpace{}%
\AgdaBound{w}\<%
\\
\>[0]\AgdaFunction{⊔⊑-invL}\AgdaSpace{}%
\AgdaSymbol{(}\AgdaInductiveConstructor{⊑-conj-L}\AgdaSpace{}%
\AgdaBound{lt1}\AgdaSpace{}%
\AgdaBound{lt2}\AgdaSymbol{)}\AgdaSpace{}%
\AgdaSymbol{=}\AgdaSpace{}%
\AgdaBound{lt1}\<%
\\
\>[0]\AgdaFunction{⊔⊑-invL}\AgdaSpace{}%
\AgdaSymbol{(}\AgdaInductiveConstructor{⊑-conj-R1}\AgdaSpace{}%
\AgdaBound{lt}\AgdaSymbol{)}\AgdaSpace{}%
\AgdaSymbol{=}\AgdaSpace{}%
\AgdaInductiveConstructor{⊑-conj-R1}\AgdaSpace{}%
\AgdaSymbol{(}\AgdaFunction{⊔⊑-invL}\AgdaSpace{}%
\AgdaBound{lt}\AgdaSymbol{)}\<%
\\
\>[0]\AgdaFunction{⊔⊑-invL}\AgdaSpace{}%
\AgdaSymbol{(}\AgdaInductiveConstructor{⊑-conj-R2}\AgdaSpace{}%
\AgdaBound{lt}\AgdaSymbol{)}\AgdaSpace{}%
\AgdaSymbol{=}\AgdaSpace{}%
\AgdaInductiveConstructor{⊑-conj-R2}\AgdaSpace{}%
\AgdaSymbol{(}\AgdaFunction{⊔⊑-invL}\AgdaSpace{}%
\AgdaBound{lt}\AgdaSymbol{)}\<%
\\
\>[0]\AgdaFunction{⊔⊑-invL}\AgdaSpace{}%
\AgdaSymbol{(}\AgdaInductiveConstructor{⊑-trans}\AgdaSpace{}%
\AgdaBound{lt1}\AgdaSpace{}%
\AgdaBound{lt2}\AgdaSymbol{)}\AgdaSpace{}%
\AgdaSymbol{=}\AgdaSpace{}%
\AgdaInductiveConstructor{⊑-trans}\AgdaSpace{}%
\AgdaSymbol{(}\AgdaFunction{⊔⊑-invL}\AgdaSpace{}%
\AgdaBound{lt1}\AgdaSymbol{)}\AgdaSpace{}%
\AgdaBound{lt2}\<%
\\
%
\\[\AgdaEmptyExtraSkip]%
\>[0]\AgdaFunction{⊔⊑-invR}\AgdaSpace{}%
\AgdaSymbol{:}\AgdaSpace{}%
\AgdaSymbol{∀\{}\AgdaBound{u}\AgdaSpace{}%
\AgdaBound{v}\AgdaSpace{}%
\AgdaBound{w}\AgdaSpace{}%
\AgdaSymbol{:}\AgdaSpace{}%
\AgdaDatatype{Value}\AgdaSymbol{\}}\<%
\\
\>[0][@{}l@{\AgdaIndent{0}}]%
\>[2]\AgdaSymbol{→}%
\>[346I]\AgdaBound{u}\AgdaSpace{}%
\AgdaOperator{\AgdaInductiveConstructor{⊔}}\AgdaSpace{}%
\AgdaBound{v}\AgdaSpace{}%
\AgdaOperator{\AgdaDatatype{⊑}}\AgdaSpace{}%
\AgdaBound{w}\<%
\\
\>[.][@{}l@{}]\<[346I]%
\>[4]\AgdaComment{---------}\<%
\\
%
\>[2]\AgdaSymbol{→}\AgdaSpace{}%
\AgdaBound{v}\AgdaSpace{}%
\AgdaOperator{\AgdaDatatype{⊑}}\AgdaSpace{}%
\AgdaBound{w}\<%
\\
\>[0]\AgdaFunction{⊔⊑-invR}\AgdaSpace{}%
\AgdaSymbol{(}\AgdaInductiveConstructor{⊑-conj-L}\AgdaSpace{}%
\AgdaBound{lt1}\AgdaSpace{}%
\AgdaBound{lt2}\AgdaSymbol{)}\AgdaSpace{}%
\AgdaSymbol{=}\AgdaSpace{}%
\AgdaBound{lt2}\<%
\\
\>[0]\AgdaFunction{⊔⊑-invR}\AgdaSpace{}%
\AgdaSymbol{(}\AgdaInductiveConstructor{⊑-conj-R1}\AgdaSpace{}%
\AgdaBound{lt}\AgdaSymbol{)}\AgdaSpace{}%
\AgdaSymbol{=}\AgdaSpace{}%
\AgdaInductiveConstructor{⊑-conj-R1}\AgdaSpace{}%
\AgdaSymbol{(}\AgdaFunction{⊔⊑-invR}\AgdaSpace{}%
\AgdaBound{lt}\AgdaSymbol{)}\<%
\\
\>[0]\AgdaFunction{⊔⊑-invR}\AgdaSpace{}%
\AgdaSymbol{(}\AgdaInductiveConstructor{⊑-conj-R2}\AgdaSpace{}%
\AgdaBound{lt}\AgdaSymbol{)}\AgdaSpace{}%
\AgdaSymbol{=}\AgdaSpace{}%
\AgdaInductiveConstructor{⊑-conj-R2}\AgdaSpace{}%
\AgdaSymbol{(}\AgdaFunction{⊔⊑-invR}\AgdaSpace{}%
\AgdaBound{lt}\AgdaSymbol{)}\<%
\\
\>[0]\AgdaFunction{⊔⊑-invR}\AgdaSpace{}%
\AgdaSymbol{(}\AgdaInductiveConstructor{⊑-trans}\AgdaSpace{}%
\AgdaBound{lt1}\AgdaSpace{}%
\AgdaBound{lt2}\AgdaSymbol{)}\AgdaSpace{}%
\AgdaSymbol{=}\AgdaSpace{}%
\AgdaInductiveConstructor{⊑-trans}\AgdaSpace{}%
\AgdaSymbol{(}\AgdaFunction{⊔⊑-invR}\AgdaSpace{}%
\AgdaBound{lt1}\AgdaSymbol{)}\AgdaSpace{}%
\AgdaBound{lt2}\<%
\end{code}
\end{fence}

\hypertarget{environments}{%
\section{Environments}\label{environments}}

An environment gives meaning to the free variables in a term by mapping
variables to values.

\begin{fence}
\begin{code}%
\>[0]\AgdaFunction{Env}\AgdaSpace{}%
\AgdaSymbol{:}\AgdaSpace{}%
\AgdaDatatype{Context}\AgdaSpace{}%
\AgdaSymbol{→}\AgdaSpace{}%
\AgdaPrimitiveType{Set}\<%
\\
\>[0]\AgdaFunction{Env}\AgdaSpace{}%
\AgdaBound{Γ}\AgdaSpace{}%
\AgdaSymbol{=}\AgdaSpace{}%
\AgdaSymbol{∀}\AgdaSpace{}%
\AgdaSymbol{(}\AgdaBound{x}\AgdaSpace{}%
\AgdaSymbol{:}\AgdaSpace{}%
\AgdaBound{Γ}\AgdaSpace{}%
\AgdaOperator{\AgdaDatatype{∋}}\AgdaSpace{}%
\AgdaInductiveConstructor{★}\AgdaSymbol{)}\AgdaSpace{}%
\AgdaSymbol{→}\AgdaSpace{}%
\AgdaDatatype{Value}\<%
\end{code}
\end{fence}

We have the empty environment, and we can extend an environment.

\begin{fence}
\begin{code}%
\>[0]\AgdaFunction{`∅}\AgdaSpace{}%
\AgdaSymbol{:}\AgdaSpace{}%
\AgdaFunction{Env}\AgdaSpace{}%
\AgdaInductiveConstructor{∅}\<%
\\
\>[0]\AgdaFunction{`∅}\AgdaSpace{}%
\AgdaSymbol{()}\<%
\\
%
\\[\AgdaEmptyExtraSkip]%
\>[0]\AgdaKeyword{infixl}\AgdaSpace{}%
\AgdaNumber{5}\AgdaSpace{}%
\AgdaOperator{\AgdaFunction{\AgdaUnderscore{}`,\AgdaUnderscore{}}}\<%
\\
%
\\[\AgdaEmptyExtraSkip]%
\>[0]\AgdaOperator{\AgdaFunction{\AgdaUnderscore{}`,\AgdaUnderscore{}}}\AgdaSpace{}%
\AgdaSymbol{:}\AgdaSpace{}%
\AgdaSymbol{∀}\AgdaSpace{}%
\AgdaSymbol{\{}\AgdaBound{Γ}\AgdaSymbol{\}}\AgdaSpace{}%
\AgdaSymbol{→}\AgdaSpace{}%
\AgdaFunction{Env}\AgdaSpace{}%
\AgdaBound{Γ}\AgdaSpace{}%
\AgdaSymbol{→}\AgdaSpace{}%
\AgdaDatatype{Value}\AgdaSpace{}%
\AgdaSymbol{→}\AgdaSpace{}%
\AgdaFunction{Env}\AgdaSpace{}%
\AgdaSymbol{(}\AgdaBound{Γ}\AgdaSpace{}%
\AgdaOperator{\AgdaInductiveConstructor{,}}\AgdaSpace{}%
\AgdaInductiveConstructor{★}\AgdaSymbol{)}\<%
\\
\>[0]\AgdaSymbol{(}\AgdaBound{γ}\AgdaSpace{}%
\AgdaOperator{\AgdaFunction{`,}}\AgdaSpace{}%
\AgdaBound{v}\AgdaSymbol{)}\AgdaSpace{}%
\AgdaInductiveConstructor{Z}\AgdaSpace{}%
\AgdaSymbol{=}\AgdaSpace{}%
\AgdaBound{v}\<%
\\
\>[0]\AgdaSymbol{(}\AgdaBound{γ}\AgdaSpace{}%
\AgdaOperator{\AgdaFunction{`,}}\AgdaSpace{}%
\AgdaBound{v}\AgdaSymbol{)}\AgdaSpace{}%
\AgdaSymbol{(}\AgdaOperator{\AgdaInductiveConstructor{S}}\AgdaSpace{}%
\AgdaBound{x}\AgdaSymbol{)}\AgdaSpace{}%
\AgdaSymbol{=}\AgdaSpace{}%
\AgdaBound{γ}\AgdaSpace{}%
\AgdaBound{x}\<%
\end{code}
\end{fence}

We can recover the previous environment from an extended environment,
and the last value. Putting them together again takes us back to where
we started.

\begin{fence}
\begin{code}%
\>[0]\AgdaFunction{init}\AgdaSpace{}%
\AgdaSymbol{:}\AgdaSpace{}%
\AgdaSymbol{∀}\AgdaSpace{}%
\AgdaSymbol{\{}\AgdaBound{Γ}\AgdaSymbol{\}}\AgdaSpace{}%
\AgdaSymbol{→}\AgdaSpace{}%
\AgdaFunction{Env}\AgdaSpace{}%
\AgdaSymbol{(}\AgdaBound{Γ}\AgdaSpace{}%
\AgdaOperator{\AgdaInductiveConstructor{,}}\AgdaSpace{}%
\AgdaInductiveConstructor{★}\AgdaSymbol{)}\AgdaSpace{}%
\AgdaSymbol{→}\AgdaSpace{}%
\AgdaFunction{Env}\AgdaSpace{}%
\AgdaBound{Γ}\<%
\\
\>[0]\AgdaFunction{init}\AgdaSpace{}%
\AgdaBound{γ}\AgdaSpace{}%
\AgdaBound{x}\AgdaSpace{}%
\AgdaSymbol{=}\AgdaSpace{}%
\AgdaBound{γ}\AgdaSpace{}%
\AgdaSymbol{(}\AgdaOperator{\AgdaInductiveConstructor{S}}\AgdaSpace{}%
\AgdaBound{x}\AgdaSymbol{)}\<%
\\
%
\\[\AgdaEmptyExtraSkip]%
\>[0]\AgdaFunction{last}\AgdaSpace{}%
\AgdaSymbol{:}\AgdaSpace{}%
\AgdaSymbol{∀}\AgdaSpace{}%
\AgdaSymbol{\{}\AgdaBound{Γ}\AgdaSymbol{\}}\AgdaSpace{}%
\AgdaSymbol{→}\AgdaSpace{}%
\AgdaFunction{Env}\AgdaSpace{}%
\AgdaSymbol{(}\AgdaBound{Γ}\AgdaSpace{}%
\AgdaOperator{\AgdaInductiveConstructor{,}}\AgdaSpace{}%
\AgdaInductiveConstructor{★}\AgdaSymbol{)}\AgdaSpace{}%
\AgdaSymbol{→}\AgdaSpace{}%
\AgdaDatatype{Value}\<%
\\
\>[0]\AgdaFunction{last}\AgdaSpace{}%
\AgdaBound{γ}\AgdaSpace{}%
\AgdaSymbol{=}\AgdaSpace{}%
\AgdaBound{γ}\AgdaSpace{}%
\AgdaInductiveConstructor{Z}\<%
\\
%
\\[\AgdaEmptyExtraSkip]%
\>[0]\AgdaFunction{init-last}\AgdaSpace{}%
\AgdaSymbol{:}\AgdaSpace{}%
\AgdaSymbol{∀}\AgdaSpace{}%
\AgdaSymbol{\{}\AgdaBound{Γ}\AgdaSymbol{\}}\AgdaSpace{}%
\AgdaSymbol{→}\AgdaSpace{}%
\AgdaSymbol{(}\AgdaBound{γ}\AgdaSpace{}%
\AgdaSymbol{:}\AgdaSpace{}%
\AgdaFunction{Env}\AgdaSpace{}%
\AgdaSymbol{(}\AgdaBound{Γ}\AgdaSpace{}%
\AgdaOperator{\AgdaInductiveConstructor{,}}\AgdaSpace{}%
\AgdaInductiveConstructor{★}\AgdaSymbol{))}\AgdaSpace{}%
\AgdaSymbol{→}\AgdaSpace{}%
\AgdaBound{γ}\AgdaSpace{}%
\AgdaOperator{\AgdaDatatype{≡}}\AgdaSpace{}%
\AgdaSymbol{(}\AgdaFunction{init}\AgdaSpace{}%
\AgdaBound{γ}\AgdaSpace{}%
\AgdaOperator{\AgdaFunction{`,}}\AgdaSpace{}%
\AgdaFunction{last}\AgdaSpace{}%
\AgdaBound{γ}\AgdaSymbol{)}\<%
\\
\>[0]\AgdaFunction{init-last}\AgdaSpace{}%
\AgdaSymbol{\{}\AgdaBound{Γ}\AgdaSymbol{\}}\AgdaSpace{}%
\AgdaBound{γ}\AgdaSpace{}%
\AgdaSymbol{=}\AgdaSpace{}%
\AgdaPostulate{extensionality}\AgdaSpace{}%
\AgdaFunction{lemma}\<%
\\
\>[0][@{}l@{\AgdaIndent{0}}]%
\>[2]\AgdaKeyword{where}%
\>[478I]\AgdaFunction{lemma}\AgdaSpace{}%
\AgdaSymbol{:}\AgdaSpace{}%
\AgdaSymbol{∀}\AgdaSpace{}%
\AgdaSymbol{(}\AgdaBound{x}\AgdaSpace{}%
\AgdaSymbol{:}\AgdaSpace{}%
\AgdaBound{Γ}\AgdaSpace{}%
\AgdaOperator{\AgdaInductiveConstructor{,}}\AgdaSpace{}%
\AgdaInductiveConstructor{★}\AgdaSpace{}%
\AgdaOperator{\AgdaDatatype{∋}}\AgdaSpace{}%
\AgdaInductiveConstructor{★}\AgdaSymbol{)}\AgdaSpace{}%
\AgdaSymbol{→}\AgdaSpace{}%
\AgdaBound{γ}\AgdaSpace{}%
\AgdaBound{x}\AgdaSpace{}%
\AgdaOperator{\AgdaDatatype{≡}}\AgdaSpace{}%
\AgdaSymbol{(}\AgdaFunction{init}\AgdaSpace{}%
\AgdaBound{γ}\AgdaSpace{}%
\AgdaOperator{\AgdaFunction{`,}}\AgdaSpace{}%
\AgdaFunction{last}\AgdaSpace{}%
\AgdaBound{γ}\AgdaSymbol{)}\AgdaSpace{}%
\AgdaBound{x}\<%
\\
\>[.][@{}l@{}]\<[478I]%
\>[8]\AgdaFunction{lemma}\AgdaSpace{}%
\AgdaInductiveConstructor{Z}%
\>[21]\AgdaSymbol{=}%
\>[24]\AgdaInductiveConstructor{refl}\<%
\\
%
\>[8]\AgdaFunction{lemma}\AgdaSpace{}%
\AgdaSymbol{(}\AgdaOperator{\AgdaInductiveConstructor{S}}\AgdaSpace{}%
\AgdaBound{x}\AgdaSymbol{)}%
\>[21]\AgdaSymbol{=}%
\>[24]\AgdaInductiveConstructor{refl}\<%
\end{code}
\end{fence}

We extend the \texttt{⊑} relation point-wise to environments with the
following definition.

\begin{fence}
\begin{code}%
\>[0]\AgdaOperator{\AgdaFunction{\AgdaUnderscore{}`⊑\AgdaUnderscore{}}}\AgdaSpace{}%
\AgdaSymbol{:}\AgdaSpace{}%
\AgdaSymbol{∀}\AgdaSpace{}%
\AgdaSymbol{\{}\AgdaBound{Γ}\AgdaSymbol{\}}\AgdaSpace{}%
\AgdaSymbol{→}\AgdaSpace{}%
\AgdaFunction{Env}\AgdaSpace{}%
\AgdaBound{Γ}\AgdaSpace{}%
\AgdaSymbol{→}\AgdaSpace{}%
\AgdaFunction{Env}\AgdaSpace{}%
\AgdaBound{Γ}\AgdaSpace{}%
\AgdaSymbol{→}\AgdaSpace{}%
\AgdaPrimitiveType{Set}\<%
\\
\>[0]\AgdaOperator{\AgdaFunction{\AgdaUnderscore{}`⊑\AgdaUnderscore{}}}\AgdaSpace{}%
\AgdaSymbol{\{}\AgdaBound{Γ}\AgdaSymbol{\}}\AgdaSpace{}%
\AgdaBound{γ}\AgdaSpace{}%
\AgdaBound{δ}\AgdaSpace{}%
\AgdaSymbol{=}\AgdaSpace{}%
\AgdaSymbol{∀}\AgdaSpace{}%
\AgdaSymbol{(}\AgdaBound{x}\AgdaSpace{}%
\AgdaSymbol{:}\AgdaSpace{}%
\AgdaBound{Γ}\AgdaSpace{}%
\AgdaOperator{\AgdaDatatype{∋}}\AgdaSpace{}%
\AgdaInductiveConstructor{★}\AgdaSymbol{)}\AgdaSpace{}%
\AgdaSymbol{→}\AgdaSpace{}%
\AgdaBound{γ}\AgdaSpace{}%
\AgdaBound{x}\AgdaSpace{}%
\AgdaOperator{\AgdaDatatype{⊑}}\AgdaSpace{}%
\AgdaBound{δ}\AgdaSpace{}%
\AgdaBound{x}\<%
\end{code}
\end{fence}

We define a bottom environment and a join operator on environments,
which takes the point-wise join of their values.

\begin{fence}
\begin{code}%
\>[0]\AgdaFunction{`⊥}\AgdaSpace{}%
\AgdaSymbol{:}\AgdaSpace{}%
\AgdaSymbol{∀}\AgdaSpace{}%
\AgdaSymbol{\{}\AgdaBound{Γ}\AgdaSymbol{\}}\AgdaSpace{}%
\AgdaSymbol{→}\AgdaSpace{}%
\AgdaFunction{Env}\AgdaSpace{}%
\AgdaBound{Γ}\<%
\\
\>[0]\AgdaFunction{`⊥}\AgdaSpace{}%
\AgdaBound{x}\AgdaSpace{}%
\AgdaSymbol{=}\AgdaSpace{}%
\AgdaInductiveConstructor{⊥}\<%
\\
%
\\[\AgdaEmptyExtraSkip]%
\>[0]\AgdaOperator{\AgdaFunction{\AgdaUnderscore{}`⊔\AgdaUnderscore{}}}\AgdaSpace{}%
\AgdaSymbol{:}\AgdaSpace{}%
\AgdaSymbol{∀}\AgdaSpace{}%
\AgdaSymbol{\{}\AgdaBound{Γ}\AgdaSymbol{\}}\AgdaSpace{}%
\AgdaSymbol{→}\AgdaSpace{}%
\AgdaFunction{Env}\AgdaSpace{}%
\AgdaBound{Γ}\AgdaSpace{}%
\AgdaSymbol{→}\AgdaSpace{}%
\AgdaFunction{Env}\AgdaSpace{}%
\AgdaBound{Γ}\AgdaSpace{}%
\AgdaSymbol{→}\AgdaSpace{}%
\AgdaFunction{Env}\AgdaSpace{}%
\AgdaBound{Γ}\<%
\\
\>[0]\AgdaSymbol{(}\AgdaBound{γ}\AgdaSpace{}%
\AgdaOperator{\AgdaFunction{`⊔}}\AgdaSpace{}%
\AgdaBound{δ}\AgdaSymbol{)}\AgdaSpace{}%
\AgdaBound{x}\AgdaSpace{}%
\AgdaSymbol{=}\AgdaSpace{}%
\AgdaBound{γ}\AgdaSpace{}%
\AgdaBound{x}\AgdaSpace{}%
\AgdaOperator{\AgdaInductiveConstructor{⊔}}\AgdaSpace{}%
\AgdaBound{δ}\AgdaSpace{}%
\AgdaBound{x}\<%
\end{code}
\end{fence}

The \texttt{⊑-refl}, \texttt{⊑-conj-R1}, and \texttt{⊑-conj-R2} rules
lift to environments. So the join of two environments \texttt{γ} and
\texttt{δ} is greater than the first environment \texttt{γ} or the
second environment \texttt{δ}.

\begin{fence}
\begin{code}%
\>[0]\AgdaFunction{`⊑-refl}\AgdaSpace{}%
\AgdaSymbol{:}\AgdaSpace{}%
\AgdaSymbol{∀}\AgdaSpace{}%
\AgdaSymbol{\{}\AgdaBound{Γ}\AgdaSymbol{\}}\AgdaSpace{}%
\AgdaSymbol{\{}\AgdaBound{γ}\AgdaSpace{}%
\AgdaSymbol{:}\AgdaSpace{}%
\AgdaFunction{Env}\AgdaSpace{}%
\AgdaBound{Γ}\AgdaSymbol{\}}\AgdaSpace{}%
\AgdaSymbol{→}\AgdaSpace{}%
\AgdaBound{γ}\AgdaSpace{}%
\AgdaOperator{\AgdaFunction{`⊑}}\AgdaSpace{}%
\AgdaBound{γ}\<%
\\
\>[0]\AgdaFunction{`⊑-refl}\AgdaSpace{}%
\AgdaSymbol{\{}\AgdaBound{Γ}\AgdaSymbol{\}}\AgdaSpace{}%
\AgdaSymbol{\{}\AgdaBound{γ}\AgdaSymbol{\}}\AgdaSpace{}%
\AgdaBound{x}\AgdaSpace{}%
\AgdaSymbol{=}\AgdaSpace{}%
\AgdaFunction{⊑-refl}\AgdaSpace{}%
\AgdaSymbol{\{}\AgdaBound{γ}\AgdaSpace{}%
\AgdaBound{x}\AgdaSymbol{\}}\<%
\\
%
\\[\AgdaEmptyExtraSkip]%
\>[0]\AgdaFunction{⊑-env-conj-R1}\AgdaSpace{}%
\AgdaSymbol{:}\AgdaSpace{}%
\AgdaSymbol{∀}\AgdaSpace{}%
\AgdaSymbol{\{}\AgdaBound{Γ}\AgdaSymbol{\}}\AgdaSpace{}%
\AgdaSymbol{→}\AgdaSpace{}%
\AgdaSymbol{(}\AgdaBound{γ}\AgdaSpace{}%
\AgdaSymbol{:}\AgdaSpace{}%
\AgdaFunction{Env}\AgdaSpace{}%
\AgdaBound{Γ}\AgdaSymbol{)}\AgdaSpace{}%
\AgdaSymbol{→}\AgdaSpace{}%
\AgdaSymbol{(}\AgdaBound{δ}\AgdaSpace{}%
\AgdaSymbol{:}\AgdaSpace{}%
\AgdaFunction{Env}\AgdaSpace{}%
\AgdaBound{Γ}\AgdaSymbol{)}\AgdaSpace{}%
\AgdaSymbol{→}\AgdaSpace{}%
\AgdaBound{γ}\AgdaSpace{}%
\AgdaOperator{\AgdaFunction{`⊑}}\AgdaSpace{}%
\AgdaSymbol{(}\AgdaBound{γ}\AgdaSpace{}%
\AgdaOperator{\AgdaFunction{`⊔}}\AgdaSpace{}%
\AgdaBound{δ}\AgdaSymbol{)}\<%
\\
\>[0]\AgdaFunction{⊑-env-conj-R1}\AgdaSpace{}%
\AgdaBound{γ}\AgdaSpace{}%
\AgdaBound{δ}\AgdaSpace{}%
\AgdaBound{x}\AgdaSpace{}%
\AgdaSymbol{=}\AgdaSpace{}%
\AgdaInductiveConstructor{⊑-conj-R1}\AgdaSpace{}%
\AgdaFunction{⊑-refl}\<%
\\
%
\\[\AgdaEmptyExtraSkip]%
\>[0]\AgdaFunction{⊑-env-conj-R2}\AgdaSpace{}%
\AgdaSymbol{:}\AgdaSpace{}%
\AgdaSymbol{∀}\AgdaSpace{}%
\AgdaSymbol{\{}\AgdaBound{Γ}\AgdaSymbol{\}}\AgdaSpace{}%
\AgdaSymbol{→}\AgdaSpace{}%
\AgdaSymbol{(}\AgdaBound{γ}\AgdaSpace{}%
\AgdaSymbol{:}\AgdaSpace{}%
\AgdaFunction{Env}\AgdaSpace{}%
\AgdaBound{Γ}\AgdaSymbol{)}\AgdaSpace{}%
\AgdaSymbol{→}\AgdaSpace{}%
\AgdaSymbol{(}\AgdaBound{δ}\AgdaSpace{}%
\AgdaSymbol{:}\AgdaSpace{}%
\AgdaFunction{Env}\AgdaSpace{}%
\AgdaBound{Γ}\AgdaSymbol{)}\AgdaSpace{}%
\AgdaSymbol{→}\AgdaSpace{}%
\AgdaBound{δ}\AgdaSpace{}%
\AgdaOperator{\AgdaFunction{`⊑}}\AgdaSpace{}%
\AgdaSymbol{(}\AgdaBound{γ}\AgdaSpace{}%
\AgdaOperator{\AgdaFunction{`⊔}}\AgdaSpace{}%
\AgdaBound{δ}\AgdaSymbol{)}\<%
\\
\>[0]\AgdaFunction{⊑-env-conj-R2}\AgdaSpace{}%
\AgdaBound{γ}\AgdaSpace{}%
\AgdaBound{δ}\AgdaSpace{}%
\AgdaBound{x}\AgdaSpace{}%
\AgdaSymbol{=}\AgdaSpace{}%
\AgdaInductiveConstructor{⊑-conj-R2}\AgdaSpace{}%
\AgdaFunction{⊑-refl}\<%
\end{code}
\end{fence}

\hypertarget{denotational-semantics}{%
\section{Denotational Semantics}\label{denotational-semantics}}

We define the semantics with a judgment of the form
\texttt{ρ\ ⊢\ M\ ↓\ v}, where \texttt{ρ} is the environment, \texttt{M}
the program, and \texttt{v} is a result value. For readers familiar with
big-step semantics, this notation will feel quite natural, but don't let
the similarity fool you. There are subtle but important differences! So
here is the definition of the semantics, which we discuss in detail in
the following paragraphs.

\begin{fence}
\begin{code}%
\>[0]\AgdaKeyword{infix}\AgdaSpace{}%
\AgdaNumber{3}\AgdaSpace{}%
\AgdaOperator{\AgdaDatatype{\AgdaUnderscore{}⊢\AgdaUnderscore{}↓\AgdaUnderscore{}}}\<%
\\
%
\\[\AgdaEmptyExtraSkip]%
\>[0]\AgdaKeyword{data}\AgdaSpace{}%
\AgdaOperator{\AgdaDatatype{\AgdaUnderscore{}⊢\AgdaUnderscore{}↓\AgdaUnderscore{}}}\AgdaSpace{}%
\AgdaSymbol{:}\AgdaSpace{}%
\AgdaSymbol{∀\{}\AgdaBound{Γ}\AgdaSymbol{\}}\AgdaSpace{}%
\AgdaSymbol{→}\AgdaSpace{}%
\AgdaFunction{Env}\AgdaSpace{}%
\AgdaBound{Γ}\AgdaSpace{}%
\AgdaSymbol{→}\AgdaSpace{}%
\AgdaSymbol{(}\AgdaBound{Γ}\AgdaSpace{}%
\AgdaOperator{\AgdaDatatype{⊢}}\AgdaSpace{}%
\AgdaInductiveConstructor{★}\AgdaSymbol{)}\AgdaSpace{}%
\AgdaSymbol{→}\AgdaSpace{}%
\AgdaDatatype{Value}\AgdaSpace{}%
\AgdaSymbol{→}\AgdaSpace{}%
\AgdaPrimitiveType{Set}\AgdaSpace{}%
\AgdaKeyword{where}\<%
\\
%
\\[\AgdaEmptyExtraSkip]%
\>[0][@{}l@{\AgdaIndent{0}}]%
\>[2]\AgdaInductiveConstructor{var}%
\>[643I]\AgdaSymbol{:}\AgdaSpace{}%
\AgdaSymbol{∀}\AgdaSpace{}%
\AgdaSymbol{\{}\AgdaBound{Γ}\AgdaSymbol{\}}\AgdaSpace{}%
\AgdaSymbol{\{}\AgdaBound{γ}\AgdaSpace{}%
\AgdaSymbol{:}\AgdaSpace{}%
\AgdaFunction{Env}\AgdaSpace{}%
\AgdaBound{Γ}\AgdaSymbol{\}}\AgdaSpace{}%
\AgdaSymbol{\{}\AgdaBound{x}\AgdaSymbol{\}}\<%
\\
\>[.][@{}l@{}]\<[643I]%
\>[6]\AgdaComment{---------------}\<%
\\
\>[2][@{}l@{\AgdaIndent{0}}]%
\>[4]\AgdaSymbol{→}\AgdaSpace{}%
\AgdaBound{γ}\AgdaSpace{}%
\AgdaOperator{\AgdaDatatype{⊢}}\AgdaSpace{}%
\AgdaSymbol{(}\AgdaOperator{\AgdaInductiveConstructor{`}}\AgdaSpace{}%
\AgdaBound{x}\AgdaSymbol{)}\AgdaSpace{}%
\AgdaOperator{\AgdaDatatype{↓}}\AgdaSpace{}%
\AgdaBound{γ}\AgdaSpace{}%
\AgdaBound{x}\<%
\\
%
\\[\AgdaEmptyExtraSkip]%
%
\>[2]\AgdaInductiveConstructor{↦-elim}\AgdaSpace{}%
\AgdaSymbol{:}\AgdaSpace{}%
\AgdaSymbol{∀}\AgdaSpace{}%
\AgdaSymbol{\{}\AgdaBound{Γ}\AgdaSymbol{\}}\AgdaSpace{}%
\AgdaSymbol{\{}\AgdaBound{γ}\AgdaSpace{}%
\AgdaSymbol{:}\AgdaSpace{}%
\AgdaFunction{Env}\AgdaSpace{}%
\AgdaBound{Γ}\AgdaSymbol{\}}\AgdaSpace{}%
\AgdaSymbol{\{}\AgdaBound{L}\AgdaSpace{}%
\AgdaBound{M}\AgdaSpace{}%
\AgdaBound{v}\AgdaSpace{}%
\AgdaBound{w}\AgdaSymbol{\}}\<%
\\
\>[2][@{}l@{\AgdaIndent{0}}]%
\>[4]\AgdaSymbol{→}\AgdaSpace{}%
\AgdaBound{γ}\AgdaSpace{}%
\AgdaOperator{\AgdaDatatype{⊢}}\AgdaSpace{}%
\AgdaBound{L}\AgdaSpace{}%
\AgdaOperator{\AgdaDatatype{↓}}\AgdaSpace{}%
\AgdaSymbol{(}\AgdaBound{v}\AgdaSpace{}%
\AgdaOperator{\AgdaInductiveConstructor{↦}}\AgdaSpace{}%
\AgdaBound{w}\AgdaSymbol{)}\<%
\\
%
\>[4]\AgdaSymbol{→}%
\>[676I]\AgdaBound{γ}\AgdaSpace{}%
\AgdaOperator{\AgdaDatatype{⊢}}\AgdaSpace{}%
\AgdaBound{M}\AgdaSpace{}%
\AgdaOperator{\AgdaDatatype{↓}}\AgdaSpace{}%
\AgdaBound{v}\<%
\\
\>[.][@{}l@{}]\<[676I]%
\>[6]\AgdaComment{---------------}\<%
\\
%
\>[4]\AgdaSymbol{→}\AgdaSpace{}%
\AgdaBound{γ}\AgdaSpace{}%
\AgdaOperator{\AgdaDatatype{⊢}}\AgdaSpace{}%
\AgdaSymbol{(}\AgdaBound{L}\AgdaSpace{}%
\AgdaOperator{\AgdaInductiveConstructor{·}}\AgdaSpace{}%
\AgdaBound{M}\AgdaSymbol{)}\AgdaSpace{}%
\AgdaOperator{\AgdaDatatype{↓}}\AgdaSpace{}%
\AgdaBound{w}\<%
\\
%
\\[\AgdaEmptyExtraSkip]%
%
\>[2]\AgdaInductiveConstructor{↦-intro}\AgdaSpace{}%
\AgdaSymbol{:}\AgdaSpace{}%
\AgdaSymbol{∀}\AgdaSpace{}%
\AgdaSymbol{\{}\AgdaBound{Γ}\AgdaSymbol{\}}\AgdaSpace{}%
\AgdaSymbol{\{}\AgdaBound{γ}\AgdaSpace{}%
\AgdaSymbol{:}\AgdaSpace{}%
\AgdaFunction{Env}\AgdaSpace{}%
\AgdaBound{Γ}\AgdaSymbol{\}}\AgdaSpace{}%
\AgdaSymbol{\{}\AgdaBound{N}\AgdaSpace{}%
\AgdaBound{v}\AgdaSpace{}%
\AgdaBound{w}\AgdaSymbol{\}}\<%
\\
\>[2][@{}l@{\AgdaIndent{0}}]%
\>[4]\AgdaSymbol{→}%
\>[698I]\AgdaBound{γ}\AgdaSpace{}%
\AgdaOperator{\AgdaFunction{`,}}\AgdaSpace{}%
\AgdaBound{v}\AgdaSpace{}%
\AgdaOperator{\AgdaDatatype{⊢}}\AgdaSpace{}%
\AgdaBound{N}\AgdaSpace{}%
\AgdaOperator{\AgdaDatatype{↓}}\AgdaSpace{}%
\AgdaBound{w}\<%
\\
\>[.][@{}l@{}]\<[698I]%
\>[6]\AgdaComment{-------------------}\<%
\\
%
\>[4]\AgdaSymbol{→}\AgdaSpace{}%
\AgdaBound{γ}\AgdaSpace{}%
\AgdaOperator{\AgdaDatatype{⊢}}\AgdaSpace{}%
\AgdaSymbol{(}\AgdaOperator{\AgdaInductiveConstructor{ƛ}}\AgdaSpace{}%
\AgdaBound{N}\AgdaSymbol{)}\AgdaSpace{}%
\AgdaOperator{\AgdaDatatype{↓}}\AgdaSpace{}%
\AgdaSymbol{(}\AgdaBound{v}\AgdaSpace{}%
\AgdaOperator{\AgdaInductiveConstructor{↦}}\AgdaSpace{}%
\AgdaBound{w}\AgdaSymbol{)}\<%
\\
%
\\[\AgdaEmptyExtraSkip]%
%
\>[2]\AgdaInductiveConstructor{⊥-intro}\AgdaSpace{}%
\AgdaSymbol{:}\AgdaSpace{}%
\AgdaSymbol{∀}\AgdaSpace{}%
\AgdaSymbol{\{}\AgdaBound{Γ}\AgdaSymbol{\}}\AgdaSpace{}%
\AgdaSymbol{\{}\AgdaBound{γ}\AgdaSpace{}%
\AgdaSymbol{:}\AgdaSpace{}%
\AgdaFunction{Env}\AgdaSpace{}%
\AgdaBound{Γ}\AgdaSymbol{\}}\AgdaSpace{}%
\AgdaSymbol{\{}\AgdaBound{M}\AgdaSymbol{\}}\<%
\\
\>[2][@{}l@{\AgdaIndent{0}}]%
\>[6]\AgdaComment{---------}\<%
\\
\>[2][@{}l@{\AgdaIndent{0}}]%
\>[4]\AgdaSymbol{→}\AgdaSpace{}%
\AgdaBound{γ}\AgdaSpace{}%
\AgdaOperator{\AgdaDatatype{⊢}}\AgdaSpace{}%
\AgdaBound{M}\AgdaSpace{}%
\AgdaOperator{\AgdaDatatype{↓}}\AgdaSpace{}%
\AgdaInductiveConstructor{⊥}\<%
\\
%
\\[\AgdaEmptyExtraSkip]%
%
\>[2]\AgdaInductiveConstructor{⊔-intro}\AgdaSpace{}%
\AgdaSymbol{:}\AgdaSpace{}%
\AgdaSymbol{∀}\AgdaSpace{}%
\AgdaSymbol{\{}\AgdaBound{Γ}\AgdaSymbol{\}}\AgdaSpace{}%
\AgdaSymbol{\{}\AgdaBound{γ}\AgdaSpace{}%
\AgdaSymbol{:}\AgdaSpace{}%
\AgdaFunction{Env}\AgdaSpace{}%
\AgdaBound{Γ}\AgdaSymbol{\}}\AgdaSpace{}%
\AgdaSymbol{\{}\AgdaBound{M}\AgdaSpace{}%
\AgdaBound{v}\AgdaSpace{}%
\AgdaBound{w}\AgdaSymbol{\}}\<%
\\
\>[2][@{}l@{\AgdaIndent{0}}]%
\>[4]\AgdaSymbol{→}\AgdaSpace{}%
\AgdaBound{γ}\AgdaSpace{}%
\AgdaOperator{\AgdaDatatype{⊢}}\AgdaSpace{}%
\AgdaBound{M}\AgdaSpace{}%
\AgdaOperator{\AgdaDatatype{↓}}\AgdaSpace{}%
\AgdaBound{v}\<%
\\
%
\>[4]\AgdaSymbol{→}%
\>[741I]\AgdaBound{γ}\AgdaSpace{}%
\AgdaOperator{\AgdaDatatype{⊢}}\AgdaSpace{}%
\AgdaBound{M}\AgdaSpace{}%
\AgdaOperator{\AgdaDatatype{↓}}\AgdaSpace{}%
\AgdaBound{w}\<%
\\
\>[.][@{}l@{}]\<[741I]%
\>[6]\AgdaComment{---------------}\<%
\\
%
\>[4]\AgdaSymbol{→}\AgdaSpace{}%
\AgdaBound{γ}\AgdaSpace{}%
\AgdaOperator{\AgdaDatatype{⊢}}\AgdaSpace{}%
\AgdaBound{M}\AgdaSpace{}%
\AgdaOperator{\AgdaDatatype{↓}}\AgdaSpace{}%
\AgdaSymbol{(}\AgdaBound{v}\AgdaSpace{}%
\AgdaOperator{\AgdaInductiveConstructor{⊔}}\AgdaSpace{}%
\AgdaBound{w}\AgdaSymbol{)}\<%
\\
%
\\[\AgdaEmptyExtraSkip]%
%
\>[2]\AgdaInductiveConstructor{sub}\AgdaSpace{}%
\AgdaSymbol{:}\AgdaSpace{}%
\AgdaSymbol{∀}\AgdaSpace{}%
\AgdaSymbol{\{}\AgdaBound{Γ}\AgdaSymbol{\}}\AgdaSpace{}%
\AgdaSymbol{\{}\AgdaBound{γ}\AgdaSpace{}%
\AgdaSymbol{:}\AgdaSpace{}%
\AgdaFunction{Env}\AgdaSpace{}%
\AgdaBound{Γ}\AgdaSymbol{\}}\AgdaSpace{}%
\AgdaSymbol{\{}\AgdaBound{M}\AgdaSpace{}%
\AgdaBound{v}\AgdaSpace{}%
\AgdaBound{w}\AgdaSymbol{\}}\<%
\\
\>[2][@{}l@{\AgdaIndent{0}}]%
\>[4]\AgdaSymbol{→}\AgdaSpace{}%
\AgdaBound{γ}\AgdaSpace{}%
\AgdaOperator{\AgdaDatatype{⊢}}\AgdaSpace{}%
\AgdaBound{M}\AgdaSpace{}%
\AgdaOperator{\AgdaDatatype{↓}}\AgdaSpace{}%
\AgdaBound{v}\<%
\\
%
\>[4]\AgdaSymbol{→}%
\>[768I]\AgdaBound{w}\AgdaSpace{}%
\AgdaOperator{\AgdaDatatype{⊑}}\AgdaSpace{}%
\AgdaBound{v}\<%
\\
\>[.][@{}l@{}]\<[768I]%
\>[6]\AgdaComment{---------}\<%
\\
%
\>[4]\AgdaSymbol{→}\AgdaSpace{}%
\AgdaBound{γ}\AgdaSpace{}%
\AgdaOperator{\AgdaDatatype{⊢}}\AgdaSpace{}%
\AgdaBound{M}\AgdaSpace{}%
\AgdaOperator{\AgdaDatatype{↓}}\AgdaSpace{}%
\AgdaBound{w}\<%
\end{code}
\end{fence}

Consider the rule for lambda abstractions, \texttt{↦-intro}. It says
that a lambda abstraction results in a single-entry table that maps the
input \texttt{v} to the output \texttt{w}, provided that evaluating the
body in an environment with \texttt{v} bound to its parameter produces
the output \texttt{w}. As a simple example of this rule, we can see that
the identity function maps \texttt{⊥} to \texttt{⊥} and also that it
maps \texttt{⊥\ ↦\ ⊥} to \texttt{⊥\ ↦\ ⊥}.

\begin{fence}
\begin{code}%
\>[0]\AgdaFunction{id}\AgdaSpace{}%
\AgdaSymbol{:}\AgdaSpace{}%
\AgdaInductiveConstructor{∅}\AgdaSpace{}%
\AgdaOperator{\AgdaDatatype{⊢}}\AgdaSpace{}%
\AgdaInductiveConstructor{★}\<%
\\
\>[0]\AgdaFunction{id}\AgdaSpace{}%
\AgdaSymbol{=}\AgdaSpace{}%
\AgdaOperator{\AgdaInductiveConstructor{ƛ}}\AgdaSpace{}%
\AgdaOperator{\AgdaFunction{\#}}\AgdaSpace{}%
\AgdaNumber{0}\<%
\end{code}
\end{fence}

\begin{fence}
\begin{code}%
\>[0]\AgdaFunction{denot-id1}\AgdaSpace{}%
\AgdaSymbol{:}\AgdaSpace{}%
\AgdaSymbol{∀}\AgdaSpace{}%
\AgdaSymbol{\{}\AgdaBound{γ}\AgdaSymbol{\}}\AgdaSpace{}%
\AgdaSymbol{→}\AgdaSpace{}%
\AgdaBound{γ}\AgdaSpace{}%
\AgdaOperator{\AgdaDatatype{⊢}}\AgdaSpace{}%
\AgdaFunction{id}\AgdaSpace{}%
\AgdaOperator{\AgdaDatatype{↓}}\AgdaSpace{}%
\AgdaInductiveConstructor{⊥}\AgdaSpace{}%
\AgdaOperator{\AgdaInductiveConstructor{↦}}\AgdaSpace{}%
\AgdaInductiveConstructor{⊥}\<%
\\
\>[0]\AgdaFunction{denot-id1}\AgdaSpace{}%
\AgdaSymbol{=}\AgdaSpace{}%
\AgdaInductiveConstructor{↦-intro}\AgdaSpace{}%
\AgdaInductiveConstructor{var}\<%
\\
%
\\[\AgdaEmptyExtraSkip]%
\>[0]\AgdaFunction{denot-id2}\AgdaSpace{}%
\AgdaSymbol{:}\AgdaSpace{}%
\AgdaSymbol{∀}\AgdaSpace{}%
\AgdaSymbol{\{}\AgdaBound{γ}\AgdaSymbol{\}}\AgdaSpace{}%
\AgdaSymbol{→}\AgdaSpace{}%
\AgdaBound{γ}\AgdaSpace{}%
\AgdaOperator{\AgdaDatatype{⊢}}\AgdaSpace{}%
\AgdaFunction{id}\AgdaSpace{}%
\AgdaOperator{\AgdaDatatype{↓}}\AgdaSpace{}%
\AgdaSymbol{(}\AgdaInductiveConstructor{⊥}\AgdaSpace{}%
\AgdaOperator{\AgdaInductiveConstructor{↦}}\AgdaSpace{}%
\AgdaInductiveConstructor{⊥}\AgdaSymbol{)}\AgdaSpace{}%
\AgdaOperator{\AgdaInductiveConstructor{↦}}\AgdaSpace{}%
\AgdaSymbol{(}\AgdaInductiveConstructor{⊥}\AgdaSpace{}%
\AgdaOperator{\AgdaInductiveConstructor{↦}}\AgdaSpace{}%
\AgdaInductiveConstructor{⊥}\AgdaSymbol{)}\<%
\\
\>[0]\AgdaFunction{denot-id2}\AgdaSpace{}%
\AgdaSymbol{=}\AgdaSpace{}%
\AgdaInductiveConstructor{↦-intro}\AgdaSpace{}%
\AgdaInductiveConstructor{var}\<%
\end{code}
\end{fence}

Of course, we will need tables with many rows to capture the meaning of
lambda abstractions. These can be constructed using the \texttt{⊔-intro}
rule. If term M (typically a lambda abstraction) can produce both tables
\texttt{v} and \texttt{w}, then it produces the combined table
\texttt{v\ ⊔\ w}. One can take an operational view of the rules
\texttt{↦-intro} and \texttt{⊔-intro} by imagining that when an
interpreter first comes to a lambda abstraction, it pre-evaluates the
function on a bunch of randomly chosen arguments, using many instances
of the rule \texttt{↦-intro}, and then joins them into a big table using
many instances of the rule \texttt{⊔-intro}. In the following we show
that the identity function produces a table containing both of the
previous results, \texttt{⊥\ ↦\ ⊥} and \texttt{(⊥\ ↦\ ⊥)\ ↦\ (⊥\ ↦\ ⊥)}.

\begin{fence}
\begin{code}%
\>[0]\AgdaFunction{denot-id3}\AgdaSpace{}%
\AgdaSymbol{:}\AgdaSpace{}%
\AgdaFunction{`∅}\AgdaSpace{}%
\AgdaOperator{\AgdaDatatype{⊢}}\AgdaSpace{}%
\AgdaFunction{id}\AgdaSpace{}%
\AgdaOperator{\AgdaDatatype{↓}}\AgdaSpace{}%
\AgdaSymbol{(}\AgdaInductiveConstructor{⊥}\AgdaSpace{}%
\AgdaOperator{\AgdaInductiveConstructor{↦}}\AgdaSpace{}%
\AgdaInductiveConstructor{⊥}\AgdaSymbol{)}\AgdaSpace{}%
\AgdaOperator{\AgdaInductiveConstructor{⊔}}\AgdaSpace{}%
\AgdaSymbol{(}\AgdaInductiveConstructor{⊥}\AgdaSpace{}%
\AgdaOperator{\AgdaInductiveConstructor{↦}}\AgdaSpace{}%
\AgdaInductiveConstructor{⊥}\AgdaSymbol{)}\AgdaSpace{}%
\AgdaOperator{\AgdaInductiveConstructor{↦}}\AgdaSpace{}%
\AgdaSymbol{(}\AgdaInductiveConstructor{⊥}\AgdaSpace{}%
\AgdaOperator{\AgdaInductiveConstructor{↦}}\AgdaSpace{}%
\AgdaInductiveConstructor{⊥}\AgdaSymbol{)}\<%
\\
\>[0]\AgdaFunction{denot-id3}\AgdaSpace{}%
\AgdaSymbol{=}\AgdaSpace{}%
\AgdaInductiveConstructor{⊔-intro}\AgdaSpace{}%
\AgdaFunction{denot-id1}\AgdaSpace{}%
\AgdaFunction{denot-id2}\<%
\end{code}
\end{fence}

We most often think of the judgment \texttt{γ\ ⊢\ M\ ↓\ v} as taking the
environment \texttt{γ} and term \texttt{M} as input, producing the
result \texttt{v}. However, it is worth emphasizing that the semantics
is a \emph{relation}. The above results for the identity function show
that the same environment and term can be mapped to different results.
However, the results for a given \texttt{γ} and \texttt{M} are not
\emph{too} different, they are all finite approximations of the same
function. Perhaps a better way of thinking about the judgment
\texttt{γ\ ⊢\ M\ ↓\ v} is that the \texttt{γ}, \texttt{M}, and
\texttt{v} are all inputs and the semantics either confirms or denies
whether \texttt{v} is an accurate partial description of the result of
\texttt{M} in environment \texttt{γ}.

Next we consider the meaning of function application as given by the
\texttt{↦-elim} rule. In the premise of the rule we have that \texttt{L}
maps \texttt{v} to \texttt{w}. So if \texttt{M} produces \texttt{v},
then the application of \texttt{L} to \texttt{M} produces \texttt{w}.

As an example of function application and the \texttt{↦-elim} rule, we
apply the identity function to itself. Indeed, we have both that
\texttt{∅\ ⊢\ id\ ↓\ (u\ ↦\ u)\ ↦\ (u\ ↦\ u)} and also
\texttt{∅\ ⊢\ id\ ↓\ (u\ ↦\ u)}, so we can apply the rule
\texttt{↦-elim}.

\begin{fence}
\begin{code}%
\>[0]\AgdaFunction{id-app-id}\AgdaSpace{}%
\AgdaSymbol{:}\AgdaSpace{}%
\AgdaSymbol{∀}\AgdaSpace{}%
\AgdaSymbol{\{}\AgdaBound{u}\AgdaSpace{}%
\AgdaSymbol{:}\AgdaSpace{}%
\AgdaDatatype{Value}\AgdaSymbol{\}}\AgdaSpace{}%
\AgdaSymbol{→}\AgdaSpace{}%
\AgdaFunction{`∅}\AgdaSpace{}%
\AgdaOperator{\AgdaDatatype{⊢}}\AgdaSpace{}%
\AgdaFunction{id}\AgdaSpace{}%
\AgdaOperator{\AgdaInductiveConstructor{·}}\AgdaSpace{}%
\AgdaFunction{id}\AgdaSpace{}%
\AgdaOperator{\AgdaDatatype{↓}}\AgdaSpace{}%
\AgdaSymbol{(}\AgdaBound{u}\AgdaSpace{}%
\AgdaOperator{\AgdaInductiveConstructor{↦}}\AgdaSpace{}%
\AgdaBound{u}\AgdaSymbol{)}\<%
\\
\>[0]\AgdaFunction{id-app-id}\AgdaSpace{}%
\AgdaSymbol{\{}\AgdaBound{u}\AgdaSymbol{\}}\AgdaSpace{}%
\AgdaSymbol{=}\AgdaSpace{}%
\AgdaInductiveConstructor{↦-elim}\AgdaSpace{}%
\AgdaSymbol{(}\AgdaInductiveConstructor{↦-intro}\AgdaSpace{}%
\AgdaInductiveConstructor{var}\AgdaSymbol{)}\AgdaSpace{}%
\AgdaSymbol{(}\AgdaInductiveConstructor{↦-intro}\AgdaSpace{}%
\AgdaInductiveConstructor{var}\AgdaSymbol{)}\<%
\end{code}
\end{fence}

Next we revisit the Church numeral two:
\texttt{λ\ f.\ λ\ u.\ (f\ (f\ u))}. This function has two parameters: a
function \texttt{f} and an arbitrary value \texttt{u}, and it applies
\texttt{f} twice. So \texttt{f} must map \texttt{u} to some value, which
we'll name \texttt{v}. Then for the second application, \texttt{f} must
map \texttt{v} to some value. Let's name it \texttt{w}. So the
function's table must include two entries, both \texttt{u\ ↦\ v} and
\texttt{v\ ↦\ w}. For each application of the table, we extract the
appropriate entry from it using the \texttt{sub} rule. In particular, we
use the ⊑-conj-R1 and ⊑-conj-R2 to select \texttt{u\ ↦\ v} and
\texttt{v\ ↦\ w}, respectively, from the table
\texttt{u\ ↦\ v\ ⊔\ v\ ↦\ w}. So the meaning of twoᶜ is that it takes
this table and parameter \texttt{u}, and it returns \texttt{w}. Indeed
we derive this as follows.

\begin{fence}
\begin{code}%
\>[0]\AgdaFunction{denot-twoᶜ}\AgdaSpace{}%
\AgdaSymbol{:}\AgdaSpace{}%
\AgdaSymbol{∀\{}\AgdaBound{u}\AgdaSpace{}%
\AgdaBound{v}\AgdaSpace{}%
\AgdaBound{w}\AgdaSpace{}%
\AgdaSymbol{:}\AgdaSpace{}%
\AgdaDatatype{Value}\AgdaSymbol{\}}\AgdaSpace{}%
\AgdaSymbol{→}\AgdaSpace{}%
\AgdaFunction{`∅}\AgdaSpace{}%
\AgdaOperator{\AgdaDatatype{⊢}}\AgdaSpace{}%
\AgdaFunction{twoᶜ}\AgdaSpace{}%
\AgdaOperator{\AgdaDatatype{↓}}\AgdaSpace{}%
\AgdaSymbol{((}\AgdaBound{u}\AgdaSpace{}%
\AgdaOperator{\AgdaInductiveConstructor{↦}}\AgdaSpace{}%
\AgdaBound{v}\AgdaSpace{}%
\AgdaOperator{\AgdaInductiveConstructor{⊔}}\AgdaSpace{}%
\AgdaBound{v}\AgdaSpace{}%
\AgdaOperator{\AgdaInductiveConstructor{↦}}\AgdaSpace{}%
\AgdaBound{w}\AgdaSymbol{)}\AgdaSpace{}%
\AgdaOperator{\AgdaInductiveConstructor{↦}}\AgdaSpace{}%
\AgdaBound{u}\AgdaSpace{}%
\AgdaOperator{\AgdaInductiveConstructor{↦}}\AgdaSpace{}%
\AgdaBound{w}\AgdaSymbol{)}\<%
\\
\>[0]\AgdaFunction{denot-twoᶜ}\AgdaSpace{}%
\AgdaSymbol{\{}\AgdaBound{u}\AgdaSymbol{\}\{}\AgdaBound{v}\AgdaSymbol{\}\{}\AgdaBound{w}\AgdaSymbol{\}}\AgdaSpace{}%
\AgdaSymbol{=}\<%
\\
\>[0][@{}l@{\AgdaIndent{0}}]%
\>[2]\AgdaInductiveConstructor{↦-intro}\AgdaSpace{}%
\AgdaSymbol{(}\AgdaInductiveConstructor{↦-intro}\AgdaSpace{}%
\AgdaSymbol{(}\AgdaInductiveConstructor{↦-elim}\AgdaSpace{}%
\AgdaSymbol{(}\AgdaInductiveConstructor{sub}\AgdaSpace{}%
\AgdaInductiveConstructor{var}\AgdaSpace{}%
\AgdaFunction{lt1}\AgdaSymbol{)}\AgdaSpace{}%
\AgdaSymbol{(}\AgdaInductiveConstructor{↦-elim}\AgdaSpace{}%
\AgdaSymbol{(}\AgdaInductiveConstructor{sub}\AgdaSpace{}%
\AgdaInductiveConstructor{var}\AgdaSpace{}%
\AgdaFunction{lt2}\AgdaSymbol{)}\AgdaSpace{}%
\AgdaInductiveConstructor{var}\AgdaSymbol{)))}\<%
\\
%
\>[2]\AgdaKeyword{where}%
\>[892I]\AgdaFunction{lt1}\AgdaSpace{}%
\AgdaSymbol{:}\AgdaSpace{}%
\AgdaBound{v}\AgdaSpace{}%
\AgdaOperator{\AgdaInductiveConstructor{↦}}\AgdaSpace{}%
\AgdaBound{w}\AgdaSpace{}%
\AgdaOperator{\AgdaDatatype{⊑}}\AgdaSpace{}%
\AgdaBound{u}\AgdaSpace{}%
\AgdaOperator{\AgdaInductiveConstructor{↦}}\AgdaSpace{}%
\AgdaBound{v}\AgdaSpace{}%
\AgdaOperator{\AgdaInductiveConstructor{⊔}}\AgdaSpace{}%
\AgdaBound{v}\AgdaSpace{}%
\AgdaOperator{\AgdaInductiveConstructor{↦}}\AgdaSpace{}%
\AgdaBound{w}\<%
\\
\>[.][@{}l@{}]\<[892I]%
\>[8]\AgdaFunction{lt1}\AgdaSpace{}%
\AgdaSymbol{=}\AgdaSpace{}%
\AgdaInductiveConstructor{⊑-conj-R2}\AgdaSpace{}%
\AgdaSymbol{(}\AgdaInductiveConstructor{⊑-fun}\AgdaSpace{}%
\AgdaFunction{⊑-refl}\AgdaSpace{}%
\AgdaFunction{⊑-refl}\AgdaSymbol{)}\<%
\\
%
\\[\AgdaEmptyExtraSkip]%
%
\>[8]\AgdaFunction{lt2}\AgdaSpace{}%
\AgdaSymbol{:}\AgdaSpace{}%
\AgdaBound{u}\AgdaSpace{}%
\AgdaOperator{\AgdaInductiveConstructor{↦}}\AgdaSpace{}%
\AgdaBound{v}\AgdaSpace{}%
\AgdaOperator{\AgdaDatatype{⊑}}\AgdaSpace{}%
\AgdaBound{u}\AgdaSpace{}%
\AgdaOperator{\AgdaInductiveConstructor{↦}}\AgdaSpace{}%
\AgdaBound{v}\AgdaSpace{}%
\AgdaOperator{\AgdaInductiveConstructor{⊔}}\AgdaSpace{}%
\AgdaBound{v}\AgdaSpace{}%
\AgdaOperator{\AgdaInductiveConstructor{↦}}\AgdaSpace{}%
\AgdaBound{w}\<%
\\
%
\>[8]\AgdaFunction{lt2}\AgdaSpace{}%
\AgdaSymbol{=}\AgdaSpace{}%
\AgdaSymbol{(}\AgdaInductiveConstructor{⊑-conj-R1}\AgdaSpace{}%
\AgdaSymbol{(}\AgdaInductiveConstructor{⊑-fun}\AgdaSpace{}%
\AgdaFunction{⊑-refl}\AgdaSpace{}%
\AgdaFunction{⊑-refl}\AgdaSymbol{))}\<%
\end{code}
\end{fence}

Next we have a classic example of self application:
\texttt{Δ\ =\ λx.\ (x\ x)}. The input value for \texttt{x} needs to be a
table, and it needs to have an entry that maps a smaller version of
itself, call it \texttt{v}, to some value \texttt{w}. So the input value
looks like \texttt{v\ ↦\ w\ ⊔\ v}. Of course, then the output of
\texttt{Δ} is \texttt{w}. The derivation is given below. The first
occurrences of \texttt{x} evaluates to \texttt{v\ ↦\ w}, the second
occurrence of \texttt{x} evaluates to \texttt{v}, and then the result of
the application is \texttt{w}.

\begin{fence}
\begin{code}%
\>[0]\AgdaFunction{Δ}\AgdaSpace{}%
\AgdaSymbol{:}\AgdaSpace{}%
\AgdaInductiveConstructor{∅}\AgdaSpace{}%
\AgdaOperator{\AgdaDatatype{⊢}}\AgdaSpace{}%
\AgdaInductiveConstructor{★}\<%
\\
\>[0]\AgdaFunction{Δ}\AgdaSpace{}%
\AgdaSymbol{=}\AgdaSpace{}%
\AgdaSymbol{(}\AgdaOperator{\AgdaInductiveConstructor{ƛ}}\AgdaSpace{}%
\AgdaSymbol{(}\AgdaOperator{\AgdaFunction{\#}}\AgdaSpace{}%
\AgdaNumber{0}\AgdaSymbol{)}\AgdaSpace{}%
\AgdaOperator{\AgdaInductiveConstructor{·}}\AgdaSpace{}%
\AgdaSymbol{(}\AgdaOperator{\AgdaFunction{\#}}\AgdaSpace{}%
\AgdaNumber{0}\AgdaSymbol{))}\<%
\\
%
\\[\AgdaEmptyExtraSkip]%
\>[0]\AgdaFunction{denot-Δ}\AgdaSpace{}%
\AgdaSymbol{:}\AgdaSpace{}%
\AgdaSymbol{∀}\AgdaSpace{}%
\AgdaSymbol{\{}\AgdaBound{v}\AgdaSpace{}%
\AgdaBound{w}\AgdaSymbol{\}}\AgdaSpace{}%
\AgdaSymbol{→}\AgdaSpace{}%
\AgdaFunction{`∅}\AgdaSpace{}%
\AgdaOperator{\AgdaDatatype{⊢}}\AgdaSpace{}%
\AgdaFunction{Δ}\AgdaSpace{}%
\AgdaOperator{\AgdaDatatype{↓}}\AgdaSpace{}%
\AgdaSymbol{((}\AgdaBound{v}\AgdaSpace{}%
\AgdaOperator{\AgdaInductiveConstructor{↦}}\AgdaSpace{}%
\AgdaBound{w}\AgdaSpace{}%
\AgdaOperator{\AgdaInductiveConstructor{⊔}}\AgdaSpace{}%
\AgdaBound{v}\AgdaSymbol{)}\AgdaSpace{}%
\AgdaOperator{\AgdaInductiveConstructor{↦}}\AgdaSpace{}%
\AgdaBound{w}\AgdaSymbol{)}\<%
\\
\>[0]\AgdaFunction{denot-Δ}\AgdaSpace{}%
\AgdaSymbol{=}\AgdaSpace{}%
\AgdaInductiveConstructor{↦-intro}\AgdaSpace{}%
\AgdaSymbol{(}\AgdaInductiveConstructor{↦-elim}%
\>[957I]\AgdaSymbol{(}\AgdaInductiveConstructor{sub}\AgdaSpace{}%
\AgdaInductiveConstructor{var}\AgdaSpace{}%
\AgdaSymbol{(}\AgdaInductiveConstructor{⊑-conj-R1}\AgdaSpace{}%
\AgdaFunction{⊑-refl}\AgdaSymbol{))}\<%
\\
\>[.][@{}l@{}]\<[957I]%
\>[26]\AgdaSymbol{(}\AgdaInductiveConstructor{sub}\AgdaSpace{}%
\AgdaInductiveConstructor{var}\AgdaSpace{}%
\AgdaSymbol{(}\AgdaInductiveConstructor{⊑-conj-R2}\AgdaSpace{}%
\AgdaFunction{⊑-refl}\AgdaSymbol{)))}\<%
\end{code}
\end{fence}

One might worry whether this semantics can deal with diverging programs.
The \texttt{⊥} value and the \texttt{⊥-intro} rule provide a way to
handle them. (The \texttt{⊥-intro} rule is also what enables β reduction
on non-terminating arguments.) The classic \texttt{Ω} program is a
particularly simple program that diverges. It applies \texttt{Δ} to
itself. The semantics assigns to \texttt{Ω} the meaning \texttt{⊥}.
There are several ways to derive this, we shall start with one that
makes use of the \texttt{⊔-intro} rule. First, \texttt{denot-Δ} tells us
that \texttt{Δ} evaluates to \texttt{((⊥\ ↦\ ⊥)\ ⊔\ ⊥)\ ↦\ ⊥} (choose
\texttt{v₁\ =\ v₂\ =\ ⊥}). Next, \texttt{Δ} also evaluates to
\texttt{⊥\ ↦\ ⊥} by use of \texttt{↦-intro} and \texttt{⊥-intro} and to
\texttt{⊥} by \texttt{⊥-intro}. As we saw previously, whenever we can
show that a program evaluates to two values, we can apply
\texttt{⊔-intro} to join them together, so \texttt{Δ} evaluates to
\texttt{(⊥\ ↦\ ⊥)\ ⊔\ ⊥}. This matches the input of the first occurrence
of \texttt{Δ}, so we can conclude that the result of the application is
\texttt{⊥}.

\begin{fence}
\begin{code}%
\>[0]\AgdaFunction{Ω}\AgdaSpace{}%
\AgdaSymbol{:}\AgdaSpace{}%
\AgdaInductiveConstructor{∅}\AgdaSpace{}%
\AgdaOperator{\AgdaDatatype{⊢}}\AgdaSpace{}%
\AgdaInductiveConstructor{★}\<%
\\
\>[0]\AgdaFunction{Ω}\AgdaSpace{}%
\AgdaSymbol{=}\AgdaSpace{}%
\AgdaFunction{Δ}\AgdaSpace{}%
\AgdaOperator{\AgdaInductiveConstructor{·}}\AgdaSpace{}%
\AgdaFunction{Δ}\<%
\\
%
\\[\AgdaEmptyExtraSkip]%
\>[0]\AgdaFunction{denot-Ω}\AgdaSpace{}%
\AgdaSymbol{:}\AgdaSpace{}%
\AgdaFunction{`∅}\AgdaSpace{}%
\AgdaOperator{\AgdaDatatype{⊢}}\AgdaSpace{}%
\AgdaFunction{Ω}\AgdaSpace{}%
\AgdaOperator{\AgdaDatatype{↓}}\AgdaSpace{}%
\AgdaInductiveConstructor{⊥}\<%
\\
\>[0]\AgdaFunction{denot-Ω}\AgdaSpace{}%
\AgdaSymbol{=}\AgdaSpace{}%
\AgdaInductiveConstructor{↦-elim}\AgdaSpace{}%
\AgdaFunction{denot-Δ}\AgdaSpace{}%
\AgdaSymbol{(}\AgdaInductiveConstructor{⊔-intro}\AgdaSpace{}%
\AgdaSymbol{(}\AgdaInductiveConstructor{↦-intro}\AgdaSpace{}%
\AgdaInductiveConstructor{⊥-intro}\AgdaSymbol{)}\AgdaSpace{}%
\AgdaInductiveConstructor{⊥-intro}\AgdaSymbol{)}\<%
\end{code}
\end{fence}

A shorter derivation of the same result is by just one use of the
\texttt{⊥-intro} rule.

\begin{fence}
\begin{code}%
\>[0]\AgdaFunction{denot-Ω'}\AgdaSpace{}%
\AgdaSymbol{:}\AgdaSpace{}%
\AgdaFunction{`∅}\AgdaSpace{}%
\AgdaOperator{\AgdaDatatype{⊢}}\AgdaSpace{}%
\AgdaFunction{Ω}\AgdaSpace{}%
\AgdaOperator{\AgdaDatatype{↓}}\AgdaSpace{}%
\AgdaInductiveConstructor{⊥}\<%
\\
\>[0]\AgdaFunction{denot-Ω'}\AgdaSpace{}%
\AgdaSymbol{=}\AgdaSpace{}%
\AgdaInductiveConstructor{⊥-intro}\<%
\end{code}
\end{fence}

Just because one can derive \texttt{∅\ ⊢\ M\ ↓\ ⊥} for some closed term
\texttt{M} doesn't mean that \texttt{M} necessarily diverges. There may
be other derivations that conclude with \texttt{M} producing some more
informative value. However, if the only thing that a term evaluates to
is \texttt{⊥}, then it indeed diverges.

An attentive reader may have noticed a disconnect earlier in the way we
planned to solve the self-application problem and the actual
\texttt{↦-elim} rule for application. We said at the beginning that we
would relax the notion of table lookup, allowing an argument to match an
input entry if the argument is equal or greater than the input entry.
Instead, the \texttt{↦-elim} rule seems to require an exact match.
However, because of the \texttt{sub} rule, application really does allow
larger arguments.

\begin{fence}
\begin{code}%
\>[0]\AgdaFunction{↦-elim2}\AgdaSpace{}%
\AgdaSymbol{:}\AgdaSpace{}%
\AgdaSymbol{∀}\AgdaSpace{}%
\AgdaSymbol{\{}\AgdaBound{Γ}\AgdaSymbol{\}}\AgdaSpace{}%
\AgdaSymbol{\{}\AgdaBound{γ}\AgdaSpace{}%
\AgdaSymbol{:}\AgdaSpace{}%
\AgdaFunction{Env}\AgdaSpace{}%
\AgdaBound{Γ}\AgdaSymbol{\}}\AgdaSpace{}%
\AgdaSymbol{\{}\AgdaBound{M₁}\AgdaSpace{}%
\AgdaBound{M₂}\AgdaSpace{}%
\AgdaBound{v₁}\AgdaSpace{}%
\AgdaBound{v₂}\AgdaSpace{}%
\AgdaBound{v₃}\AgdaSymbol{\}}\<%
\\
\>[0][@{}l@{\AgdaIndent{0}}]%
\>[2]\AgdaSymbol{→}\AgdaSpace{}%
\AgdaBound{γ}\AgdaSpace{}%
\AgdaOperator{\AgdaDatatype{⊢}}\AgdaSpace{}%
\AgdaBound{M₁}\AgdaSpace{}%
\AgdaOperator{\AgdaDatatype{↓}}\AgdaSpace{}%
\AgdaSymbol{(}\AgdaBound{v₁}\AgdaSpace{}%
\AgdaOperator{\AgdaInductiveConstructor{↦}}\AgdaSpace{}%
\AgdaBound{v₃}\AgdaSymbol{)}\<%
\\
%
\>[2]\AgdaSymbol{→}\AgdaSpace{}%
\AgdaBound{γ}\AgdaSpace{}%
\AgdaOperator{\AgdaDatatype{⊢}}\AgdaSpace{}%
\AgdaBound{M₂}\AgdaSpace{}%
\AgdaOperator{\AgdaDatatype{↓}}\AgdaSpace{}%
\AgdaBound{v₂}\<%
\\
%
\>[2]\AgdaSymbol{→}%
\>[1017I]\AgdaBound{v₁}\AgdaSpace{}%
\AgdaOperator{\AgdaDatatype{⊑}}\AgdaSpace{}%
\AgdaBound{v₂}\<%
\\
\>[.][@{}l@{}]\<[1017I]%
\>[4]\AgdaComment{------------------}\<%
\\
%
\>[2]\AgdaSymbol{→}\AgdaSpace{}%
\AgdaBound{γ}\AgdaSpace{}%
\AgdaOperator{\AgdaDatatype{⊢}}\AgdaSpace{}%
\AgdaSymbol{(}\AgdaBound{M₁}\AgdaSpace{}%
\AgdaOperator{\AgdaInductiveConstructor{·}}\AgdaSpace{}%
\AgdaBound{M₂}\AgdaSymbol{)}\AgdaSpace{}%
\AgdaOperator{\AgdaDatatype{↓}}\AgdaSpace{}%
\AgdaBound{v₃}\<%
\\
\>[0]\AgdaFunction{↦-elim2}\AgdaSpace{}%
\AgdaBound{d₁}\AgdaSpace{}%
\AgdaBound{d₂}\AgdaSpace{}%
\AgdaBound{lt}\AgdaSpace{}%
\AgdaSymbol{=}\AgdaSpace{}%
\AgdaInductiveConstructor{↦-elim}\AgdaSpace{}%
\AgdaBound{d₁}\AgdaSpace{}%
\AgdaSymbol{(}\AgdaInductiveConstructor{sub}\AgdaSpace{}%
\AgdaBound{d₂}\AgdaSpace{}%
\AgdaBound{lt}\AgdaSymbol{)}\<%
\end{code}
\end{fence}

\hypertarget{exercise-denot-plusux1d9c-practice}{%
\subsubsection{\texorpdfstring{Exercise \texttt{denot-plusᶜ}
(practice)}{Exercise denot-plusᶜ (practice)}}\label{exercise-denot-plusux1d9c-practice}}

What is a denotation for \texttt{plusᶜ}? That is, find a value
\texttt{v} (other than \texttt{⊥}) such that \texttt{∅\ ⊢\ plusᶜ\ ↓\ v}.
Also, give the proof of \texttt{∅\ ⊢\ plusᶜ\ ↓\ v} for your choice of
\texttt{v}.

\begin{fence}
\begin{code}%
\>[0]\AgdaComment{-- Your code goes here}\<%
\end{code}
\end{fence}

\hypertarget{denotations-and-denotational-equality}{%
\section{Denotations and denotational
equality}\label{denotations-and-denotational-equality}}

Next we define a notion of denotational equality based on the above
semantics. Its statement makes use of an if-and-only-if, which we define
as follows.

\begin{fence}
\begin{code}%
\>[0]\AgdaOperator{\AgdaFunction{\AgdaUnderscore{}iff\AgdaUnderscore{}}}\AgdaSpace{}%
\AgdaSymbol{:}\AgdaSpace{}%
\AgdaPrimitiveType{Set}\AgdaSpace{}%
\AgdaSymbol{→}\AgdaSpace{}%
\AgdaPrimitiveType{Set}\AgdaSpace{}%
\AgdaSymbol{→}\AgdaSpace{}%
\AgdaPrimitiveType{Set}\<%
\\
\>[0]\AgdaBound{P}\AgdaSpace{}%
\AgdaOperator{\AgdaFunction{iff}}\AgdaSpace{}%
\AgdaBound{Q}\AgdaSpace{}%
\AgdaSymbol{=}\AgdaSpace{}%
\AgdaSymbol{(}\AgdaBound{P}\AgdaSpace{}%
\AgdaSymbol{→}\AgdaSpace{}%
\AgdaBound{Q}\AgdaSymbol{)}\AgdaSpace{}%
\AgdaOperator{\AgdaFunction{×}}\AgdaSpace{}%
\AgdaSymbol{(}\AgdaBound{Q}\AgdaSpace{}%
\AgdaSymbol{→}\AgdaSpace{}%
\AgdaBound{P}\AgdaSymbol{)}\<%
\end{code}
\end{fence}

Another way to view the denotational semantics is as a function that
maps a term to a relation from environments to values. That is, the
\emph{denotation} of a term is a relation from environments to values.

\begin{fence}
\begin{code}%
\>[0]\AgdaFunction{Denotation}\AgdaSpace{}%
\AgdaSymbol{:}\AgdaSpace{}%
\AgdaDatatype{Context}\AgdaSpace{}%
\AgdaSymbol{→}\AgdaSpace{}%
\AgdaPrimitiveType{Set₁}\<%
\\
\>[0]\AgdaFunction{Denotation}\AgdaSpace{}%
\AgdaBound{Γ}\AgdaSpace{}%
\AgdaSymbol{=}\AgdaSpace{}%
\AgdaSymbol{(}\AgdaFunction{Env}\AgdaSpace{}%
\AgdaBound{Γ}\AgdaSpace{}%
\AgdaSymbol{→}\AgdaSpace{}%
\AgdaDatatype{Value}\AgdaSpace{}%
\AgdaSymbol{→}\AgdaSpace{}%
\AgdaPrimitiveType{Set}\AgdaSymbol{)}\<%
\end{code}
\end{fence}

The following function ℰ gives this alternative view of the semantics,
which really just amounts to changing the order of the parameters.

\begin{fence}
\begin{code}%
\>[0]\AgdaFunction{ℰ}\AgdaSpace{}%
\AgdaSymbol{:}\AgdaSpace{}%
\AgdaSymbol{∀\{}\AgdaBound{Γ}\AgdaSymbol{\}}\AgdaSpace{}%
\AgdaSymbol{→}\AgdaSpace{}%
\AgdaSymbol{(}\AgdaBound{M}\AgdaSpace{}%
\AgdaSymbol{:}\AgdaSpace{}%
\AgdaBound{Γ}\AgdaSpace{}%
\AgdaOperator{\AgdaDatatype{⊢}}\AgdaSpace{}%
\AgdaInductiveConstructor{★}\AgdaSymbol{)}\AgdaSpace{}%
\AgdaSymbol{→}\AgdaSpace{}%
\AgdaFunction{Denotation}\AgdaSpace{}%
\AgdaBound{Γ}\<%
\\
\>[0]\AgdaFunction{ℰ}\AgdaSpace{}%
\AgdaBound{M}\AgdaSpace{}%
\AgdaSymbol{=}\AgdaSpace{}%
\AgdaSymbol{λ}\AgdaSpace{}%
\AgdaBound{γ}\AgdaSpace{}%
\AgdaBound{v}\AgdaSpace{}%
\AgdaSymbol{→}\AgdaSpace{}%
\AgdaBound{γ}\AgdaSpace{}%
\AgdaOperator{\AgdaDatatype{⊢}}\AgdaSpace{}%
\AgdaBound{M}\AgdaSpace{}%
\AgdaOperator{\AgdaDatatype{↓}}\AgdaSpace{}%
\AgdaBound{v}\<%
\end{code}
\end{fence}

In general, two denotations are equal when they produce the same values
in the same environment.

\begin{fence}
\begin{code}%
\>[0]\AgdaKeyword{infix}\AgdaSpace{}%
\AgdaNumber{3}\AgdaSpace{}%
\AgdaOperator{\AgdaFunction{\AgdaUnderscore{}≃\AgdaUnderscore{}}}\<%
\\
%
\\[\AgdaEmptyExtraSkip]%
\>[0]\AgdaOperator{\AgdaFunction{\AgdaUnderscore{}≃\AgdaUnderscore{}}}\AgdaSpace{}%
\AgdaSymbol{:}\AgdaSpace{}%
\AgdaSymbol{∀}\AgdaSpace{}%
\AgdaSymbol{\{}\AgdaBound{Γ}\AgdaSymbol{\}}\AgdaSpace{}%
\AgdaSymbol{→}\AgdaSpace{}%
\AgdaSymbol{(}\AgdaFunction{Denotation}\AgdaSpace{}%
\AgdaBound{Γ}\AgdaSymbol{)}\AgdaSpace{}%
\AgdaSymbol{→}\AgdaSpace{}%
\AgdaSymbol{(}\AgdaFunction{Denotation}\AgdaSpace{}%
\AgdaBound{Γ}\AgdaSymbol{)}\AgdaSpace{}%
\AgdaSymbol{→}\AgdaSpace{}%
\AgdaPrimitiveType{Set}\<%
\\
\>[0]\AgdaSymbol{(}\AgdaOperator{\AgdaFunction{\AgdaUnderscore{}≃\AgdaUnderscore{}}}\AgdaSpace{}%
\AgdaSymbol{\{}\AgdaBound{Γ}\AgdaSymbol{\}}\AgdaSpace{}%
\AgdaBound{D₁}\AgdaSpace{}%
\AgdaBound{D₂}\AgdaSymbol{)}\AgdaSpace{}%
\AgdaSymbol{=}\AgdaSpace{}%
\AgdaSymbol{(}\AgdaBound{γ}\AgdaSpace{}%
\AgdaSymbol{:}\AgdaSpace{}%
\AgdaFunction{Env}\AgdaSpace{}%
\AgdaBound{Γ}\AgdaSymbol{)}\AgdaSpace{}%
\AgdaSymbol{→}\AgdaSpace{}%
\AgdaSymbol{(}\AgdaBound{v}\AgdaSpace{}%
\AgdaSymbol{:}\AgdaSpace{}%
\AgdaDatatype{Value}\AgdaSymbol{)}\AgdaSpace{}%
\AgdaSymbol{→}\AgdaSpace{}%
\AgdaBound{D₁}\AgdaSpace{}%
\AgdaBound{γ}\AgdaSpace{}%
\AgdaBound{v}\AgdaSpace{}%
\AgdaOperator{\AgdaFunction{iff}}\AgdaSpace{}%
\AgdaBound{D₂}\AgdaSpace{}%
\AgdaBound{γ}\AgdaSpace{}%
\AgdaBound{v}\<%
\end{code}
\end{fence}

Denotational equality is an equivalence relation.

\begin{fence}
\begin{code}%
\>[0]\AgdaFunction{≃-refl}\AgdaSpace{}%
\AgdaSymbol{:}\AgdaSpace{}%
\AgdaSymbol{∀}\AgdaSpace{}%
\AgdaSymbol{\{}\AgdaBound{Γ}\AgdaSpace{}%
\AgdaSymbol{:}\AgdaSpace{}%
\AgdaDatatype{Context}\AgdaSymbol{\}}\AgdaSpace{}%
\AgdaSymbol{→}\AgdaSpace{}%
\AgdaSymbol{\{}\AgdaBound{M}\AgdaSpace{}%
\AgdaSymbol{:}\AgdaSpace{}%
\AgdaFunction{Denotation}\AgdaSpace{}%
\AgdaBound{Γ}\AgdaSymbol{\}}\<%
\\
\>[0][@{}l@{\AgdaIndent{0}}]%
\>[2]\AgdaSymbol{→}\AgdaSpace{}%
\AgdaBound{M}\AgdaSpace{}%
\AgdaOperator{\AgdaFunction{≃}}\AgdaSpace{}%
\AgdaBound{M}\<%
\\
\>[0]\AgdaFunction{≃-refl}\AgdaSpace{}%
\AgdaBound{γ}\AgdaSpace{}%
\AgdaBound{v}\AgdaSpace{}%
\AgdaSymbol{=}\AgdaSpace{}%
\AgdaOperator{\AgdaInductiveConstructor{⟨}}\AgdaSpace{}%
\AgdaSymbol{(λ}\AgdaSpace{}%
\AgdaBound{x}\AgdaSpace{}%
\AgdaSymbol{→}\AgdaSpace{}%
\AgdaBound{x}\AgdaSymbol{)}\AgdaSpace{}%
\AgdaOperator{\AgdaInductiveConstructor{,}}\AgdaSpace{}%
\AgdaSymbol{(λ}\AgdaSpace{}%
\AgdaBound{x}\AgdaSpace{}%
\AgdaSymbol{→}\AgdaSpace{}%
\AgdaBound{x}\AgdaSymbol{)}\AgdaSpace{}%
\AgdaOperator{\AgdaInductiveConstructor{⟩}}\<%
\\
%
\\[\AgdaEmptyExtraSkip]%
\>[0]\AgdaFunction{≃-sym}\AgdaSpace{}%
\AgdaSymbol{:}\AgdaSpace{}%
\AgdaSymbol{∀}\AgdaSpace{}%
\AgdaSymbol{\{}\AgdaBound{Γ}\AgdaSpace{}%
\AgdaSymbol{:}\AgdaSpace{}%
\AgdaDatatype{Context}\AgdaSymbol{\}}\AgdaSpace{}%
\AgdaSymbol{→}\AgdaSpace{}%
\AgdaSymbol{\{}\AgdaBound{M}\AgdaSpace{}%
\AgdaBound{N}\AgdaSpace{}%
\AgdaSymbol{:}\AgdaSpace{}%
\AgdaFunction{Denotation}\AgdaSpace{}%
\AgdaBound{Γ}\AgdaSymbol{\}}\<%
\\
\>[0][@{}l@{\AgdaIndent{0}}]%
\>[2]\AgdaSymbol{→}%
\>[1157I]\AgdaBound{M}\AgdaSpace{}%
\AgdaOperator{\AgdaFunction{≃}}\AgdaSpace{}%
\AgdaBound{N}\<%
\\
\>[.][@{}l@{}]\<[1157I]%
\>[4]\AgdaComment{-----}\<%
\\
%
\>[2]\AgdaSymbol{→}\AgdaSpace{}%
\AgdaBound{N}\AgdaSpace{}%
\AgdaOperator{\AgdaFunction{≃}}\AgdaSpace{}%
\AgdaBound{M}\<%
\\
\>[0]\AgdaFunction{≃-sym}\AgdaSpace{}%
\AgdaBound{eq}\AgdaSpace{}%
\AgdaBound{γ}\AgdaSpace{}%
\AgdaBound{v}\AgdaSpace{}%
\AgdaSymbol{=}\AgdaSpace{}%
\AgdaOperator{\AgdaInductiveConstructor{⟨}}\AgdaSpace{}%
\AgdaSymbol{(}\AgdaField{proj₂}\AgdaSpace{}%
\AgdaSymbol{(}\AgdaBound{eq}\AgdaSpace{}%
\AgdaBound{γ}\AgdaSpace{}%
\AgdaBound{v}\AgdaSymbol{))}\AgdaSpace{}%
\AgdaOperator{\AgdaInductiveConstructor{,}}\AgdaSpace{}%
\AgdaSymbol{(}\AgdaField{proj₁}\AgdaSpace{}%
\AgdaSymbol{(}\AgdaBound{eq}\AgdaSpace{}%
\AgdaBound{γ}\AgdaSpace{}%
\AgdaBound{v}\AgdaSymbol{))}\AgdaSpace{}%
\AgdaOperator{\AgdaInductiveConstructor{⟩}}\<%
\\
%
\\[\AgdaEmptyExtraSkip]%
\>[0]\AgdaFunction{≃-trans}\AgdaSpace{}%
\AgdaSymbol{:}\AgdaSpace{}%
\AgdaSymbol{∀}\AgdaSpace{}%
\AgdaSymbol{\{}\AgdaBound{Γ}\AgdaSpace{}%
\AgdaSymbol{:}\AgdaSpace{}%
\AgdaDatatype{Context}\AgdaSymbol{\}}\AgdaSpace{}%
\AgdaSymbol{→}\AgdaSpace{}%
\AgdaSymbol{\{}\AgdaBound{M₁}\AgdaSpace{}%
\AgdaBound{M₂}\AgdaSpace{}%
\AgdaBound{M₃}\AgdaSpace{}%
\AgdaSymbol{:}\AgdaSpace{}%
\AgdaFunction{Denotation}\AgdaSpace{}%
\AgdaBound{Γ}\AgdaSymbol{\}}\<%
\\
\>[0][@{}l@{\AgdaIndent{0}}]%
\>[2]\AgdaSymbol{→}\AgdaSpace{}%
\AgdaBound{M₁}\AgdaSpace{}%
\AgdaOperator{\AgdaFunction{≃}}\AgdaSpace{}%
\AgdaBound{M₂}\<%
\\
%
\>[2]\AgdaSymbol{→}%
\>[1193I]\AgdaBound{M₂}\AgdaSpace{}%
\AgdaOperator{\AgdaFunction{≃}}\AgdaSpace{}%
\AgdaBound{M₃}\<%
\\
\>[.][@{}l@{}]\<[1193I]%
\>[4]\AgdaComment{-------}\<%
\\
%
\>[2]\AgdaSymbol{→}\AgdaSpace{}%
\AgdaBound{M₁}\AgdaSpace{}%
\AgdaOperator{\AgdaFunction{≃}}\AgdaSpace{}%
\AgdaBound{M₃}\<%
\\
\>[0]\AgdaFunction{≃-trans}\AgdaSpace{}%
\AgdaBound{eq1}\AgdaSpace{}%
\AgdaBound{eq2}\AgdaSpace{}%
\AgdaBound{γ}\AgdaSpace{}%
\AgdaBound{v}\AgdaSpace{}%
\AgdaSymbol{=}\AgdaSpace{}%
\AgdaOperator{\AgdaInductiveConstructor{⟨}}%
\>[1205I]\AgdaSymbol{(λ}\AgdaSpace{}%
\AgdaBound{z}\AgdaSpace{}%
\AgdaSymbol{→}\AgdaSpace{}%
\AgdaField{proj₁}\AgdaSpace{}%
\AgdaSymbol{(}\AgdaBound{eq2}\AgdaSpace{}%
\AgdaBound{γ}\AgdaSpace{}%
\AgdaBound{v}\AgdaSymbol{)}\AgdaSpace{}%
\AgdaSymbol{(}\AgdaField{proj₁}\AgdaSpace{}%
\AgdaSymbol{(}\AgdaBound{eq1}\AgdaSpace{}%
\AgdaBound{γ}\AgdaSpace{}%
\AgdaBound{v}\AgdaSymbol{)}\AgdaSpace{}%
\AgdaBound{z}\AgdaSymbol{))}\AgdaSpace{}%
\AgdaOperator{\AgdaInductiveConstructor{,}}\<%
\\
\>[.][@{}l@{}]\<[1205I]%
\>[24]\AgdaSymbol{(λ}\AgdaSpace{}%
\AgdaBound{z}\AgdaSpace{}%
\AgdaSymbol{→}\AgdaSpace{}%
\AgdaField{proj₂}\AgdaSpace{}%
\AgdaSymbol{(}\AgdaBound{eq1}\AgdaSpace{}%
\AgdaBound{γ}\AgdaSpace{}%
\AgdaBound{v}\AgdaSymbol{)}\AgdaSpace{}%
\AgdaSymbol{(}\AgdaField{proj₂}\AgdaSpace{}%
\AgdaSymbol{(}\AgdaBound{eq2}\AgdaSpace{}%
\AgdaBound{γ}\AgdaSpace{}%
\AgdaBound{v}\AgdaSymbol{)}\AgdaSpace{}%
\AgdaBound{z}\AgdaSymbol{))}\AgdaSpace{}%
\AgdaOperator{\AgdaInductiveConstructor{⟩}}\<%
\end{code}
\end{fence}

Two terms \texttt{M} and \texttt{N} are denotational equal when their
denotations are equal, that is, \texttt{ℰ\ M\ ≃\ ℰ\ N}.

The following submodule introduces equational reasoning for the
\texttt{≃} relation.

\begin{fence}
\begin{code}%
\>[0]\<%
\\
\>[0]\AgdaKeyword{module}\AgdaSpace{}%
\AgdaModule{≃-Reasoning}\AgdaSpace{}%
\AgdaSymbol{\{}\AgdaBound{Γ}\AgdaSpace{}%
\AgdaSymbol{:}\AgdaSpace{}%
\AgdaDatatype{Context}\AgdaSymbol{\}}\AgdaSpace{}%
\AgdaKeyword{where}\<%
\\
%
\\[\AgdaEmptyExtraSkip]%
\>[0][@{}l@{\AgdaIndent{0}}]%
\>[2]\AgdaKeyword{infix}%
\>[9]\AgdaNumber{1}\AgdaSpace{}%
\AgdaOperator{\AgdaFunction{start\AgdaUnderscore{}}}\<%
\\
%
\>[2]\AgdaKeyword{infixr}\AgdaSpace{}%
\AgdaNumber{2}\AgdaSpace{}%
\AgdaOperator{\AgdaFunction{\AgdaUnderscore{}≃⟨⟩\AgdaUnderscore{}}}\AgdaSpace{}%
\AgdaOperator{\AgdaFunction{\AgdaUnderscore{}≃⟨\AgdaUnderscore{}⟩\AgdaUnderscore{}}}\<%
\\
%
\>[2]\AgdaKeyword{infix}%
\>[9]\AgdaNumber{3}\AgdaSpace{}%
\AgdaOperator{\AgdaFunction{\AgdaUnderscore{}☐}}\<%
\\
%
\\[\AgdaEmptyExtraSkip]%
%
\>[2]\AgdaOperator{\AgdaFunction{start\AgdaUnderscore{}}}\AgdaSpace{}%
\AgdaSymbol{:}\AgdaSpace{}%
\AgdaSymbol{∀}\AgdaSpace{}%
\AgdaSymbol{\{}\AgdaBound{x}\AgdaSpace{}%
\AgdaBound{y}\AgdaSpace{}%
\AgdaSymbol{:}\AgdaSpace{}%
\AgdaFunction{Denotation}\AgdaSpace{}%
\AgdaBound{Γ}\AgdaSymbol{\}}\<%
\\
\>[2][@{}l@{\AgdaIndent{0}}]%
\>[4]\AgdaSymbol{→}%
\>[1247I]\AgdaBound{x}\AgdaSpace{}%
\AgdaOperator{\AgdaFunction{≃}}\AgdaSpace{}%
\AgdaBound{y}\<%
\\
\>[.][@{}l@{}]\<[1247I]%
\>[6]\AgdaComment{-----}\<%
\\
%
\>[4]\AgdaSymbol{→}\AgdaSpace{}%
\AgdaBound{x}\AgdaSpace{}%
\AgdaOperator{\AgdaFunction{≃}}\AgdaSpace{}%
\AgdaBound{y}\<%
\\
%
\>[2]\AgdaOperator{\AgdaFunction{start}}\AgdaSpace{}%
\AgdaBound{x≃y}%
\>[13]\AgdaSymbol{=}%
\>[16]\AgdaBound{x≃y}\<%
\\
%
\\[\AgdaEmptyExtraSkip]%
%
\>[2]\AgdaOperator{\AgdaFunction{\AgdaUnderscore{}≃⟨\AgdaUnderscore{}⟩\AgdaUnderscore{}}}\AgdaSpace{}%
\AgdaSymbol{:}\AgdaSpace{}%
\AgdaSymbol{∀}\AgdaSpace{}%
\AgdaSymbol{(}\AgdaBound{x}\AgdaSpace{}%
\AgdaSymbol{:}\AgdaSpace{}%
\AgdaFunction{Denotation}\AgdaSpace{}%
\AgdaBound{Γ}\AgdaSymbol{)}\AgdaSpace{}%
\AgdaSymbol{\{}\AgdaBound{y}\AgdaSpace{}%
\AgdaBound{z}\AgdaSpace{}%
\AgdaSymbol{:}\AgdaSpace{}%
\AgdaFunction{Denotation}\AgdaSpace{}%
\AgdaBound{Γ}\AgdaSymbol{\}}\<%
\\
\>[2][@{}l@{\AgdaIndent{0}}]%
\>[4]\AgdaSymbol{→}\AgdaSpace{}%
\AgdaBound{x}\AgdaSpace{}%
\AgdaOperator{\AgdaFunction{≃}}\AgdaSpace{}%
\AgdaBound{y}\<%
\\
%
\>[4]\AgdaSymbol{→}%
\>[1268I]\AgdaBound{y}\AgdaSpace{}%
\AgdaOperator{\AgdaFunction{≃}}\AgdaSpace{}%
\AgdaBound{z}\<%
\\
\>[.][@{}l@{}]\<[1268I]%
\>[6]\AgdaComment{-----}\<%
\\
%
\>[4]\AgdaSymbol{→}\AgdaSpace{}%
\AgdaBound{x}\AgdaSpace{}%
\AgdaOperator{\AgdaFunction{≃}}\AgdaSpace{}%
\AgdaBound{z}\<%
\\
%
\>[2]\AgdaSymbol{(}\AgdaBound{x}\AgdaSpace{}%
\AgdaOperator{\AgdaFunction{≃⟨}}\AgdaSpace{}%
\AgdaBound{x≃y}\AgdaSpace{}%
\AgdaOperator{\AgdaFunction{⟩}}\AgdaSpace{}%
\AgdaBound{y≃z}\AgdaSymbol{)}\AgdaSpace{}%
\AgdaSymbol{=}%
\>[22]\AgdaFunction{≃-trans}\AgdaSpace{}%
\AgdaBound{x≃y}\AgdaSpace{}%
\AgdaBound{y≃z}\<%
\\
%
\\[\AgdaEmptyExtraSkip]%
%
\>[2]\AgdaOperator{\AgdaFunction{\AgdaUnderscore{}≃⟨⟩\AgdaUnderscore{}}}\AgdaSpace{}%
\AgdaSymbol{:}\AgdaSpace{}%
\AgdaSymbol{∀}\AgdaSpace{}%
\AgdaSymbol{(}\AgdaBound{x}\AgdaSpace{}%
\AgdaSymbol{:}\AgdaSpace{}%
\AgdaFunction{Denotation}\AgdaSpace{}%
\AgdaBound{Γ}\AgdaSymbol{)}\AgdaSpace{}%
\AgdaSymbol{\{}\AgdaBound{y}\AgdaSpace{}%
\AgdaSymbol{:}\AgdaSpace{}%
\AgdaFunction{Denotation}\AgdaSpace{}%
\AgdaBound{Γ}\AgdaSymbol{\}}\<%
\\
\>[2][@{}l@{\AgdaIndent{0}}]%
\>[4]\AgdaSymbol{→}%
\>[1291I]\AgdaBound{x}\AgdaSpace{}%
\AgdaOperator{\AgdaFunction{≃}}\AgdaSpace{}%
\AgdaBound{y}\<%
\\
\>[.][@{}l@{}]\<[1291I]%
\>[6]\AgdaComment{-----}\<%
\\
%
\>[4]\AgdaSymbol{→}\AgdaSpace{}%
\AgdaBound{x}\AgdaSpace{}%
\AgdaOperator{\AgdaFunction{≃}}\AgdaSpace{}%
\AgdaBound{y}\<%
\\
%
\>[2]\AgdaBound{x}\AgdaSpace{}%
\AgdaOperator{\AgdaFunction{≃⟨⟩}}\AgdaSpace{}%
\AgdaBound{x≃y}%
\>[13]\AgdaSymbol{=}%
\>[16]\AgdaBound{x≃y}\<%
\\
%
\\[\AgdaEmptyExtraSkip]%
%
\>[2]\AgdaOperator{\AgdaFunction{\AgdaUnderscore{}☐}}%
\>[1299I]\AgdaSymbol{:}\AgdaSpace{}%
\AgdaSymbol{∀}\AgdaSpace{}%
\AgdaSymbol{(}\AgdaBound{x}\AgdaSpace{}%
\AgdaSymbol{:}\AgdaSpace{}%
\AgdaFunction{Denotation}\AgdaSpace{}%
\AgdaBound{Γ}\AgdaSymbol{)}\<%
\\
\>[1299I][@{}l@{\AgdaIndent{0}}]%
\>[6]\AgdaComment{-----}\<%
\\
\>[2][@{}l@{\AgdaIndent{0}}]%
\>[4]\AgdaSymbol{→}\AgdaSpace{}%
\AgdaBound{x}\AgdaSpace{}%
\AgdaOperator{\AgdaFunction{≃}}\AgdaSpace{}%
\AgdaBound{x}\<%
\\
%
\>[2]\AgdaSymbol{(}\AgdaBound{x}\AgdaSpace{}%
\AgdaOperator{\AgdaFunction{☐}}\AgdaSymbol{)}%
\>[9]\AgdaSymbol{=}%
\>[12]\AgdaFunction{≃-refl}\<%
\end{code}
\end{fence}

\hypertarget{road-map-for-the-following-chapters}{%
\section{Road map for the following
chapters}\label{road-map-for-the-following-chapters}}

The subsequent chapters prove that the denotational semantics has
several desirable properties. First, we prove that the semantics is
compositional, i.e., that the denotation of a term is a function of the
denotations of its subterms. To do this we shall prove equations of the
following shape.

\begin{myDisplay}
ℰ (` x) ≃ ...
ℰ (ƛ M) ≃ ... ℰ M ...
ℰ (M · N) ≃ ... ℰ M ... ℰ N ...
\end{myDisplay}

The compositionality property is not trivial because the semantics we
have defined includes three rules that are not syntax directed:
\texttt{⊥-intro}, \texttt{⊔-intro}, and \texttt{sub}. The above
equations suggest that the denotational semantics can be defined as a
recursive function, and indeed, we give such a definition and prove that
it is equivalent to ℰ.

Next we investigate whether the denotational semantics and the reduction
semantics are equivalent. Recall that the job of a language semantics is
to describe the observable behavior of a given program \texttt{M}. For
the lambda calculus there are several choices that one can make, but
they usually boil down to a single bit of information:

\begin{itemize}
\tightlist
\item
  divergence: the program \texttt{M} executes forever.
\item
  termination: the program \texttt{M} halts.
\end{itemize}

We can characterize divergence and termination in terms of reduction.

\begin{itemize}
\tightlist
\item
  divergence: \texttt{¬\ (M\ —↠\ ƛ\ N)} for any term \texttt{N}.
\item
  termination: \texttt{M\ —↠\ ƛ\ N} for some term \texttt{N}.
\end{itemize}

We can also characterize divergence and termination using denotations.

\begin{itemize}
\tightlist
\item
  divergence: \texttt{¬\ (∅\ ⊢\ M\ ↓\ v\ ↦\ w)} for any \texttt{v} and
  \texttt{w}.
\item
  termination: \texttt{∅\ ⊢\ M\ ↓\ v\ ↦\ w} for some \texttt{v} and
  \texttt{w}.
\end{itemize}

Alternatively, we can use the denotation function \texttt{ℰ}.

\begin{itemize}
\tightlist
\item
  divergence: \texttt{¬\ (ℰ\ M\ ≃\ ℰ\ (ƛ\ N))} for any term \texttt{N}.
\item
  termination: \texttt{ℰ\ M\ ≃\ ℰ\ (ƛ\ N)} for some term \texttt{N}.
\end{itemize}

So the question is whether the reduction semantics and denotational
semantics are equivalent.

\begin{myDisplay}
(∃ N. M —↠ ƛ N)  iff  (∃ N. ℰ M ≃ ℰ (ƛ N))
\end{myDisplay}

We address each direction of the equivalence in the second and third
chapters. In the second chapter we prove that reduction to a lambda
abstraction implies denotational equality to a lambda abstraction. This
property is called the \emph{soundness} in the literature.

\begin{myDisplay}
M —↠ ƛ N  implies  ℰ M ≃ ℰ (ƛ N)
\end{myDisplay}

In the third chapter we prove that denotational equality to a lambda
abstraction implies reduction to a lambda abstraction. This property is
called \emph{adequacy} in the literature.

\begin{myDisplay}
ℰ M ≃ ℰ (ƛ N)  implies M —↠ ƛ N′ for some N′
\end{myDisplay}

The fourth chapter applies the results of the three preceding chapters
(compositionality, soundness, and adequacy) to prove that denotational
equality implies a property called \emph{contextual equivalence}. This
property is important because it justifies the use of denotational
equality in proving the correctness of program transformations such as
performance optimizations.

The proofs of all of these properties rely on some basic results about
the denotational semantics, which we establish in the rest of this
chapter. We start with some lemmas about renaming, which are quite
similar to the renaming lemmas that we have seen in previous chapters.
We conclude with a proof of an important inversion lemma for the
less-than relation regarding function values.

\hypertarget{renaming-preserves-denotations}{%
\section{Renaming preserves
denotations}\label{renaming-preserves-denotations}}

We shall prove that renaming variables, and changing the environment
accordingly, preserves the meaning of a term. We generalize the renaming
lemma to allow the values in the new environment to be the same or
larger than the original values. This generalization is useful in
proving that reduction implies denotational equality.

As before, we need an extension lemma to handle the case where we
proceed underneath a lambda abstraction. Suppose that \texttt{ρ} is a
renaming that maps variables in \texttt{γ} into variables with equal or
larger values in \texttt{δ}. This lemmas says that extending the
renaming producing a renaming \texttt{ext\ r} that maps \texttt{γ\ ,\ v}
to \texttt{δ\ ,\ v}.

\begin{fence}
\begin{code}%
\>[0]\AgdaFunction{ext-⊑}\AgdaSpace{}%
\AgdaSymbol{:}\AgdaSpace{}%
\AgdaSymbol{∀}\AgdaSpace{}%
\AgdaSymbol{\{}\AgdaBound{Γ}\AgdaSpace{}%
\AgdaBound{Δ}\AgdaSpace{}%
\AgdaBound{v}\AgdaSymbol{\}}\AgdaSpace{}%
\AgdaSymbol{\{}\AgdaBound{γ}\AgdaSpace{}%
\AgdaSymbol{:}\AgdaSpace{}%
\AgdaFunction{Env}\AgdaSpace{}%
\AgdaBound{Γ}\AgdaSymbol{\}}\AgdaSpace{}%
\AgdaSymbol{\{}\AgdaBound{δ}\AgdaSpace{}%
\AgdaSymbol{:}\AgdaSpace{}%
\AgdaFunction{Env}\AgdaSpace{}%
\AgdaBound{Δ}\AgdaSymbol{\}}\<%
\\
\>[0][@{}l@{\AgdaIndent{0}}]%
\>[2]\AgdaSymbol{→}\AgdaSpace{}%
\AgdaSymbol{(}\AgdaBound{ρ}\AgdaSpace{}%
\AgdaSymbol{:}\AgdaSpace{}%
\AgdaFunction{Rename}\AgdaSpace{}%
\AgdaBound{Γ}\AgdaSpace{}%
\AgdaBound{Δ}\AgdaSymbol{)}\<%
\\
%
\>[2]\AgdaSymbol{→}%
\>[1327I]\AgdaBound{γ}\AgdaSpace{}%
\AgdaOperator{\AgdaFunction{`⊑}}\AgdaSpace{}%
\AgdaSymbol{(}\AgdaBound{δ}\AgdaSpace{}%
\AgdaOperator{\AgdaFunction{∘}}\AgdaSpace{}%
\AgdaBound{ρ}\AgdaSymbol{)}\<%
\\
\>[.][@{}l@{}]\<[1327I]%
\>[4]\AgdaComment{------------------------------}\<%
\\
%
\>[2]\AgdaSymbol{→}\AgdaSpace{}%
\AgdaSymbol{(}\AgdaBound{γ}\AgdaSpace{}%
\AgdaOperator{\AgdaFunction{`,}}\AgdaSpace{}%
\AgdaBound{v}\AgdaSymbol{)}\AgdaSpace{}%
\AgdaOperator{\AgdaFunction{`⊑}}\AgdaSpace{}%
\AgdaSymbol{((}\AgdaBound{δ}\AgdaSpace{}%
\AgdaOperator{\AgdaFunction{`,}}\AgdaSpace{}%
\AgdaBound{v}\AgdaSymbol{)}\AgdaSpace{}%
\AgdaOperator{\AgdaFunction{∘}}\AgdaSpace{}%
\AgdaFunction{ext}\AgdaSpace{}%
\AgdaBound{ρ}\AgdaSymbol{)}\<%
\\
\>[0]\AgdaFunction{ext-⊑}\AgdaSpace{}%
\AgdaBound{ρ}\AgdaSpace{}%
\AgdaBound{lt}\AgdaSpace{}%
\AgdaInductiveConstructor{Z}\AgdaSpace{}%
\AgdaSymbol{=}\AgdaSpace{}%
\AgdaFunction{⊑-refl}\<%
\\
\>[0]\AgdaFunction{ext-⊑}\AgdaSpace{}%
\AgdaBound{ρ}\AgdaSpace{}%
\AgdaBound{lt}\AgdaSpace{}%
\AgdaSymbol{(}\AgdaOperator{\AgdaInductiveConstructor{S}}\AgdaSpace{}%
\AgdaBound{n′}\AgdaSymbol{)}\AgdaSpace{}%
\AgdaSymbol{=}\AgdaSpace{}%
\AgdaBound{lt}\AgdaSpace{}%
\AgdaBound{n′}\<%
\end{code}
\end{fence}

We proceed by cases on the de Bruijn index \texttt{n}.

\begin{itemize}
\item
  If it is \texttt{Z}, then we just need to show that \texttt{v\ ≡\ v},
  which we have by \texttt{refl}.
\item
  If it is \texttt{S\ n′}, then the goal simplifies to
  \texttt{γ\ n′\ ≡\ δ\ (ρ\ n′)}, which is an instance of the premise.
\end{itemize}

Now for the renaming lemma. Suppose we have a renaming that maps
variables in \texttt{γ} into variables with the same values in
\texttt{δ}. If \texttt{M} results in \texttt{v} when evaluated in
environment \texttt{γ}, then applying the renaming to \texttt{M}
produces a program that results in the same value \texttt{v} when
evaluated in \texttt{δ}.

\begin{fence}
\begin{code}%
\>[0]\AgdaFunction{rename-pres}\AgdaSpace{}%
\AgdaSymbol{:}\AgdaSpace{}%
\AgdaSymbol{∀}\AgdaSpace{}%
\AgdaSymbol{\{}\AgdaBound{Γ}\AgdaSpace{}%
\AgdaBound{Δ}\AgdaSpace{}%
\AgdaBound{v}\AgdaSymbol{\}}\AgdaSpace{}%
\AgdaSymbol{\{}\AgdaBound{γ}\AgdaSpace{}%
\AgdaSymbol{:}\AgdaSpace{}%
\AgdaFunction{Env}\AgdaSpace{}%
\AgdaBound{Γ}\AgdaSymbol{\}}\AgdaSpace{}%
\AgdaSymbol{\{}\AgdaBound{δ}\AgdaSpace{}%
\AgdaSymbol{:}\AgdaSpace{}%
\AgdaFunction{Env}\AgdaSpace{}%
\AgdaBound{Δ}\AgdaSymbol{\}}\AgdaSpace{}%
\AgdaSymbol{\{}\AgdaBound{M}\AgdaSpace{}%
\AgdaSymbol{:}\AgdaSpace{}%
\AgdaBound{Γ}\AgdaSpace{}%
\AgdaOperator{\AgdaDatatype{⊢}}\AgdaSpace{}%
\AgdaInductiveConstructor{★}\AgdaSymbol{\}}\<%
\\
\>[0][@{}l@{\AgdaIndent{0}}]%
\>[2]\AgdaSymbol{→}\AgdaSpace{}%
\AgdaSymbol{(}\AgdaBound{ρ}\AgdaSpace{}%
\AgdaSymbol{:}\AgdaSpace{}%
\AgdaFunction{Rename}\AgdaSpace{}%
\AgdaBound{Γ}\AgdaSpace{}%
\AgdaBound{Δ}\AgdaSymbol{)}\<%
\\
%
\>[2]\AgdaSymbol{→}\AgdaSpace{}%
\AgdaBound{γ}\AgdaSpace{}%
\AgdaOperator{\AgdaFunction{`⊑}}\AgdaSpace{}%
\AgdaSymbol{(}\AgdaBound{δ}\AgdaSpace{}%
\AgdaOperator{\AgdaFunction{∘}}\AgdaSpace{}%
\AgdaBound{ρ}\AgdaSymbol{)}\<%
\\
%
\>[2]\AgdaSymbol{→}%
\>[1382I]\AgdaBound{γ}\AgdaSpace{}%
\AgdaOperator{\AgdaDatatype{⊢}}\AgdaSpace{}%
\AgdaBound{M}\AgdaSpace{}%
\AgdaOperator{\AgdaDatatype{↓}}\AgdaSpace{}%
\AgdaBound{v}\<%
\\
\>[.][@{}l@{}]\<[1382I]%
\>[4]\AgdaComment{---------------------}\<%
\\
%
\>[2]\AgdaSymbol{→}\AgdaSpace{}%
\AgdaBound{δ}\AgdaSpace{}%
\AgdaOperator{\AgdaDatatype{⊢}}\AgdaSpace{}%
\AgdaSymbol{(}\AgdaFunction{rename}\AgdaSpace{}%
\AgdaBound{ρ}\AgdaSpace{}%
\AgdaBound{M}\AgdaSymbol{)}\AgdaSpace{}%
\AgdaOperator{\AgdaDatatype{↓}}\AgdaSpace{}%
\AgdaBound{v}\<%
\\
\>[0]\AgdaFunction{rename-pres}\AgdaSpace{}%
\AgdaBound{ρ}\AgdaSpace{}%
\AgdaBound{lt}\AgdaSpace{}%
\AgdaSymbol{(}\AgdaInductiveConstructor{var}\AgdaSpace{}%
\AgdaSymbol{\{}\AgdaArgument{x}\AgdaSpace{}%
\AgdaSymbol{=}\AgdaSpace{}%
\AgdaBound{x}\AgdaSymbol{\})}\AgdaSpace{}%
\AgdaSymbol{=}\AgdaSpace{}%
\AgdaInductiveConstructor{sub}\AgdaSpace{}%
\AgdaInductiveConstructor{var}\AgdaSpace{}%
\AgdaSymbol{(}\AgdaBound{lt}\AgdaSpace{}%
\AgdaBound{x}\AgdaSymbol{)}\<%
\\
\>[0]\AgdaFunction{rename-pres}\AgdaSpace{}%
\AgdaBound{ρ}\AgdaSpace{}%
\AgdaBound{lt}\AgdaSpace{}%
\AgdaSymbol{(}\AgdaInductiveConstructor{↦-elim}\AgdaSpace{}%
\AgdaBound{d}\AgdaSpace{}%
\AgdaBound{d₁}\AgdaSymbol{)}\AgdaSpace{}%
\AgdaSymbol{=}\<%
\\
\>[0][@{}l@{\AgdaIndent{0}}]%
\>[3]\AgdaInductiveConstructor{↦-elim}\AgdaSpace{}%
\AgdaSymbol{(}\AgdaFunction{rename-pres}\AgdaSpace{}%
\AgdaBound{ρ}\AgdaSpace{}%
\AgdaBound{lt}\AgdaSpace{}%
\AgdaBound{d}\AgdaSymbol{)}\AgdaSpace{}%
\AgdaSymbol{(}\AgdaFunction{rename-pres}\AgdaSpace{}%
\AgdaBound{ρ}\AgdaSpace{}%
\AgdaBound{lt}\AgdaSpace{}%
\AgdaBound{d₁}\AgdaSymbol{)}\<%
\\
\>[0]\AgdaFunction{rename-pres}\AgdaSpace{}%
\AgdaBound{ρ}\AgdaSpace{}%
\AgdaBound{lt}\AgdaSpace{}%
\AgdaSymbol{(}\AgdaInductiveConstructor{↦-intro}\AgdaSpace{}%
\AgdaBound{d}\AgdaSymbol{)}\AgdaSpace{}%
\AgdaSymbol{=}\<%
\\
\>[0][@{}l@{\AgdaIndent{0}}]%
\>[3]\AgdaInductiveConstructor{↦-intro}\AgdaSpace{}%
\AgdaSymbol{(}\AgdaFunction{rename-pres}\AgdaSpace{}%
\AgdaSymbol{(}\AgdaFunction{ext}\AgdaSpace{}%
\AgdaBound{ρ}\AgdaSymbol{)}\AgdaSpace{}%
\AgdaSymbol{(}\AgdaFunction{ext-⊑}\AgdaSpace{}%
\AgdaBound{ρ}\AgdaSpace{}%
\AgdaBound{lt}\AgdaSymbol{)}\AgdaSpace{}%
\AgdaBound{d}\AgdaSymbol{)}\<%
\\
\>[0]\AgdaFunction{rename-pres}\AgdaSpace{}%
\AgdaBound{ρ}\AgdaSpace{}%
\AgdaBound{lt}\AgdaSpace{}%
\AgdaInductiveConstructor{⊥-intro}\AgdaSpace{}%
\AgdaSymbol{=}\AgdaSpace{}%
\AgdaInductiveConstructor{⊥-intro}\<%
\\
\>[0]\AgdaFunction{rename-pres}\AgdaSpace{}%
\AgdaBound{ρ}\AgdaSpace{}%
\AgdaBound{lt}\AgdaSpace{}%
\AgdaSymbol{(}\AgdaInductiveConstructor{⊔-intro}\AgdaSpace{}%
\AgdaBound{d}\AgdaSpace{}%
\AgdaBound{d₁}\AgdaSymbol{)}\AgdaSpace{}%
\AgdaSymbol{=}\<%
\\
\>[0][@{}l@{\AgdaIndent{0}}]%
\>[3]\AgdaInductiveConstructor{⊔-intro}\AgdaSpace{}%
\AgdaSymbol{(}\AgdaFunction{rename-pres}\AgdaSpace{}%
\AgdaBound{ρ}\AgdaSpace{}%
\AgdaBound{lt}\AgdaSpace{}%
\AgdaBound{d}\AgdaSymbol{)}\AgdaSpace{}%
\AgdaSymbol{(}\AgdaFunction{rename-pres}\AgdaSpace{}%
\AgdaBound{ρ}\AgdaSpace{}%
\AgdaBound{lt}\AgdaSpace{}%
\AgdaBound{d₁}\AgdaSymbol{)}\<%
\\
\>[0]\AgdaFunction{rename-pres}\AgdaSpace{}%
\AgdaBound{ρ}\AgdaSpace{}%
\AgdaBound{lt}\AgdaSpace{}%
\AgdaSymbol{(}\AgdaInductiveConstructor{sub}\AgdaSpace{}%
\AgdaBound{d}\AgdaSpace{}%
\AgdaBound{lt′}\AgdaSymbol{)}\AgdaSpace{}%
\AgdaSymbol{=}\<%
\\
\>[0][@{}l@{\AgdaIndent{0}}]%
\>[3]\AgdaInductiveConstructor{sub}\AgdaSpace{}%
\AgdaSymbol{(}\AgdaFunction{rename-pres}\AgdaSpace{}%
\AgdaBound{ρ}\AgdaSpace{}%
\AgdaBound{lt}\AgdaSpace{}%
\AgdaBound{d}\AgdaSymbol{)}\AgdaSpace{}%
\AgdaBound{lt′}\<%
\end{code}
\end{fence}

The proof is by induction on the semantics of \texttt{M}. As you can
see, all of the cases are trivial except the cases for variables and
lambda.

\begin{itemize}
\item
  For a variable \texttt{x}, we make use of the premise to show that
  \texttt{γ\ x\ ≡\ δ\ (ρ\ x)}.
\item
  For a lambda abstraction, the induction hypothesis requires us to
  extend the renaming. We do so, and use the \texttt{ext-⊑} lemma to
  show that the extended renaming maps variables to ones with equivalent
  values.
\end{itemize}

\hypertarget{environment-strengthening-and-identity-renaming}{%
\section{Environment strengthening and identity
renaming}\label{environment-strengthening-and-identity-renaming}}

We shall need a corollary of the renaming lemma that says that replacing
the environment with a larger one (a stronger one) does not change
whether a term \texttt{M} results in particular value \texttt{v}. In
particular, if \texttt{γ\ ⊢\ M\ ↓\ v} and \texttt{γ\ ⊑\ δ}, then
\texttt{δ\ ⊢\ M\ ↓\ v}. What does this have to do with renaming? It's
renaming with the identity function. We apply the renaming lemma with
the identity renaming, which gives us
\texttt{δ\ ⊢\ rename\ (λ\ \{A\}\ x\ →\ x)\ M\ ↓\ v}, and then we apply
the \texttt{rename-id} lemma to obtain \texttt{δ\ ⊢\ M\ ↓\ v}.

\begin{fence}
\begin{code}%
\>[0]\AgdaFunction{⊑-env}\AgdaSpace{}%
\AgdaSymbol{:}\AgdaSpace{}%
\AgdaSymbol{∀}\AgdaSpace{}%
\AgdaSymbol{\{}\AgdaBound{Γ}\AgdaSymbol{\}}\AgdaSpace{}%
\AgdaSymbol{\{}\AgdaBound{γ}\AgdaSpace{}%
\AgdaSymbol{:}\AgdaSpace{}%
\AgdaFunction{Env}\AgdaSpace{}%
\AgdaBound{Γ}\AgdaSymbol{\}}\AgdaSpace{}%
\AgdaSymbol{\{}\AgdaBound{δ}\AgdaSpace{}%
\AgdaSymbol{:}\AgdaSpace{}%
\AgdaFunction{Env}\AgdaSpace{}%
\AgdaBound{Γ}\AgdaSymbol{\}}\AgdaSpace{}%
\AgdaSymbol{\{}\AgdaBound{M}\AgdaSpace{}%
\AgdaBound{v}\AgdaSymbol{\}}\<%
\\
\>[0][@{}l@{\AgdaIndent{0}}]%
\>[2]\AgdaSymbol{→}\AgdaSpace{}%
\AgdaBound{γ}\AgdaSpace{}%
\AgdaOperator{\AgdaDatatype{⊢}}\AgdaSpace{}%
\AgdaBound{M}\AgdaSpace{}%
\AgdaOperator{\AgdaDatatype{↓}}\AgdaSpace{}%
\AgdaBound{v}\<%
\\
%
\>[2]\AgdaSymbol{→}%
\>[1479I]\AgdaBound{γ}\AgdaSpace{}%
\AgdaOperator{\AgdaFunction{`⊑}}\AgdaSpace{}%
\AgdaBound{δ}\<%
\\
\>[.][@{}l@{}]\<[1479I]%
\>[4]\AgdaComment{----------}\<%
\\
%
\>[2]\AgdaSymbol{→}\AgdaSpace{}%
\AgdaBound{δ}\AgdaSpace{}%
\AgdaOperator{\AgdaDatatype{⊢}}\AgdaSpace{}%
\AgdaBound{M}\AgdaSpace{}%
\AgdaOperator{\AgdaDatatype{↓}}\AgdaSpace{}%
\AgdaBound{v}\<%
\\
\>[0]\AgdaFunction{⊑-env}\AgdaSymbol{\{}\AgdaBound{Γ}\AgdaSymbol{\}\{}\AgdaBound{γ}\AgdaSymbol{\}\{}\AgdaBound{δ}\AgdaSymbol{\}\{}\AgdaBound{M}\AgdaSymbol{\}\{}\AgdaBound{v}\AgdaSymbol{\}}\AgdaSpace{}%
\AgdaBound{d}\AgdaSpace{}%
\AgdaBound{lt}\<%
\\
\>[0][@{}l@{\AgdaIndent{0}}]%
\>[6]\AgdaKeyword{with}\AgdaSpace{}%
\AgdaFunction{rename-pres}\AgdaSymbol{\{}\AgdaBound{Γ}\AgdaSymbol{\}\{}\AgdaBound{Γ}\AgdaSymbol{\}\{}\AgdaBound{v}\AgdaSymbol{\}\{}\AgdaBound{γ}\AgdaSymbol{\}\{}\AgdaBound{δ}\AgdaSymbol{\}\{}\AgdaBound{M}\AgdaSymbol{\}}\AgdaSpace{}%
\AgdaSymbol{(λ}\AgdaSpace{}%
\AgdaSymbol{\{}\AgdaBound{A}\AgdaSymbol{\}}\AgdaSpace{}%
\AgdaBound{x}\AgdaSpace{}%
\AgdaSymbol{→}\AgdaSpace{}%
\AgdaBound{x}\AgdaSymbol{)}\AgdaSpace{}%
\AgdaBound{lt}\AgdaSpace{}%
\AgdaBound{d}\<%
\\
\>[0]\AgdaSymbol{...}\AgdaSpace{}%
\AgdaSymbol{|}%
\>[1498I]\AgdaBound{δ⊢id[M]↓v}\AgdaSpace{}%
\AgdaKeyword{rewrite}\AgdaSpace{}%
\AgdaFunction{rename-id}\AgdaSpace{}%
\AgdaSymbol{\{}\AgdaBound{Γ}\AgdaSymbol{\}\{}\AgdaInductiveConstructor{★}\AgdaSymbol{\}\{}\AgdaBound{M}\AgdaSymbol{\}}\AgdaSpace{}%
\AgdaSymbol{=}\<%
\\
\>[.][@{}l@{}]\<[1498I]%
\>[6]\AgdaBound{δ⊢id[M]↓v}\<%
\end{code}
\end{fence}

In the proof that substitution reflects denotations, in the case for
lambda abstraction, we use a minor variation of \texttt{⊑-env}, in which
just the last element of the environment gets larger.

\begin{fence}
\begin{code}%
\>[0]\AgdaFunction{up-env}\AgdaSpace{}%
\AgdaSymbol{:}\AgdaSpace{}%
\AgdaSymbol{∀}\AgdaSpace{}%
\AgdaSymbol{\{}\AgdaBound{Γ}\AgdaSymbol{\}}\AgdaSpace{}%
\AgdaSymbol{\{}\AgdaBound{γ}\AgdaSpace{}%
\AgdaSymbol{:}\AgdaSpace{}%
\AgdaFunction{Env}\AgdaSpace{}%
\AgdaBound{Γ}\AgdaSymbol{\}}\AgdaSpace{}%
\AgdaSymbol{\{}\AgdaBound{M}\AgdaSpace{}%
\AgdaBound{v}\AgdaSpace{}%
\AgdaBound{u₁}\AgdaSpace{}%
\AgdaBound{u₂}\AgdaSymbol{\}}\<%
\\
\>[0][@{}l@{\AgdaIndent{0}}]%
\>[2]\AgdaSymbol{→}\AgdaSpace{}%
\AgdaSymbol{(}\AgdaBound{γ}\AgdaSpace{}%
\AgdaOperator{\AgdaFunction{`,}}\AgdaSpace{}%
\AgdaBound{u₁}\AgdaSymbol{)}\AgdaSpace{}%
\AgdaOperator{\AgdaDatatype{⊢}}\AgdaSpace{}%
\AgdaBound{M}\AgdaSpace{}%
\AgdaOperator{\AgdaDatatype{↓}}\AgdaSpace{}%
\AgdaBound{v}\<%
\\
%
\>[2]\AgdaSymbol{→}%
\>[1521I]\AgdaBound{u₁}\AgdaSpace{}%
\AgdaOperator{\AgdaDatatype{⊑}}\AgdaSpace{}%
\AgdaBound{u₂}\<%
\\
\>[.][@{}l@{}]\<[1521I]%
\>[4]\AgdaComment{-----------------}\<%
\\
%
\>[2]\AgdaSymbol{→}\AgdaSpace{}%
\AgdaSymbol{(}\AgdaBound{γ}\AgdaSpace{}%
\AgdaOperator{\AgdaFunction{`,}}\AgdaSpace{}%
\AgdaBound{u₂}\AgdaSymbol{)}\AgdaSpace{}%
\AgdaOperator{\AgdaDatatype{⊢}}\AgdaSpace{}%
\AgdaBound{M}\AgdaSpace{}%
\AgdaOperator{\AgdaDatatype{↓}}\AgdaSpace{}%
\AgdaBound{v}\<%
\\
\>[0]\AgdaFunction{up-env}\AgdaSpace{}%
\AgdaBound{d}\AgdaSpace{}%
\AgdaBound{lt}\AgdaSpace{}%
\AgdaSymbol{=}\AgdaSpace{}%
\AgdaFunction{⊑-env}\AgdaSpace{}%
\AgdaBound{d}\AgdaSpace{}%
\AgdaSymbol{(}\AgdaFunction{ext-le}\AgdaSpace{}%
\AgdaBound{lt}\AgdaSymbol{)}\<%
\\
\>[0][@{}l@{\AgdaIndent{0}}]%
\>[2]\AgdaKeyword{where}\<%
\\
%
\>[2]\AgdaFunction{ext-le}\AgdaSpace{}%
\AgdaSymbol{:}\AgdaSpace{}%
\AgdaSymbol{∀}\AgdaSpace{}%
\AgdaSymbol{\{}\AgdaBound{γ}\AgdaSpace{}%
\AgdaBound{u₁}\AgdaSpace{}%
\AgdaBound{u₂}\AgdaSymbol{\}}\AgdaSpace{}%
\AgdaSymbol{→}\AgdaSpace{}%
\AgdaBound{u₁}\AgdaSpace{}%
\AgdaOperator{\AgdaDatatype{⊑}}\AgdaSpace{}%
\AgdaBound{u₂}\AgdaSpace{}%
\AgdaSymbol{→}\AgdaSpace{}%
\AgdaSymbol{(}\AgdaBound{γ}\AgdaSpace{}%
\AgdaOperator{\AgdaFunction{`,}}\AgdaSpace{}%
\AgdaBound{u₁}\AgdaSymbol{)}\AgdaSpace{}%
\AgdaOperator{\AgdaFunction{`⊑}}\AgdaSpace{}%
\AgdaSymbol{(}\AgdaBound{γ}\AgdaSpace{}%
\AgdaOperator{\AgdaFunction{`,}}\AgdaSpace{}%
\AgdaBound{u₂}\AgdaSymbol{)}\<%
\\
%
\>[2]\AgdaFunction{ext-le}\AgdaSpace{}%
\AgdaBound{lt}\AgdaSpace{}%
\AgdaInductiveConstructor{Z}\AgdaSpace{}%
\AgdaSymbol{=}\AgdaSpace{}%
\AgdaBound{lt}\<%
\\
%
\>[2]\AgdaFunction{ext-le}\AgdaSpace{}%
\AgdaBound{lt}\AgdaSpace{}%
\AgdaSymbol{(}\AgdaOperator{\AgdaInductiveConstructor{S}}\AgdaSpace{}%
\AgdaBound{n}\AgdaSymbol{)}\AgdaSpace{}%
\AgdaSymbol{=}\AgdaSpace{}%
\AgdaFunction{⊑-refl}\<%
\end{code}
\end{fence}

\hypertarget{exercise-denot-church-recommended}{%
\subsubsection{\texorpdfstring{Exercise \texttt{denot-church}
(recommended)}{Exercise denot-church (recommended)}}\label{exercise-denot-church-recommended}}

Church numerals are more general than natural numbers in that they
represent paths. A path consists of \texttt{n} edges and
\texttt{n\ +\ 1} vertices. We store the vertices in a vector of length
\texttt{n\ +\ 1} in reverse order. The edges in the path map the ith
vertex to the \texttt{i\ +\ 1} vertex. The following function
\texttt{D\^{}suc} (for denotation of successor) constructs a table whose
entries are all the edges in the path.

\begin{fence}
\begin{code}%
\>[0]\AgdaFunction{D\textasciicircum{}suc}\AgdaSpace{}%
\AgdaSymbol{:}\AgdaSpace{}%
\AgdaSymbol{(}\AgdaBound{n}\AgdaSpace{}%
\AgdaSymbol{:}\AgdaSpace{}%
\AgdaDatatype{ℕ}\AgdaSymbol{)}\AgdaSpace{}%
\AgdaSymbol{→}\AgdaSpace{}%
\AgdaDatatype{Vec}\AgdaSpace{}%
\AgdaDatatype{Value}\AgdaSpace{}%
\AgdaSymbol{(}\AgdaInductiveConstructor{suc}\AgdaSpace{}%
\AgdaBound{n}\AgdaSymbol{)}\AgdaSpace{}%
\AgdaSymbol{→}\AgdaSpace{}%
\AgdaDatatype{Value}\<%
\\
\>[0]\AgdaFunction{D\textasciicircum{}suc}\AgdaSpace{}%
\AgdaInductiveConstructor{zero}\AgdaSpace{}%
\AgdaSymbol{(}\AgdaBound{a[0]}\AgdaSpace{}%
\AgdaOperator{\AgdaInductiveConstructor{∷}}\AgdaSpace{}%
\AgdaInductiveConstructor{[]}\AgdaSymbol{)}\AgdaSpace{}%
\AgdaSymbol{=}\AgdaSpace{}%
\AgdaInductiveConstructor{⊥}\<%
\\
\>[0]\AgdaFunction{D\textasciicircum{}suc}\AgdaSpace{}%
\AgdaSymbol{(}\AgdaInductiveConstructor{suc}\AgdaSpace{}%
\AgdaBound{i}\AgdaSymbol{)}\AgdaSpace{}%
\AgdaSymbol{(}\AgdaBound{a[i+1]}\AgdaSpace{}%
\AgdaOperator{\AgdaInductiveConstructor{∷}}\AgdaSpace{}%
\AgdaBound{a[i]}\AgdaSpace{}%
\AgdaOperator{\AgdaInductiveConstructor{∷}}\AgdaSpace{}%
\AgdaBound{ls}\AgdaSymbol{)}\AgdaSpace{}%
\AgdaSymbol{=}%
\>[38]\AgdaBound{a[i]}\AgdaSpace{}%
\AgdaOperator{\AgdaInductiveConstructor{↦}}\AgdaSpace{}%
\AgdaBound{a[i+1]}%
\>[53]\AgdaOperator{\AgdaInductiveConstructor{⊔}}%
\>[56]\AgdaFunction{D\textasciicircum{}suc}\AgdaSpace{}%
\AgdaBound{i}\AgdaSpace{}%
\AgdaSymbol{(}\AgdaBound{a[i]}\AgdaSpace{}%
\AgdaOperator{\AgdaInductiveConstructor{∷}}\AgdaSpace{}%
\AgdaBound{ls}\AgdaSymbol{)}\<%
\end{code}
\end{fence}

We use the following auxiliary function to obtain the last element of a
non-empty vector. (This formulation is more convenient for our purposes
than the one in the Agda standard library.)

\begin{fence}
\begin{code}%
\>[0]\AgdaFunction{vec-last}\AgdaSpace{}%
\AgdaSymbol{:}\AgdaSpace{}%
\AgdaSymbol{∀\{}\AgdaBound{n}\AgdaSpace{}%
\AgdaSymbol{:}\AgdaSpace{}%
\AgdaDatatype{ℕ}\AgdaSymbol{\}}\AgdaSpace{}%
\AgdaSymbol{→}\AgdaSpace{}%
\AgdaDatatype{Vec}\AgdaSpace{}%
\AgdaDatatype{Value}\AgdaSpace{}%
\AgdaSymbol{(}\AgdaInductiveConstructor{suc}\AgdaSpace{}%
\AgdaBound{n}\AgdaSymbol{)}\AgdaSpace{}%
\AgdaSymbol{→}\AgdaSpace{}%
\AgdaDatatype{Value}\<%
\\
\>[0]\AgdaFunction{vec-last}\AgdaSpace{}%
\AgdaSymbol{\{}\AgdaNumber{0}\AgdaSymbol{\}}\AgdaSpace{}%
\AgdaSymbol{(}\AgdaBound{a}\AgdaSpace{}%
\AgdaOperator{\AgdaInductiveConstructor{∷}}\AgdaSpace{}%
\AgdaInductiveConstructor{[]}\AgdaSymbol{)}\AgdaSpace{}%
\AgdaSymbol{=}\AgdaSpace{}%
\AgdaBound{a}\<%
\\
\>[0]\AgdaFunction{vec-last}\AgdaSpace{}%
\AgdaSymbol{\{}\AgdaInductiveConstructor{suc}\AgdaSpace{}%
\AgdaBound{n}\AgdaSymbol{\}}\AgdaSpace{}%
\AgdaSymbol{(}\AgdaBound{a}\AgdaSpace{}%
\AgdaOperator{\AgdaInductiveConstructor{∷}}\AgdaSpace{}%
\AgdaBound{b}\AgdaSpace{}%
\AgdaOperator{\AgdaInductiveConstructor{∷}}\AgdaSpace{}%
\AgdaBound{ls}\AgdaSymbol{)}\AgdaSpace{}%
\AgdaSymbol{=}\AgdaSpace{}%
\AgdaFunction{vec-last}\AgdaSpace{}%
\AgdaSymbol{(}\AgdaBound{b}\AgdaSpace{}%
\AgdaOperator{\AgdaInductiveConstructor{∷}}\AgdaSpace{}%
\AgdaBound{ls}\AgdaSymbol{)}\<%
\end{code}
\end{fence}

The function \texttt{Dᶜ} computes the denotation of the nth Church
numeral for a given path.

\begin{fence}
\begin{code}%
\>[0]\AgdaFunction{Dᶜ}\AgdaSpace{}%
\AgdaSymbol{:}\AgdaSpace{}%
\AgdaSymbol{(}\AgdaBound{n}\AgdaSpace{}%
\AgdaSymbol{:}\AgdaSpace{}%
\AgdaDatatype{ℕ}\AgdaSymbol{)}\AgdaSpace{}%
\AgdaSymbol{→}\AgdaSpace{}%
\AgdaDatatype{Vec}\AgdaSpace{}%
\AgdaDatatype{Value}\AgdaSpace{}%
\AgdaSymbol{(}\AgdaInductiveConstructor{suc}\AgdaSpace{}%
\AgdaBound{n}\AgdaSymbol{)}\AgdaSpace{}%
\AgdaSymbol{→}\AgdaSpace{}%
\AgdaDatatype{Value}\<%
\\
\>[0]\AgdaFunction{Dᶜ}\AgdaSpace{}%
\AgdaBound{n}\AgdaSpace{}%
\AgdaSymbol{(}\AgdaBound{a[n]}\AgdaSpace{}%
\AgdaOperator{\AgdaInductiveConstructor{∷}}\AgdaSpace{}%
\AgdaBound{ls}\AgdaSymbol{)}\AgdaSpace{}%
\AgdaSymbol{=}\AgdaSpace{}%
\AgdaSymbol{(}\AgdaFunction{D\textasciicircum{}suc}\AgdaSpace{}%
\AgdaBound{n}\AgdaSpace{}%
\AgdaSymbol{(}\AgdaBound{a[n]}\AgdaSpace{}%
\AgdaOperator{\AgdaInductiveConstructor{∷}}\AgdaSpace{}%
\AgdaBound{ls}\AgdaSymbol{))}\AgdaSpace{}%
\AgdaOperator{\AgdaInductiveConstructor{↦}}\AgdaSpace{}%
\AgdaSymbol{(}\AgdaFunction{vec-last}\AgdaSpace{}%
\AgdaSymbol{(}\AgdaBound{a[n]}\AgdaSpace{}%
\AgdaOperator{\AgdaInductiveConstructor{∷}}\AgdaSpace{}%
\AgdaBound{ls}\AgdaSymbol{))}\AgdaSpace{}%
\AgdaOperator{\AgdaInductiveConstructor{↦}}\AgdaSpace{}%
\AgdaBound{a[n]}\<%
\end{code}
\end{fence}

\begin{itemize}
\item
  The Church numeral for 0 ignores its first argument and returns its
  second argument, so for the singleton path consisting of just
  \texttt{a{[}0{]}}, its denotation is

  \begin{myDisplay}
    ⊥ ↦ a[0] ↦ a[0]
  \end{myDisplay}
\item
  The Church numeral for \texttt{suc\ n} takes two arguments: a
  successor function whose denotation is given by \texttt{D\^{}suc}, and
  the start of the path (last of the vector). It returns the
  \texttt{n\ +\ 1} vertex in the path.

  \begin{myDisplay}
    (D^suc (suc n) (a[n+1] ∷ a[n] ∷ ls)) ↦ (vec-last (a[n] ∷ ls)) ↦ a[n+1]
  \end{myDisplay}
\end{itemize}

The exercise is to prove that for any path \texttt{ls}, the meaning of
the Church numeral \texttt{n} is \texttt{Dᶜ\ n\ ls}.

To facilitate talking about arbitrary Church numerals, the following
\texttt{church} function builds the term for the nth Church numeral,
using the auxiliary function \texttt{apply-n}.

\begin{fence}
\begin{code}%
\>[0]\AgdaFunction{apply-n}\AgdaSpace{}%
\AgdaSymbol{:}\AgdaSpace{}%
\AgdaSymbol{(}\AgdaBound{n}\AgdaSpace{}%
\AgdaSymbol{:}\AgdaSpace{}%
\AgdaDatatype{ℕ}\AgdaSymbol{)}\AgdaSpace{}%
\AgdaSymbol{→}\AgdaSpace{}%
\AgdaInductiveConstructor{∅}\AgdaSpace{}%
\AgdaOperator{\AgdaInductiveConstructor{,}}\AgdaSpace{}%
\AgdaInductiveConstructor{★}\AgdaSpace{}%
\AgdaOperator{\AgdaInductiveConstructor{,}}\AgdaSpace{}%
\AgdaInductiveConstructor{★}\AgdaSpace{}%
\AgdaOperator{\AgdaDatatype{⊢}}\AgdaSpace{}%
\AgdaInductiveConstructor{★}\<%
\\
\>[0]\AgdaFunction{apply-n}\AgdaSpace{}%
\AgdaInductiveConstructor{zero}\AgdaSpace{}%
\AgdaSymbol{=}\AgdaSpace{}%
\AgdaOperator{\AgdaFunction{\#}}\AgdaSpace{}%
\AgdaNumber{0}\<%
\\
\>[0]\AgdaFunction{apply-n}\AgdaSpace{}%
\AgdaSymbol{(}\AgdaInductiveConstructor{suc}\AgdaSpace{}%
\AgdaBound{n}\AgdaSymbol{)}\AgdaSpace{}%
\AgdaSymbol{=}\AgdaSpace{}%
\AgdaOperator{\AgdaFunction{\#}}\AgdaSpace{}%
\AgdaNumber{1}\AgdaSpace{}%
\AgdaOperator{\AgdaInductiveConstructor{·}}\AgdaSpace{}%
\AgdaFunction{apply-n}\AgdaSpace{}%
\AgdaBound{n}\<%
\\
%
\\[\AgdaEmptyExtraSkip]%
\>[0]\AgdaFunction{church}\AgdaSpace{}%
\AgdaSymbol{:}\AgdaSpace{}%
\AgdaSymbol{(}\AgdaBound{n}\AgdaSpace{}%
\AgdaSymbol{:}\AgdaSpace{}%
\AgdaDatatype{ℕ}\AgdaSymbol{)}\AgdaSpace{}%
\AgdaSymbol{→}\AgdaSpace{}%
\AgdaInductiveConstructor{∅}\AgdaSpace{}%
\AgdaOperator{\AgdaDatatype{⊢}}\AgdaSpace{}%
\AgdaInductiveConstructor{★}\<%
\\
\>[0]\AgdaFunction{church}\AgdaSpace{}%
\AgdaBound{n}\AgdaSpace{}%
\AgdaSymbol{=}\AgdaSpace{}%
\AgdaOperator{\AgdaInductiveConstructor{ƛ}}\AgdaSpace{}%
\AgdaOperator{\AgdaInductiveConstructor{ƛ}}\AgdaSpace{}%
\AgdaFunction{apply-n}\AgdaSpace{}%
\AgdaBound{n}\<%
\end{code}
\end{fence}

Prove the following theorem.

\begin{myDisplay}
denot-church : ∀{n : ℕ}{ls : Vec Value (suc n)}
   → `∅ ⊢ church n ↓ Dᶜ n ls
\end{myDisplay}

\begin{fence}
\begin{code}%
\>[0]\AgdaComment{-- Your code goes here}\<%
\end{code}
\end{fence}

\hypertarget{inversion-of-the-less-than-relation-for-functions}{%
\section{Inversion of the less-than relation for
functions}\label{inversion-of-the-less-than-relation-for-functions}}

What can we deduce from knowing that a function \texttt{v\ ↦\ w} is less
than some value \texttt{u}? What can we deduce about \texttt{u}? The
answer to this question is called the inversion property of less-than
for functions. This question is not easy to answer because of the
\texttt{⊑-dist} rule, which relates a function on the left to a pair of
functions on the right. So \texttt{u} may include several functions
that, as a group, relate to \texttt{v\ ↦\ w}. Furthermore, because of
the rules \texttt{⊑-conj-R1} and \texttt{⊑-conj-R2}, there may be other
values inside \texttt{u}, such as \texttt{⊥}, that have nothing to do
with \texttt{v\ ↦\ w}. But in general, we can deduce that \texttt{u}
includes a collection of functions where the join of their domains is
less than \texttt{v} and the join of their codomains is greater than
\texttt{w}.

To precisely state and prove this inversion property, we need to define
what it means for a value to \emph{include} a collection of values. We
also need to define how to compute the join of their domains and
codomains.

\hypertarget{value-membership-and-inclusion}{%
\subsection{Value membership and
inclusion}\label{value-membership-and-inclusion}}

Recall that we think of a value as a set of entries with the join
operator \texttt{v\ ⊔\ w} acting like set union. The function value
\texttt{v\ ↦\ w} and bottom value \texttt{⊥} constitute the two kinds of
elements of the set. (In other contexts one can instead think of
\texttt{⊥} as the empty set, but here we must think of it as an
element.) We write \texttt{u\ ∈\ v} to say that \texttt{u} is an element
of \texttt{v}, as defined below.

\begin{fence}
\begin{code}%
\>[0]\AgdaKeyword{infix}\AgdaSpace{}%
\AgdaNumber{5}\AgdaSpace{}%
\AgdaOperator{\AgdaFunction{\AgdaUnderscore{}∈\AgdaUnderscore{}}}\<%
\\
%
\\[\AgdaEmptyExtraSkip]%
\>[0]\AgdaOperator{\AgdaFunction{\AgdaUnderscore{}∈\AgdaUnderscore{}}}\AgdaSpace{}%
\AgdaSymbol{:}\AgdaSpace{}%
\AgdaDatatype{Value}\AgdaSpace{}%
\AgdaSymbol{→}\AgdaSpace{}%
\AgdaDatatype{Value}\AgdaSpace{}%
\AgdaSymbol{→}\AgdaSpace{}%
\AgdaPrimitiveType{Set}\<%
\\
\>[0]\AgdaBound{u}\AgdaSpace{}%
\AgdaOperator{\AgdaFunction{∈}}\AgdaSpace{}%
\AgdaInductiveConstructor{⊥}\AgdaSpace{}%
\AgdaSymbol{=}\AgdaSpace{}%
\AgdaBound{u}\AgdaSpace{}%
\AgdaOperator{\AgdaDatatype{≡}}\AgdaSpace{}%
\AgdaInductiveConstructor{⊥}\<%
\\
\>[0]\AgdaBound{u}\AgdaSpace{}%
\AgdaOperator{\AgdaFunction{∈}}\AgdaSpace{}%
\AgdaBound{v}\AgdaSpace{}%
\AgdaOperator{\AgdaInductiveConstructor{↦}}\AgdaSpace{}%
\AgdaBound{w}\AgdaSpace{}%
\AgdaSymbol{=}\AgdaSpace{}%
\AgdaBound{u}\AgdaSpace{}%
\AgdaOperator{\AgdaDatatype{≡}}\AgdaSpace{}%
\AgdaBound{v}\AgdaSpace{}%
\AgdaOperator{\AgdaInductiveConstructor{↦}}\AgdaSpace{}%
\AgdaBound{w}\<%
\\
\>[0]\AgdaBound{u}\AgdaSpace{}%
\AgdaOperator{\AgdaFunction{∈}}\AgdaSpace{}%
\AgdaSymbol{(}\AgdaBound{v}\AgdaSpace{}%
\AgdaOperator{\AgdaInductiveConstructor{⊔}}\AgdaSpace{}%
\AgdaBound{w}\AgdaSymbol{)}\AgdaSpace{}%
\AgdaSymbol{=}\AgdaSpace{}%
\AgdaBound{u}\AgdaSpace{}%
\AgdaOperator{\AgdaFunction{∈}}\AgdaSpace{}%
\AgdaBound{v}\AgdaSpace{}%
\AgdaOperator{\AgdaDatatype{⊎}}\AgdaSpace{}%
\AgdaBound{u}\AgdaSpace{}%
\AgdaOperator{\AgdaFunction{∈}}\AgdaSpace{}%
\AgdaBound{w}\<%
\end{code}
\end{fence}

So we can represent a collection of values simply as a value. We write
\texttt{v\ ⊆\ w} to say that all the elements of \texttt{v} are also in
\texttt{w}.

\begin{fence}
\begin{code}%
\>[0]\AgdaKeyword{infix}\AgdaSpace{}%
\AgdaNumber{5}\AgdaSpace{}%
\AgdaOperator{\AgdaFunction{\AgdaUnderscore{}⊆\AgdaUnderscore{}}}\<%
\\
%
\\[\AgdaEmptyExtraSkip]%
\>[0]\AgdaOperator{\AgdaFunction{\AgdaUnderscore{}⊆\AgdaUnderscore{}}}\AgdaSpace{}%
\AgdaSymbol{:}\AgdaSpace{}%
\AgdaDatatype{Value}\AgdaSpace{}%
\AgdaSymbol{→}\AgdaSpace{}%
\AgdaDatatype{Value}\AgdaSpace{}%
\AgdaSymbol{→}\AgdaSpace{}%
\AgdaPrimitiveType{Set}\<%
\\
\>[0]\AgdaBound{v}\AgdaSpace{}%
\AgdaOperator{\AgdaFunction{⊆}}\AgdaSpace{}%
\AgdaBound{w}\AgdaSpace{}%
\AgdaSymbol{=}\AgdaSpace{}%
\AgdaSymbol{∀\{}\AgdaBound{u}\AgdaSymbol{\}}\AgdaSpace{}%
\AgdaSymbol{→}\AgdaSpace{}%
\AgdaBound{u}\AgdaSpace{}%
\AgdaOperator{\AgdaFunction{∈}}\AgdaSpace{}%
\AgdaBound{v}\AgdaSpace{}%
\AgdaSymbol{→}\AgdaSpace{}%
\AgdaBound{u}\AgdaSpace{}%
\AgdaOperator{\AgdaFunction{∈}}\AgdaSpace{}%
\AgdaBound{w}\<%
\end{code}
\end{fence}

The notions of membership and inclusion for values are closely related
to the less-than relation. They are narrower relations in that they
imply the less-than relation but not the other way around.

\begin{fence}
\begin{code}%
\>[0]\AgdaFunction{∈→⊑}%
\>[1746I]\AgdaSymbol{:}\AgdaSpace{}%
\AgdaSymbol{∀\{}\AgdaBound{u}\AgdaSpace{}%
\AgdaBound{v}\AgdaSpace{}%
\AgdaSymbol{:}\AgdaSpace{}%
\AgdaDatatype{Value}\AgdaSymbol{\}}\<%
\\
\>[.][@{}l@{}]\<[1746I]%
\>[4]\AgdaSymbol{→}%
\>[1751I]\AgdaBound{u}\AgdaSpace{}%
\AgdaOperator{\AgdaFunction{∈}}\AgdaSpace{}%
\AgdaBound{v}\<%
\\
\>[.][@{}l@{}]\<[1751I]%
\>[6]\AgdaComment{-----}\<%
\\
%
\>[4]\AgdaSymbol{→}\AgdaSpace{}%
\AgdaBound{u}\AgdaSpace{}%
\AgdaOperator{\AgdaDatatype{⊑}}\AgdaSpace{}%
\AgdaBound{v}\<%
\\
\>[0]\AgdaFunction{∈→⊑}\AgdaSpace{}%
\AgdaSymbol{\{}\AgdaDottedPattern{\AgdaSymbol{.}}\AgdaDottedPattern{\AgdaInductiveConstructor{⊥}}\AgdaSymbol{\}}\AgdaSpace{}%
\AgdaSymbol{\{}\AgdaInductiveConstructor{⊥}\AgdaSymbol{\}}\AgdaSpace{}%
\AgdaInductiveConstructor{refl}\AgdaSpace{}%
\AgdaSymbol{=}\AgdaSpace{}%
\AgdaInductiveConstructor{⊑-bot}\<%
\\
\>[0]\AgdaFunction{∈→⊑}\AgdaSpace{}%
\AgdaSymbol{\{}\AgdaBound{v}\AgdaSpace{}%
\AgdaOperator{\AgdaInductiveConstructor{↦}}\AgdaSpace{}%
\AgdaBound{w}\AgdaSymbol{\}}\AgdaSpace{}%
\AgdaSymbol{\{}\AgdaBound{v}\AgdaSpace{}%
\AgdaOperator{\AgdaInductiveConstructor{↦}}\AgdaSpace{}%
\AgdaBound{w}\AgdaSymbol{\}}\AgdaSpace{}%
\AgdaInductiveConstructor{refl}\AgdaSpace{}%
\AgdaSymbol{=}\AgdaSpace{}%
\AgdaFunction{⊑-refl}\<%
\\
\>[0]\AgdaFunction{∈→⊑}\AgdaSpace{}%
\AgdaSymbol{\{}\AgdaBound{u}\AgdaSymbol{\}}\AgdaSpace{}%
\AgdaSymbol{\{}\AgdaBound{v}\AgdaSpace{}%
\AgdaOperator{\AgdaInductiveConstructor{⊔}}\AgdaSpace{}%
\AgdaBound{w}\AgdaSymbol{\}}\AgdaSpace{}%
\AgdaSymbol{(}\AgdaInductiveConstructor{inj₁}\AgdaSpace{}%
\AgdaBound{x}\AgdaSymbol{)}\AgdaSpace{}%
\AgdaSymbol{=}\AgdaSpace{}%
\AgdaInductiveConstructor{⊑-conj-R1}\AgdaSpace{}%
\AgdaSymbol{(}\AgdaFunction{∈→⊑}\AgdaSpace{}%
\AgdaBound{x}\AgdaSymbol{)}\<%
\\
\>[0]\AgdaFunction{∈→⊑}\AgdaSpace{}%
\AgdaSymbol{\{}\AgdaBound{u}\AgdaSymbol{\}}\AgdaSpace{}%
\AgdaSymbol{\{}\AgdaBound{v}\AgdaSpace{}%
\AgdaOperator{\AgdaInductiveConstructor{⊔}}\AgdaSpace{}%
\AgdaBound{w}\AgdaSymbol{\}}\AgdaSpace{}%
\AgdaSymbol{(}\AgdaInductiveConstructor{inj₂}\AgdaSpace{}%
\AgdaBound{y}\AgdaSymbol{)}\AgdaSpace{}%
\AgdaSymbol{=}\AgdaSpace{}%
\AgdaInductiveConstructor{⊑-conj-R2}\AgdaSpace{}%
\AgdaSymbol{(}\AgdaFunction{∈→⊑}\AgdaSpace{}%
\AgdaBound{y}\AgdaSymbol{)}\<%
\\
%
\\[\AgdaEmptyExtraSkip]%
\>[0]\AgdaFunction{⊆→⊑}%
\>[1791I]\AgdaSymbol{:}\AgdaSpace{}%
\AgdaSymbol{∀\{}\AgdaBound{u}\AgdaSpace{}%
\AgdaBound{v}\AgdaSpace{}%
\AgdaSymbol{:}\AgdaSpace{}%
\AgdaDatatype{Value}\AgdaSymbol{\}}\<%
\\
\>[.][@{}l@{}]\<[1791I]%
\>[4]\AgdaSymbol{→}%
\>[1796I]\AgdaBound{u}\AgdaSpace{}%
\AgdaOperator{\AgdaFunction{⊆}}\AgdaSpace{}%
\AgdaBound{v}\<%
\\
\>[.][@{}l@{}]\<[1796I]%
\>[6]\AgdaComment{-----}\<%
\\
%
\>[4]\AgdaSymbol{→}\AgdaSpace{}%
\AgdaBound{u}\AgdaSpace{}%
\AgdaOperator{\AgdaDatatype{⊑}}\AgdaSpace{}%
\AgdaBound{v}\<%
\\
\>[0]\AgdaFunction{⊆→⊑}\AgdaSpace{}%
\AgdaSymbol{\{}\AgdaInductiveConstructor{⊥}\AgdaSymbol{\}}\AgdaSpace{}%
\AgdaBound{s}\AgdaSpace{}%
\AgdaKeyword{with}\AgdaSpace{}%
\AgdaBound{s}\AgdaSpace{}%
\AgdaSymbol{\{}\AgdaInductiveConstructor{⊥}\AgdaSymbol{\}}\AgdaSpace{}%
\AgdaInductiveConstructor{refl}\<%
\\
\>[0]\AgdaSymbol{...}\AgdaSpace{}%
\AgdaSymbol{|}\AgdaSpace{}%
\AgdaBound{x}\AgdaSpace{}%
\AgdaSymbol{=}\AgdaSpace{}%
\AgdaInductiveConstructor{⊑-bot}\<%
\\
\>[0]\AgdaFunction{⊆→⊑}\AgdaSpace{}%
\AgdaSymbol{\{}\AgdaBound{u}\AgdaSpace{}%
\AgdaOperator{\AgdaInductiveConstructor{↦}}\AgdaSpace{}%
\AgdaBound{u′}\AgdaSymbol{\}}\AgdaSpace{}%
\AgdaBound{s}\AgdaSpace{}%
\AgdaKeyword{with}\AgdaSpace{}%
\AgdaBound{s}\AgdaSpace{}%
\AgdaSymbol{\{}\AgdaBound{u}\AgdaSpace{}%
\AgdaOperator{\AgdaInductiveConstructor{↦}}\AgdaSpace{}%
\AgdaBound{u′}\AgdaSymbol{\}}\AgdaSpace{}%
\AgdaInductiveConstructor{refl}\<%
\\
\>[0]\AgdaSymbol{...}\AgdaSpace{}%
\AgdaSymbol{|}\AgdaSpace{}%
\AgdaBound{x}\AgdaSpace{}%
\AgdaSymbol{=}\AgdaSpace{}%
\AgdaFunction{∈→⊑}\AgdaSpace{}%
\AgdaBound{x}\<%
\\
\>[0]\AgdaFunction{⊆→⊑}\AgdaSpace{}%
\AgdaSymbol{\{}\AgdaBound{u}\AgdaSpace{}%
\AgdaOperator{\AgdaInductiveConstructor{⊔}}\AgdaSpace{}%
\AgdaBound{u′}\AgdaSymbol{\}}\AgdaSpace{}%
\AgdaBound{s}\AgdaSpace{}%
\AgdaSymbol{=}\AgdaSpace{}%
\AgdaInductiveConstructor{⊑-conj-L}\AgdaSpace{}%
\AgdaSymbol{(}\AgdaFunction{⊆→⊑}\AgdaSpace{}%
\AgdaSymbol{(λ}\AgdaSpace{}%
\AgdaBound{z}\AgdaSpace{}%
\AgdaSymbol{→}\AgdaSpace{}%
\AgdaBound{s}\AgdaSpace{}%
\AgdaSymbol{(}\AgdaInductiveConstructor{inj₁}\AgdaSpace{}%
\AgdaBound{z}\AgdaSymbol{)))}\AgdaSpace{}%
\AgdaSymbol{(}\AgdaFunction{⊆→⊑}\AgdaSpace{}%
\AgdaSymbol{(λ}\AgdaSpace{}%
\AgdaBound{z}\AgdaSpace{}%
\AgdaSymbol{→}\AgdaSpace{}%
\AgdaBound{s}\AgdaSpace{}%
\AgdaSymbol{(}\AgdaInductiveConstructor{inj₂}\AgdaSpace{}%
\AgdaBound{z}\AgdaSymbol{)))}\<%
\end{code}
\end{fence}

We shall also need some inversion principles for value inclusion. If the
union of \texttt{u} and \texttt{v} is included in \texttt{w}, then of
course both \texttt{u} and \texttt{v} are each included in \texttt{w}.

\begin{fence}
\begin{code}%
\>[0]\AgdaFunction{⊔⊆-inv}%
\>[1847I]\AgdaSymbol{:}\AgdaSpace{}%
\AgdaSymbol{∀\{}\AgdaBound{u}\AgdaSpace{}%
\AgdaBound{v}\AgdaSpace{}%
\AgdaBound{w}\AgdaSpace{}%
\AgdaSymbol{:}\AgdaSpace{}%
\AgdaDatatype{Value}\AgdaSymbol{\}}\<%
\\
\>[.][@{}l@{}]\<[1847I]%
\>[7]\AgdaSymbol{→}%
\>[1853I]\AgdaSymbol{(}\AgdaBound{u}\AgdaSpace{}%
\AgdaOperator{\AgdaInductiveConstructor{⊔}}\AgdaSpace{}%
\AgdaBound{v}\AgdaSymbol{)}\AgdaSpace{}%
\AgdaOperator{\AgdaFunction{⊆}}\AgdaSpace{}%
\AgdaBound{w}\<%
\\
\>[.][@{}l@{}]\<[1853I]%
\>[9]\AgdaComment{---------------}\<%
\\
%
\>[7]\AgdaSymbol{→}\AgdaSpace{}%
\AgdaBound{u}\AgdaSpace{}%
\AgdaOperator{\AgdaFunction{⊆}}\AgdaSpace{}%
\AgdaBound{w}%
\>[16]\AgdaOperator{\AgdaFunction{×}}%
\>[19]\AgdaBound{v}\AgdaSpace{}%
\AgdaOperator{\AgdaFunction{⊆}}\AgdaSpace{}%
\AgdaBound{w}\<%
\\
\>[0]\AgdaFunction{⊔⊆-inv}\AgdaSpace{}%
\AgdaBound{uvw}\AgdaSpace{}%
\AgdaSymbol{=}\AgdaSpace{}%
\AgdaOperator{\AgdaInductiveConstructor{⟨}}\AgdaSpace{}%
\AgdaSymbol{(λ}\AgdaSpace{}%
\AgdaBound{x}\AgdaSpace{}%
\AgdaSymbol{→}\AgdaSpace{}%
\AgdaBound{uvw}\AgdaSpace{}%
\AgdaSymbol{(}\AgdaInductiveConstructor{inj₁}\AgdaSpace{}%
\AgdaBound{x}\AgdaSymbol{))}\AgdaSpace{}%
\AgdaOperator{\AgdaInductiveConstructor{,}}\AgdaSpace{}%
\AgdaSymbol{(λ}\AgdaSpace{}%
\AgdaBound{x}\AgdaSpace{}%
\AgdaSymbol{→}\AgdaSpace{}%
\AgdaBound{uvw}\AgdaSpace{}%
\AgdaSymbol{(}\AgdaInductiveConstructor{inj₂}\AgdaSpace{}%
\AgdaBound{x}\AgdaSymbol{))}\AgdaSpace{}%
\AgdaOperator{\AgdaInductiveConstructor{⟩}}\<%
\end{code}
\end{fence}

In our value representation, the function value \texttt{v\ ↦\ w} is both
an element and also a singleton set. So if \texttt{v\ ↦\ w} is a subset
of \texttt{u}, then \texttt{v\ ↦\ w} must be a member of \texttt{u}.

\begin{fence}
\begin{code}%
\>[0]\AgdaFunction{↦⊆→∈}%
\>[1880I]\AgdaSymbol{:}\AgdaSpace{}%
\AgdaSymbol{∀\{}\AgdaBound{v}\AgdaSpace{}%
\AgdaBound{w}\AgdaSpace{}%
\AgdaBound{u}\AgdaSpace{}%
\AgdaSymbol{:}\AgdaSpace{}%
\AgdaDatatype{Value}\AgdaSymbol{\}}\<%
\\
\>[.][@{}l@{}]\<[1880I]%
\>[5]\AgdaSymbol{→}%
\>[1886I]\AgdaBound{v}\AgdaSpace{}%
\AgdaOperator{\AgdaInductiveConstructor{↦}}\AgdaSpace{}%
\AgdaBound{w}\AgdaSpace{}%
\AgdaOperator{\AgdaFunction{⊆}}\AgdaSpace{}%
\AgdaBound{u}\<%
\\
\>[.][@{}l@{}]\<[1886I]%
\>[7]\AgdaComment{---------}\<%
\\
%
\>[5]\AgdaSymbol{→}\AgdaSpace{}%
\AgdaBound{v}\AgdaSpace{}%
\AgdaOperator{\AgdaInductiveConstructor{↦}}\AgdaSpace{}%
\AgdaBound{w}\AgdaSpace{}%
\AgdaOperator{\AgdaFunction{∈}}\AgdaSpace{}%
\AgdaBound{u}\<%
\\
\>[0]\AgdaFunction{↦⊆→∈}\AgdaSpace{}%
\AgdaBound{incl}\AgdaSpace{}%
\AgdaSymbol{=}\AgdaSpace{}%
\AgdaBound{incl}\AgdaSpace{}%
\AgdaInductiveConstructor{refl}\<%
\end{code}
\end{fence}

\hypertarget{function-values}{%
\subsection{Function values}\label{function-values}}

To identify collections of functions, we define the following two
predicates. We write \texttt{Fun\ u} if \texttt{u} is a function value,
that is, if \texttt{u\ ≡\ v\ ↦\ w} for some values \texttt{v} and
\texttt{w}. We write \texttt{all-funs\ v} if all the elements of
\texttt{v} are functions.

\begin{fence}
\begin{code}%
\>[0]\AgdaKeyword{data}\AgdaSpace{}%
\AgdaDatatype{Fun}\AgdaSpace{}%
\AgdaSymbol{:}\AgdaSpace{}%
\AgdaDatatype{Value}\AgdaSpace{}%
\AgdaSymbol{→}\AgdaSpace{}%
\AgdaPrimitiveType{Set}\AgdaSpace{}%
\AgdaKeyword{where}\<%
\\
\>[0][@{}l@{\AgdaIndent{0}}]%
\>[2]\AgdaInductiveConstructor{fun}\AgdaSpace{}%
\AgdaSymbol{:}\AgdaSpace{}%
\AgdaSymbol{∀\{}\AgdaBound{u}\AgdaSpace{}%
\AgdaBound{v}\AgdaSpace{}%
\AgdaBound{w}\AgdaSymbol{\}}\AgdaSpace{}%
\AgdaSymbol{→}\AgdaSpace{}%
\AgdaBound{u}\AgdaSpace{}%
\AgdaOperator{\AgdaDatatype{≡}}\AgdaSpace{}%
\AgdaSymbol{(}\AgdaBound{v}\AgdaSpace{}%
\AgdaOperator{\AgdaInductiveConstructor{↦}}\AgdaSpace{}%
\AgdaBound{w}\AgdaSymbol{)}\AgdaSpace{}%
\AgdaSymbol{→}\AgdaSpace{}%
\AgdaDatatype{Fun}\AgdaSpace{}%
\AgdaBound{u}\<%
\\
%
\\[\AgdaEmptyExtraSkip]%
\>[0]\AgdaFunction{all-funs}\AgdaSpace{}%
\AgdaSymbol{:}\AgdaSpace{}%
\AgdaDatatype{Value}\AgdaSpace{}%
\AgdaSymbol{→}\AgdaSpace{}%
\AgdaPrimitiveType{Set}\<%
\\
\>[0]\AgdaFunction{all-funs}\AgdaSpace{}%
\AgdaBound{v}\AgdaSpace{}%
\AgdaSymbol{=}\AgdaSpace{}%
\AgdaSymbol{∀\{}\AgdaBound{u}\AgdaSymbol{\}}\AgdaSpace{}%
\AgdaSymbol{→}\AgdaSpace{}%
\AgdaBound{u}\AgdaSpace{}%
\AgdaOperator{\AgdaFunction{∈}}\AgdaSpace{}%
\AgdaBound{v}\AgdaSpace{}%
\AgdaSymbol{→}\AgdaSpace{}%
\AgdaDatatype{Fun}\AgdaSpace{}%
\AgdaBound{u}\<%
\end{code}
\end{fence}

The value \texttt{⊥} is not a function.

\begin{fence}
\begin{code}%
\>[0]\AgdaFunction{¬Fun⊥}\AgdaSpace{}%
\AgdaSymbol{:}\AgdaSpace{}%
\AgdaOperator{\AgdaFunction{¬}}\AgdaSpace{}%
\AgdaSymbol{(}\AgdaDatatype{Fun}\AgdaSpace{}%
\AgdaInductiveConstructor{⊥}\AgdaSymbol{)}\<%
\\
\>[0]\AgdaFunction{¬Fun⊥}\AgdaSpace{}%
\AgdaSymbol{(}\AgdaInductiveConstructor{fun}\AgdaSpace{}%
\AgdaSymbol{())}\<%
\end{code}
\end{fence}

In our values-as-sets representation, our sets always include at least
one element. Thus, if all the elements are functions, there is at least
one that is a function.

\begin{fence}
\begin{code}%
\>[0]\AgdaFunction{all-funs∈}\AgdaSpace{}%
\AgdaSymbol{:}\AgdaSpace{}%
\AgdaSymbol{∀\{}\AgdaBound{u}\AgdaSymbol{\}}\<%
\\
\>[0][@{}l@{\AgdaIndent{0}}]%
\>[6]\AgdaSymbol{→}\AgdaSpace{}%
\AgdaFunction{all-funs}\AgdaSpace{}%
\AgdaBound{u}\<%
\\
%
\>[6]\AgdaSymbol{→}\AgdaSpace{}%
\AgdaFunction{Σ[}\AgdaSpace{}%
\AgdaBound{v}\AgdaSpace{}%
\AgdaFunction{∈}\AgdaSpace{}%
\AgdaDatatype{Value}\AgdaSpace{}%
\AgdaFunction{]}\AgdaSpace{}%
\AgdaFunction{Σ[}\AgdaSpace{}%
\AgdaBound{w}\AgdaSpace{}%
\AgdaFunction{∈}\AgdaSpace{}%
\AgdaDatatype{Value}\AgdaSpace{}%
\AgdaFunction{]}\AgdaSpace{}%
\AgdaBound{v}\AgdaSpace{}%
\AgdaOperator{\AgdaInductiveConstructor{↦}}\AgdaSpace{}%
\AgdaBound{w}\AgdaSpace{}%
\AgdaOperator{\AgdaFunction{∈}}\AgdaSpace{}%
\AgdaBound{u}\<%
\\
\>[0]\AgdaFunction{all-funs∈}\AgdaSpace{}%
\AgdaSymbol{\{}\AgdaInductiveConstructor{⊥}\AgdaSymbol{\}}\AgdaSpace{}%
\AgdaBound{f}\AgdaSpace{}%
\AgdaKeyword{with}\AgdaSpace{}%
\AgdaBound{f}\AgdaSpace{}%
\AgdaSymbol{\{}\AgdaInductiveConstructor{⊥}\AgdaSymbol{\}}\AgdaSpace{}%
\AgdaInductiveConstructor{refl}\<%
\\
\>[0]\AgdaSymbol{...}\AgdaSpace{}%
\AgdaSymbol{|}\AgdaSpace{}%
\AgdaInductiveConstructor{fun}\AgdaSpace{}%
\AgdaSymbol{()}\<%
\\
\>[0]\AgdaFunction{all-funs∈}\AgdaSpace{}%
\AgdaSymbol{\{}\AgdaBound{v}\AgdaSpace{}%
\AgdaOperator{\AgdaInductiveConstructor{↦}}\AgdaSpace{}%
\AgdaBound{w}\AgdaSymbol{\}}\AgdaSpace{}%
\AgdaBound{f}\AgdaSpace{}%
\AgdaSymbol{=}\AgdaSpace{}%
\AgdaOperator{\AgdaInductiveConstructor{⟨}}\AgdaSpace{}%
\AgdaBound{v}\AgdaSpace{}%
\AgdaOperator{\AgdaInductiveConstructor{,}}\AgdaSpace{}%
\AgdaOperator{\AgdaInductiveConstructor{⟨}}\AgdaSpace{}%
\AgdaBound{w}\AgdaSpace{}%
\AgdaOperator{\AgdaInductiveConstructor{,}}\AgdaSpace{}%
\AgdaInductiveConstructor{refl}\AgdaSpace{}%
\AgdaOperator{\AgdaInductiveConstructor{⟩}}\AgdaSpace{}%
\AgdaOperator{\AgdaInductiveConstructor{⟩}}\<%
\\
\>[0]\AgdaFunction{all-funs∈}\AgdaSpace{}%
\AgdaSymbol{\{}\AgdaBound{u}\AgdaSpace{}%
\AgdaOperator{\AgdaInductiveConstructor{⊔}}\AgdaSpace{}%
\AgdaBound{u′}\AgdaSymbol{\}}\AgdaSpace{}%
\AgdaBound{f}\<%
\\
\>[0][@{}l@{\AgdaIndent{0}}]%
\>[4]\AgdaKeyword{with}\AgdaSpace{}%
\AgdaFunction{all-funs∈}\AgdaSpace{}%
\AgdaSymbol{λ}\AgdaSpace{}%
\AgdaBound{z}\AgdaSpace{}%
\AgdaSymbol{→}\AgdaSpace{}%
\AgdaBound{f}\AgdaSpace{}%
\AgdaSymbol{(}\AgdaInductiveConstructor{inj₁}\AgdaSpace{}%
\AgdaBound{z}\AgdaSymbol{)}\<%
\\
\>[0]\AgdaSymbol{...}\AgdaSpace{}%
\AgdaSymbol{|}\AgdaSpace{}%
\AgdaOperator{\AgdaInductiveConstructor{⟨}}\AgdaSpace{}%
\AgdaBound{v}\AgdaSpace{}%
\AgdaOperator{\AgdaInductiveConstructor{,}}\AgdaSpace{}%
\AgdaOperator{\AgdaInductiveConstructor{⟨}}\AgdaSpace{}%
\AgdaBound{w}\AgdaSpace{}%
\AgdaOperator{\AgdaInductiveConstructor{,}}\AgdaSpace{}%
\AgdaBound{m}\AgdaSpace{}%
\AgdaOperator{\AgdaInductiveConstructor{⟩}}\AgdaSpace{}%
\AgdaOperator{\AgdaInductiveConstructor{⟩}}\AgdaSpace{}%
\AgdaSymbol{=}\AgdaSpace{}%
\AgdaOperator{\AgdaInductiveConstructor{⟨}}\AgdaSpace{}%
\AgdaBound{v}\AgdaSpace{}%
\AgdaOperator{\AgdaInductiveConstructor{,}}\AgdaSpace{}%
\AgdaOperator{\AgdaInductiveConstructor{⟨}}\AgdaSpace{}%
\AgdaBound{w}\AgdaSpace{}%
\AgdaOperator{\AgdaInductiveConstructor{,}}\AgdaSpace{}%
\AgdaSymbol{(}\AgdaInductiveConstructor{inj₁}\AgdaSpace{}%
\AgdaBound{m}\AgdaSymbol{)}\AgdaSpace{}%
\AgdaOperator{\AgdaInductiveConstructor{⟩}}\AgdaSpace{}%
\AgdaOperator{\AgdaInductiveConstructor{⟩}}\<%
\end{code}
\end{fence}

\hypertarget{domains-and-codomains}{%
\subsection{Domains and codomains}\label{domains-and-codomains}}

Returning to our goal, the inversion principle for less-than a function,
we want to show that \texttt{v\ ↦\ w\ ⊑\ u} implies that \texttt{u}
includes a set of function values such that the join of their domains is
less than \texttt{v} and the join of their codomains is greater than
\texttt{w}.

To this end we define the following \texttt{⨆dom} and \texttt{⨆cod}
functions. Given some value \texttt{u} (that represents a set of
entries), \texttt{⨆dom\ u} returns the join of their domains and
\texttt{⨆cod\ u} returns the join of their codomains.

\begin{fence}
\begin{code}%
\>[0]\AgdaFunction{⨆dom}\AgdaSpace{}%
\AgdaSymbol{:}\AgdaSpace{}%
\AgdaSymbol{(}\AgdaBound{u}\AgdaSpace{}%
\AgdaSymbol{:}\AgdaSpace{}%
\AgdaDatatype{Value}\AgdaSymbol{)}\AgdaSpace{}%
\AgdaSymbol{→}\AgdaSpace{}%
\AgdaDatatype{Value}\<%
\\
\>[0]\AgdaFunction{⨆dom}\AgdaSpace{}%
\AgdaInductiveConstructor{⊥}%
\>[8]\AgdaSymbol{=}\AgdaSpace{}%
\AgdaInductiveConstructor{⊥}\<%
\\
\>[0]\AgdaFunction{⨆dom}\AgdaSpace{}%
\AgdaSymbol{(}\AgdaBound{v}\AgdaSpace{}%
\AgdaOperator{\AgdaInductiveConstructor{↦}}\AgdaSpace{}%
\AgdaBound{w}\AgdaSymbol{)}\AgdaSpace{}%
\AgdaSymbol{=}\AgdaSpace{}%
\AgdaBound{v}\<%
\\
\>[0]\AgdaFunction{⨆dom}\AgdaSpace{}%
\AgdaSymbol{(}\AgdaBound{u}\AgdaSpace{}%
\AgdaOperator{\AgdaInductiveConstructor{⊔}}\AgdaSpace{}%
\AgdaBound{u′}\AgdaSymbol{)}\AgdaSpace{}%
\AgdaSymbol{=}\AgdaSpace{}%
\AgdaFunction{⨆dom}\AgdaSpace{}%
\AgdaBound{u}\AgdaSpace{}%
\AgdaOperator{\AgdaInductiveConstructor{⊔}}\AgdaSpace{}%
\AgdaFunction{⨆dom}\AgdaSpace{}%
\AgdaBound{u′}\<%
\\
%
\\[\AgdaEmptyExtraSkip]%
\>[0]\AgdaFunction{⨆cod}\AgdaSpace{}%
\AgdaSymbol{:}\AgdaSpace{}%
\AgdaSymbol{(}\AgdaBound{u}\AgdaSpace{}%
\AgdaSymbol{:}\AgdaSpace{}%
\AgdaDatatype{Value}\AgdaSymbol{)}\AgdaSpace{}%
\AgdaSymbol{→}\AgdaSpace{}%
\AgdaDatatype{Value}\<%
\\
\>[0]\AgdaFunction{⨆cod}\AgdaSpace{}%
\AgdaInductiveConstructor{⊥}%
\>[8]\AgdaSymbol{=}\AgdaSpace{}%
\AgdaInductiveConstructor{⊥}\<%
\\
\>[0]\AgdaFunction{⨆cod}\AgdaSpace{}%
\AgdaSymbol{(}\AgdaBound{v}\AgdaSpace{}%
\AgdaOperator{\AgdaInductiveConstructor{↦}}\AgdaSpace{}%
\AgdaBound{w}\AgdaSymbol{)}\AgdaSpace{}%
\AgdaSymbol{=}\AgdaSpace{}%
\AgdaBound{w}\<%
\\
\>[0]\AgdaFunction{⨆cod}\AgdaSpace{}%
\AgdaSymbol{(}\AgdaBound{u}\AgdaSpace{}%
\AgdaOperator{\AgdaInductiveConstructor{⊔}}\AgdaSpace{}%
\AgdaBound{u′}\AgdaSymbol{)}\AgdaSpace{}%
\AgdaSymbol{=}\AgdaSpace{}%
\AgdaFunction{⨆cod}\AgdaSpace{}%
\AgdaBound{u}\AgdaSpace{}%
\AgdaOperator{\AgdaInductiveConstructor{⊔}}\AgdaSpace{}%
\AgdaFunction{⨆cod}\AgdaSpace{}%
\AgdaBound{u′}\<%
\end{code}
\end{fence}

We need just one property each for \texttt{⨆dom} and \texttt{⨆cod}.
Given a collection of functions represented by value \texttt{u}, and an
entry \texttt{v\ ↦\ w\ ∈\ u}, we know that \texttt{v} is included in the
domain of \texttt{v}.

\begin{fence}
\begin{code}%
\>[0]\AgdaFunction{↦∈→⊆⨆dom}%
\>[2057I]\AgdaSymbol{:}\AgdaSpace{}%
\AgdaSymbol{∀\{}\AgdaBound{u}\AgdaSpace{}%
\AgdaBound{v}\AgdaSpace{}%
\AgdaBound{w}\AgdaSpace{}%
\AgdaSymbol{:}\AgdaSpace{}%
\AgdaDatatype{Value}\AgdaSymbol{\}}\<%
\\
\>[2057I][@{}l@{\AgdaIndent{0}}]%
\>[10]\AgdaSymbol{→}%
\>[2063I]\AgdaFunction{all-funs}\AgdaSpace{}%
\AgdaBound{u}%
\>[24]\AgdaSymbol{→}%
\>[27]\AgdaSymbol{(}\AgdaBound{v}\AgdaSpace{}%
\AgdaOperator{\AgdaInductiveConstructor{↦}}\AgdaSpace{}%
\AgdaBound{w}\AgdaSymbol{)}\AgdaSpace{}%
\AgdaOperator{\AgdaFunction{∈}}\AgdaSpace{}%
\AgdaBound{u}\<%
\\
\>[.][@{}l@{}]\<[2063I]%
\>[12]\AgdaComment{----------------------}\<%
\\
%
\>[10]\AgdaSymbol{→}\AgdaSpace{}%
\AgdaBound{v}\AgdaSpace{}%
\AgdaOperator{\AgdaFunction{⊆}}\AgdaSpace{}%
\AgdaFunction{⨆dom}\AgdaSpace{}%
\AgdaBound{u}\<%
\\
\>[0]\AgdaFunction{↦∈→⊆⨆dom}\AgdaSpace{}%
\AgdaSymbol{\{}\AgdaInductiveConstructor{⊥}\AgdaSymbol{\}}\AgdaSpace{}%
\AgdaBound{fg}\AgdaSpace{}%
\AgdaSymbol{()}\AgdaSpace{}%
\AgdaBound{u∈v}\<%
\\
\>[0]\AgdaFunction{↦∈→⊆⨆dom}\AgdaSpace{}%
\AgdaSymbol{\{}\AgdaBound{v}\AgdaSpace{}%
\AgdaOperator{\AgdaInductiveConstructor{↦}}\AgdaSpace{}%
\AgdaBound{w}\AgdaSymbol{\}}\AgdaSpace{}%
\AgdaBound{fg}\AgdaSpace{}%
\AgdaInductiveConstructor{refl}\AgdaSpace{}%
\AgdaBound{u∈v}\AgdaSpace{}%
\AgdaSymbol{=}\AgdaSpace{}%
\AgdaBound{u∈v}\<%
\\
\>[0]\AgdaFunction{↦∈→⊆⨆dom}\AgdaSpace{}%
\AgdaSymbol{\{}\AgdaBound{u}\AgdaSpace{}%
\AgdaOperator{\AgdaInductiveConstructor{⊔}}\AgdaSpace{}%
\AgdaBound{u′}\AgdaSymbol{\}}\AgdaSpace{}%
\AgdaBound{fg}\AgdaSpace{}%
\AgdaSymbol{(}\AgdaInductiveConstructor{inj₁}\AgdaSpace{}%
\AgdaBound{v↦w∈u}\AgdaSymbol{)}\AgdaSpace{}%
\AgdaBound{u∈v}\AgdaSpace{}%
\AgdaSymbol{=}\<%
\\
\>[0][@{}l@{\AgdaIndent{0}}]%
\>[3]\AgdaKeyword{let}\AgdaSpace{}%
\AgdaBound{ih}\AgdaSpace{}%
\AgdaSymbol{=}\AgdaSpace{}%
\AgdaFunction{↦∈→⊆⨆dom}\AgdaSpace{}%
\AgdaSymbol{(λ}\AgdaSpace{}%
\AgdaBound{z}\AgdaSpace{}%
\AgdaSymbol{→}\AgdaSpace{}%
\AgdaBound{fg}\AgdaSpace{}%
\AgdaSymbol{(}\AgdaInductiveConstructor{inj₁}\AgdaSpace{}%
\AgdaBound{z}\AgdaSymbol{))}\AgdaSpace{}%
\AgdaBound{v↦w∈u}\AgdaSpace{}%
\AgdaKeyword{in}\<%
\\
%
\>[3]\AgdaInductiveConstructor{inj₁}\AgdaSpace{}%
\AgdaSymbol{(}\AgdaBound{ih}\AgdaSpace{}%
\AgdaBound{u∈v}\AgdaSymbol{)}\<%
\\
\>[0]\AgdaFunction{↦∈→⊆⨆dom}\AgdaSpace{}%
\AgdaSymbol{\{}\AgdaBound{u}\AgdaSpace{}%
\AgdaOperator{\AgdaInductiveConstructor{⊔}}\AgdaSpace{}%
\AgdaBound{u′}\AgdaSymbol{\}}\AgdaSpace{}%
\AgdaBound{fg}\AgdaSpace{}%
\AgdaSymbol{(}\AgdaInductiveConstructor{inj₂}\AgdaSpace{}%
\AgdaBound{v↦w∈u′}\AgdaSymbol{)}\AgdaSpace{}%
\AgdaBound{u∈v}\AgdaSpace{}%
\AgdaSymbol{=}\<%
\\
\>[0][@{}l@{\AgdaIndent{0}}]%
\>[3]\AgdaKeyword{let}\AgdaSpace{}%
\AgdaBound{ih}\AgdaSpace{}%
\AgdaSymbol{=}\AgdaSpace{}%
\AgdaFunction{↦∈→⊆⨆dom}\AgdaSpace{}%
\AgdaSymbol{(λ}\AgdaSpace{}%
\AgdaBound{z}\AgdaSpace{}%
\AgdaSymbol{→}\AgdaSpace{}%
\AgdaBound{fg}\AgdaSpace{}%
\AgdaSymbol{(}\AgdaInductiveConstructor{inj₂}\AgdaSpace{}%
\AgdaBound{z}\AgdaSymbol{))}\AgdaSpace{}%
\AgdaBound{v↦w∈u′}\AgdaSpace{}%
\AgdaKeyword{in}\<%
\\
%
\>[3]\AgdaInductiveConstructor{inj₂}\AgdaSpace{}%
\AgdaSymbol{(}\AgdaBound{ih}\AgdaSpace{}%
\AgdaBound{u∈v}\AgdaSymbol{)}\<%
\end{code}
\end{fence}

Regarding \texttt{⨆cod}, suppose we have a collection of functions
represented by \texttt{u}, but all of them are just copies of
\texttt{v\ ↦\ w}. Then the \texttt{⨆cod\ u} is included in \texttt{w}.

\begin{fence}
\begin{code}%
\>[0]\AgdaFunction{⊆↦→⨆cod⊆}\AgdaSpace{}%
\AgdaSymbol{:}\AgdaSpace{}%
\AgdaSymbol{∀\{}\AgdaBound{u}\AgdaSpace{}%
\AgdaBound{v}\AgdaSpace{}%
\AgdaBound{w}\AgdaSpace{}%
\AgdaSymbol{:}\AgdaSpace{}%
\AgdaDatatype{Value}\AgdaSymbol{\}}\<%
\\
\>[0][@{}l@{\AgdaIndent{0}}]%
\>[8]\AgdaSymbol{→}%
\>[2133I]\AgdaBound{u}\AgdaSpace{}%
\AgdaOperator{\AgdaFunction{⊆}}\AgdaSpace{}%
\AgdaBound{v}\AgdaSpace{}%
\AgdaOperator{\AgdaInductiveConstructor{↦}}\AgdaSpace{}%
\AgdaBound{w}\<%
\\
\>[.][@{}l@{}]\<[2133I]%
\>[10]\AgdaComment{---------}\<%
\\
%
\>[8]\AgdaSymbol{→}\AgdaSpace{}%
\AgdaFunction{⨆cod}\AgdaSpace{}%
\AgdaBound{u}\AgdaSpace{}%
\AgdaOperator{\AgdaFunction{⊆}}\AgdaSpace{}%
\AgdaBound{w}\<%
\\
\>[0]\AgdaFunction{⊆↦→⨆cod⊆}\AgdaSpace{}%
\AgdaSymbol{\{}\AgdaInductiveConstructor{⊥}\AgdaSymbol{\}}\AgdaSpace{}%
\AgdaBound{s}\AgdaSpace{}%
\AgdaInductiveConstructor{refl}\AgdaSpace{}%
\AgdaKeyword{with}\AgdaSpace{}%
\AgdaBound{s}\AgdaSpace{}%
\AgdaSymbol{\{}\AgdaInductiveConstructor{⊥}\AgdaSymbol{\}}\AgdaSpace{}%
\AgdaInductiveConstructor{refl}\<%
\\
\>[0]\AgdaSymbol{...}\AgdaSpace{}%
\AgdaSymbol{|}\AgdaSpace{}%
\AgdaSymbol{()}\<%
\\
\>[0]\AgdaFunction{⊆↦→⨆cod⊆}\AgdaSpace{}%
\AgdaSymbol{\{}\AgdaBound{C}\AgdaSpace{}%
\AgdaOperator{\AgdaInductiveConstructor{↦}}\AgdaSpace{}%
\AgdaBound{C′}\AgdaSymbol{\}}\AgdaSpace{}%
\AgdaBound{s}\AgdaSpace{}%
\AgdaBound{m}\AgdaSpace{}%
\AgdaKeyword{with}\AgdaSpace{}%
\AgdaBound{s}\AgdaSpace{}%
\AgdaSymbol{\{}\AgdaBound{C}\AgdaSpace{}%
\AgdaOperator{\AgdaInductiveConstructor{↦}}\AgdaSpace{}%
\AgdaBound{C′}\AgdaSymbol{\}}\AgdaSpace{}%
\AgdaInductiveConstructor{refl}\<%
\\
\>[0]\AgdaSymbol{...}\AgdaSpace{}%
\AgdaSymbol{|}\AgdaSpace{}%
\AgdaInductiveConstructor{refl}\AgdaSpace{}%
\AgdaSymbol{=}\AgdaSpace{}%
\AgdaBound{m}\<%
\\
\>[0]\AgdaFunction{⊆↦→⨆cod⊆}\AgdaSpace{}%
\AgdaSymbol{\{}\AgdaBound{u}\AgdaSpace{}%
\AgdaOperator{\AgdaInductiveConstructor{⊔}}\AgdaSpace{}%
\AgdaBound{u′}\AgdaSymbol{\}}\AgdaSpace{}%
\AgdaBound{s}\AgdaSpace{}%
\AgdaSymbol{(}\AgdaInductiveConstructor{inj₁}\AgdaSpace{}%
\AgdaBound{x}\AgdaSymbol{)}\AgdaSpace{}%
\AgdaSymbol{=}\AgdaSpace{}%
\AgdaFunction{⊆↦→⨆cod⊆}\AgdaSpace{}%
\AgdaSymbol{(λ}\AgdaSpace{}%
\AgdaSymbol{\{}\AgdaBound{C}\AgdaSymbol{\}}\AgdaSpace{}%
\AgdaBound{z}\AgdaSpace{}%
\AgdaSymbol{→}\AgdaSpace{}%
\AgdaBound{s}\AgdaSpace{}%
\AgdaSymbol{(}\AgdaInductiveConstructor{inj₁}\AgdaSpace{}%
\AgdaBound{z}\AgdaSymbol{))}\AgdaSpace{}%
\AgdaBound{x}\<%
\\
\>[0]\AgdaFunction{⊆↦→⨆cod⊆}\AgdaSpace{}%
\AgdaSymbol{\{}\AgdaBound{u}\AgdaSpace{}%
\AgdaOperator{\AgdaInductiveConstructor{⊔}}\AgdaSpace{}%
\AgdaBound{u′}\AgdaSymbol{\}}\AgdaSpace{}%
\AgdaBound{s}\AgdaSpace{}%
\AgdaSymbol{(}\AgdaInductiveConstructor{inj₂}\AgdaSpace{}%
\AgdaBound{y}\AgdaSymbol{)}\AgdaSpace{}%
\AgdaSymbol{=}\AgdaSpace{}%
\AgdaFunction{⊆↦→⨆cod⊆}\AgdaSpace{}%
\AgdaSymbol{(λ}\AgdaSpace{}%
\AgdaSymbol{\{}\AgdaBound{C}\AgdaSymbol{\}}\AgdaSpace{}%
\AgdaBound{z}\AgdaSpace{}%
\AgdaSymbol{→}\AgdaSpace{}%
\AgdaBound{s}\AgdaSpace{}%
\AgdaSymbol{(}\AgdaInductiveConstructor{inj₂}\AgdaSpace{}%
\AgdaBound{z}\AgdaSymbol{))}\AgdaSpace{}%
\AgdaBound{y}\<%
\end{code}
\end{fence}

With the \texttt{⨆dom} and \texttt{⨆cod} functions in hand, we can make
precise the conclusion of the inversion principle for functions, which
we package into the following predicate named \texttt{factor}. We say
that \texttt{v\ ↦\ w} \emph{factors} \texttt{u} into \texttt{u′} if
\texttt{u′} is a included in \texttt{u}, if \texttt{u′} contains only
functions, its domain is less than \texttt{v}, and its codomain is
greater than \texttt{w}.

\begin{fence}
\begin{code}%
\>[0]\AgdaFunction{factor}\AgdaSpace{}%
\AgdaSymbol{:}\AgdaSpace{}%
\AgdaSymbol{(}\AgdaBound{u}\AgdaSpace{}%
\AgdaSymbol{:}\AgdaSpace{}%
\AgdaDatatype{Value}\AgdaSymbol{)}\AgdaSpace{}%
\AgdaSymbol{→}\AgdaSpace{}%
\AgdaSymbol{(}\AgdaBound{u′}\AgdaSpace{}%
\AgdaSymbol{:}\AgdaSpace{}%
\AgdaDatatype{Value}\AgdaSymbol{)}\AgdaSpace{}%
\AgdaSymbol{→}\AgdaSpace{}%
\AgdaSymbol{(}\AgdaBound{v}\AgdaSpace{}%
\AgdaSymbol{:}\AgdaSpace{}%
\AgdaDatatype{Value}\AgdaSymbol{)}\AgdaSpace{}%
\AgdaSymbol{→}\AgdaSpace{}%
\AgdaSymbol{(}\AgdaBound{w}\AgdaSpace{}%
\AgdaSymbol{:}\AgdaSpace{}%
\AgdaDatatype{Value}\AgdaSymbol{)}\AgdaSpace{}%
\AgdaSymbol{→}\AgdaSpace{}%
\AgdaPrimitiveType{Set}\<%
\\
\>[0]\AgdaFunction{factor}\AgdaSpace{}%
\AgdaBound{u}\AgdaSpace{}%
\AgdaBound{u′}\AgdaSpace{}%
\AgdaBound{v}\AgdaSpace{}%
\AgdaBound{w}\AgdaSpace{}%
\AgdaSymbol{=}\AgdaSpace{}%
\AgdaFunction{all-funs}\AgdaSpace{}%
\AgdaBound{u′}%
\>[31]\AgdaOperator{\AgdaFunction{×}}%
\>[34]\AgdaBound{u′}\AgdaSpace{}%
\AgdaOperator{\AgdaFunction{⊆}}\AgdaSpace{}%
\AgdaBound{u}%
\>[42]\AgdaOperator{\AgdaFunction{×}}%
\>[45]\AgdaFunction{⨆dom}\AgdaSpace{}%
\AgdaBound{u′}\AgdaSpace{}%
\AgdaOperator{\AgdaDatatype{⊑}}\AgdaSpace{}%
\AgdaBound{v}%
\>[58]\AgdaOperator{\AgdaFunction{×}}%
\>[61]\AgdaBound{w}\AgdaSpace{}%
\AgdaOperator{\AgdaDatatype{⊑}}\AgdaSpace{}%
\AgdaFunction{⨆cod}\AgdaSpace{}%
\AgdaBound{u′}\<%
\end{code}
\end{fence}

So the inversion principle for functions can be stated as

\begin{myDisplay}
  v ↦ w ⊑ u
  ---------------
→ factor u u′ v w
\end{myDisplay}

We prove the inversion principle for functions by induction on the
derivation of the less-than relation. To make the induction hypothesis
stronger, we broaden the premise \texttt{v\ ↦\ w\ ⊑\ u} to
\texttt{u₁\ ⊑\ u₂}, and strengthen the conclusion to say that for
\emph{every} function value \texttt{v\ ↦\ w\ ∈\ u₁}, we have that
\texttt{v\ ↦\ w} factors \texttt{u₂} into some value \texttt{u₃}.

\begin{myDisplay}
→ u₁ ⊑ u₂
  ------------------------------------------------------
→ ∀{v w} → v ↦ w ∈ u₁ → Σ[ u₃ ∈ Value ] factor u₂ u₃ v w
\end{myDisplay}

\hypertarget{inversion-of-less-than-for-functions-the-case-for--trans}{%
\subsection{Inversion of less-than for functions, the case for
⊑-trans}\label{inversion-of-less-than-for-functions-the-case-for--trans}}

The crux of the proof is the case for \texttt{⊑-trans}.

\begin{myDisplay}
u₁ ⊑ u   u ⊑ u₂
--------------- (⊑-trans)
    u₁ ⊑ u₂
\end{myDisplay}

By the induction hypothesis for \texttt{u₁\ ⊑\ u}, we know that
\texttt{v\ ↦\ w\ factors\ u\ into\ u′}, for some value \texttt{u′}, so
we have \texttt{all-funs\ u′} and \texttt{u′\ ⊆\ u}. By the induction
hypothesis for \texttt{u\ ⊑\ u₂}, we know that for any
\texttt{v′\ ↦\ w′\ ∈\ u}, \texttt{v′\ ↦\ w′} factors \texttt{u₂} into
\texttt{u₃}. With these facts in hand, we proceed by induction on
\texttt{u′} to prove that \texttt{(⨆dom\ u′)\ ↦\ (⨆cod\ u′)} factors
\texttt{u₂} into \texttt{u₃}. We discuss each case of the proof in the
text below.

\begin{fence}
\begin{code}%
\>[0]\AgdaFunction{sub-inv-trans}\AgdaSpace{}%
\AgdaSymbol{:}\AgdaSpace{}%
\AgdaSymbol{∀\{}\AgdaBound{u′}\AgdaSpace{}%
\AgdaBound{u₂}\AgdaSpace{}%
\AgdaBound{u}\AgdaSpace{}%
\AgdaSymbol{:}\AgdaSpace{}%
\AgdaDatatype{Value}\AgdaSymbol{\}}\<%
\\
\>[0][@{}l@{\AgdaIndent{0}}]%
\>[4]\AgdaSymbol{→}\AgdaSpace{}%
\AgdaFunction{all-funs}\AgdaSpace{}%
\AgdaBound{u′}%
\>[19]\AgdaSymbol{→}%
\>[22]\AgdaBound{u′}\AgdaSpace{}%
\AgdaOperator{\AgdaFunction{⊆}}\AgdaSpace{}%
\AgdaBound{u}\<%
\\
%
\>[4]\AgdaSymbol{→}%
\>[2241I]\AgdaSymbol{(∀\{}\AgdaBound{v′}\AgdaSpace{}%
\AgdaBound{w′}\AgdaSymbol{\}}\AgdaSpace{}%
\AgdaSymbol{→}\AgdaSpace{}%
\AgdaBound{v′}\AgdaSpace{}%
\AgdaOperator{\AgdaInductiveConstructor{↦}}\AgdaSpace{}%
\AgdaBound{w′}\AgdaSpace{}%
\AgdaOperator{\AgdaFunction{∈}}\AgdaSpace{}%
\AgdaBound{u}\AgdaSpace{}%
\AgdaSymbol{→}\AgdaSpace{}%
\AgdaFunction{Σ[}\AgdaSpace{}%
\AgdaBound{u₃}\AgdaSpace{}%
\AgdaFunction{∈}\AgdaSpace{}%
\AgdaDatatype{Value}\AgdaSpace{}%
\AgdaFunction{]}\AgdaSpace{}%
\AgdaFunction{factor}\AgdaSpace{}%
\AgdaBound{u₂}\AgdaSpace{}%
\AgdaBound{u₃}\AgdaSpace{}%
\AgdaBound{v′}\AgdaSpace{}%
\AgdaBound{w′}\AgdaSymbol{)}\<%
\\
\>[.][@{}l@{}]\<[2241I]%
\>[6]\AgdaComment{---------------------------------------------------------------}\<%
\\
%
\>[4]\AgdaSymbol{→}\AgdaSpace{}%
\AgdaFunction{Σ[}\AgdaSpace{}%
\AgdaBound{u₃}\AgdaSpace{}%
\AgdaFunction{∈}\AgdaSpace{}%
\AgdaDatatype{Value}\AgdaSpace{}%
\AgdaFunction{]}\AgdaSpace{}%
\AgdaFunction{factor}\AgdaSpace{}%
\AgdaBound{u₂}\AgdaSpace{}%
\AgdaBound{u₃}\AgdaSpace{}%
\AgdaSymbol{(}\AgdaFunction{⨆dom}\AgdaSpace{}%
\AgdaBound{u′}\AgdaSymbol{)}\AgdaSpace{}%
\AgdaSymbol{(}\AgdaFunction{⨆cod}\AgdaSpace{}%
\AgdaBound{u′}\AgdaSymbol{)}\<%
\\
\>[0]\AgdaFunction{sub-inv-trans}\AgdaSpace{}%
\AgdaSymbol{\{}\AgdaInductiveConstructor{⊥}\AgdaSymbol{\}}\AgdaSpace{}%
\AgdaSymbol{\{}\AgdaBound{u₂}\AgdaSymbol{\}}\AgdaSpace{}%
\AgdaSymbol{\{}\AgdaBound{u}\AgdaSymbol{\}}\AgdaSpace{}%
\AgdaBound{fu′}\AgdaSpace{}%
\AgdaBound{u′⊆u}\AgdaSpace{}%
\AgdaBound{IH}\AgdaSpace{}%
\AgdaSymbol{=}\<%
\\
\>[0][@{}l@{\AgdaIndent{0}}]%
\>[3]\AgdaFunction{⊥-elim}\AgdaSpace{}%
\AgdaSymbol{(}\AgdaFunction{contradiction}\AgdaSpace{}%
\AgdaSymbol{(}\AgdaBound{fu′}\AgdaSpace{}%
\AgdaInductiveConstructor{refl}\AgdaSymbol{)}\AgdaSpace{}%
\AgdaFunction{¬Fun⊥}\AgdaSymbol{)}\<%
\\
\>[0]\AgdaFunction{sub-inv-trans}\AgdaSpace{}%
\AgdaSymbol{\{}\AgdaBound{u₁′}\AgdaSpace{}%
\AgdaOperator{\AgdaInductiveConstructor{↦}}\AgdaSpace{}%
\AgdaBound{u₂′}\AgdaSymbol{\}}\AgdaSpace{}%
\AgdaSymbol{\{}\AgdaBound{u₂}\AgdaSymbol{\}}\AgdaSpace{}%
\AgdaSymbol{\{}\AgdaBound{u}\AgdaSymbol{\}}\AgdaSpace{}%
\AgdaBound{fg}\AgdaSpace{}%
\AgdaBound{u′⊆u}\AgdaSpace{}%
\AgdaBound{IH}\AgdaSpace{}%
\AgdaSymbol{=}\AgdaSpace{}%
\AgdaBound{IH}\AgdaSpace{}%
\AgdaSymbol{(}\AgdaFunction{↦⊆→∈}\AgdaSpace{}%
\AgdaBound{u′⊆u}\AgdaSymbol{)}\<%
\\
\>[0]\AgdaFunction{sub-inv-trans}\AgdaSpace{}%
\AgdaSymbol{\{}\AgdaBound{u₁′}\AgdaSpace{}%
\AgdaOperator{\AgdaInductiveConstructor{⊔}}\AgdaSpace{}%
\AgdaBound{u₂′}\AgdaSymbol{\}}\AgdaSpace{}%
\AgdaSymbol{\{}\AgdaBound{u₂}\AgdaSymbol{\}}\AgdaSpace{}%
\AgdaSymbol{\{}\AgdaBound{u}\AgdaSymbol{\}}\AgdaSpace{}%
\AgdaBound{fg}\AgdaSpace{}%
\AgdaBound{u′⊆u}\AgdaSpace{}%
\AgdaBound{IH}\<%
\\
\>[0][@{}l@{\AgdaIndent{0}}]%
\>[4]\AgdaKeyword{with}\AgdaSpace{}%
\AgdaFunction{⊔⊆-inv}\AgdaSpace{}%
\AgdaBound{u′⊆u}\<%
\\
\>[0]\AgdaSymbol{...}%
\>[2305I]\AgdaSymbol{|}\AgdaSpace{}%
\AgdaOperator{\AgdaInductiveConstructor{⟨}}\AgdaSpace{}%
\AgdaBound{u₁′⊆u}\AgdaSpace{}%
\AgdaOperator{\AgdaInductiveConstructor{,}}\AgdaSpace{}%
\AgdaBound{u₂′⊆u}\AgdaSpace{}%
\AgdaOperator{\AgdaInductiveConstructor{⟩}}\<%
\\
\>[.][@{}l@{}]\<[2305I]%
\>[4]\AgdaKeyword{with}\AgdaSpace{}%
\AgdaFunction{sub-inv-trans}\AgdaSpace{}%
\AgdaSymbol{\{}\AgdaBound{u₁′}\AgdaSymbol{\}}\AgdaSpace{}%
\AgdaSymbol{\{}\AgdaBound{u₂}\AgdaSymbol{\}}\AgdaSpace{}%
\AgdaSymbol{\{}\AgdaBound{u}\AgdaSymbol{\}}\AgdaSpace{}%
\AgdaSymbol{(λ}\AgdaSpace{}%
\AgdaSymbol{\{}\AgdaBound{v′}\AgdaSymbol{\}}\AgdaSpace{}%
\AgdaBound{z}\AgdaSpace{}%
\AgdaSymbol{→}\AgdaSpace{}%
\AgdaBound{fg}\AgdaSpace{}%
\AgdaSymbol{(}\AgdaInductiveConstructor{inj₁}\AgdaSpace{}%
\AgdaBound{z}\AgdaSymbol{))}\AgdaSpace{}%
\AgdaBound{u₁′⊆u}\AgdaSpace{}%
\AgdaBound{IH}\<%
\\
\>[4][@{}l@{\AgdaIndent{0}}]%
\>[7]\AgdaSymbol{|}\AgdaSpace{}%
\AgdaFunction{sub-inv-trans}\AgdaSpace{}%
\AgdaSymbol{\{}\AgdaBound{u₂′}\AgdaSymbol{\}}\AgdaSpace{}%
\AgdaSymbol{\{}\AgdaBound{u₂}\AgdaSymbol{\}}\AgdaSpace{}%
\AgdaSymbol{\{}\AgdaBound{u}\AgdaSymbol{\}}\AgdaSpace{}%
\AgdaSymbol{(λ}\AgdaSpace{}%
\AgdaSymbol{\{}\AgdaBound{v′}\AgdaSymbol{\}}\AgdaSpace{}%
\AgdaBound{z}\AgdaSpace{}%
\AgdaSymbol{→}\AgdaSpace{}%
\AgdaBound{fg}\AgdaSpace{}%
\AgdaSymbol{(}\AgdaInductiveConstructor{inj₂}\AgdaSpace{}%
\AgdaBound{z}\AgdaSymbol{))}\AgdaSpace{}%
\AgdaBound{u₂′⊆u}\AgdaSpace{}%
\AgdaBound{IH}\<%
\\
\>[0]\AgdaSymbol{...}%
\>[2337I]\AgdaSymbol{|}\AgdaSpace{}%
\AgdaOperator{\AgdaInductiveConstructor{⟨}}\AgdaSpace{}%
\AgdaBound{u₃₁}\AgdaSpace{}%
\AgdaOperator{\AgdaInductiveConstructor{,}}\AgdaSpace{}%
\AgdaOperator{\AgdaInductiveConstructor{⟨}}\AgdaSpace{}%
\AgdaBound{fu21'}\AgdaSpace{}%
\AgdaOperator{\AgdaInductiveConstructor{,}}\AgdaSpace{}%
\AgdaOperator{\AgdaInductiveConstructor{⟨}}\AgdaSpace{}%
\AgdaBound{u₃₁⊆u₂}\AgdaSpace{}%
\AgdaOperator{\AgdaInductiveConstructor{,}}\AgdaSpace{}%
\AgdaOperator{\AgdaInductiveConstructor{⟨}}\AgdaSpace{}%
\AgdaBound{du₃₁⊑du₁′}\AgdaSpace{}%
\AgdaOperator{\AgdaInductiveConstructor{,}}\AgdaSpace{}%
\AgdaBound{cu₁′⊑cu₃₁}\AgdaSpace{}%
\AgdaOperator{\AgdaInductiveConstructor{⟩}}\AgdaSpace{}%
\AgdaOperator{\AgdaInductiveConstructor{⟩}}\AgdaSpace{}%
\AgdaOperator{\AgdaInductiveConstructor{⟩}}\AgdaSpace{}%
\AgdaOperator{\AgdaInductiveConstructor{⟩}}\<%
\\
\>[.][@{}l@{}]\<[2337I]%
\>[4]\AgdaSymbol{|}%
\>[2355I]\AgdaOperator{\AgdaInductiveConstructor{⟨}}\AgdaSpace{}%
\AgdaBound{u₃₂}\AgdaSpace{}%
\AgdaOperator{\AgdaInductiveConstructor{,}}\AgdaSpace{}%
\AgdaOperator{\AgdaInductiveConstructor{⟨}}\AgdaSpace{}%
\AgdaBound{fu22'}\AgdaSpace{}%
\AgdaOperator{\AgdaInductiveConstructor{,}}\AgdaSpace{}%
\AgdaOperator{\AgdaInductiveConstructor{⟨}}\AgdaSpace{}%
\AgdaBound{u₃₂⊆u₂}\AgdaSpace{}%
\AgdaOperator{\AgdaInductiveConstructor{,}}\AgdaSpace{}%
\AgdaOperator{\AgdaInductiveConstructor{⟨}}\AgdaSpace{}%
\AgdaBound{du₃₂⊑du₂′}\AgdaSpace{}%
\AgdaOperator{\AgdaInductiveConstructor{,}}\AgdaSpace{}%
\AgdaBound{cu₁′⊑cu₃₂}\AgdaSpace{}%
\AgdaOperator{\AgdaInductiveConstructor{⟩}}\AgdaSpace{}%
\AgdaOperator{\AgdaInductiveConstructor{⟩}}\AgdaSpace{}%
\AgdaOperator{\AgdaInductiveConstructor{⟩}}\AgdaSpace{}%
\AgdaOperator{\AgdaInductiveConstructor{⟩}}\AgdaSpace{}%
\AgdaSymbol{=}\<%
\\
\>[.][@{}l@{}]\<[2355I]%
\>[6]\AgdaOperator{\AgdaInductiveConstructor{⟨}}\AgdaSpace{}%
\AgdaSymbol{(}\AgdaBound{u₃₁}\AgdaSpace{}%
\AgdaOperator{\AgdaInductiveConstructor{⊔}}\AgdaSpace{}%
\AgdaBound{u₃₂}\AgdaSymbol{)}\AgdaSpace{}%
\AgdaOperator{\AgdaInductiveConstructor{,}}\AgdaSpace{}%
\AgdaOperator{\AgdaInductiveConstructor{⟨}}\AgdaSpace{}%
\AgdaFunction{fu₂′}\AgdaSpace{}%
\AgdaOperator{\AgdaInductiveConstructor{,}}\AgdaSpace{}%
\AgdaOperator{\AgdaInductiveConstructor{⟨}}\AgdaSpace{}%
\AgdaFunction{u₂′⊆u₂}\AgdaSpace{}%
\AgdaOperator{\AgdaInductiveConstructor{,}}\<%
\\
%
\>[6]\AgdaOperator{\AgdaInductiveConstructor{⟨}}%
\>[2383I]\AgdaFunction{⊔⊑⊔}\AgdaSpace{}%
\AgdaBound{du₃₁⊑du₁′}\AgdaSpace{}%
\AgdaBound{du₃₂⊑du₂′}\AgdaSpace{}%
\AgdaOperator{\AgdaInductiveConstructor{,}}\<%
\\
\>[.][@{}l@{}]\<[2383I]%
\>[8]\AgdaFunction{⊔⊑⊔}\AgdaSpace{}%
\AgdaBound{cu₁′⊑cu₃₁}\AgdaSpace{}%
\AgdaBound{cu₁′⊑cu₃₂}\AgdaSpace{}%
\AgdaOperator{\AgdaInductiveConstructor{⟩}}\AgdaSpace{}%
\AgdaOperator{\AgdaInductiveConstructor{⟩}}\AgdaSpace{}%
\AgdaOperator{\AgdaInductiveConstructor{⟩}}\AgdaSpace{}%
\AgdaOperator{\AgdaInductiveConstructor{⟩}}\<%
\\
%
\>[4]\AgdaKeyword{where}%
\>[2393I]\AgdaFunction{fu₂′}\AgdaSpace{}%
\AgdaSymbol{:}\AgdaSpace{}%
\AgdaSymbol{\{}\AgdaBound{v′}\AgdaSpace{}%
\AgdaSymbol{:}\AgdaSpace{}%
\AgdaDatatype{Value}\AgdaSymbol{\}}\AgdaSpace{}%
\AgdaSymbol{→}\AgdaSpace{}%
\AgdaBound{v′}\AgdaSpace{}%
\AgdaOperator{\AgdaFunction{∈}}\AgdaSpace{}%
\AgdaBound{u₃₁}\AgdaSpace{}%
\AgdaOperator{\AgdaDatatype{⊎}}\AgdaSpace{}%
\AgdaBound{v′}\AgdaSpace{}%
\AgdaOperator{\AgdaFunction{∈}}\AgdaSpace{}%
\AgdaBound{u₃₂}\AgdaSpace{}%
\AgdaSymbol{→}\AgdaSpace{}%
\AgdaDatatype{Fun}\AgdaSpace{}%
\AgdaBound{v′}\<%
\\
\>[.][@{}l@{}]\<[2393I]%
\>[10]\AgdaFunction{fu₂′}\AgdaSpace{}%
\AgdaSymbol{\{}\AgdaBound{v′}\AgdaSymbol{\}}\AgdaSpace{}%
\AgdaSymbol{(}\AgdaInductiveConstructor{inj₁}\AgdaSpace{}%
\AgdaBound{x}\AgdaSymbol{)}\AgdaSpace{}%
\AgdaSymbol{=}\AgdaSpace{}%
\AgdaBound{fu21'}\AgdaSpace{}%
\AgdaBound{x}\<%
\\
%
\>[10]\AgdaFunction{fu₂′}\AgdaSpace{}%
\AgdaSymbol{\{}\AgdaBound{v′}\AgdaSymbol{\}}\AgdaSpace{}%
\AgdaSymbol{(}\AgdaInductiveConstructor{inj₂}\AgdaSpace{}%
\AgdaBound{y}\AgdaSymbol{)}\AgdaSpace{}%
\AgdaSymbol{=}\AgdaSpace{}%
\AgdaBound{fu22'}\AgdaSpace{}%
\AgdaBound{y}\<%
\\
%
\>[10]\AgdaFunction{u₂′⊆u₂}\AgdaSpace{}%
\AgdaSymbol{:}\AgdaSpace{}%
\AgdaSymbol{\{}\AgdaBound{C}\AgdaSpace{}%
\AgdaSymbol{:}\AgdaSpace{}%
\AgdaDatatype{Value}\AgdaSymbol{\}}\AgdaSpace{}%
\AgdaSymbol{→}\AgdaSpace{}%
\AgdaBound{C}\AgdaSpace{}%
\AgdaOperator{\AgdaFunction{∈}}\AgdaSpace{}%
\AgdaBound{u₃₁}\AgdaSpace{}%
\AgdaOperator{\AgdaDatatype{⊎}}\AgdaSpace{}%
\AgdaBound{C}\AgdaSpace{}%
\AgdaOperator{\AgdaFunction{∈}}\AgdaSpace{}%
\AgdaBound{u₃₂}\AgdaSpace{}%
\AgdaSymbol{→}\AgdaSpace{}%
\AgdaBound{C}\AgdaSpace{}%
\AgdaOperator{\AgdaFunction{∈}}\AgdaSpace{}%
\AgdaBound{u₂}\<%
\\
%
\>[10]\AgdaFunction{u₂′⊆u₂}\AgdaSpace{}%
\AgdaSymbol{\{}\AgdaBound{C}\AgdaSymbol{\}}\AgdaSpace{}%
\AgdaSymbol{(}\AgdaInductiveConstructor{inj₁}\AgdaSpace{}%
\AgdaBound{x}\AgdaSymbol{)}\AgdaSpace{}%
\AgdaSymbol{=}\AgdaSpace{}%
\AgdaBound{u₃₁⊆u₂}\AgdaSpace{}%
\AgdaBound{x}\<%
\\
%
\>[10]\AgdaFunction{u₂′⊆u₂}\AgdaSpace{}%
\AgdaSymbol{\{}\AgdaBound{C}\AgdaSymbol{\}}\AgdaSpace{}%
\AgdaSymbol{(}\AgdaInductiveConstructor{inj₂}\AgdaSpace{}%
\AgdaBound{y}\AgdaSymbol{)}\AgdaSpace{}%
\AgdaSymbol{=}\AgdaSpace{}%
\AgdaBound{u₃₂⊆u₂}\AgdaSpace{}%
\AgdaBound{y}\<%
\end{code}
\end{fence}

\begin{itemize}
\item
  Suppose \texttt{u′\ ≡\ ⊥}. Then we have a contradiction because it is
  not the case that \texttt{Fun\ ⊥}.
\item
  Suppose \texttt{u′\ ≡\ u₁′\ ↦\ u₂′}. Then \texttt{u₁′\ ↦\ u₂′\ ∈\ u}
  and we can apply the premise (the induction hypothesis from
  \texttt{u\ ⊑\ u₂}) to obtain that \texttt{u₁′\ ↦\ u₂′} factors of
  \texttt{u₂\ into\ u₂′}. This case is complete because
  \texttt{⨆dom\ u′\ ≡\ u₁′} and \texttt{⨆cod\ u′\ ≡\ u₂′}.
\item
  Suppose \texttt{u′\ ≡\ u₁′\ ⊔\ u₂′}. Then we have \texttt{u₁′\ ⊆\ u}
  and \texttt{u₂′\ ⊆\ u}. We also have \texttt{all-funs\ u₁′} and
  \texttt{all-funs\ u₂′}, so we can apply the induction hypothesis for
  both \texttt{u₁′} and \texttt{u₂′}. So there exists values
  \texttt{u₃₁} and \texttt{u₃₂} such that
  \texttt{(⨆dom\ u₁′)\ ↦\ (⨆cod\ u₁′)} factors \texttt{u} into
  \texttt{u₃₁} and \texttt{(⨆dom\ u₂′)\ ↦\ (⨆cod\ u₂′)} factors
  \texttt{u} into \texttt{u₃₂}. We will show that
  \texttt{(⨆dom\ u)\ ↦\ (⨆cod\ u)} factors \texttt{u} into
  \texttt{u₃₁\ ⊔\ u₃₂}. So we need to show that

  \begin{myDisplay}
    ⨆dom (u₃₁ ⊔ u₃₂) ⊑ ⨆dom (u₁′ ⊔ u₂′)
    ⨆cod (u₁′ ⊔ u₂′) ⊑ ⨆cod (u₃₁ ⊔ u₃₂)
  \end{myDisplay}

  But those both follow directly from the factoring of \texttt{u} into
  \texttt{u₃₁} and \texttt{u₃₂}, using the monotonicity of \texttt{⊔}
  with respect to \texttt{⊑}.
\end{itemize}

\hypertarget{inversion-of-less-than-for-functions}{%
\subsection{Inversion of less-than for
functions}\label{inversion-of-less-than-for-functions}}

We come to the proof of the main lemma concerning the inversion of
less-than for functions. We show that if \texttt{u₁\ ⊑\ u₂}, then for
any \texttt{v\ ↦\ w\ ∈\ u₁}, we can factor \texttt{u₂} into \texttt{u₃}
according to \texttt{v\ ↦\ w}. We proceed by induction on the derivation
of \texttt{u₁\ ⊑\ u₂}, and describe each case in the text after the Agda
proof.

\begin{fence}
\begin{code}%
\>[0]\AgdaFunction{sub-inv}%
\>[2449I]\AgdaSymbol{:}\AgdaSpace{}%
\AgdaSymbol{∀\{}\AgdaBound{u₁}\AgdaSpace{}%
\AgdaBound{u₂}\AgdaSpace{}%
\AgdaSymbol{:}\AgdaSpace{}%
\AgdaDatatype{Value}\AgdaSymbol{\}}\<%
\\
\>[.][@{}l@{}]\<[2449I]%
\>[8]\AgdaSymbol{→}\AgdaSpace{}%
\AgdaBound{u₁}\AgdaSpace{}%
\AgdaOperator{\AgdaDatatype{⊑}}\AgdaSpace{}%
\AgdaBound{u₂}\<%
\\
%
\>[8]\AgdaSymbol{→}%
\>[2457I]\AgdaSymbol{∀\{}\AgdaBound{v}\AgdaSpace{}%
\AgdaBound{w}\AgdaSymbol{\}}\AgdaSpace{}%
\AgdaSymbol{→}\AgdaSpace{}%
\AgdaBound{v}\AgdaSpace{}%
\AgdaOperator{\AgdaInductiveConstructor{↦}}\AgdaSpace{}%
\AgdaBound{w}\AgdaSpace{}%
\AgdaOperator{\AgdaFunction{∈}}\AgdaSpace{}%
\AgdaBound{u₁}\<%
\\
\>[.][@{}l@{}]\<[2457I]%
\>[10]\AgdaComment{-------------------------------------}\<%
\\
%
\>[8]\AgdaSymbol{→}\AgdaSpace{}%
\AgdaFunction{Σ[}\AgdaSpace{}%
\AgdaBound{u₃}\AgdaSpace{}%
\AgdaFunction{∈}\AgdaSpace{}%
\AgdaDatatype{Value}\AgdaSpace{}%
\AgdaFunction{]}\AgdaSpace{}%
\AgdaFunction{factor}\AgdaSpace{}%
\AgdaBound{u₂}\AgdaSpace{}%
\AgdaBound{u₃}\AgdaSpace{}%
\AgdaBound{v}\AgdaSpace{}%
\AgdaBound{w}\<%
\\
\>[0]\AgdaFunction{sub-inv}\AgdaSpace{}%
\AgdaSymbol{\{}\AgdaInductiveConstructor{⊥}\AgdaSymbol{\}}\AgdaSpace{}%
\AgdaSymbol{\{}\AgdaBound{u₂}\AgdaSymbol{\}}\AgdaSpace{}%
\AgdaInductiveConstructor{⊑-bot}\AgdaSpace{}%
\AgdaSymbol{\{}\AgdaBound{v}\AgdaSymbol{\}}\AgdaSpace{}%
\AgdaSymbol{\{}\AgdaBound{w}\AgdaSymbol{\}}\AgdaSpace{}%
\AgdaSymbol{()}\<%
\\
\>[0]\AgdaFunction{sub-inv}\AgdaSpace{}%
\AgdaSymbol{\{}\AgdaBound{u₁₁}\AgdaSpace{}%
\AgdaOperator{\AgdaInductiveConstructor{⊔}}\AgdaSpace{}%
\AgdaBound{u₁₂}\AgdaSymbol{\}}\AgdaSpace{}%
\AgdaSymbol{\{}\AgdaBound{u₂}\AgdaSymbol{\}}\AgdaSpace{}%
\AgdaSymbol{(}\AgdaInductiveConstructor{⊑-conj-L}\AgdaSpace{}%
\AgdaBound{lt1}\AgdaSpace{}%
\AgdaBound{lt2}\AgdaSymbol{)}\AgdaSpace{}%
\AgdaSymbol{\{}\AgdaBound{v}\AgdaSymbol{\}}\AgdaSpace{}%
\AgdaSymbol{\{}\AgdaBound{w}\AgdaSymbol{\}}\AgdaSpace{}%
\AgdaSymbol{(}\AgdaInductiveConstructor{inj₁}\AgdaSpace{}%
\AgdaBound{x}\AgdaSymbol{)}\AgdaSpace{}%
\AgdaSymbol{=}\AgdaSpace{}%
\AgdaFunction{sub-inv}\AgdaSpace{}%
\AgdaBound{lt1}\AgdaSpace{}%
\AgdaBound{x}\<%
\\
\>[0]\AgdaFunction{sub-inv}\AgdaSpace{}%
\AgdaSymbol{\{}\AgdaBound{u₁₁}\AgdaSpace{}%
\AgdaOperator{\AgdaInductiveConstructor{⊔}}\AgdaSpace{}%
\AgdaBound{u₁₂}\AgdaSymbol{\}}\AgdaSpace{}%
\AgdaSymbol{\{}\AgdaBound{u₂}\AgdaSymbol{\}}\AgdaSpace{}%
\AgdaSymbol{(}\AgdaInductiveConstructor{⊑-conj-L}\AgdaSpace{}%
\AgdaBound{lt1}\AgdaSpace{}%
\AgdaBound{lt2}\AgdaSymbol{)}\AgdaSpace{}%
\AgdaSymbol{\{}\AgdaBound{v}\AgdaSymbol{\}}\AgdaSpace{}%
\AgdaSymbol{\{}\AgdaBound{w}\AgdaSymbol{\}}\AgdaSpace{}%
\AgdaSymbol{(}\AgdaInductiveConstructor{inj₂}\AgdaSpace{}%
\AgdaBound{y}\AgdaSymbol{)}\AgdaSpace{}%
\AgdaSymbol{=}\AgdaSpace{}%
\AgdaFunction{sub-inv}\AgdaSpace{}%
\AgdaBound{lt2}\AgdaSpace{}%
\AgdaBound{y}\<%
\\
\>[0]\AgdaCatchallClause{\AgdaFunction{sub-inv}}\AgdaSpace{}%
\AgdaCatchallClause{\AgdaSymbol{\{}}\AgdaCatchallClause{\AgdaBound{u₁}}\AgdaCatchallClause{\AgdaSymbol{\}}}\AgdaSpace{}%
\AgdaCatchallClause{\AgdaSymbol{\{}}\AgdaCatchallClause{\AgdaBound{u₂₁}}\AgdaSpace{}%
\AgdaCatchallClause{\AgdaOperator{\AgdaInductiveConstructor{⊔}}}\AgdaSpace{}%
\AgdaCatchallClause{\AgdaBound{u₂₂}}\AgdaCatchallClause{\AgdaSymbol{\}}}\AgdaSpace{}%
\AgdaCatchallClause{\AgdaSymbol{(}}\AgdaCatchallClause{\AgdaInductiveConstructor{⊑-conj-R1}}\AgdaSpace{}%
\AgdaCatchallClause{\AgdaBound{lt}}\AgdaCatchallClause{\AgdaSymbol{)}}\AgdaSpace{}%
\AgdaCatchallClause{\AgdaSymbol{\{}}\AgdaCatchallClause{\AgdaBound{v}}\AgdaCatchallClause{\AgdaSymbol{\}}}\AgdaSpace{}%
\AgdaCatchallClause{\AgdaSymbol{\{}}\AgdaCatchallClause{\AgdaBound{w}}\AgdaCatchallClause{\AgdaSymbol{\}}}\AgdaSpace{}%
\AgdaCatchallClause{\AgdaBound{m}}\<%
\\
\>[0][@{}l@{\AgdaIndent{0}}]%
\>[4]\AgdaKeyword{with}\AgdaSpace{}%
\AgdaFunction{sub-inv}\AgdaSpace{}%
\AgdaBound{lt}\AgdaSpace{}%
\AgdaBound{m}\<%
\\
\>[0]\AgdaSymbol{...}\AgdaSpace{}%
\AgdaSymbol{|}%
\>[2524I]\AgdaOperator{\AgdaInductiveConstructor{⟨}}\AgdaSpace{}%
\AgdaBound{u₃₁}\AgdaSpace{}%
\AgdaOperator{\AgdaInductiveConstructor{,}}\AgdaSpace{}%
\AgdaOperator{\AgdaInductiveConstructor{⟨}}\AgdaSpace{}%
\AgdaBound{fu₃₁}\AgdaSpace{}%
\AgdaOperator{\AgdaInductiveConstructor{,}}\AgdaSpace{}%
\AgdaOperator{\AgdaInductiveConstructor{⟨}}\AgdaSpace{}%
\AgdaBound{u₃₁⊆u₂₁}\AgdaSpace{}%
\AgdaOperator{\AgdaInductiveConstructor{,}}\AgdaSpace{}%
\AgdaOperator{\AgdaInductiveConstructor{⟨}}\AgdaSpace{}%
\AgdaBound{domu₃₁⊑v}\AgdaSpace{}%
\AgdaOperator{\AgdaInductiveConstructor{,}}\AgdaSpace{}%
\AgdaBound{w⊑codu₃₁}\AgdaSpace{}%
\AgdaOperator{\AgdaInductiveConstructor{⟩}}\AgdaSpace{}%
\AgdaOperator{\AgdaInductiveConstructor{⟩}}\AgdaSpace{}%
\AgdaOperator{\AgdaInductiveConstructor{⟩}}\AgdaSpace{}%
\AgdaOperator{\AgdaInductiveConstructor{⟩}}\AgdaSpace{}%
\AgdaSymbol{=}\<%
\\
\>[.][@{}l@{}]\<[2524I]%
\>[6]\AgdaOperator{\AgdaInductiveConstructor{⟨}}\AgdaSpace{}%
\AgdaBound{u₃₁}\AgdaSpace{}%
\AgdaOperator{\AgdaInductiveConstructor{,}}\AgdaSpace{}%
\AgdaOperator{\AgdaInductiveConstructor{⟨}}\AgdaSpace{}%
\AgdaBound{fu₃₁}\AgdaSpace{}%
\AgdaOperator{\AgdaInductiveConstructor{,}}\AgdaSpace{}%
\AgdaOperator{\AgdaInductiveConstructor{⟨}}\AgdaSpace{}%
\AgdaSymbol{(λ}\AgdaSpace{}%
\AgdaSymbol{\{}\AgdaBound{w}\AgdaSymbol{\}}\AgdaSpace{}%
\AgdaBound{z}%
\>[2551I]\AgdaSymbol{→}\AgdaSpace{}%
\AgdaInductiveConstructor{inj₁}\AgdaSpace{}%
\AgdaSymbol{(}\AgdaBound{u₃₁⊆u₂₁}\AgdaSpace{}%
\AgdaBound{z}\AgdaSymbol{))}\AgdaSpace{}%
\AgdaOperator{\AgdaInductiveConstructor{,}}\<%
\\
\>[2551I][@{}l@{\AgdaIndent{0}}]%
\>[35]\AgdaOperator{\AgdaInductiveConstructor{⟨}}\AgdaSpace{}%
\AgdaBound{domu₃₁⊑v}\AgdaSpace{}%
\AgdaOperator{\AgdaInductiveConstructor{,}}\AgdaSpace{}%
\AgdaBound{w⊑codu₃₁}\AgdaSpace{}%
\AgdaOperator{\AgdaInductiveConstructor{⟩}}\AgdaSpace{}%
\AgdaOperator{\AgdaInductiveConstructor{⟩}}\AgdaSpace{}%
\AgdaOperator{\AgdaInductiveConstructor{⟩}}\AgdaSpace{}%
\AgdaOperator{\AgdaInductiveConstructor{⟩}}\<%
\\
\>[0]\AgdaCatchallClause{\AgdaFunction{sub-inv}}\AgdaSpace{}%
\AgdaCatchallClause{\AgdaSymbol{\{}}\AgdaCatchallClause{\AgdaBound{u₁}}\AgdaCatchallClause{\AgdaSymbol{\}}}\AgdaSpace{}%
\AgdaCatchallClause{\AgdaSymbol{\{}}\AgdaCatchallClause{\AgdaBound{u₂₁}}\AgdaSpace{}%
\AgdaCatchallClause{\AgdaOperator{\AgdaInductiveConstructor{⊔}}}\AgdaSpace{}%
\AgdaCatchallClause{\AgdaBound{u₂₂}}\AgdaCatchallClause{\AgdaSymbol{\}}}\AgdaSpace{}%
\AgdaCatchallClause{\AgdaSymbol{(}}\AgdaCatchallClause{\AgdaInductiveConstructor{⊑-conj-R2}}\AgdaSpace{}%
\AgdaCatchallClause{\AgdaBound{lt}}\AgdaCatchallClause{\AgdaSymbol{)}}\AgdaSpace{}%
\AgdaCatchallClause{\AgdaSymbol{\{}}\AgdaCatchallClause{\AgdaBound{v}}\AgdaCatchallClause{\AgdaSymbol{\}}}\AgdaSpace{}%
\AgdaCatchallClause{\AgdaSymbol{\{}}\AgdaCatchallClause{\AgdaBound{w}}\AgdaCatchallClause{\AgdaSymbol{\}}}\AgdaSpace{}%
\AgdaCatchallClause{\AgdaBound{m}}\<%
\\
\>[0][@{}l@{\AgdaIndent{0}}]%
\>[4]\AgdaKeyword{with}\AgdaSpace{}%
\AgdaFunction{sub-inv}\AgdaSpace{}%
\AgdaBound{lt}\AgdaSpace{}%
\AgdaBound{m}\<%
\\
\>[0]\AgdaSymbol{...}\AgdaSpace{}%
\AgdaSymbol{|}%
\>[2576I]\AgdaOperator{\AgdaInductiveConstructor{⟨}}\AgdaSpace{}%
\AgdaBound{u₃₂}\AgdaSpace{}%
\AgdaOperator{\AgdaInductiveConstructor{,}}\AgdaSpace{}%
\AgdaOperator{\AgdaInductiveConstructor{⟨}}\AgdaSpace{}%
\AgdaBound{fu₃₂}\AgdaSpace{}%
\AgdaOperator{\AgdaInductiveConstructor{,}}\AgdaSpace{}%
\AgdaOperator{\AgdaInductiveConstructor{⟨}}\AgdaSpace{}%
\AgdaBound{u₃₂⊆u₂₂}\AgdaSpace{}%
\AgdaOperator{\AgdaInductiveConstructor{,}}\AgdaSpace{}%
\AgdaOperator{\AgdaInductiveConstructor{⟨}}\AgdaSpace{}%
\AgdaBound{domu₃₂⊑v}\AgdaSpace{}%
\AgdaOperator{\AgdaInductiveConstructor{,}}\AgdaSpace{}%
\AgdaBound{w⊑codu₃₂}\AgdaSpace{}%
\AgdaOperator{\AgdaInductiveConstructor{⟩}}\AgdaSpace{}%
\AgdaOperator{\AgdaInductiveConstructor{⟩}}\AgdaSpace{}%
\AgdaOperator{\AgdaInductiveConstructor{⟩}}\AgdaSpace{}%
\AgdaOperator{\AgdaInductiveConstructor{⟩}}\AgdaSpace{}%
\AgdaSymbol{=}\<%
\\
\>[.][@{}l@{}]\<[2576I]%
\>[6]\AgdaOperator{\AgdaInductiveConstructor{⟨}}\AgdaSpace{}%
\AgdaBound{u₃₂}\AgdaSpace{}%
\AgdaOperator{\AgdaInductiveConstructor{,}}\AgdaSpace{}%
\AgdaOperator{\AgdaInductiveConstructor{⟨}}\AgdaSpace{}%
\AgdaBound{fu₃₂}\AgdaSpace{}%
\AgdaOperator{\AgdaInductiveConstructor{,}}\AgdaSpace{}%
\AgdaOperator{\AgdaInductiveConstructor{⟨}}\AgdaSpace{}%
\AgdaSymbol{(λ}\AgdaSpace{}%
\AgdaSymbol{\{}\AgdaBound{C}\AgdaSymbol{\}}\AgdaSpace{}%
\AgdaBound{z}%
\>[2603I]\AgdaSymbol{→}\AgdaSpace{}%
\AgdaInductiveConstructor{inj₂}\AgdaSpace{}%
\AgdaSymbol{(}\AgdaBound{u₃₂⊆u₂₂}\AgdaSpace{}%
\AgdaBound{z}\AgdaSymbol{))}\AgdaSpace{}%
\AgdaOperator{\AgdaInductiveConstructor{,}}\<%
\\
\>[2603I][@{}l@{\AgdaIndent{0}}]%
\>[35]\AgdaOperator{\AgdaInductiveConstructor{⟨}}\AgdaSpace{}%
\AgdaBound{domu₃₂⊑v}\AgdaSpace{}%
\AgdaOperator{\AgdaInductiveConstructor{,}}\AgdaSpace{}%
\AgdaBound{w⊑codu₃₂}\AgdaSpace{}%
\AgdaOperator{\AgdaInductiveConstructor{⟩}}\AgdaSpace{}%
\AgdaOperator{\AgdaInductiveConstructor{⟩}}\AgdaSpace{}%
\AgdaOperator{\AgdaInductiveConstructor{⟩}}\AgdaSpace{}%
\AgdaOperator{\AgdaInductiveConstructor{⟩}}\<%
\\
\>[0]\AgdaCatchallClause{\AgdaFunction{sub-inv}}\AgdaSpace{}%
\AgdaCatchallClause{\AgdaSymbol{\{}}\AgdaCatchallClause{\AgdaBound{u₁}}\AgdaCatchallClause{\AgdaSymbol{\}}}\AgdaSpace{}%
\AgdaCatchallClause{\AgdaSymbol{\{}}\AgdaCatchallClause{\AgdaBound{u₂}}\AgdaCatchallClause{\AgdaSymbol{\}}}\AgdaSpace{}%
\AgdaCatchallClause{\AgdaSymbol{(}}\AgdaCatchallClause{\AgdaInductiveConstructor{⊑-trans}}\AgdaCatchallClause{\AgdaSymbol{\{}}\AgdaCatchallClause{\AgdaArgument{v}}\AgdaSpace{}%
\AgdaCatchallClause{\AgdaSymbol{=}}\AgdaSpace{}%
\AgdaCatchallClause{\AgdaBound{u}}\AgdaCatchallClause{\AgdaSymbol{\}}}\AgdaSpace{}%
\AgdaCatchallClause{\AgdaBound{u₁⊑u}}\AgdaSpace{}%
\AgdaCatchallClause{\AgdaBound{u⊑u₂}}\AgdaCatchallClause{\AgdaSymbol{)}}\AgdaSpace{}%
\AgdaCatchallClause{\AgdaSymbol{\{}}\AgdaCatchallClause{\AgdaBound{v}}\AgdaCatchallClause{\AgdaSymbol{\}}}\AgdaSpace{}%
\AgdaCatchallClause{\AgdaSymbol{\{}}\AgdaCatchallClause{\AgdaBound{w}}\AgdaCatchallClause{\AgdaSymbol{\}}}\AgdaSpace{}%
\AgdaCatchallClause{\AgdaBound{v↦w∈u₁}}\<%
\\
\>[0][@{}l@{\AgdaIndent{0}}]%
\>[4]\AgdaKeyword{with}\AgdaSpace{}%
\AgdaFunction{sub-inv}\AgdaSpace{}%
\AgdaBound{u₁⊑u}\AgdaSpace{}%
\AgdaBound{v↦w∈u₁}\<%
\\
\>[0]\AgdaSymbol{...}%
\>[2628I]\AgdaSymbol{|}\AgdaSpace{}%
\AgdaOperator{\AgdaInductiveConstructor{⟨}}\AgdaSpace{}%
\AgdaBound{u′}\AgdaSpace{}%
\AgdaOperator{\AgdaInductiveConstructor{,}}\AgdaSpace{}%
\AgdaOperator{\AgdaInductiveConstructor{⟨}}\AgdaSpace{}%
\AgdaBound{fu′}\AgdaSpace{}%
\AgdaOperator{\AgdaInductiveConstructor{,}}\AgdaSpace{}%
\AgdaOperator{\AgdaInductiveConstructor{⟨}}\AgdaSpace{}%
\AgdaBound{u′⊆u}\AgdaSpace{}%
\AgdaOperator{\AgdaInductiveConstructor{,}}\AgdaSpace{}%
\AgdaOperator{\AgdaInductiveConstructor{⟨}}\AgdaSpace{}%
\AgdaBound{domu′⊑v}\AgdaSpace{}%
\AgdaOperator{\AgdaInductiveConstructor{,}}\AgdaSpace{}%
\AgdaBound{w⊑codu′}\AgdaSpace{}%
\AgdaOperator{\AgdaInductiveConstructor{⟩}}\AgdaSpace{}%
\AgdaOperator{\AgdaInductiveConstructor{⟩}}\AgdaSpace{}%
\AgdaOperator{\AgdaInductiveConstructor{⟩}}\AgdaSpace{}%
\AgdaOperator{\AgdaInductiveConstructor{⟩}}\<%
\\
\>[.][@{}l@{}]\<[2628I]%
\>[4]\AgdaKeyword{with}\AgdaSpace{}%
\AgdaFunction{sub-inv-trans}\AgdaSpace{}%
\AgdaSymbol{\{}\AgdaBound{u′}\AgdaSymbol{\}}\AgdaSpace{}%
\AgdaBound{fu′}\AgdaSpace{}%
\AgdaBound{u′⊆u}\AgdaSpace{}%
\AgdaSymbol{(}\AgdaFunction{sub-inv}\AgdaSpace{}%
\AgdaBound{u⊑u₂}\AgdaSymbol{)}\<%
\\
\>[0]\AgdaSymbol{...}\AgdaSpace{}%
\AgdaSymbol{|}%
\>[2653I]\AgdaOperator{\AgdaInductiveConstructor{⟨}}\AgdaSpace{}%
\AgdaBound{u₃}\AgdaSpace{}%
\AgdaOperator{\AgdaInductiveConstructor{,}}\AgdaSpace{}%
\AgdaOperator{\AgdaInductiveConstructor{⟨}}\AgdaSpace{}%
\AgdaBound{fu₃}\AgdaSpace{}%
\AgdaOperator{\AgdaInductiveConstructor{,}}\AgdaSpace{}%
\AgdaOperator{\AgdaInductiveConstructor{⟨}}\AgdaSpace{}%
\AgdaBound{u₃⊆u₂}\AgdaSpace{}%
\AgdaOperator{\AgdaInductiveConstructor{,}}\AgdaSpace{}%
\AgdaOperator{\AgdaInductiveConstructor{⟨}}\AgdaSpace{}%
\AgdaBound{domu₃⊑domu′}\AgdaSpace{}%
\AgdaOperator{\AgdaInductiveConstructor{,}}\AgdaSpace{}%
\AgdaBound{codu′⊑codu₃}\AgdaSpace{}%
\AgdaOperator{\AgdaInductiveConstructor{⟩}}\AgdaSpace{}%
\AgdaOperator{\AgdaInductiveConstructor{⟩}}\AgdaSpace{}%
\AgdaOperator{\AgdaInductiveConstructor{⟩}}\AgdaSpace{}%
\AgdaOperator{\AgdaInductiveConstructor{⟩}}\AgdaSpace{}%
\AgdaSymbol{=}\<%
\\
\>[.][@{}l@{}]\<[2653I]%
\>[6]\AgdaOperator{\AgdaInductiveConstructor{⟨}}\AgdaSpace{}%
\AgdaBound{u₃}\AgdaSpace{}%
\AgdaOperator{\AgdaInductiveConstructor{,}}\AgdaSpace{}%
\AgdaOperator{\AgdaInductiveConstructor{⟨}}\AgdaSpace{}%
\AgdaBound{fu₃}\AgdaSpace{}%
\AgdaOperator{\AgdaInductiveConstructor{,}}\AgdaSpace{}%
\AgdaOperator{\AgdaInductiveConstructor{⟨}}\AgdaSpace{}%
\AgdaBound{u₃⊆u₂}\AgdaSpace{}%
\AgdaOperator{\AgdaInductiveConstructor{,}}\AgdaSpace{}%
\AgdaOperator{\AgdaInductiveConstructor{⟨}}%
\>[2680I]\AgdaInductiveConstructor{⊑-trans}\AgdaSpace{}%
\AgdaBound{domu₃⊑domu′}\AgdaSpace{}%
\AgdaBound{domu′⊑v}\AgdaSpace{}%
\AgdaOperator{\AgdaInductiveConstructor{,}}\<%
\\
\>[2680I][@{}l@{\AgdaIndent{0}}]%
\>[36]\AgdaInductiveConstructor{⊑-trans}\AgdaSpace{}%
\AgdaBound{w⊑codu′}\AgdaSpace{}%
\AgdaBound{codu′⊑codu₃}\AgdaSpace{}%
\AgdaOperator{\AgdaInductiveConstructor{⟩}}\AgdaSpace{}%
\AgdaOperator{\AgdaInductiveConstructor{⟩}}\AgdaSpace{}%
\AgdaOperator{\AgdaInductiveConstructor{⟩}}\AgdaSpace{}%
\AgdaOperator{\AgdaInductiveConstructor{⟩}}\<%
\\
\>[0]\AgdaFunction{sub-inv}\AgdaSpace{}%
\AgdaSymbol{\{}\AgdaBound{u₁₁}\AgdaSpace{}%
\AgdaOperator{\AgdaInductiveConstructor{↦}}\AgdaSpace{}%
\AgdaBound{u₁₂}\AgdaSymbol{\}}\AgdaSpace{}%
\AgdaSymbol{\{}\AgdaBound{u₂₁}\AgdaSpace{}%
\AgdaOperator{\AgdaInductiveConstructor{↦}}\AgdaSpace{}%
\AgdaBound{u₂₂}\AgdaSymbol{\}}\AgdaSpace{}%
\AgdaSymbol{(}\AgdaInductiveConstructor{⊑-fun}\AgdaSpace{}%
\AgdaBound{lt1}\AgdaSpace{}%
\AgdaBound{lt2}\AgdaSymbol{)}\AgdaSpace{}%
\AgdaInductiveConstructor{refl}\AgdaSpace{}%
\AgdaSymbol{=}\<%
\\
\>[0][@{}l@{\AgdaIndent{0}}]%
\>[4]\AgdaOperator{\AgdaInductiveConstructor{⟨}}\AgdaSpace{}%
\AgdaBound{u₂₁}\AgdaSpace{}%
\AgdaOperator{\AgdaInductiveConstructor{↦}}\AgdaSpace{}%
\AgdaBound{u₂₂}\AgdaSpace{}%
\AgdaOperator{\AgdaInductiveConstructor{,}}\AgdaSpace{}%
\AgdaOperator{\AgdaInductiveConstructor{⟨}}\AgdaSpace{}%
\AgdaSymbol{(λ}\AgdaSpace{}%
\AgdaSymbol{\{}\AgdaBound{w}\AgdaSymbol{\}}\AgdaSpace{}%
\AgdaSymbol{→}\AgdaSpace{}%
\AgdaInductiveConstructor{fun}\AgdaSymbol{)}\AgdaSpace{}%
\AgdaOperator{\AgdaInductiveConstructor{,}}\AgdaSpace{}%
\AgdaOperator{\AgdaInductiveConstructor{⟨}}\AgdaSpace{}%
\AgdaSymbol{(λ}\AgdaSpace{}%
\AgdaSymbol{\{}\AgdaBound{C}\AgdaSymbol{\}}\AgdaSpace{}%
\AgdaBound{z}\AgdaSpace{}%
\AgdaSymbol{→}\AgdaSpace{}%
\AgdaBound{z}\AgdaSymbol{)}\AgdaSpace{}%
\AgdaOperator{\AgdaInductiveConstructor{,}}\AgdaSpace{}%
\AgdaOperator{\AgdaInductiveConstructor{⟨}}\AgdaSpace{}%
\AgdaBound{lt1}\AgdaSpace{}%
\AgdaOperator{\AgdaInductiveConstructor{,}}\AgdaSpace{}%
\AgdaBound{lt2}\AgdaSpace{}%
\AgdaOperator{\AgdaInductiveConstructor{⟩}}\AgdaSpace{}%
\AgdaOperator{\AgdaInductiveConstructor{⟩}}\AgdaSpace{}%
\AgdaOperator{\AgdaInductiveConstructor{⟩}}\AgdaSpace{}%
\AgdaOperator{\AgdaInductiveConstructor{⟩}}\<%
\\
\>[0]\AgdaFunction{sub-inv}\AgdaSpace{}%
\AgdaSymbol{\{}\AgdaBound{u₂₁}\AgdaSpace{}%
\AgdaOperator{\AgdaInductiveConstructor{↦}}\AgdaSpace{}%
\AgdaSymbol{(}\AgdaBound{u₂₂}\AgdaSpace{}%
\AgdaOperator{\AgdaInductiveConstructor{⊔}}\AgdaSpace{}%
\AgdaBound{u₂₃}\AgdaSymbol{)\}}\AgdaSpace{}%
\AgdaSymbol{\{}\AgdaBound{u₂₁}\AgdaSpace{}%
\AgdaOperator{\AgdaInductiveConstructor{↦}}\AgdaSpace{}%
\AgdaBound{u₂₂}\AgdaSpace{}%
\AgdaOperator{\AgdaInductiveConstructor{⊔}}\AgdaSpace{}%
\AgdaBound{u₂₁}\AgdaSpace{}%
\AgdaOperator{\AgdaInductiveConstructor{↦}}\AgdaSpace{}%
\AgdaBound{u₂₃}\AgdaSymbol{\}}\AgdaSpace{}%
\AgdaInductiveConstructor{⊑-dist}\<%
\\
\>[0][@{}l@{\AgdaIndent{0}}]%
\>[4]\AgdaSymbol{\{}\AgdaDottedPattern{\AgdaSymbol{.}}\AgdaDottedPattern{\AgdaBound{u₂₁}}\AgdaSymbol{\}}\AgdaSpace{}%
\AgdaSymbol{\{}\AgdaDottedPattern{\AgdaSymbol{.(}}\AgdaDottedPattern{\AgdaBound{u₂₂}}\AgdaSpace{}%
\AgdaDottedPattern{\AgdaOperator{\AgdaInductiveConstructor{⊔}}}\AgdaSpace{}%
\AgdaDottedPattern{\AgdaBound{u₂₃}}\AgdaDottedPattern{\AgdaSymbol{)}}\AgdaSymbol{\}}\AgdaSpace{}%
\AgdaInductiveConstructor{refl}\AgdaSpace{}%
\AgdaSymbol{=}\<%
\\
%
\>[4]\AgdaOperator{\AgdaInductiveConstructor{⟨}}\AgdaSpace{}%
\AgdaBound{u₂₁}\AgdaSpace{}%
\AgdaOperator{\AgdaInductiveConstructor{↦}}\AgdaSpace{}%
\AgdaBound{u₂₂}\AgdaSpace{}%
\AgdaOperator{\AgdaInductiveConstructor{⊔}}\AgdaSpace{}%
\AgdaBound{u₂₁}\AgdaSpace{}%
\AgdaOperator{\AgdaInductiveConstructor{↦}}\AgdaSpace{}%
\AgdaBound{u₂₃}\AgdaSpace{}%
\AgdaOperator{\AgdaInductiveConstructor{,}}\AgdaSpace{}%
\AgdaOperator{\AgdaInductiveConstructor{⟨}}\AgdaSpace{}%
\AgdaFunction{f}\AgdaSpace{}%
\AgdaOperator{\AgdaInductiveConstructor{,}}\AgdaSpace{}%
\AgdaOperator{\AgdaInductiveConstructor{⟨}}\AgdaSpace{}%
\AgdaFunction{g}\AgdaSpace{}%
\AgdaOperator{\AgdaInductiveConstructor{,}}\AgdaSpace{}%
\AgdaOperator{\AgdaInductiveConstructor{⟨}}\AgdaSpace{}%
\AgdaInductiveConstructor{⊑-conj-L}\AgdaSpace{}%
\AgdaFunction{⊑-refl}\AgdaSpace{}%
\AgdaFunction{⊑-refl}\AgdaSpace{}%
\AgdaOperator{\AgdaInductiveConstructor{,}}\AgdaSpace{}%
\AgdaFunction{⊑-refl}\AgdaSpace{}%
\AgdaOperator{\AgdaInductiveConstructor{⟩}}\AgdaSpace{}%
\AgdaOperator{\AgdaInductiveConstructor{⟩}}\AgdaSpace{}%
\AgdaOperator{\AgdaInductiveConstructor{⟩}}\AgdaSpace{}%
\AgdaOperator{\AgdaInductiveConstructor{⟩}}\<%
\\
\>[0][@{}l@{\AgdaIndent{0}}]%
\>[2]\AgdaKeyword{where}%
\>[2768I]\AgdaFunction{f}\AgdaSpace{}%
\AgdaSymbol{:}\AgdaSpace{}%
\AgdaFunction{all-funs}\AgdaSpace{}%
\AgdaSymbol{(}\AgdaBound{u₂₁}\AgdaSpace{}%
\AgdaOperator{\AgdaInductiveConstructor{↦}}\AgdaSpace{}%
\AgdaBound{u₂₂}\AgdaSpace{}%
\AgdaOperator{\AgdaInductiveConstructor{⊔}}\AgdaSpace{}%
\AgdaBound{u₂₁}\AgdaSpace{}%
\AgdaOperator{\AgdaInductiveConstructor{↦}}\AgdaSpace{}%
\AgdaBound{u₂₃}\AgdaSymbol{)}\<%
\\
\>[.][@{}l@{}]\<[2768I]%
\>[8]\AgdaFunction{f}\AgdaSpace{}%
\AgdaSymbol{(}\AgdaInductiveConstructor{inj₁}\AgdaSpace{}%
\AgdaBound{x}\AgdaSymbol{)}\AgdaSpace{}%
\AgdaSymbol{=}\AgdaSpace{}%
\AgdaInductiveConstructor{fun}\AgdaSpace{}%
\AgdaBound{x}\<%
\\
%
\>[8]\AgdaFunction{f}\AgdaSpace{}%
\AgdaSymbol{(}\AgdaInductiveConstructor{inj₂}\AgdaSpace{}%
\AgdaBound{y}\AgdaSymbol{)}\AgdaSpace{}%
\AgdaSymbol{=}\AgdaSpace{}%
\AgdaInductiveConstructor{fun}\AgdaSpace{}%
\AgdaBound{y}\<%
\\
%
\>[8]\AgdaFunction{g}\AgdaSpace{}%
\AgdaSymbol{:}\AgdaSpace{}%
\AgdaSymbol{(}\AgdaBound{u₂₁}\AgdaSpace{}%
\AgdaOperator{\AgdaInductiveConstructor{↦}}\AgdaSpace{}%
\AgdaBound{u₂₂}\AgdaSpace{}%
\AgdaOperator{\AgdaInductiveConstructor{⊔}}\AgdaSpace{}%
\AgdaBound{u₂₁}\AgdaSpace{}%
\AgdaOperator{\AgdaInductiveConstructor{↦}}\AgdaSpace{}%
\AgdaBound{u₂₃}\AgdaSymbol{)}\AgdaSpace{}%
\AgdaOperator{\AgdaFunction{⊆}}\AgdaSpace{}%
\AgdaSymbol{(}\AgdaBound{u₂₁}\AgdaSpace{}%
\AgdaOperator{\AgdaInductiveConstructor{↦}}\AgdaSpace{}%
\AgdaBound{u₂₂}\AgdaSpace{}%
\AgdaOperator{\AgdaInductiveConstructor{⊔}}\AgdaSpace{}%
\AgdaBound{u₂₁}\AgdaSpace{}%
\AgdaOperator{\AgdaInductiveConstructor{↦}}\AgdaSpace{}%
\AgdaBound{u₂₃}\AgdaSymbol{)}\<%
\\
%
\>[8]\AgdaFunction{g}\AgdaSpace{}%
\AgdaSymbol{(}\AgdaInductiveConstructor{inj₁}\AgdaSpace{}%
\AgdaBound{x}\AgdaSymbol{)}\AgdaSpace{}%
\AgdaSymbol{=}\AgdaSpace{}%
\AgdaInductiveConstructor{inj₁}\AgdaSpace{}%
\AgdaBound{x}\<%
\\
%
\>[8]\AgdaFunction{g}\AgdaSpace{}%
\AgdaSymbol{(}\AgdaInductiveConstructor{inj₂}\AgdaSpace{}%
\AgdaBound{y}\AgdaSymbol{)}\AgdaSpace{}%
\AgdaSymbol{=}\AgdaSpace{}%
\AgdaInductiveConstructor{inj₂}\AgdaSpace{}%
\AgdaBound{y}\<%
\end{code}
\end{fence}

Let \texttt{v} and \texttt{w} be arbitrary values.

\begin{itemize}
\item
  Case \texttt{⊑-bot}. So \texttt{u₁\ ≡\ ⊥}. We have
  \texttt{v\ ↦\ w\ ∈\ ⊥}, but that is impossible.
\item
  Case \texttt{⊑-conj-L}.

  \begin{myDisplay}
    u₁₁ ⊑ u₂   u₁₂ ⊑ u₂
    -------------------
    u₁₁ ⊔ u₁₂ ⊑ u₂
  \end{myDisplay}

  Given that \texttt{v\ ↦\ w\ ∈\ u₁₁\ ⊔\ u₁₂}, there are two subcases to
  consider.

  \begin{itemize}
  \item
    Subcase \texttt{v\ ↦\ w\ ∈\ u₁₁}. We conclude by the induction
    hypothesis for \texttt{u₁₁\ ⊑\ u₂}.
  \item
    Subcase \texttt{v\ ↦\ w\ ∈\ u₁₂}. We conclude by the induction
    hypothesis for \texttt{u₁₂\ ⊑\ u₂}.
  \end{itemize}
\item
  Case \texttt{⊑-conj-R1}.

  \begin{myDisplay}
    u₁ ⊑ u₂₁
    --------------
    u₁ ⊑ u₂₁ ⊔ u₂₂
  \end{myDisplay}

  Given that \texttt{v\ ↦\ w\ ∈\ u₁}, the induction hypothesis for
  \texttt{u₁\ ⊑\ u₂₁} gives us that \texttt{v\ ↦\ w} factors
  \texttt{u₂₁} into \texttt{u₃₁} for some \texttt{u₃₁}. To show that
  \texttt{v\ ↦\ w} also factors \texttt{u₂₁\ ⊔\ u₂₂} into \texttt{u₃₁},
  we just need to show that \texttt{u₃₁\ ⊆\ u₂₁\ ⊔\ u₂₂}, but that
  follows directly from \texttt{u₃₁\ ⊆\ u₂₁}.
\item
  Case \texttt{⊑-conj-R2}. This case follows by reasoning similar to the
  case for \texttt{⊑-conj-R1}.
\item
  Case \texttt{⊑-trans}.

  \begin{myDisplay}
    u₁ ⊑ u   u ⊑ u₂
    ---------------
        u₁ ⊑ u₂
  \end{myDisplay}

  By the induction hypothesis for \texttt{u₁\ ⊑\ u}, we know that
  \texttt{v\ ↦\ w} factors \texttt{u} into \texttt{u′}, for some value
  \texttt{u′}, so we have \texttt{all-funs\ u′} and \texttt{u′\ ⊆\ u}.
  By the induction hypothesis for \texttt{u\ ⊑\ u₂}, we know that for
  any \texttt{v′\ ↦\ w′\ ∈\ u}, \texttt{v′\ ↦\ w′} factors \texttt{u₂}.
  Now we apply the lemma sub-inv-trans, which gives us some \texttt{u₃}
  such that \texttt{(⨆dom\ u′)\ ↦\ (⨆cod\ u′)} factors \texttt{u₂} into
  \texttt{u₃}. We show that \texttt{v\ ↦\ w} also factors \texttt{u₂}
  into \texttt{u₃}. From \texttt{⨆dom\ u₃\ ⊑\ ⨆dom\ u′} and
  \texttt{⨆dom\ u′\ ⊑\ v}, we have \texttt{⨆dom\ u₃\ ⊑\ v}. From
  \texttt{w\ ⊑\ ⨆cod\ u′} and \texttt{⨆cod\ u′\ ⊑\ ⨆cod\ u₃}, we have
  \texttt{w\ ⊑\ ⨆cod\ u₃}, and this case is complete.
\item
  Case \texttt{⊑-fun}.

  \begin{myDisplay}
    u₂₁ ⊑ u₁₁  u₁₂ ⊑ u₂₂
    ---------------------
    u₁₁ ↦ u₁₂ ⊑ u₂₁ ↦ u₂₂
  \end{myDisplay}

  Given that \texttt{v\ ↦\ w\ ∈\ u₁₁\ ↦\ u₁₂}, we have
  \texttt{v\ ≡\ u₁₁} and \texttt{w\ ≡\ u₁₂}. We show that
  \texttt{u₁₁\ ↦\ u₁₂} factors \texttt{u₂₁\ ↦\ u₂₂} into itself. We need
  to show that \texttt{⨆dom\ (u₂₁\ ↦\ u₂₂)\ ⊑\ u₁₁} and
  \texttt{u₁₂\ ⊑\ ⨆cod\ (u₂₁\ ↦\ u₂₂)}, but that is equivalent to our
  premises \texttt{u₂₁\ ⊑\ u₁₁} and \texttt{u₁₂\ ⊑\ u₂₂}.
\item
  Case \texttt{⊑-dist}.

  \begin{myDisplay}
    ---------------------------------------------
    u₂₁ ↦ (u₂₂ ⊔ u₂₃) ⊑ (u₂₁ ↦ u₂₂) ⊔ (u₂₁ ↦ u₂₃)
  \end{myDisplay}

  Given that \texttt{v\ ↦\ w\ ∈\ u₂₁\ ↦\ (u₂₂\ ⊔\ u₂₃)}, we have
  \texttt{v\ ≡\ u₂₁} and \texttt{w\ ≡\ u₂₂\ ⊔\ u₂₃}. We show that
  \texttt{u₂₁\ ↦\ (u₂₂\ ⊔\ u₂₃)} factors
  \texttt{(u₂₁\ ↦\ u₂₂)\ ⊔\ (u₂₁\ ↦\ u₂₃)} into itself. We have
  \texttt{u₂₁\ ⊔\ u₂₁\ ⊑\ u₂₁}, and also
  \texttt{u₂₂\ ⊔\ u₂₃\ ⊑\ u₂₂\ ⊔\ u₂₃}, so the proof is complete.
\end{itemize}

We conclude this section with two corollaries of the sub-inv lemma.
First, we have the following property that is convenient to use in later
proofs. We specialize the premise to just \texttt{v\ ↦\ w\ ⊑\ u₁} and we
modify the conclusion to say that for every \texttt{v′\ ↦\ w′\ ∈\ u₂},
we have \texttt{v′\ ⊑\ v}.

\begin{fence}
\begin{code}%
\>[0]\AgdaFunction{sub-inv-fun}\AgdaSpace{}%
\AgdaSymbol{:}\AgdaSpace{}%
\AgdaSymbol{∀\{}\AgdaBound{v}\AgdaSpace{}%
\AgdaBound{w}\AgdaSpace{}%
\AgdaBound{u₁}\AgdaSpace{}%
\AgdaSymbol{:}\AgdaSpace{}%
\AgdaDatatype{Value}\AgdaSymbol{\}}\<%
\\
\>[0][@{}l@{\AgdaIndent{0}}]%
\>[4]\AgdaSymbol{→}%
\>[2820I]\AgdaSymbol{(}\AgdaBound{v}\AgdaSpace{}%
\AgdaOperator{\AgdaInductiveConstructor{↦}}\AgdaSpace{}%
\AgdaBound{w}\AgdaSymbol{)}\AgdaSpace{}%
\AgdaOperator{\AgdaDatatype{⊑}}\AgdaSpace{}%
\AgdaBound{u₁}\<%
\\
\>[.][@{}l@{}]\<[2820I]%
\>[6]\AgdaComment{-----------------------------------------------------}\<%
\\
%
\>[4]\AgdaSymbol{→}%
\>[2825I]\AgdaFunction{Σ[}\AgdaSpace{}%
\AgdaBound{u₂}\AgdaSpace{}%
\AgdaFunction{∈}\AgdaSpace{}%
\AgdaDatatype{Value}\AgdaSpace{}%
\AgdaFunction{]}\AgdaSpace{}%
\AgdaFunction{all-funs}\AgdaSpace{}%
\AgdaBound{u₂}\AgdaSpace{}%
\AgdaOperator{\AgdaFunction{×}}\AgdaSpace{}%
\AgdaBound{u₂}\AgdaSpace{}%
\AgdaOperator{\AgdaFunction{⊆}}\AgdaSpace{}%
\AgdaBound{u₁}\<%
\\
\>[2825I][@{}l@{\AgdaIndent{0}}]%
\>[8]\AgdaOperator{\AgdaFunction{×}}\AgdaSpace{}%
\AgdaSymbol{(∀\{}\AgdaBound{v′}\AgdaSpace{}%
\AgdaBound{w′}\AgdaSymbol{\}}\AgdaSpace{}%
\AgdaSymbol{→}\AgdaSpace{}%
\AgdaSymbol{(}\AgdaBound{v′}\AgdaSpace{}%
\AgdaOperator{\AgdaInductiveConstructor{↦}}\AgdaSpace{}%
\AgdaBound{w′}\AgdaSymbol{)}\AgdaSpace{}%
\AgdaOperator{\AgdaFunction{∈}}\AgdaSpace{}%
\AgdaBound{u₂}\AgdaSpace{}%
\AgdaSymbol{→}\AgdaSpace{}%
\AgdaBound{v′}\AgdaSpace{}%
\AgdaOperator{\AgdaDatatype{⊑}}\AgdaSpace{}%
\AgdaBound{v}\AgdaSymbol{)}\AgdaSpace{}%
\AgdaOperator{\AgdaFunction{×}}\AgdaSpace{}%
\AgdaBound{w}\AgdaSpace{}%
\AgdaOperator{\AgdaDatatype{⊑}}\AgdaSpace{}%
\AgdaFunction{⨆cod}\AgdaSpace{}%
\AgdaBound{u₂}\<%
\\
\>[0]\AgdaFunction{sub-inv-fun}\AgdaSymbol{\{}\AgdaBound{v}\AgdaSymbol{\}\{}\AgdaBound{w}\AgdaSymbol{\}\{}\AgdaBound{u₁}\AgdaSymbol{\}}\AgdaSpace{}%
\AgdaBound{abc}\<%
\\
\>[0][@{}l@{\AgdaIndent{0}}]%
\>[4]\AgdaKeyword{with}\AgdaSpace{}%
\AgdaFunction{sub-inv}\AgdaSpace{}%
\AgdaBound{abc}\AgdaSpace{}%
\AgdaSymbol{\{}\AgdaBound{v}\AgdaSymbol{\}\{}\AgdaBound{w}\AgdaSymbol{\}}\AgdaSpace{}%
\AgdaInductiveConstructor{refl}\<%
\\
\>[0]\AgdaSymbol{...}\AgdaSpace{}%
\AgdaSymbol{|}%
\>[2859I]\AgdaOperator{\AgdaInductiveConstructor{⟨}}\AgdaSpace{}%
\AgdaBound{u₂}\AgdaSpace{}%
\AgdaOperator{\AgdaInductiveConstructor{,}}\AgdaSpace{}%
\AgdaOperator{\AgdaInductiveConstructor{⟨}}\AgdaSpace{}%
\AgdaBound{f}\AgdaSpace{}%
\AgdaOperator{\AgdaInductiveConstructor{,}}\AgdaSpace{}%
\AgdaOperator{\AgdaInductiveConstructor{⟨}}\AgdaSpace{}%
\AgdaBound{u₂⊆u₁}\AgdaSpace{}%
\AgdaOperator{\AgdaInductiveConstructor{,}}\AgdaSpace{}%
\AgdaOperator{\AgdaInductiveConstructor{⟨}}\AgdaSpace{}%
\AgdaBound{db}\AgdaSpace{}%
\AgdaOperator{\AgdaInductiveConstructor{,}}\AgdaSpace{}%
\AgdaBound{cc}\AgdaSpace{}%
\AgdaOperator{\AgdaInductiveConstructor{⟩}}\AgdaSpace{}%
\AgdaOperator{\AgdaInductiveConstructor{⟩}}\AgdaSpace{}%
\AgdaOperator{\AgdaInductiveConstructor{⟩}}\AgdaSpace{}%
\AgdaOperator{\AgdaInductiveConstructor{⟩}}\AgdaSpace{}%
\AgdaSymbol{=}\<%
\\
\>[.][@{}l@{}]\<[2859I]%
\>[6]\AgdaOperator{\AgdaInductiveConstructor{⟨}}\AgdaSpace{}%
\AgdaBound{u₂}\AgdaSpace{}%
\AgdaOperator{\AgdaInductiveConstructor{,}}\AgdaSpace{}%
\AgdaOperator{\AgdaInductiveConstructor{⟨}}\AgdaSpace{}%
\AgdaBound{f}\AgdaSpace{}%
\AgdaOperator{\AgdaInductiveConstructor{,}}\AgdaSpace{}%
\AgdaOperator{\AgdaInductiveConstructor{⟨}}\AgdaSpace{}%
\AgdaBound{u₂⊆u₁}\AgdaSpace{}%
\AgdaOperator{\AgdaInductiveConstructor{,}}\AgdaSpace{}%
\AgdaOperator{\AgdaInductiveConstructor{⟨}}\AgdaSpace{}%
\AgdaFunction{G}\AgdaSpace{}%
\AgdaOperator{\AgdaInductiveConstructor{,}}\AgdaSpace{}%
\AgdaBound{cc}\AgdaSpace{}%
\AgdaOperator{\AgdaInductiveConstructor{⟩}}\AgdaSpace{}%
\AgdaOperator{\AgdaInductiveConstructor{⟩}}\AgdaSpace{}%
\AgdaOperator{\AgdaInductiveConstructor{⟩}}\AgdaSpace{}%
\AgdaOperator{\AgdaInductiveConstructor{⟩}}\<%
\\
\>[0][@{}l@{\AgdaIndent{0}}]%
\>[3]\AgdaKeyword{where}%
\>[2893I]\AgdaFunction{G}\AgdaSpace{}%
\AgdaSymbol{:}\AgdaSpace{}%
\AgdaSymbol{∀\{}\AgdaBound{D}\AgdaSpace{}%
\AgdaBound{E}\AgdaSymbol{\}}\AgdaSpace{}%
\AgdaSymbol{→}\AgdaSpace{}%
\AgdaSymbol{(}\AgdaBound{D}\AgdaSpace{}%
\AgdaOperator{\AgdaInductiveConstructor{↦}}\AgdaSpace{}%
\AgdaBound{E}\AgdaSymbol{)}\AgdaSpace{}%
\AgdaOperator{\AgdaFunction{∈}}\AgdaSpace{}%
\AgdaBound{u₂}\AgdaSpace{}%
\AgdaSymbol{→}\AgdaSpace{}%
\AgdaBound{D}\AgdaSpace{}%
\AgdaOperator{\AgdaDatatype{⊑}}\AgdaSpace{}%
\AgdaBound{v}\<%
\\
\>[.][@{}l@{}]\<[2893I]%
\>[9]\AgdaFunction{G}\AgdaSymbol{\{}\AgdaBound{D}\AgdaSymbol{\}\{}\AgdaBound{E}\AgdaSymbol{\}}\AgdaSpace{}%
\AgdaBound{m}\AgdaSpace{}%
\AgdaSymbol{=}\AgdaSpace{}%
\AgdaInductiveConstructor{⊑-trans}\AgdaSpace{}%
\AgdaSymbol{(}\AgdaFunction{⊆→⊑}\AgdaSpace{}%
\AgdaSymbol{(}\AgdaFunction{↦∈→⊆⨆dom}\AgdaSpace{}%
\AgdaBound{f}\AgdaSpace{}%
\AgdaBound{m}\AgdaSymbol{))}\AgdaSpace{}%
\AgdaBound{db}\<%
\end{code}
\end{fence}

The second corollary is the inversion rule that one would expect for
less-than with functions on the left and right-hand sides.

\begin{fence}
\begin{code}%
\>[0]\AgdaFunction{↦⊑↦-inv}%
\>[2915I]\AgdaSymbol{:}\AgdaSpace{}%
\AgdaSymbol{∀\{}\AgdaBound{v}\AgdaSpace{}%
\AgdaBound{w}\AgdaSpace{}%
\AgdaBound{v′}\AgdaSpace{}%
\AgdaBound{w′}\AgdaSymbol{\}}\<%
\\
\>[.][@{}l@{}]\<[2915I]%
\>[8]\AgdaSymbol{→}%
\>[2920I]\AgdaBound{v}\AgdaSpace{}%
\AgdaOperator{\AgdaInductiveConstructor{↦}}\AgdaSpace{}%
\AgdaBound{w}\AgdaSpace{}%
\AgdaOperator{\AgdaDatatype{⊑}}\AgdaSpace{}%
\AgdaBound{v′}\AgdaSpace{}%
\AgdaOperator{\AgdaInductiveConstructor{↦}}\AgdaSpace{}%
\AgdaBound{w′}\<%
\\
\>[.][@{}l@{}]\<[2920I]%
\>[10]\AgdaComment{-----------------}\<%
\\
%
\>[8]\AgdaSymbol{→}\AgdaSpace{}%
\AgdaBound{v′}\AgdaSpace{}%
\AgdaOperator{\AgdaDatatype{⊑}}\AgdaSpace{}%
\AgdaBound{v}\AgdaSpace{}%
\AgdaOperator{\AgdaFunction{×}}\AgdaSpace{}%
\AgdaBound{w}\AgdaSpace{}%
\AgdaOperator{\AgdaDatatype{⊑}}\AgdaSpace{}%
\AgdaBound{w′}\<%
\\
\>[0]\AgdaFunction{↦⊑↦-inv}\AgdaSymbol{\{}\AgdaBound{v}\AgdaSymbol{\}\{}\AgdaBound{w}\AgdaSymbol{\}\{}\AgdaBound{v′}\AgdaSymbol{\}\{}\AgdaBound{w′}\AgdaSymbol{\}}\AgdaSpace{}%
\AgdaBound{lt}\<%
\\
\>[0][@{}l@{\AgdaIndent{0}}]%
\>[4]\AgdaKeyword{with}\AgdaSpace{}%
\AgdaFunction{sub-inv-fun}\AgdaSpace{}%
\AgdaBound{lt}\<%
\\
\>[0]\AgdaSymbol{...}%
\>[2937I]\AgdaSymbol{|}\AgdaSpace{}%
\AgdaOperator{\AgdaInductiveConstructor{⟨}}\AgdaSpace{}%
\AgdaBound{Γ}\AgdaSpace{}%
\AgdaOperator{\AgdaInductiveConstructor{,}}\AgdaSpace{}%
\AgdaOperator{\AgdaInductiveConstructor{⟨}}\AgdaSpace{}%
\AgdaBound{f}\AgdaSpace{}%
\AgdaOperator{\AgdaInductiveConstructor{,}}\AgdaSpace{}%
\AgdaOperator{\AgdaInductiveConstructor{⟨}}\AgdaSpace{}%
\AgdaBound{Γ⊆v34}\AgdaSpace{}%
\AgdaOperator{\AgdaInductiveConstructor{,}}\AgdaSpace{}%
\AgdaOperator{\AgdaInductiveConstructor{⟨}}\AgdaSpace{}%
\AgdaBound{lt1}\AgdaSpace{}%
\AgdaOperator{\AgdaInductiveConstructor{,}}\AgdaSpace{}%
\AgdaBound{lt2}\AgdaSpace{}%
\AgdaOperator{\AgdaInductiveConstructor{⟩}}\AgdaSpace{}%
\AgdaOperator{\AgdaInductiveConstructor{⟩}}\AgdaSpace{}%
\AgdaOperator{\AgdaInductiveConstructor{⟩}}\AgdaSpace{}%
\AgdaOperator{\AgdaInductiveConstructor{⟩}}\<%
\\
\>[.][@{}l@{}]\<[2937I]%
\>[4]\AgdaKeyword{with}\AgdaSpace{}%
\AgdaFunction{all-funs∈}\AgdaSpace{}%
\AgdaBound{f}\<%
\\
\>[0]\AgdaSymbol{...}%
\>[2957I]\AgdaSymbol{|}\AgdaSpace{}%
\AgdaOperator{\AgdaInductiveConstructor{⟨}}\AgdaSpace{}%
\AgdaBound{u}\AgdaSpace{}%
\AgdaOperator{\AgdaInductiveConstructor{,}}\AgdaSpace{}%
\AgdaOperator{\AgdaInductiveConstructor{⟨}}\AgdaSpace{}%
\AgdaBound{u′}\AgdaSpace{}%
\AgdaOperator{\AgdaInductiveConstructor{,}}\AgdaSpace{}%
\AgdaBound{u↦u′∈Γ}\AgdaSpace{}%
\AgdaOperator{\AgdaInductiveConstructor{⟩}}\AgdaSpace{}%
\AgdaOperator{\AgdaInductiveConstructor{⟩}}\<%
\\
\>[.][@{}l@{}]\<[2957I]%
\>[4]\AgdaKeyword{with}\AgdaSpace{}%
\AgdaBound{Γ⊆v34}\AgdaSpace{}%
\AgdaBound{u↦u′∈Γ}\<%
\\
\>[0]\AgdaSymbol{...}\AgdaSpace{}%
\AgdaSymbol{|}\AgdaSpace{}%
\AgdaInductiveConstructor{refl}\AgdaSpace{}%
\AgdaSymbol{=}\<%
\\
\>[0][@{}l@{\AgdaIndent{0}}]%
\>[2]\AgdaKeyword{let}\AgdaSpace{}%
\AgdaBound{⨆codΓ⊆w′}\AgdaSpace{}%
\AgdaSymbol{=}\AgdaSpace{}%
\AgdaFunction{⊆↦→⨆cod⊆}\AgdaSpace{}%
\AgdaBound{Γ⊆v34}\AgdaSpace{}%
\AgdaKeyword{in}\<%
\\
%
\>[2]\AgdaOperator{\AgdaInductiveConstructor{⟨}}\AgdaSpace{}%
\AgdaBound{lt1}\AgdaSpace{}%
\AgdaBound{u↦u′∈Γ}\AgdaSpace{}%
\AgdaOperator{\AgdaInductiveConstructor{,}}\AgdaSpace{}%
\AgdaInductiveConstructor{⊑-trans}\AgdaSpace{}%
\AgdaBound{lt2}\AgdaSpace{}%
\AgdaSymbol{(}\AgdaFunction{⊆→⊑}\AgdaSpace{}%
\AgdaBound{⨆codΓ⊆w′}\AgdaSymbol{)}\AgdaSpace{}%
\AgdaOperator{\AgdaInductiveConstructor{⟩}}\<%
\end{code}
\end{fence}

\hypertarget{notes}{%
\section{Notes}\label{notes}}

The denotational semantics presented in this chapter is an example of a
\emph{filter model} (Barendregt, Coppo, Dezani-Ciancaglini, 1983).
Filter models use type systems with intersection types to precisely
characterize runtime behavior (Coppo, Dezani-Ciancaglini, and Salle,
1979). The notation that we use in this chapter is not that of type
systems and intersection types, but the \texttt{Value} data type is
isomorphic to types (\texttt{↦} is \texttt{→}, \texttt{⊔} is \texttt{∧},
\texttt{⊥} is \texttt{⊤}), the \texttt{⊑} relation is the inverse of
subtyping \texttt{\textless{}:}, and the evaluation relation
\texttt{ρ\ ⊢\ M\ ↓\ v} is isomorphic to a type system. Write \texttt{Γ}
instead of \texttt{ρ}, \texttt{A} instead of \texttt{v}, and replace
\texttt{↓} with \texttt{:} and one has a typing judgement
\texttt{Γ\ ⊢\ M\ :\ A}. By varying the definition of subtyping and using
different choices of type atoms, intersection type systems provide
semantics for many different untyped λ calculi, from full beta to the
lazy and call-by-value calculi (Alessi, Barbanera, and
Dezani-Ciancaglini, 2006) (Rocca and Paolini, 2004). The denotational
semantics in this chapter corresponds to the BCD system (Barendregt,
Coppo, Dezani-Ciancaglini, 1983). Part 3 of the book \emph{Lambda
Calculus with Types} describes a framework for intersection type systems
that enables results similar to the ones in this chapter, but for the
entire family of intersection type systems (Barendregt, Dekkers, and
Statman, 2013).

The two ideas of using finite tables to represent functions and of
relaxing table lookup to enable self application first appeared in a
technical report by Gordon Plotkin (1972) and are later described in an
article in Theoretical Computer Science (Plotkin 1993). In that work,
the inductive definition of \texttt{Value} is a bit different than the
one we use:

\begin{myDisplay}
Value = C + ℘f(Value) × ℘f(Value)
\end{myDisplay}

where \texttt{C} is a set of constants and \texttt{℘f} means finite
powerset. The pairs in \texttt{℘f(Value)\ ×\ ℘f(Value)} represent
input-output mappings, just as in this chapter. The finite powersets are
used to enable a function table to appear in the input and in the
output. These differences amount to changing where the recursion appears
in the definition of \texttt{Value}. Plotkin's model is an example of a
\emph{graph model} of the untyped lambda calculus (Barendregt, 1984). In
a graph model, the semantics is presented as a function from programs
and environments to (possibly infinite) sets of values. The semantics in
this chapter is instead defined as a relation, but set-valued functions
are isomorphic to relations. Indeed, we present the semantics as a
function in the next chapter and prove that it is equivalent to the
relational version.

Dana Scott's ℘(ω) (1976) and Engeler's B(A) (1981) are two more examples
of graph models. Both use the following inductive definition of
\texttt{Value}.

\begin{myDisplay}
Value = C + ℘f(Value) × Value
\end{myDisplay}

The use of \texttt{Value} instead of \texttt{℘f(Value)} in the output
does not restrict expressiveness compared to Plotkin's model because the
semantics use sets of values and a pair of sets \texttt{(V,\ V′)} can be
represented as a set of pairs
\texttt{\{\ (V,\ v′)\ \textbar{}\ v′\ ∈\ V′\ \}}. In Scott's ℘(ω), the
above values are mapped to and from the natural numbers using a kind of
Godel encoding.

\hypertarget{references}{%
\section{References}\label{references}}

\begin{itemize}
\item
  Intersection Types and Lambda Models. Fabio Alessi, Franco Barbanera,
  and Mariangiola Dezani-Ciancaglini, Theoretical Compututer Science,
  vol.~355, pages 108-126, 2006.
\item
  The Lambda Calculus. H.P. Barendregt, 1984.
\item
  A filter lambda model and the completeness of type assignment. Henk
  Barendregt, Mario Coppo, and Mariangiola Dezani-Ciancaglini, Journal
  of Symbolic Logic, vol.~48, pages 931-940, 1983.
\item
  Lambda Calculus with Types. Henk Barendregt, Wil Dekkers, and Richard
  Statman, Cambridge University Press, Perspectives in Logic,

  \begin{enumerate}
  \def\labelenumi{\arabic{enumi}.}
  \setcounter{enumi}{2012}
  \tightlist
  \item
  \end{enumerate}
\item
  Functional characterization of some semantic equalities inside
  λ-calculus. Mario Coppo, Mariangiola Dezani-Ciancaglini, and Patrick
  Salle, in Sixth Colloquium on Automata, Languages and Programming.
  Springer, pages 133--146, 1979.
\item
  Algebras and combinators. Erwin Engeler, Algebra Universalis, vol.~13,
  pages 389-392, 1981.
\item
  A Set-Theoretical Definition of Application. Gordon D. Plotkin,
  University of Edinburgh, Technical Report MIP-R-95, 1972.
\item
  Set-theoretical and other elementary models of the λ-calculus. Gordon
  D. Plotkin, Theoretical Computer Science, vol.~121, pages 351-409,
  1993.
\item
  The Parametric Lambda Calculus. Simona Ronchi Della Rocca and Luca
  Paolini, Springer, 2004.
\item
  Data Types as Lattices. Dana Scott, SIAM Journal on Computing, vol.~5,
  pages 522-587, 1976.
\end{itemize}

\hypertarget{unicode}{%
\section{Unicode}\label{unicode}}

This chapter uses the following unicode:

\begin{myDisplay}
⊥  U+22A5  UP TACK (\bot)
↦  U+21A6  RIGHTWARDS ARROW FROM BAR (\mapsto)
⊔  U+2294  SQUARE CUP (\lub)
⊑  U+2291  SQUARE IMAGE OF OR EQUAL TO (\sqsubseteq)
⨆ U+2A06  N-ARY SQUARE UNION OPERATOR (\Lub)
⊢  U+22A2  RIGHT TACK (\|- or \vdash)
↓  U+2193  DOWNWARDS ARROW (\d)
ᶜ  U+1D9C  MODIFIER LETTER SMALL C (\^c)
ℰ  U+2130  SCRIPT CAPITAL E (\McE)
≃  U+2243  ASYMPTOTICALLY EQUAL TO (\~- or \simeq)
∈  U+2208  ELEMENT OF (\in)
⊆  U+2286  SUBSET OF OR EQUAL TO (\sub= or \subseteq)
\end{myDisplay}

