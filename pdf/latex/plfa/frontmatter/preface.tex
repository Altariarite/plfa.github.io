\hypertarget{Preface}{%
\chapter{Preface}\label{Preface}}

The most profound connection between logic and computation is a pun. The
doctrine of Propositions as Types asserts that a certain kind of formal
structure may be read in two ways: either as a proposition in logic or
as a type in computing. Further, a related structure may be read as
either the proof of the proposition or as a programme of the
corresponding type. Further still, simplification of proofs corresponds
to evaluation of programs.

Accordingly, the title of this book also has two readings. It may be
parsed as ``(Programming Language) Foundations in Agda'' or
``Programming (Language Foundations) in Agda'' --- the specifications we
will write in the proof assistant Agda both describe programming
languages and are themselves programmes.

The book is aimed at students in the last year of an undergraduate
honours programme or the first year of a master or doctorate degree. It
aims to teach the fundamentals of operational semantics of programming
languages, with simply-typed lambda calculus as the central example. The
textbook is written as a literate script in Agda. The hope is that using
a proof assistant will make the development more concrete and accessible
to students, and give them rapid feedback to find and correct
misapprehensions.

The book is broken into two parts. The first part, Logical Foundations,
develops the needed formalisms. The second part, Programming Language
Foundations, introduces basic methods of operational semantics.

\hypertarget{personal-remarks}{%
\section{Personal remarks}\label{personal-remarks}}

Since 2013, I have taught a course on Types and Semantics for
Programming Languages to fourth-year undergraduates and masters students
at the University of Edinburgh. An earlier version of that course was
based on Benjamin Pierce's excellent
\href{https://www.cis.upenn.edu/~bcpierce/tapl/}{TAPL}. My version was
based of Pierce's subsequent textbook,
\href{https://softwarefoundations.cis.upenn.edu/}{Software Foundations},
written in collaboration with others and based on Coq. I am convinced of
Pierce's claim that basing a course around a proof assistant aids
learning, as summarised in his ICFP Keynote,
\href{https://www.cis.upenn.edu/~bcpierce/papers/plcurriculum.pdf}{Lambda,
The Ultimate TA}.

However, after five years of experience, I have come to the conclusion
that Coq is not the best vehicle. Too much of the course needs to focus
on learning tactics for proof derivation, to the cost of learning the
fundamentals of programming language theory. Every concept has to be
learned twice: e.g., both the product data type, and the corresponding
tactics for introduction and elimination of conjunctions. The rules Coq
applies to generate induction hypotheses can sometimes seem mysterious.
While the \texttt{notation} construct permits pleasingly flexible
syntax, it can be confusing that the same concept must always be given
two names, e.g., both \texttt{subst\ N\ x\ M} and
\texttt{N\ {[}x\ :=\ M{]}}. Names of tactics are sometimes short and
sometimes long; naming conventions in the standard library can be wildly
inconsistent. \emph{Propositions as types} as a foundation of proof is
present but hidden.

I found myself keen to recast the course in Agda. In Agda, there is no
longer any need to learn about tactics: there is just dependently-typed
programming, plain and simple. Introduction is always by a constructor,
elimination is always by pattern matching. Induction is no longer a
mysterious separate concept, but corresponds to the familiar notion of
recursion. Mixfix syntax is flexible while using just one name for each
concept, e.g., substitution is \texttt{\_{[}\_:=\_{]}}. The standard
library is not perfect, but there is a fair attempt at consistency.
\emph{Propositions as types} as a foundation of proof is on proud
display.

Alas, there is no textbook for programming language theory in Agda.
Stump's
\href{https://www.morganclaypoolpublishers.com/catalog_Orig/product_info.php?cPath=24\&products_id=908}{Verified
Functional Programming in Agda} covers related ground, but focusses more
on programming with dependent types than on the theory of programming
languages.

The original goal was to simply adapt \emph{Software Foundations},
maintaining the same text but transposing the code from Coq to Agda. But
it quickly became clear to me that after five years in the classroom I
had my own ideas about how to present the material. They say you should
never write a book unless you cannot \emph{not} write the book, and I
soon found that this was a book I could not not write.

I am fortunate that my student, \href{https://github.com/wenkokke}{Wen
Kokke}, was keen to help. She guided me as a newbie to Agda and provided
an infrastructure that is easy to use and produces pages that are a
pleasure to view.

Most of the text was written during a sabbatical in the first half of
2018.

--- Philip Wadler, Rio de Janeiro, January--June 2018

\hypertarget{a-word-on-the-exercises}{%
\section{A word on the exercises}\label{a-word-on-the-exercises}}

Exercises labelled ``(recommended)'' are the ones students are required
to do in the class taught at Edinburgh from this textbook.

Exercises labelled ``(stretch)'' are there to provide an extra
challenge. Few students do all of these, but most attempt at least a
few.

Exercises labelled ``(practice)'' are included for those who want extra
practice.

You may need to import library functions required for the solution.

Please do not post answers to the exercises in a public place.

There is a private repository of answers to selected questions on
github. Please contact Philip Wadler if you would like to access it.

