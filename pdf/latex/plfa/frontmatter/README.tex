\hypertarget{GettingStarted}{%
\chapter{Getting Started}\label{GettingStarted}}

\hypertarget{dependencies-for-users}{%
\section{Dependencies for users}\label{dependencies-for-users}}

You can read PLFA \href{http://plfa.inf.ed.ac.uk}{online} without
installing anything. However, if you wish to interact with the code or
complete the exercises, you need several things:

\begin{itemize}
\tightlist
\item
  \protect\hyperlink{install-the-haskell-tool-stack}{Stack}
\item
  \protect\hyperlink{install-git}{Git}
\item
  \protect\hyperlink{install-agda-using-stack}{Agda}
\item
  \protect\hyperlink{install-plfa-and-the-agda-standard-library}{Agda
  standard library}
\item
  \protect\hyperlink{install-plfa-and-the-agda-standard-library}{PLFA}
\end{itemize}

PLFA is tested against specific versions of Agda and the standard
library, which are shown in the badges above. Agda and the standard
library change rapidly, and these changes often break PLFA, so using
older or newer versions usually causes problems.

There are several versions of Agda and its standard library online. If
you are using a package manager, like Homebrew or Debian apt, the
version of Agda available there may be out-of date. Furthermore, Agda is
under active development, so if you install the development version from
the GitHub, you might find the developers have introduced changes which
break the code here. Therefore, it's~important to have the specific
versions of Agda and the standard library shown above.

\hypertarget{on-macos-install-the-xcode-command-line-tools}{%
\subsection{On macOS: Install the XCode Command Line
Tools}\label{on-macos-install-the-xcode-command-line-tools}}

On macOS, you'll need to install the
\href{https://developer.apple.com/xcode/}{XCode Command Line Tools}. For
most versions of macOS, you can install these by running the following
command:

\begin{myDisplay}
xcode-select --install
\end{myDisplay}

\hypertarget{install-the-haskell-tool-stack}{%
\subsection{Install the Haskell Tool
Stack}\label{install-the-haskell-tool-stack}}

Agda is written in Haskell, so to install it we'll need the
\emph{Haskell Tool Stack}, or \emph{Stack} for short. Stack is a program
for managing different Haskell compilers and packages:

\begin{itemize}
\item
  \emph{On UNIX and macOS.} If your package manager has a package for
  Stack, that's~probably your easiest option. For instance, Homebrew on
  macOS and APT on Debian offer the ``haskell-stack'' package.
  Otherwise, you can follow the instructions on
  \href{https://docs.haskellstack.org/en/stable/README/}{the Stack
  website}. Usually, Stack installs binaries at
  \texttt{HOME/.local/bin}. Please ensure this is on your PATH, by
  including the following in your shell configuration, e.g., in
  \texttt{HOME/.bash\_profile}:

  \begin{myDisplay}
  export PATH="${HOME}/.local/bin:${PATH}"
  \end{myDisplay}

  Finally, ensure that you've got the latest version of Stack, by
  running:

  \begin{myDisplay}
  stack upgrade
  \end{myDisplay}
\item
  \emph{On Windows.} There is a Windows installer on
  \href{https://docs.haskellstack.org/en/stable/README/}{the Stack
  website}.
\end{itemize}

\hypertarget{install-git}{%
\subsection{Install Git}\label{install-git}}

If you do not already have Git installed, see
\href{https://git-scm.com/downloads}{the Git downloads page}.

\hypertarget{install-agda-using-stack}{%
\subsection{Install Agda using Stack}\label{install-agda-using-stack}}

The easiest way to install a \emph{specific version} of Agda is using
\href{https://docs.haskellstack.org/en/stable/README/}{Stack}. You can
get the required version of Agda from GitHub, either by cloning the
repository and switching to the correct branch, or by downloading
\href{https://github.com/agda/agda/releases/tag/v2.6.1.3}{the zip
archive}:

\begin{myDisplay}
git clone https://github.com/agda/agda.git
cd agda
git checkout v2.6.1.3
\end{myDisplay}

To install Agda, run Stack from the Agda source directory:

\begin{myDisplay}
stack install --stack-yaml stack-8.8.3.yaml
\end{myDisplay}

\emph{This step will take a long time and a lot of memory to complete.}

\hypertarget{using-an-existing-installation-of-ghc}{%
\subsubsection{Using an existing installation of
GHC}\label{using-an-existing-installation-of-ghc}}

Stack is perfectly capable of installing and managing versions of the
\href{https://www.haskell.org/ghc/}{Glasgow Haskell Compiler} for you.
However, if you already have a copy of GHC installed, and you want Stack
to use your system installation, you can pass the
\texttt{-\/-system-ghc} flag and select the appropriate
\texttt{stack-*.yaml} file. For instance, if you have GHC 8.2.2
installed, run:

\begin{myDisplay}
stack install --system-ghc --stack-yaml stack-8.2.2.yaml
\end{myDisplay}

\hypertarget{check-if-agda-was-installed-correctly}{%
\subsubsection{Check if Agda was installed
correctly}\label{check-if-agda-was-installed-correctly}}

If you'd like, you can test to see if you've installed Agda correctly.
Create a file called \texttt{hello.agda} with these lines:

\begin{myDisplay}
data Greeting : Set where
  hello : Greeting

greet : Greeting
greet = hello
\end{myDisplay}

From a command line, change to the same directory where your
\texttt{hello.agda} file is located. Then run:

\begin{myDisplay}
agda -v 2 hello.agda
\end{myDisplay}

You should see a short message like the following, but no errors:

\begin{myDisplay}
Checking hello (/path/to/hello.agda).
Finished hello.
\end{myDisplay}

\hypertarget{install-plfa-and-the-agda-standard-library}{%
\subsection{Install PLFA and the Agda standard
library}\label{install-plfa-and-the-agda-standard-library}}

You can get the latest version of Programming Language Foundations in
Agda from GitHub, either by cloning the repository, or by downloading
\href{https://github.com/plfa/plfa.github.io/archive/dev.zip}{the zip
archive}:

\begin{myDisplay}
git clone --recurse-submodules https://github.com/plfa/plfa.github.io plfa
\end{myDisplay}

PLFA ships with the required version of the Agda standard library, so if
you cloned with the \texttt{-\/-recurse-submodules} flag, you've already
got, in the \texttt{standard-library} directory!

If you forgot to add the \texttt{-\/-recurse-submodules} flag, no
worries, we can fix that!

\begin{myDisplay}
cd plfa/
git submodule update --recursive
\end{myDisplay}

If you obtained PLFA by downloading the zip archive, you can get the
required version of the Agda standard library from GitHub. You can
either clone the repository and switch to the correct branch, or you can
download the
\href{https://github.com/agda/agda-stdlib/releases/tag/v1.3}{the zip
archive}:

\begin{myDisplay}
git clone https://github.com/agda/agda-stdlib.git agda-stdlib
cd agda-stdlib
git checkout v1.3
\end{myDisplay}

Finally, we need to let Agda know where to find the Agda standard
library. You'll need the path where you installed the standard library.
Check to see that the file ``standard-library.agda-lib'' exists, and
make a note of the path to this file. You will need to create two
configuration files in \texttt{AGDA\_DIR}. On UNIX and macOS,
\texttt{AGDA\_DIR} defaults to \texttt{\textasciitilde{}/.agda}. On
Windows, \texttt{AGDA\_DIR} usually defaults to
\texttt{\%AppData\%\textbackslash{}agda}, where \texttt{\%AppData\%}
usually defaults to
\texttt{C:\textbackslash{}Users\textbackslash{}USERNAME\textbackslash{}AppData\textbackslash{}Roaming}.

\begin{itemize}
\tightlist
\item
  If the \texttt{AGDA\_DIR} directory does not already exist, create it.
\item
  In \texttt{AGDA\_DIR}, create a plain-text file called
  \texttt{libraries} containing the
  \texttt{/path/to/standard-library.agda-lib}. This lets Agda know that
  an Agda library called \texttt{standard-library} is available.
\item
  In \texttt{AGDA\_DIR}, create a plain-text file called
  \texttt{defaults} containing \emph{just} the line
  \texttt{standard-library}.
\end{itemize}

More information about placing the standard libraries is available from
\href{https://agda.readthedocs.io/en/v2.6.1.3/tools/package-system.html\#example-using-the-standard-library}{the
Library Management page} of the Agda documentation.

It is possible to set up PLFA as an Agda library as well. If you want to
complete the exercises found in the \texttt{courses} folder, or to
import modules from the book, you need to do this. To do so, add the
path to \texttt{plfa.agda-lib} to \texttt{AGDA\_DIR/libraries} and add
\texttt{plfa} to \texttt{AGDA\_DIR/defaults}, each on a line of their
own.

\hypertarget{check-if-the-agda-standard-library-was-installed-correctly}{%
\subsubsection{Check if the Agda standard library was installed
correctly}\label{check-if-the-agda-standard-library-was-installed-correctly}}

If you'd like, you can test to see if you've installed the Agda standard
library correctly. Create a file called \texttt{nats.agda} with these
lines:

\begin{myDisplay}
open import Data.Nat

ten : ℕ
ten = 10
\end{myDisplay}

(Note that the ℕ is a Unicode character, not a plain capital N. You
should be able to just copy-and-paste it from this page into your file.)

From a command line, change to the same directory where your
\texttt{nats.agda} file is located. Then run:

\begin{myDisplay}
agda -v 2 nats.agda
\end{myDisplay}

You should see a several lines describing the files which Agda loads
while checking your file, but no errors:

\begin{myDisplay}
Checking nats (/path/to/nats.agda).
Loading  Agda.Builtin.Equality (…).
…
Loading  Data.Nat (…).
Finished nats.
\end{myDisplay}

\hypertarget{setting-up-an-editor-for-agda}{%
\section{Setting up an editor for
Agda}\label{setting-up-an-editor-for-agda}}

\hypertarget{emacs}{%
\subsection{Emacs}\label{emacs}}

The recommended editor for Agda is Emacs. To install Emacs:

\begin{itemize}
\item
  \emph{On UNIX}, the version of Emacs in your repository is probably
  fine as long as it is fairly recent. There are also links to the most
  recent release on the
  \href{https://www.gnu.org/software/emacs/download.html}{GNU Emacs
  downloads page}.
\item
  \emph{On MacOS}, \href{http://aquamacs.org/}{Aquamacs} is probably the
  preferred version of Emacs, but GNU Emacs can also be installed via
  Homebrew or MacPorts. See the
  \href{https://www.gnu.org/software/emacs/download.html}{GNU Emacs
  downloads page} for instructions.
\item
  \emph{On Windows}. See the
  \href{https://www.gnu.org/software/emacs/download.html}{GNU Emacs
  downloads page} for instructions.
\end{itemize}

Make sure that you are able to open, edit, and save text files with your
installation. The \href{https://www.gnu.org/software/emacs/tour/}{tour
of Emacs} page on the GNU Emacs site describes how to access the
tutorial within your Emacs installation.

Agda ships with the editor support for Emacs built-in, so if you've
installed Agda, all you have to do to configure Emacs is run:

\begin{myDisplay}
agda-mode setup
agda-mode compile
\end{myDisplay}

If you are already an Emacs user and have customized your setup, you may
want to note the configuration which the \texttt{setup} appends to your
\texttt{.emacs} file, and integrate it with your own preferred setup.

\hypertarget{check-if-agda-mode-was-installed-correctly}{%
\subsubsection{\texorpdfstring{Check if \texttt{agda-mode} was installed
correctly}{Check if agda-mode was installed correctly}}\label{check-if-agda-mode-was-installed-correctly}}

Open the \texttt{nats.agda} file you created earlier, and load and
type-check the file by typing
\href{https://agda.readthedocs.io/en/v2.6.1.3/tools/emacs-mode.html\#notation-for-key-combinations}{\texttt{C-c\ C-l}}.

\hypertarget{auto-loading-agda-mode-in-emacs}{%
\subsubsection{\texorpdfstring{Auto-loading \texttt{agda-mode} in
Emacs}{Auto-loading agda-mode in Emacs}}\label{auto-loading-agda-mode-in-emacs}}

Since version 2.6.0, Agda has had support for literate editing with
Markdown, using the \texttt{.lagda.md} extension. One issue is that
Emacs will default to Markdown editing mode for files with a
\texttt{.md} suffix. In order to have \texttt{agda-mode} automatically
loaded whenever you open a file ending with \texttt{.agda} or
\texttt{.lagda.md}, add the following line to your Emacs configuration
file:

\begin{myDisplay}
;; auto-load agda-mode for .agda and .lagda.md
(setq auto-mode-alist
   (append
     '(("\\.agda\\'" . agda2-mode)
       ("\\.lagda.md\\'" . agda2-mode))
     auto-mode-alist))
\end{myDisplay}

If you already have settings which change your \texttt{auto-mode-alist}
in your configuration, put these \emph{after} the ones you already have
or combine them if you are comfortable with Emacs Lisp. The
configuration file for Emacs is normally located in \texttt{HOME/.emacs}
or \texttt{HOME/.emacs.d/init.el}, but Aquamacs users might need to move
their startup settings to the ``Preferences.el'' file in
\texttt{HOME/Library/Preferences/Aquamacs\ Emacs/Preferences}. For
Windows, see
\href{https://www.gnu.org/software/emacs/manual/html_node/efaq-w32/Location-of-init-file.html}{the
GNU Emacs documentation} for a description of where the Emacs
configuration is located.

\hypertarget{optional-using-the-mononoki-font-with-emacs}{%
\subsubsection{Optional: using the mononoki font with
Emacs}\label{optional-using-the-mononoki-font-with-emacs}}

Agda uses Unicode characters for many key symbols, and it is important
that the font which you use to view and edit Agda programs shows these
symbols correctly. The most important part is that the font you use has
good Unicode support, so while we recommend
\href{https://madmalik.github.io/mononoki/}{mononoki}, fonts such as
\href{https://github.com/adobe-fonts/source-code-pro}{Source Code Pro},
\href{https://dejavu-fonts.github.io/}{DejaVu Sans Mono}, and
\href{https://www.gnu.org/software/freefont/}{FreeMono} are all good
alternatives.

You can download and install mononoki directly from
\href{https://github.com/madmalik/mononoki}{GitHub}. For most systems,
installing a font is merely a matter of clicking the downloaded
\texttt{.otf} or \texttt{.ttf} file. If your package manager offers a
package for mononoki, that might be easier. For instance, Homebrew on
macOS offers the \texttt{font-mononoki} package in the
\href{https://github.com/Homebrew/homebrew-cask-fonts}{\texttt{cask-fonts}
cask}, and APT on Debian offers the
\href{https://packages.debian.org/sid/fonts/fonts-mononoki}{\texttt{fonts-mononoki}
package}. To configure Emacs to use mononoki as its default font, add
the following to the end of your Emacs configuration file:

\begin{myDisplay}
;; default to mononoki
(set-face-attribute 'default nil
                    :family "mononoki"
                    :height 120
                    :weight 'normal
                    :width  'normal)
\end{myDisplay}

\hypertarget{using-agda-mode-in-emacs}{%
\subsubsection{\texorpdfstring{Using \texttt{agda-mode} in
Emacs}{Using agda-mode in Emacs}}\label{using-agda-mode-in-emacs}}

To load and type-check the file, use
\href{https://agda.readthedocs.io/en/v2.6.1.3/tools/emacs-mode.html\#notation-for-key-combinations}{\texttt{C-c\ C-l}}.

Agda is edited interactively, using
\href{https://agda.readthedocs.io/en/v2.6.1.3/getting-started/quick-guide.html}{``holes''},
which are bits of the program that are not yet filled in. If you use a
question mark as an expression, and load the buffer using
\texttt{C-c\ C-l}, Agda replaces the question mark with a hole. There
are several things you can to while the cursor is in a hole:

\begin{itemize}
\tightlist
\item
  \texttt{C-c\ C-c}: \textbf{c}ase split (asks for variable name)
\item
  \texttt{C-c\ C-space}: fill in hole
\item
  \texttt{C-c\ C-r}: \textbf{r}efine with constructor
\item
  \texttt{C-c\ C-a}: \textbf{a}utomatically fill in hole
\item
  \texttt{C-c\ C-,}: goal type and context
\item
  \texttt{C-c\ C-.}: goal type, context, and inferred type
\end{itemize}

See
\href{https://agda.readthedocs.io/en/v2.6.1.3/tools/emacs-mode.html}{the
emacs-mode docs} for more details.

If you want to see messages beside rather than below your Agda code, you
can do the following:

\begin{itemize}
\tightlist
\item
  Open your Agda file, and load it using \texttt{C-c\ C-l};
\item
  type \texttt{C-x\ 1} to get only your Agda file showing;
\item
  type \texttt{C-x\ 3} to split the window horizontally;
\item
  move your cursor to the right-hand half of your frame;
\item
  type \texttt{C-x\ b} and switch to the buffer called ``Agda
  information''.
\end{itemize}

Now, error messages from Agda will appear next to your file, rather than
squished beneath it.

\hypertarget{entering-unicode-characters-in-emacs-with-agda-mode}{%
\subsubsection{\texorpdfstring{Entering Unicode characters in Emacs with
\texttt{agda-mode}}{Entering Unicode characters in Emacs with agda-mode}}\label{entering-unicode-characters-in-emacs-with-agda-mode}}

When you write Agda code, you will need to insert characters which are
not found on standard keyboards. Emacs ``agda-mode'' makes it easier to
do this by defining character translations: when you enter certain
sequences of ordinary characters (the kind you find on any keyboard),
Emacs will replace them in your Agda file with the corresponding special
character.

For example, we can add a comment line to one of the \texttt{.agda} test
files. Let's say we want to add a comment line that reads:

\begin{myDisplay}
{- I am excited to type ∀ and → and ≤ and ≡ !! -}
\end{myDisplay}

The first few characters are ordinary, so we would just type them as
usual\ldots{}

\begin{myDisplay}
{- I am excited to type
\end{myDisplay}

But after that last space, we do not find ∀ on the keyboard. The code
for this character is the four characters \texttt{\textbackslash{}all},
so we type those four characters, and when we finish, Emacs will replace
them with what we want\ldots{}

\begin{myDisplay}
{- I am excited to type ∀
\end{myDisplay}

We can continue with the codes for the other characters. Sometimes the
characters will change as we type them, because a prefix of our
character's code is the code of another character. This happens with the
arrow, whose code is \texttt{\textbackslash{}-\textgreater{}}. After
typing \texttt{\textbackslash{}-} we see\ldots{}

\begin{myDisplay}
{- I am excited to type ∀ and
\end{myDisplay}

\ldots because the code \texttt{\textbackslash{}-} corresponds to a
hyphen of a certain width. When we add the \texttt{\textgreater{}}, the
\texttt{­} becomes \texttt{→}! The code for \texttt{≤} is
\texttt{\textbackslash{}\textless{}=}, and the code for \texttt{≡} is
\texttt{\textbackslash{}==}.

\begin{myDisplay}
{- I am excited to type ∀ and → and ≤ and ≡
\end{myDisplay}

Finally the last few characters are ordinary again\ldots{}

\begin{myDisplay}
{- I am excited to type ∀ and → and ≤ and ≡ !! -}
\end{myDisplay}

If you're having trouble typing the Unicode characters into Emacs, the
end of each chapter should provide a list of the Unicode characters
introduced in that chapter.

Emacs with \texttt{agda-mode} offers a number of useful commands, and
two of them are especially useful when it comes to working with Unicode
characters. For a full list of supported characters, use
\texttt{agda-input-show-translations} with:

\begin{myDisplay}
M-x agda-input-show-translations
\end{myDisplay}

All the supported characters in \texttt{agda-mode} are shown.

If you want to know how you input a specific Unicode character in agda
file, move the cursor onto the character and type the following command:

\begin{myDisplay}
M-x quail-show-key
\end{myDisplay}

You'll see the key sequence of the character in mini buffer.

\hypertarget{spacemacs}{%
\subsection{Spacemacs}\label{spacemacs}}

\href{https://www.spacemacs.org/}{Spacemacs} is a ``community-driven
Emacs distribution'' with native support for both Emacs and Vim editing
styles. It comes with
\href{https://develop.spacemacs.org/layers/+lang/agda/README.html}{integration
for \texttt{agda-mode}} out of the box. All that is required is that you
turn it on.

\hypertarget{visual-studio-code}{%
\subsection{Visual Studio Code}\label{visual-studio-code}}

\href{https://code.visualstudio.com/}{Visual Studio Code} is a free
source code editor developed by Microsoft. There is
\href{https://marketplace.visualstudio.com/items?itemName=banacorn.agda-mode}{a
plugin for Agda support} available on the Visual Studio Marketplace.

\hypertarget{atom}{%
\subsection{Atom}\label{atom}}

\href{https://atom.io/}{Atom} is a free source code editor developed by
GitHub. There is \href{https://atom.io/packages/agda-mode}{a plugin for
Agda support} available on the Atom package manager.

\hypertarget{dependencies-for-developers}{%
\section{Dependencies for
developers}\label{dependencies-for-developers}}

PLFA is written in literate Agda with
\href{https://pandoc.org/MANUAL.html\#pandocs-markdown}{Pandoc
Markdown}. PLFA is available as both a website and an EPUB e-book, both
of which can be built on UNIX and macOS. Finally, to help developers
avoid common mistakes, we provide a set of Git hooks.

\hypertarget{building-the-website-and-e-book}{%
\subsection{Building the website and
e-book}\label{building-the-website-and-e-book}}

If you'd like to build the web version of PLFA locally,
\protect\hyperlink{install-the-haskell-tool-stack}{Stack} is all you
need! PLFA is built using \href{https://jaspervdj.be/hakyll/}{Hakyll},
a~Haskell library for building static websites. We've setup a Makefile
to help you run common tasks. For instance, to build PLFA, run:

\begin{myDisplay}
make build
\end{myDisplay}

If you'd like to serve PLFA locally, rebuilding the website when any of
the source files are changed, run:

\begin{myDisplay}
make watch
\end{myDisplay}

The Makefile offers more than just building and watching, it also offers
the following useful options:

\begin{myDisplay}
build                      # Build PLFA
watch                      # Build and serve PLFA, monitor for changes and rebuild
test                       # Test web version for broken links, invalid HTML, etc.
test-epub                  # Test EPUB for compliance to the EPUB3 standard
clean                      # Clean PLFA build
init                       # Setup the Git hooks (see below)
update-contributors        # Pull in new contributors from GitHub to contributors/
list                       # List all build targets
\end{myDisplay}

For completeness, the Makefile also offers the following options, but
you're unlikely to need these:

\begin{myDisplay}
legacy-versions            # Build legacy versions of PLFA
setup-install-bundler      # Install Ruby Bundler (needed for ‘legacy-versions’)
setup-install-htmlproofer  # Install HTMLProofer (needed for ‘test’ and Git hooks)
setup-check-fix-whitespace # Check if fix-whitespace is installed (needed for Git hooks)
setup-check-epubcheck      # Check if epubcheck is installed (needed for EPUB tests)
setup-check-gem            # Check if RubyGems is installed
setup-check-npm            # Check if the Node Package Manager is installed
setup-check-stack          # Check if the Haskell Tool Stack is installed
\end{myDisplay}

The \href{https://plfa.github.io/plfa.epub}{EPUB version} of the book is
built as part of the website, since it's~hosted on the website.

\hypertarget{git-hooks}{%
\subsection{Git hooks}\label{git-hooks}}

The repository comes with several Git hooks:

\begin{enumerate}
\def\labelenumi{\arabic{enumi}.}
\item
  The \href{https://github.com/agda/fix-whitespace}{fix-whitespace}
  program is run to check for whitespace violations.
\item
  The test suite is run to check if everything type checks.
\end{enumerate}

You can install these Git hooks by calling \texttt{make\ init}. You can
install \href{https://github.com/agda/fix-whitespace}{fix-whitespace} by
running:

\begin{myDisplay}
git clone https://github.com/agda/fix-whitespace
cd fix-whitespace/
stack install --stack-yaml stack-8.8.3.yaml
\end{myDisplay}

If you want Stack to use your system installation of GHC, follow the
instructions for
\protect\hyperlink{using-an-existing-installation-of-ghc}{Using an
existing installation of GHC}.

