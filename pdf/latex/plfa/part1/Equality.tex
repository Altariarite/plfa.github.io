\hypertarget{Equality}{%
\chapter{Equality: Equality and equational reasoning}\label{Equality}}

\begin{fence}
\begin{code}%
\>[0]\AgdaKeyword{module}\AgdaSpace{}%
\AgdaModule{plfa.part1.Equality}\AgdaSpace{}%
\AgdaKeyword{where}\<%
\end{code}
\end{fence}

Much of our reasoning has involved equality. Given two terms \texttt{M}
and \texttt{N}, both of type \texttt{A}, we write \texttt{M\ ≡\ N} to
assert that \texttt{M} and \texttt{N} are interchangeable. So far we
have treated equality as a primitive, here we show how to define it as
an inductive datatype.

\hypertarget{imports}{%
\section{Imports}\label{imports}}

This chapter has no imports. Every chapter in this book, and nearly
every module in the Agda standard library, imports equality. Since we
define equality here, any import would create a conflict.

\hypertarget{equality}{%
\section{Equality}\label{equality}}

We declare equality as follows:

\begin{fence}
\begin{code}%
\>[0]\AgdaKeyword{data}\AgdaSpace{}%
\AgdaOperator{\AgdaDatatype{\AgdaUnderscore{}≡\AgdaUnderscore{}}}\AgdaSpace{}%
\AgdaSymbol{\{}\AgdaBound{A}\AgdaSpace{}%
\AgdaSymbol{:}\AgdaSpace{}%
\AgdaPrimitiveType{Set}\AgdaSymbol{\}}\AgdaSpace{}%
\AgdaSymbol{(}\AgdaBound{x}\AgdaSpace{}%
\AgdaSymbol{:}\AgdaSpace{}%
\AgdaBound{A}\AgdaSymbol{)}\AgdaSpace{}%
\AgdaSymbol{:}\AgdaSpace{}%
\AgdaBound{A}\AgdaSpace{}%
\AgdaSymbol{→}\AgdaSpace{}%
\AgdaPrimitiveType{Set}\AgdaSpace{}%
\AgdaKeyword{where}\<%
\\
\>[0][@{}l@{\AgdaIndent{0}}]%
\>[2]\AgdaInductiveConstructor{refl}\AgdaSpace{}%
\AgdaSymbol{:}\AgdaSpace{}%
\AgdaBound{x}\AgdaSpace{}%
\AgdaOperator{\AgdaDatatype{≡}}\AgdaSpace{}%
\AgdaBound{x}\<%
\end{code}
\end{fence}

In other words, for any type \texttt{A} and for any \texttt{x} of type
\texttt{A}, the constructor \texttt{refl} provides evidence that
\texttt{x\ ≡\ x}. Hence, every value is equal to itself, and we have no
other way of showing values equal. The definition features an asymmetry,
in that the first argument to \texttt{\_≡\_} is given by the parameter
\texttt{x\ :\ A}, while the second is given by an index in
\texttt{A\ →\ Set}. This follows our policy of using parameters wherever
possible. The first argument to \texttt{\_≡\_} can be a parameter
because it doesn't vary, while the second must be an index, so it can be
required to be equal to the first.

We declare the precedence of equality as follows:

\begin{fence}
\begin{code}%
\>[0]\AgdaKeyword{infix}\AgdaSpace{}%
\AgdaNumber{4}\AgdaSpace{}%
\AgdaOperator{\AgdaDatatype{\AgdaUnderscore{}≡\AgdaUnderscore{}}}\<%
\end{code}
\end{fence}

We set the precedence of \texttt{\_≡\_} at level 4, the same as
\texttt{\_≤\_}, which means it binds less tightly than any arithmetic
operator. It associates neither to left nor right; writing
\texttt{x\ ≡\ y\ ≡\ z} is illegal.

\hypertarget{equality-is-an-equivalence-relation}{%
\section{Equality is an equivalence
relation}\label{equality-is-an-equivalence-relation}}

An equivalence relation is one which is reflexive, symmetric, and
transitive. Reflexivity is built-in to the definition of equality, via
the constructor \texttt{refl}. It is straightforward to show symmetry:

\begin{fence}
\begin{code}%
\>[0]\AgdaFunction{sym}\AgdaSpace{}%
\AgdaSymbol{:}\AgdaSpace{}%
\AgdaSymbol{∀}\AgdaSpace{}%
\AgdaSymbol{\{}\AgdaBound{A}\AgdaSpace{}%
\AgdaSymbol{:}\AgdaSpace{}%
\AgdaPrimitiveType{Set}\AgdaSymbol{\}}\AgdaSpace{}%
\AgdaSymbol{\{}\AgdaBound{x}\AgdaSpace{}%
\AgdaBound{y}\AgdaSpace{}%
\AgdaSymbol{:}\AgdaSpace{}%
\AgdaBound{A}\AgdaSymbol{\}}\<%
\\
\>[0][@{}l@{\AgdaIndent{0}}]%
\>[2]\AgdaSymbol{→}%
\>[29I]\AgdaBound{x}\AgdaSpace{}%
\AgdaOperator{\AgdaDatatype{≡}}\AgdaSpace{}%
\AgdaBound{y}\<%
\\
\>[.][@{}l@{}]\<[29I]%
\>[4]\AgdaComment{-----}\<%
\\
%
\>[2]\AgdaSymbol{→}\AgdaSpace{}%
\AgdaBound{y}\AgdaSpace{}%
\AgdaOperator{\AgdaDatatype{≡}}\AgdaSpace{}%
\AgdaBound{x}\<%
\\
\>[0]\AgdaFunction{sym}\AgdaSpace{}%
\AgdaInductiveConstructor{refl}\AgdaSpace{}%
\AgdaSymbol{=}\AgdaSpace{}%
\AgdaInductiveConstructor{refl}\<%
\end{code}
\end{fence}

How does this proof work? The argument to \texttt{sym} has type
\texttt{x\ ≡\ y}, but on the left-hand side of the equation the argument
has been instantiated to the pattern \texttt{refl}, which requires that
\texttt{x} and \texttt{y} are the same. Hence, for the right-hand side
of the equation we need a term of type \texttt{x\ ≡\ x}, and
\texttt{refl} will do.

It is instructive to develop \texttt{sym} interactively. To start, we
supply a variable for the argument on the left, and a hole for the body
on the right:

\begin{myDisplay}
sym : ∀ {A : Set} {x y : A}
  → x ≡ y
    -----
  → y ≡ x
sym e = {! !}
\end{myDisplay}

If we go into the hole and type \texttt{C-c\ C-,} then Agda reports:

\begin{myDisplay}
Goal: .y ≡ .x
————————————————————————————————————————————————————————————
e  : .x ≡ .y
.y : .A
.x : .A
.A : Set
\end{myDisplay}

If in the hole we type \texttt{C-c\ C-c\ e} then Agda will instantiate
\texttt{e} to all possible constructors, with one equation for each.
There is only one possible constructor:

\begin{myDisplay}
sym : ∀ {A : Set} {x y : A}
  → x ≡ y
    -----
  → y ≡ x
sym refl = {! !}
\end{myDisplay}

If we go into the hole again and type \texttt{C-c\ C-,} then Agda now
reports:

\begin{myDisplay}
 Goal: .x ≡ .x
 ————————————————————————————————————————————————————————————
 .x : .A
 .A : Set
\end{myDisplay}

This is the key step---Agda has worked out that \texttt{x} and
\texttt{y} must be the same to match the pattern \texttt{refl}!

Finally, if we go back into the hole and type \texttt{C-c\ C-r} it will
instantiate the hole with the one constructor that yields a value of the
expected type:

\begin{myDisplay}
sym : ∀ {A : Set} {x y : A}
  → x ≡ y
    -----
  → y ≡ x
sym refl = refl
\end{myDisplay}

This completes the definition as given above.

Transitivity is equally straightforward:

\begin{fence}
\begin{code}%
\>[0]\AgdaFunction{trans}\AgdaSpace{}%
\AgdaSymbol{:}\AgdaSpace{}%
\AgdaSymbol{∀}\AgdaSpace{}%
\AgdaSymbol{\{}\AgdaBound{A}\AgdaSpace{}%
\AgdaSymbol{:}\AgdaSpace{}%
\AgdaPrimitiveType{Set}\AgdaSymbol{\}}\AgdaSpace{}%
\AgdaSymbol{\{}\AgdaBound{x}\AgdaSpace{}%
\AgdaBound{y}\AgdaSpace{}%
\AgdaBound{z}\AgdaSpace{}%
\AgdaSymbol{:}\AgdaSpace{}%
\AgdaBound{A}\AgdaSymbol{\}}\<%
\\
\>[0][@{}l@{\AgdaIndent{0}}]%
\>[2]\AgdaSymbol{→}\AgdaSpace{}%
\AgdaBound{x}\AgdaSpace{}%
\AgdaOperator{\AgdaDatatype{≡}}\AgdaSpace{}%
\AgdaBound{y}\<%
\\
%
\>[2]\AgdaSymbol{→}%
\>[51I]\AgdaBound{y}\AgdaSpace{}%
\AgdaOperator{\AgdaDatatype{≡}}\AgdaSpace{}%
\AgdaBound{z}\<%
\\
\>[.][@{}l@{}]\<[51I]%
\>[4]\AgdaComment{-----}\<%
\\
%
\>[2]\AgdaSymbol{→}\AgdaSpace{}%
\AgdaBound{x}\AgdaSpace{}%
\AgdaOperator{\AgdaDatatype{≡}}\AgdaSpace{}%
\AgdaBound{z}\<%
\\
\>[0]\AgdaFunction{trans}\AgdaSpace{}%
\AgdaInductiveConstructor{refl}\AgdaSpace{}%
\AgdaInductiveConstructor{refl}%
\>[17]\AgdaSymbol{=}%
\>[20]\AgdaInductiveConstructor{refl}\<%
\end{code}
\end{fence}

Again, a useful exercise is to carry out an interactive development,
checking how Agda's knowledge changes as each of the two arguments is
instantiated.

\hypertarget{Equality-cong}{%
\section{Congruence and substitution}\label{Equality-cong}}

Equality satisfies \emph{congruence}. If two terms are equal, they
remain so after the same function is applied to both:

\begin{fence}
\begin{code}%
\>[0]\AgdaFunction{cong}\AgdaSpace{}%
\AgdaSymbol{:}\AgdaSpace{}%
\AgdaSymbol{∀}\AgdaSpace{}%
\AgdaSymbol{\{}\AgdaBound{A}\AgdaSpace{}%
\AgdaBound{B}\AgdaSpace{}%
\AgdaSymbol{:}\AgdaSpace{}%
\AgdaPrimitiveType{Set}\AgdaSymbol{\}}\AgdaSpace{}%
\AgdaSymbol{(}\AgdaBound{f}\AgdaSpace{}%
\AgdaSymbol{:}\AgdaSpace{}%
\AgdaBound{A}\AgdaSpace{}%
\AgdaSymbol{→}\AgdaSpace{}%
\AgdaBound{B}\AgdaSymbol{)}\AgdaSpace{}%
\AgdaSymbol{\{}\AgdaBound{x}\AgdaSpace{}%
\AgdaBound{y}\AgdaSpace{}%
\AgdaSymbol{:}\AgdaSpace{}%
\AgdaBound{A}\AgdaSymbol{\}}\<%
\\
\>[0][@{}l@{\AgdaIndent{0}}]%
\>[2]\AgdaSymbol{→}%
\>[74I]\AgdaBound{x}\AgdaSpace{}%
\AgdaOperator{\AgdaDatatype{≡}}\AgdaSpace{}%
\AgdaBound{y}\<%
\\
\>[.][@{}l@{}]\<[74I]%
\>[4]\AgdaComment{---------}\<%
\\
%
\>[2]\AgdaSymbol{→}\AgdaSpace{}%
\AgdaBound{f}\AgdaSpace{}%
\AgdaBound{x}\AgdaSpace{}%
\AgdaOperator{\AgdaDatatype{≡}}\AgdaSpace{}%
\AgdaBound{f}\AgdaSpace{}%
\AgdaBound{y}\<%
\\
\>[0]\AgdaFunction{cong}\AgdaSpace{}%
\AgdaBound{f}\AgdaSpace{}%
\AgdaInductiveConstructor{refl}%
\>[13]\AgdaSymbol{=}%
\>[16]\AgdaInductiveConstructor{refl}\<%
\end{code}
\end{fence}

Congruence of functions with two arguments is similar:

\begin{fence}
\begin{code}%
\>[0]\AgdaFunction{cong₂}\AgdaSpace{}%
\AgdaSymbol{:}\AgdaSpace{}%
\AgdaSymbol{∀}\AgdaSpace{}%
\AgdaSymbol{\{}\AgdaBound{A}\AgdaSpace{}%
\AgdaBound{B}\AgdaSpace{}%
\AgdaBound{C}\AgdaSpace{}%
\AgdaSymbol{:}\AgdaSpace{}%
\AgdaPrimitiveType{Set}\AgdaSymbol{\}}\AgdaSpace{}%
\AgdaSymbol{(}\AgdaBound{f}\AgdaSpace{}%
\AgdaSymbol{:}\AgdaSpace{}%
\AgdaBound{A}\AgdaSpace{}%
\AgdaSymbol{→}\AgdaSpace{}%
\AgdaBound{B}\AgdaSpace{}%
\AgdaSymbol{→}\AgdaSpace{}%
\AgdaBound{C}\AgdaSymbol{)}\AgdaSpace{}%
\AgdaSymbol{\{}\AgdaBound{u}\AgdaSpace{}%
\AgdaBound{x}\AgdaSpace{}%
\AgdaSymbol{:}\AgdaSpace{}%
\AgdaBound{A}\AgdaSymbol{\}}\AgdaSpace{}%
\AgdaSymbol{\{}\AgdaBound{v}\AgdaSpace{}%
\AgdaBound{y}\AgdaSpace{}%
\AgdaSymbol{:}\AgdaSpace{}%
\AgdaBound{B}\AgdaSymbol{\}}\<%
\\
\>[0][@{}l@{\AgdaIndent{0}}]%
\>[2]\AgdaSymbol{→}\AgdaSpace{}%
\AgdaBound{u}\AgdaSpace{}%
\AgdaOperator{\AgdaDatatype{≡}}\AgdaSpace{}%
\AgdaBound{x}\<%
\\
%
\>[2]\AgdaSymbol{→}%
\>[109I]\AgdaBound{v}\AgdaSpace{}%
\AgdaOperator{\AgdaDatatype{≡}}\AgdaSpace{}%
\AgdaBound{y}\<%
\\
\>[.][@{}l@{}]\<[109I]%
\>[4]\AgdaComment{-------------}\<%
\\
%
\>[2]\AgdaSymbol{→}\AgdaSpace{}%
\AgdaBound{f}\AgdaSpace{}%
\AgdaBound{u}\AgdaSpace{}%
\AgdaBound{v}\AgdaSpace{}%
\AgdaOperator{\AgdaDatatype{≡}}\AgdaSpace{}%
\AgdaBound{f}\AgdaSpace{}%
\AgdaBound{x}\AgdaSpace{}%
\AgdaBound{y}\<%
\\
\>[0]\AgdaFunction{cong₂}\AgdaSpace{}%
\AgdaBound{f}\AgdaSpace{}%
\AgdaInductiveConstructor{refl}\AgdaSpace{}%
\AgdaInductiveConstructor{refl}%
\>[19]\AgdaSymbol{=}%
\>[22]\AgdaInductiveConstructor{refl}\<%
\end{code}
\end{fence}

Equality is also a congruence in the function position of an
application. If two functions are equal, then applying them to the same
term yields equal terms:

\begin{fence}
\begin{code}%
\>[0]\AgdaFunction{cong-app}\AgdaSpace{}%
\AgdaSymbol{:}\AgdaSpace{}%
\AgdaSymbol{∀}\AgdaSpace{}%
\AgdaSymbol{\{}\AgdaBound{A}\AgdaSpace{}%
\AgdaBound{B}\AgdaSpace{}%
\AgdaSymbol{:}\AgdaSpace{}%
\AgdaPrimitiveType{Set}\AgdaSymbol{\}}\AgdaSpace{}%
\AgdaSymbol{\{}\AgdaBound{f}\AgdaSpace{}%
\AgdaBound{g}\AgdaSpace{}%
\AgdaSymbol{:}\AgdaSpace{}%
\AgdaBound{A}\AgdaSpace{}%
\AgdaSymbol{→}\AgdaSpace{}%
\AgdaBound{B}\AgdaSymbol{\}}\<%
\\
\>[0][@{}l@{\AgdaIndent{0}}]%
\>[2]\AgdaSymbol{→}%
\>[134I]\AgdaBound{f}\AgdaSpace{}%
\AgdaOperator{\AgdaDatatype{≡}}\AgdaSpace{}%
\AgdaBound{g}\<%
\\
\>[.][@{}l@{}]\<[134I]%
\>[4]\AgdaComment{---------------------}\<%
\\
%
\>[2]\AgdaSymbol{→}\AgdaSpace{}%
\AgdaSymbol{∀}\AgdaSpace{}%
\AgdaSymbol{(}\AgdaBound{x}\AgdaSpace{}%
\AgdaSymbol{:}\AgdaSpace{}%
\AgdaBound{A}\AgdaSymbol{)}\AgdaSpace{}%
\AgdaSymbol{→}\AgdaSpace{}%
\AgdaBound{f}\AgdaSpace{}%
\AgdaBound{x}\AgdaSpace{}%
\AgdaOperator{\AgdaDatatype{≡}}\AgdaSpace{}%
\AgdaBound{g}\AgdaSpace{}%
\AgdaBound{x}\<%
\\
\>[0]\AgdaFunction{cong-app}\AgdaSpace{}%
\AgdaInductiveConstructor{refl}\AgdaSpace{}%
\AgdaBound{x}\AgdaSpace{}%
\AgdaSymbol{=}\AgdaSpace{}%
\AgdaInductiveConstructor{refl}\<%
\end{code}
\end{fence}

Equality also satisfies \emph{substitution}. If two values are equal and
a predicate holds of the first then it also holds of the second:

\begin{fence}
\begin{code}%
\>[0]\AgdaFunction{subst}\AgdaSpace{}%
\AgdaSymbol{:}\AgdaSpace{}%
\AgdaSymbol{∀}\AgdaSpace{}%
\AgdaSymbol{\{}\AgdaBound{A}\AgdaSpace{}%
\AgdaSymbol{:}\AgdaSpace{}%
\AgdaPrimitiveType{Set}\AgdaSymbol{\}}\AgdaSpace{}%
\AgdaSymbol{\{}\AgdaBound{x}\AgdaSpace{}%
\AgdaBound{y}\AgdaSpace{}%
\AgdaSymbol{:}\AgdaSpace{}%
\AgdaBound{A}\AgdaSymbol{\}}\AgdaSpace{}%
\AgdaSymbol{(}\AgdaBound{P}\AgdaSpace{}%
\AgdaSymbol{:}\AgdaSpace{}%
\AgdaBound{A}\AgdaSpace{}%
\AgdaSymbol{→}\AgdaSpace{}%
\AgdaPrimitiveType{Set}\AgdaSymbol{)}\<%
\\
\>[0][@{}l@{\AgdaIndent{0}}]%
\>[2]\AgdaSymbol{→}%
\>[165I]\AgdaBound{x}\AgdaSpace{}%
\AgdaOperator{\AgdaDatatype{≡}}\AgdaSpace{}%
\AgdaBound{y}\<%
\\
\>[.][@{}l@{}]\<[165I]%
\>[4]\AgdaComment{---------}\<%
\\
%
\>[2]\AgdaSymbol{→}\AgdaSpace{}%
\AgdaBound{P}\AgdaSpace{}%
\AgdaBound{x}\AgdaSpace{}%
\AgdaSymbol{→}\AgdaSpace{}%
\AgdaBound{P}\AgdaSpace{}%
\AgdaBound{y}\<%
\\
\>[0]\AgdaFunction{subst}\AgdaSpace{}%
\AgdaBound{P}\AgdaSpace{}%
\AgdaInductiveConstructor{refl}\AgdaSpace{}%
\AgdaBound{px}\AgdaSpace{}%
\AgdaSymbol{=}\AgdaSpace{}%
\AgdaBound{px}\<%
\end{code}
\end{fence}

\hypertarget{chains-of-equations}{%
\section{Chains of equations}\label{chains-of-equations}}

Here we show how to support reasoning with chains of equations, as used
throughout the book. We package the declarations into a module, named
\texttt{≡-Reasoning}, to match the format used in Agda's standard
library:

\begin{fence}
\begin{code}%
\>[0]\AgdaKeyword{module}\AgdaSpace{}%
\AgdaModule{≡-Reasoning}\AgdaSpace{}%
\AgdaSymbol{\{}\AgdaBound{A}\AgdaSpace{}%
\AgdaSymbol{:}\AgdaSpace{}%
\AgdaPrimitiveType{Set}\AgdaSymbol{\}}\AgdaSpace{}%
\AgdaKeyword{where}\<%
\\
%
\\[\AgdaEmptyExtraSkip]%
\>[0][@{}l@{\AgdaIndent{0}}]%
\>[2]\AgdaKeyword{infix}%
\>[9]\AgdaNumber{1}\AgdaSpace{}%
\AgdaOperator{\AgdaFunction{begin\AgdaUnderscore{}}}\<%
\\
%
\>[2]\AgdaKeyword{infixr}\AgdaSpace{}%
\AgdaNumber{2}\AgdaSpace{}%
\AgdaOperator{\AgdaFunction{\AgdaUnderscore{}≡⟨⟩\AgdaUnderscore{}}}\AgdaSpace{}%
\AgdaOperator{\AgdaFunction{\AgdaUnderscore{}≡⟨\AgdaUnderscore{}⟩\AgdaUnderscore{}}}\<%
\\
%
\>[2]\AgdaKeyword{infix}%
\>[9]\AgdaNumber{3}\AgdaSpace{}%
\AgdaOperator{\AgdaFunction{\AgdaUnderscore{}∎}}\<%
\\
%
\\[\AgdaEmptyExtraSkip]%
%
\>[2]\AgdaOperator{\AgdaFunction{begin\AgdaUnderscore{}}}\AgdaSpace{}%
\AgdaSymbol{:}\AgdaSpace{}%
\AgdaSymbol{∀}\AgdaSpace{}%
\AgdaSymbol{\{}\AgdaBound{x}\AgdaSpace{}%
\AgdaBound{y}\AgdaSpace{}%
\AgdaSymbol{:}\AgdaSpace{}%
\AgdaBound{A}\AgdaSymbol{\}}\<%
\\
\>[2][@{}l@{\AgdaIndent{0}}]%
\>[4]\AgdaSymbol{→}%
\>[194I]\AgdaBound{x}\AgdaSpace{}%
\AgdaOperator{\AgdaDatatype{≡}}\AgdaSpace{}%
\AgdaBound{y}\<%
\\
\>[.][@{}l@{}]\<[194I]%
\>[6]\AgdaComment{-----}\<%
\\
%
\>[4]\AgdaSymbol{→}\AgdaSpace{}%
\AgdaBound{x}\AgdaSpace{}%
\AgdaOperator{\AgdaDatatype{≡}}\AgdaSpace{}%
\AgdaBound{y}\<%
\\
%
\>[2]\AgdaOperator{\AgdaFunction{begin}}\AgdaSpace{}%
\AgdaBound{x≡y}%
\>[13]\AgdaSymbol{=}%
\>[16]\AgdaBound{x≡y}\<%
\\
%
\\[\AgdaEmptyExtraSkip]%
%
\>[2]\AgdaOperator{\AgdaFunction{\AgdaUnderscore{}≡⟨⟩\AgdaUnderscore{}}}\AgdaSpace{}%
\AgdaSymbol{:}\AgdaSpace{}%
\AgdaSymbol{∀}\AgdaSpace{}%
\AgdaSymbol{(}\AgdaBound{x}\AgdaSpace{}%
\AgdaSymbol{:}\AgdaSpace{}%
\AgdaBound{A}\AgdaSymbol{)}\AgdaSpace{}%
\AgdaSymbol{\{}\AgdaBound{y}\AgdaSpace{}%
\AgdaSymbol{:}\AgdaSpace{}%
\AgdaBound{A}\AgdaSymbol{\}}\<%
\\
\>[2][@{}l@{\AgdaIndent{0}}]%
\>[4]\AgdaSymbol{→}%
\>[209I]\AgdaBound{x}\AgdaSpace{}%
\AgdaOperator{\AgdaDatatype{≡}}\AgdaSpace{}%
\AgdaBound{y}\<%
\\
\>[.][@{}l@{}]\<[209I]%
\>[6]\AgdaComment{-----}\<%
\\
%
\>[4]\AgdaSymbol{→}\AgdaSpace{}%
\AgdaBound{x}\AgdaSpace{}%
\AgdaOperator{\AgdaDatatype{≡}}\AgdaSpace{}%
\AgdaBound{y}\<%
\\
%
\>[2]\AgdaBound{x}\AgdaSpace{}%
\AgdaOperator{\AgdaFunction{≡⟨⟩}}\AgdaSpace{}%
\AgdaBound{x≡y}%
\>[13]\AgdaSymbol{=}%
\>[16]\AgdaBound{x≡y}\<%
\\
%
\\[\AgdaEmptyExtraSkip]%
%
\>[2]\AgdaOperator{\AgdaFunction{\AgdaUnderscore{}≡⟨\AgdaUnderscore{}⟩\AgdaUnderscore{}}}\AgdaSpace{}%
\AgdaSymbol{:}\AgdaSpace{}%
\AgdaSymbol{∀}\AgdaSpace{}%
\AgdaSymbol{(}\AgdaBound{x}\AgdaSpace{}%
\AgdaSymbol{:}\AgdaSpace{}%
\AgdaBound{A}\AgdaSymbol{)}\AgdaSpace{}%
\AgdaSymbol{\{}\AgdaBound{y}\AgdaSpace{}%
\AgdaBound{z}\AgdaSpace{}%
\AgdaSymbol{:}\AgdaSpace{}%
\AgdaBound{A}\AgdaSymbol{\}}\<%
\\
\>[2][@{}l@{\AgdaIndent{0}}]%
\>[4]\AgdaSymbol{→}\AgdaSpace{}%
\AgdaBound{x}\AgdaSpace{}%
\AgdaOperator{\AgdaDatatype{≡}}\AgdaSpace{}%
\AgdaBound{y}\<%
\\
%
\>[4]\AgdaSymbol{→}%
\>[229I]\AgdaBound{y}\AgdaSpace{}%
\AgdaOperator{\AgdaDatatype{≡}}\AgdaSpace{}%
\AgdaBound{z}\<%
\\
\>[.][@{}l@{}]\<[229I]%
\>[6]\AgdaComment{-----}\<%
\\
%
\>[4]\AgdaSymbol{→}\AgdaSpace{}%
\AgdaBound{x}\AgdaSpace{}%
\AgdaOperator{\AgdaDatatype{≡}}\AgdaSpace{}%
\AgdaBound{z}\<%
\\
%
\>[2]\AgdaBound{x}\AgdaSpace{}%
\AgdaOperator{\AgdaFunction{≡⟨}}\AgdaSpace{}%
\AgdaBound{x≡y}\AgdaSpace{}%
\AgdaOperator{\AgdaFunction{⟩}}\AgdaSpace{}%
\AgdaBound{y≡z}%
\>[18]\AgdaSymbol{=}%
\>[21]\AgdaFunction{trans}\AgdaSpace{}%
\AgdaBound{x≡y}\AgdaSpace{}%
\AgdaBound{y≡z}\<%
\\
%
\\[\AgdaEmptyExtraSkip]%
%
\>[2]\AgdaOperator{\AgdaFunction{\AgdaUnderscore{}∎}}%
\>[241I]\AgdaSymbol{:}\AgdaSpace{}%
\AgdaSymbol{∀}\AgdaSpace{}%
\AgdaSymbol{(}\AgdaBound{x}\AgdaSpace{}%
\AgdaSymbol{:}\AgdaSpace{}%
\AgdaBound{A}\AgdaSymbol{)}\<%
\\
\>[241I][@{}l@{\AgdaIndent{0}}]%
\>[6]\AgdaComment{-----}\<%
\\
\>[2][@{}l@{\AgdaIndent{0}}]%
\>[4]\AgdaSymbol{→}\AgdaSpace{}%
\AgdaBound{x}\AgdaSpace{}%
\AgdaOperator{\AgdaDatatype{≡}}\AgdaSpace{}%
\AgdaBound{x}\<%
\\
%
\>[2]\AgdaBound{x}\AgdaSpace{}%
\AgdaOperator{\AgdaFunction{∎}}%
\>[7]\AgdaSymbol{=}%
\>[10]\AgdaInductiveConstructor{refl}\<%
\\
%
\\[\AgdaEmptyExtraSkip]%
\>[0]\AgdaKeyword{open}\AgdaSpace{}%
\AgdaModule{≡-Reasoning}\<%
\end{code}
\end{fence}

This is our first use of a nested module. It consists of the keyword
\texttt{module} followed by the module name and any parameters, explicit
or implicit, the keyword \texttt{where}, and the contents of the module
indented. Modules may contain any sort of declaration, including other
nested modules. Nested modules are similar to the top-level modules that
constitute each chapter of this book, save that the body of a top-level
module need not be indented. Opening the module makes all of the
definitions available in the current environment.

As an example, let's look at a proof of transitivity as a chain of
equations:

\begin{fence}
\begin{code}%
\>[0]\AgdaFunction{trans′}\AgdaSpace{}%
\AgdaSymbol{:}\AgdaSpace{}%
\AgdaSymbol{∀}\AgdaSpace{}%
\AgdaSymbol{\{}\AgdaBound{A}\AgdaSpace{}%
\AgdaSymbol{:}\AgdaSpace{}%
\AgdaPrimitiveType{Set}\AgdaSymbol{\}}\AgdaSpace{}%
\AgdaSymbol{\{}\AgdaBound{x}\AgdaSpace{}%
\AgdaBound{y}\AgdaSpace{}%
\AgdaBound{z}\AgdaSpace{}%
\AgdaSymbol{:}\AgdaSpace{}%
\AgdaBound{A}\AgdaSymbol{\}}\<%
\\
\>[0][@{}l@{\AgdaIndent{0}}]%
\>[2]\AgdaSymbol{→}\AgdaSpace{}%
\AgdaBound{x}\AgdaSpace{}%
\AgdaOperator{\AgdaDatatype{≡}}\AgdaSpace{}%
\AgdaBound{y}\<%
\\
%
\>[2]\AgdaSymbol{→}%
\>[264I]\AgdaBound{y}\AgdaSpace{}%
\AgdaOperator{\AgdaDatatype{≡}}\AgdaSpace{}%
\AgdaBound{z}\<%
\\
\>[.][@{}l@{}]\<[264I]%
\>[4]\AgdaComment{-----}\<%
\\
%
\>[2]\AgdaSymbol{→}\AgdaSpace{}%
\AgdaBound{x}\AgdaSpace{}%
\AgdaOperator{\AgdaDatatype{≡}}\AgdaSpace{}%
\AgdaBound{z}\<%
\\
\>[0]\AgdaFunction{trans′}\AgdaSpace{}%
\AgdaSymbol{\{}\AgdaBound{A}\AgdaSymbol{\}}\AgdaSpace{}%
\AgdaSymbol{\{}\AgdaBound{x}\AgdaSymbol{\}}\AgdaSpace{}%
\AgdaSymbol{\{}\AgdaBound{y}\AgdaSymbol{\}}\AgdaSpace{}%
\AgdaSymbol{\{}\AgdaBound{z}\AgdaSymbol{\}}\AgdaSpace{}%
\AgdaBound{x≡y}\AgdaSpace{}%
\AgdaBound{y≡z}\AgdaSpace{}%
\AgdaSymbol{=}\<%
\\
\>[0][@{}l@{\AgdaIndent{0}}]%
\>[2]\AgdaOperator{\AgdaFunction{begin}}\<%
\\
\>[2][@{}l@{\AgdaIndent{0}}]%
\>[4]\AgdaBound{x}\<%
\\
%
\>[2]\AgdaOperator{\AgdaFunction{≡⟨}}\AgdaSpace{}%
\AgdaBound{x≡y}\AgdaSpace{}%
\AgdaOperator{\AgdaFunction{⟩}}\<%
\\
\>[2][@{}l@{\AgdaIndent{0}}]%
\>[4]\AgdaBound{y}\<%
\\
%
\>[2]\AgdaOperator{\AgdaFunction{≡⟨}}\AgdaSpace{}%
\AgdaBound{y≡z}\AgdaSpace{}%
\AgdaOperator{\AgdaFunction{⟩}}\<%
\\
\>[2][@{}l@{\AgdaIndent{0}}]%
\>[4]\AgdaBound{z}\<%
\\
%
\>[2]\AgdaOperator{\AgdaFunction{∎}}\<%
\end{code}
\end{fence}

According to the fixity declarations, the body parses as follows:

\begin{myDisplay}
begin (x ≡⟨ x≡y ⟩ (y ≡⟨ y≡z ⟩ (z ∎)))
\end{myDisplay}

The application of \texttt{begin} is purely cosmetic, as it simply
returns its argument. That argument consists of \texttt{\_≡⟨\_⟩\_}
applied to \texttt{x}, \texttt{x≡y}, and \texttt{y\ ≡⟨\ y≡z\ ⟩\ (z\ ∎)}.
The first argument is a term, \texttt{x}, while the second and third
arguments are both proofs of equations, in particular proofs of
\texttt{x\ ≡\ y} and \texttt{y\ ≡\ z} respectively, which are combined
by \texttt{trans} in the body of \texttt{\_≡⟨\_⟩\_} to yield a proof of
\texttt{x\ ≡\ z}. The proof of \texttt{y\ ≡\ z} consists of
\texttt{\_≡⟨\_⟩\_} applied to \texttt{y}, \texttt{y≡z}, and
\texttt{z\ ∎}. The first argument is a term, \texttt{y}, while the
second and third arguments are both proofs of equations, in particular
proofs of \texttt{y\ ≡\ z} and \texttt{z\ ≡\ z} respectively, which are
combined by \texttt{trans} in the body of \texttt{\_≡⟨\_⟩\_} to yield a
proof of \texttt{y\ ≡\ z}. Finally, the proof of \texttt{z\ ≡\ z}
consists of \texttt{\_∎} applied to the term \texttt{z}, which yields
\texttt{refl}. After simplification, the body is equivalent to the term:

\begin{myDisplay}
trans x≡y (trans y≡z refl)
\end{myDisplay}

We could replace any use of a chain of equations by a chain of
applications of \texttt{trans}; the result would be more compact but
harder to read. The trick behind \texttt{∎} means that a chain of
equalities simplifies to a chain of applications of \texttt{trans} that
ends in \texttt{trans\ e\ refl}, where \texttt{e} is a term that proves
some equality, even though \texttt{e} alone would do.

\hypertarget{exercise-trans-and--reasoning-practice}{%
\subsubsection{\texorpdfstring{Exercise \texttt{trans} and
\texttt{≡-Reasoning}
(practice)}{Exercise trans and ≡-Reasoning (practice)}}\label{exercise-trans-and--reasoning-practice}}

Sadly, we cannot use the definition of trans' using ≡-Reasoning as the
definition for trans. Can you see why? (Hint: look at the definition of
\texttt{\_≡⟨\_⟩\_})

\begin{fence}
\begin{code}%
\>[0]\AgdaComment{-- Your code goes here}\<%
\end{code}
\end{fence}

\hypertarget{chains-of-equations-another-example}{%
\section{Chains of equations, another
example}\label{chains-of-equations-another-example}}

As a second example of chains of equations, we repeat the proof that
addition is commutative. We first repeat the definitions of naturals and
addition. We cannot import them because (as noted at the beginning of
this chapter) it would cause a conflict:

\begin{fence}
\begin{code}%
\>[0]\AgdaKeyword{data}\AgdaSpace{}%
\AgdaDatatype{ℕ}\AgdaSpace{}%
\AgdaSymbol{:}\AgdaSpace{}%
\AgdaPrimitiveType{Set}\AgdaSpace{}%
\AgdaKeyword{where}\<%
\\
\>[0][@{}l@{\AgdaIndent{0}}]%
\>[2]\AgdaInductiveConstructor{zero}\AgdaSpace{}%
\AgdaSymbol{:}\AgdaSpace{}%
\AgdaDatatype{ℕ}\<%
\\
%
\>[2]\AgdaInductiveConstructor{suc}%
\>[7]\AgdaSymbol{:}\AgdaSpace{}%
\AgdaDatatype{ℕ}\AgdaSpace{}%
\AgdaSymbol{→}\AgdaSpace{}%
\AgdaDatatype{ℕ}\<%
\\
%
\\[\AgdaEmptyExtraSkip]%
\>[0]\AgdaOperator{\AgdaFunction{\AgdaUnderscore{}+\AgdaUnderscore{}}}\AgdaSpace{}%
\AgdaSymbol{:}\AgdaSpace{}%
\AgdaDatatype{ℕ}\AgdaSpace{}%
\AgdaSymbol{→}\AgdaSpace{}%
\AgdaDatatype{ℕ}\AgdaSpace{}%
\AgdaSymbol{→}\AgdaSpace{}%
\AgdaDatatype{ℕ}\<%
\\
\>[0]\AgdaInductiveConstructor{zero}%
\>[8]\AgdaOperator{\AgdaFunction{+}}\AgdaSpace{}%
\AgdaBound{n}%
\>[13]\AgdaSymbol{=}%
\>[16]\AgdaBound{n}\<%
\\
\>[0]\AgdaSymbol{(}\AgdaInductiveConstructor{suc}\AgdaSpace{}%
\AgdaBound{m}\AgdaSymbol{)}\AgdaSpace{}%
\AgdaOperator{\AgdaFunction{+}}\AgdaSpace{}%
\AgdaBound{n}%
\>[13]\AgdaSymbol{=}%
\>[16]\AgdaInductiveConstructor{suc}\AgdaSpace{}%
\AgdaSymbol{(}\AgdaBound{m}\AgdaSpace{}%
\AgdaOperator{\AgdaFunction{+}}\AgdaSpace{}%
\AgdaBound{n}\AgdaSymbol{)}\<%
\end{code}
\end{fence}

To save space we postulate (rather than prove in full) two lemmas:

\begin{fence}
\begin{code}%
\>[0]\AgdaKeyword{postulate}\<%
\\
\>[0][@{}l@{\AgdaIndent{0}}]%
\>[2]\AgdaPostulate{+-identity}\AgdaSpace{}%
\AgdaSymbol{:}\AgdaSpace{}%
\AgdaSymbol{∀}\AgdaSpace{}%
\AgdaSymbol{(}\AgdaBound{m}\AgdaSpace{}%
\AgdaSymbol{:}\AgdaSpace{}%
\AgdaDatatype{ℕ}\AgdaSymbol{)}\AgdaSpace{}%
\AgdaSymbol{→}\AgdaSpace{}%
\AgdaBound{m}\AgdaSpace{}%
\AgdaOperator{\AgdaFunction{+}}\AgdaSpace{}%
\AgdaInductiveConstructor{zero}\AgdaSpace{}%
\AgdaOperator{\AgdaDatatype{≡}}\AgdaSpace{}%
\AgdaBound{m}\<%
\\
%
\>[2]\AgdaPostulate{+-suc}\AgdaSpace{}%
\AgdaSymbol{:}\AgdaSpace{}%
\AgdaSymbol{∀}\AgdaSpace{}%
\AgdaSymbol{(}\AgdaBound{m}\AgdaSpace{}%
\AgdaBound{n}\AgdaSpace{}%
\AgdaSymbol{:}\AgdaSpace{}%
\AgdaDatatype{ℕ}\AgdaSymbol{)}\AgdaSpace{}%
\AgdaSymbol{→}\AgdaSpace{}%
\AgdaBound{m}\AgdaSpace{}%
\AgdaOperator{\AgdaFunction{+}}\AgdaSpace{}%
\AgdaInductiveConstructor{suc}\AgdaSpace{}%
\AgdaBound{n}\AgdaSpace{}%
\AgdaOperator{\AgdaDatatype{≡}}\AgdaSpace{}%
\AgdaInductiveConstructor{suc}\AgdaSpace{}%
\AgdaSymbol{(}\AgdaBound{m}\AgdaSpace{}%
\AgdaOperator{\AgdaFunction{+}}\AgdaSpace{}%
\AgdaBound{n}\AgdaSymbol{)}\<%
\end{code}
\end{fence}

This is our first use of a \emph{postulate}. A postulate specifies a
signature for an identifier but no definition. Here we postulate
something proved earlier to save space. Postulates must be used with
caution. If we postulate something false then we could use Agda to prove
anything whatsoever.

We then repeat the proof of commutativity:

\begin{fence}
\begin{code}%
\>[0]\AgdaFunction{+-comm}\AgdaSpace{}%
\AgdaSymbol{:}\AgdaSpace{}%
\AgdaSymbol{∀}\AgdaSpace{}%
\AgdaSymbol{(}\AgdaBound{m}\AgdaSpace{}%
\AgdaBound{n}\AgdaSpace{}%
\AgdaSymbol{:}\AgdaSpace{}%
\AgdaDatatype{ℕ}\AgdaSymbol{)}\AgdaSpace{}%
\AgdaSymbol{→}\AgdaSpace{}%
\AgdaBound{m}\AgdaSpace{}%
\AgdaOperator{\AgdaFunction{+}}\AgdaSpace{}%
\AgdaBound{n}\AgdaSpace{}%
\AgdaOperator{\AgdaDatatype{≡}}\AgdaSpace{}%
\AgdaBound{n}\AgdaSpace{}%
\AgdaOperator{\AgdaFunction{+}}\AgdaSpace{}%
\AgdaBound{m}\<%
\\
\>[0]\AgdaFunction{+-comm}\AgdaSpace{}%
\AgdaBound{m}\AgdaSpace{}%
\AgdaInductiveConstructor{zero}\AgdaSpace{}%
\AgdaSymbol{=}\<%
\\
\>[0][@{}l@{\AgdaIndent{0}}]%
\>[2]\AgdaOperator{\AgdaFunction{begin}}\<%
\\
\>[2][@{}l@{\AgdaIndent{0}}]%
\>[4]\AgdaBound{m}\AgdaSpace{}%
\AgdaOperator{\AgdaFunction{+}}\AgdaSpace{}%
\AgdaInductiveConstructor{zero}\<%
\\
%
\>[2]\AgdaOperator{\AgdaFunction{≡⟨}}\AgdaSpace{}%
\AgdaPostulate{+-identity}\AgdaSpace{}%
\AgdaBound{m}\AgdaSpace{}%
\AgdaOperator{\AgdaFunction{⟩}}\<%
\\
\>[2][@{}l@{\AgdaIndent{0}}]%
\>[4]\AgdaBound{m}\<%
\\
%
\>[2]\AgdaOperator{\AgdaFunction{≡⟨⟩}}\<%
\\
\>[2][@{}l@{\AgdaIndent{0}}]%
\>[4]\AgdaInductiveConstructor{zero}\AgdaSpace{}%
\AgdaOperator{\AgdaFunction{+}}\AgdaSpace{}%
\AgdaBound{m}\<%
\\
%
\>[2]\AgdaOperator{\AgdaFunction{∎}}\<%
\\
\>[0]\AgdaFunction{+-comm}\AgdaSpace{}%
\AgdaBound{m}\AgdaSpace{}%
\AgdaSymbol{(}\AgdaInductiveConstructor{suc}\AgdaSpace{}%
\AgdaBound{n}\AgdaSymbol{)}\AgdaSpace{}%
\AgdaSymbol{=}\<%
\\
\>[0][@{}l@{\AgdaIndent{0}}]%
\>[2]\AgdaOperator{\AgdaFunction{begin}}\<%
\\
\>[2][@{}l@{\AgdaIndent{0}}]%
\>[4]\AgdaBound{m}\AgdaSpace{}%
\AgdaOperator{\AgdaFunction{+}}\AgdaSpace{}%
\AgdaInductiveConstructor{suc}\AgdaSpace{}%
\AgdaBound{n}\<%
\\
%
\>[2]\AgdaOperator{\AgdaFunction{≡⟨}}\AgdaSpace{}%
\AgdaPostulate{+-suc}\AgdaSpace{}%
\AgdaBound{m}\AgdaSpace{}%
\AgdaBound{n}\AgdaSpace{}%
\AgdaOperator{\AgdaFunction{⟩}}\<%
\\
\>[2][@{}l@{\AgdaIndent{0}}]%
\>[4]\AgdaInductiveConstructor{suc}\AgdaSpace{}%
\AgdaSymbol{(}\AgdaBound{m}\AgdaSpace{}%
\AgdaOperator{\AgdaFunction{+}}\AgdaSpace{}%
\AgdaBound{n}\AgdaSymbol{)}\<%
\\
%
\>[2]\AgdaOperator{\AgdaFunction{≡⟨}}\AgdaSpace{}%
\AgdaFunction{cong}\AgdaSpace{}%
\AgdaInductiveConstructor{suc}\AgdaSpace{}%
\AgdaSymbol{(}\AgdaFunction{+-comm}\AgdaSpace{}%
\AgdaBound{m}\AgdaSpace{}%
\AgdaBound{n}\AgdaSymbol{)}\AgdaSpace{}%
\AgdaOperator{\AgdaFunction{⟩}}\<%
\\
\>[2][@{}l@{\AgdaIndent{0}}]%
\>[4]\AgdaInductiveConstructor{suc}\AgdaSpace{}%
\AgdaSymbol{(}\AgdaBound{n}\AgdaSpace{}%
\AgdaOperator{\AgdaFunction{+}}\AgdaSpace{}%
\AgdaBound{m}\AgdaSymbol{)}\<%
\\
%
\>[2]\AgdaOperator{\AgdaFunction{≡⟨⟩}}\<%
\\
\>[2][@{}l@{\AgdaIndent{0}}]%
\>[4]\AgdaInductiveConstructor{suc}\AgdaSpace{}%
\AgdaBound{n}\AgdaSpace{}%
\AgdaOperator{\AgdaFunction{+}}\AgdaSpace{}%
\AgdaBound{m}\<%
\\
%
\>[2]\AgdaOperator{\AgdaFunction{∎}}\<%
\end{code}
\end{fence}

The reasoning here is similar to that in the preceding section. We use
\texttt{\_≡⟨⟩\_} when no justification is required. One can think of
\texttt{\_≡⟨⟩\_} as equivalent to \texttt{\_≡⟨\ refl\ ⟩\_}.

Agda always treats a term as equivalent to its simplified term. The
reason that one can write

\begin{myDisplay}
  suc (n + m)
≡⟨⟩
  suc n + m
\end{myDisplay}

is because Agda treats both terms as the same. This also means that one
could instead interchange the lines and write

\begin{myDisplay}
  suc n + m
≡⟨⟩
  suc (n + m)
\end{myDisplay}

and Agda would not object. Agda only checks that the terms separated by
\texttt{≡⟨⟩} have the same simplified form; it's up to us to write them
in an order that will make sense to the reader.

\hypertarget{exercise--reasoning-stretch}{%
\subsubsection{\texorpdfstring{Exercise \texttt{≤-Reasoning}
(stretch)}{Exercise ≤-Reasoning (stretch)}}\label{exercise--reasoning-stretch}}

The proof of monotonicity from Chapter
\protect\hyperlink{Relations}{Relations} can be written in a more
readable form by using an analogue of our notation for
\texttt{≡-Reasoning}. Define \texttt{≤-Reasoning} analogously, and use
it to write out an alternative proof that addition is monotonic with
regard to inequality. Rewrite all of \texttt{+-monoˡ-≤},
\texttt{+-monoʳ-≤}, and \texttt{+-mono-≤}.

\begin{fence}
\begin{code}%
\>[0]\AgdaComment{-- Your code goes here}\<%
\end{code}
\end{fence}

\hypertarget{rewriting}{%
\section{Rewriting}\label{rewriting}}

Consider a property of natural numbers, such as being even. We repeat
the earlier definition:

\begin{fence}
\begin{code}%
\>[0]\AgdaKeyword{data}\AgdaSpace{}%
\AgdaDatatype{even}\AgdaSpace{}%
\AgdaSymbol{:}\AgdaSpace{}%
\AgdaDatatype{ℕ}\AgdaSpace{}%
\AgdaSymbol{→}\AgdaSpace{}%
\AgdaPrimitiveType{Set}\<%
\\
\>[0]\AgdaKeyword{data}\AgdaSpace{}%
\AgdaDatatype{odd}%
\>[10]\AgdaSymbol{:}\AgdaSpace{}%
\AgdaDatatype{ℕ}\AgdaSpace{}%
\AgdaSymbol{→}\AgdaSpace{}%
\AgdaPrimitiveType{Set}\<%
\\
%
\\[\AgdaEmptyExtraSkip]%
\>[0]\AgdaKeyword{data}\AgdaSpace{}%
\AgdaDatatype{even}\AgdaSpace{}%
\AgdaKeyword{where}\<%
\\
%
\\[\AgdaEmptyExtraSkip]%
\>[0][@{}l@{\AgdaIndent{0}}]%
\>[2]\AgdaInductiveConstructor{even-zero}\AgdaSpace{}%
\AgdaSymbol{:}\AgdaSpace{}%
\AgdaDatatype{even}\AgdaSpace{}%
\AgdaInductiveConstructor{zero}\<%
\\
%
\\[\AgdaEmptyExtraSkip]%
%
\>[2]\AgdaInductiveConstructor{even-suc}\AgdaSpace{}%
\AgdaSymbol{:}\AgdaSpace{}%
\AgdaSymbol{∀}\AgdaSpace{}%
\AgdaSymbol{\{}\AgdaBound{n}\AgdaSpace{}%
\AgdaSymbol{:}\AgdaSpace{}%
\AgdaDatatype{ℕ}\AgdaSymbol{\}}\<%
\\
\>[2][@{}l@{\AgdaIndent{0}}]%
\>[4]\AgdaSymbol{→}%
\>[399I]\AgdaDatatype{odd}\AgdaSpace{}%
\AgdaBound{n}\<%
\\
\>[.][@{}l@{}]\<[399I]%
\>[6]\AgdaComment{------------}\<%
\\
%
\>[4]\AgdaSymbol{→}\AgdaSpace{}%
\AgdaDatatype{even}\AgdaSpace{}%
\AgdaSymbol{(}\AgdaInductiveConstructor{suc}\AgdaSpace{}%
\AgdaBound{n}\AgdaSymbol{)}\<%
\\
%
\\[\AgdaEmptyExtraSkip]%
\>[0]\AgdaKeyword{data}\AgdaSpace{}%
\AgdaDatatype{odd}\AgdaSpace{}%
\AgdaKeyword{where}\<%
\\
\>[0][@{}l@{\AgdaIndent{0}}]%
\>[2]\AgdaInductiveConstructor{odd-suc}\AgdaSpace{}%
\AgdaSymbol{:}\AgdaSpace{}%
\AgdaSymbol{∀}\AgdaSpace{}%
\AgdaSymbol{\{}\AgdaBound{n}\AgdaSpace{}%
\AgdaSymbol{:}\AgdaSpace{}%
\AgdaDatatype{ℕ}\AgdaSymbol{\}}\<%
\\
\>[2][@{}l@{\AgdaIndent{0}}]%
\>[4]\AgdaSymbol{→}%
\>[411I]\AgdaDatatype{even}\AgdaSpace{}%
\AgdaBound{n}\<%
\\
\>[.][@{}l@{}]\<[411I]%
\>[6]\AgdaComment{-----------}\<%
\\
%
\>[4]\AgdaSymbol{→}\AgdaSpace{}%
\AgdaDatatype{odd}\AgdaSpace{}%
\AgdaSymbol{(}\AgdaInductiveConstructor{suc}\AgdaSpace{}%
\AgdaBound{n}\AgdaSymbol{)}\<%
\end{code}
\end{fence}

In the previous section, we proved addition is commutative. Given
evidence that \texttt{even\ (m\ +\ n)} holds, we ought also to be able
to take that as evidence that \texttt{even\ (n\ +\ m)} holds.

Agda includes special notation to support just this kind of reasoning,
the \texttt{rewrite} notation we encountered earlier. To enable this
notation, we use pragmas to tell Agda which type corresponds to
equality:

\begin{fence}
\begin{code}%
\>[0]\AgdaSymbol{\{-\#}\AgdaSpace{}%
\AgdaKeyword{BUILTIN}\AgdaSpace{}%
\AgdaKeyword{EQUALITY}\AgdaSpace{}%
\AgdaOperator{\AgdaDatatype{\AgdaUnderscore{}≡\AgdaUnderscore{}}}\AgdaSpace{}%
\AgdaSymbol{\#-\}}\<%
\end{code}
\end{fence}

We can then prove the desired property as follows:

\begin{fence}
\begin{code}%
\>[0]\AgdaFunction{even-comm}\AgdaSpace{}%
\AgdaSymbol{:}\AgdaSpace{}%
\AgdaSymbol{∀}\AgdaSpace{}%
\AgdaSymbol{(}\AgdaBound{m}\AgdaSpace{}%
\AgdaBound{n}\AgdaSpace{}%
\AgdaSymbol{:}\AgdaSpace{}%
\AgdaDatatype{ℕ}\AgdaSymbol{)}\<%
\\
\>[0][@{}l@{\AgdaIndent{0}}]%
\>[2]\AgdaSymbol{→}%
\>[426I]\AgdaDatatype{even}\AgdaSpace{}%
\AgdaSymbol{(}\AgdaBound{m}\AgdaSpace{}%
\AgdaOperator{\AgdaFunction{+}}\AgdaSpace{}%
\AgdaBound{n}\AgdaSymbol{)}\<%
\\
\>[.][@{}l@{}]\<[426I]%
\>[4]\AgdaComment{------------}\<%
\\
%
\>[2]\AgdaSymbol{→}\AgdaSpace{}%
\AgdaDatatype{even}\AgdaSpace{}%
\AgdaSymbol{(}\AgdaBound{n}\AgdaSpace{}%
\AgdaOperator{\AgdaFunction{+}}\AgdaSpace{}%
\AgdaBound{m}\AgdaSymbol{)}\<%
\\
\>[0]\AgdaFunction{even-comm}\AgdaSpace{}%
\AgdaBound{m}\AgdaSpace{}%
\AgdaBound{n}\AgdaSpace{}%
\AgdaBound{ev}%
\>[18]\AgdaKeyword{rewrite}\AgdaSpace{}%
\AgdaFunction{+-comm}\AgdaSpace{}%
\AgdaBound{n}\AgdaSpace{}%
\AgdaBound{m}%
\>[38]\AgdaSymbol{=}%
\>[41]\AgdaBound{ev}\<%
\end{code}
\end{fence}

Here \texttt{ev} ranges over evidence that \texttt{even\ (m\ +\ n)}
holds, and we show that it also provides evidence that
\texttt{even\ (n\ +\ m)} holds. In general, the keyword \texttt{rewrite}
is followed by evidence of an equality, and that equality is used to
rewrite the type of the goal and of any variable in scope.

It is instructive to develop \texttt{even-comm} interactively. To start,
we supply variables for the arguments on the left, and a hole for the
body on the right:

\begin{myDisplay}
even-comm : ∀ (m n : ℕ)
  → even (m + n)
    ------------
  → even (n + m)
even-comm m n ev = {! !}
\end{myDisplay}

If we go into the hole and type \texttt{C-c\ C-,} then Agda reports:

\begin{myDisplay}
Goal: even (n + m)
————————————————————————————————————————————————————————————
ev : even (m + n)
n  : ℕ
m  : ℕ
\end{myDisplay}

Now we add the rewrite:

\begin{myDisplay}
even-comm : ∀ (m n : ℕ)
  → even (m + n)
    ------------
  → even (n + m)
even-comm m n ev rewrite +-comm n m = {! !}
\end{myDisplay}

If we go into the hole again and type \texttt{C-c\ C-,} then Agda now
reports:

\begin{myDisplay}
Goal: even (m + n)
————————————————————————————————————————————————————————————
ev : even (m + n)
n  : ℕ
m  : ℕ
\end{myDisplay}

The arguments have been swapped in the goal. Now it is trivial to see
that \texttt{ev} satisfies the goal, and typing \texttt{C-c\ C-a} in the
hole causes it to be filled with \texttt{ev}. The command
\texttt{C-c\ C-a} performs an automated search, including checking
whether a variable in scope has the same type as the goal.

\hypertarget{multiple-rewrites}{%
\section{Multiple rewrites}\label{multiple-rewrites}}

One may perform multiple rewrites, each separated by a vertical bar. For
instance, here is a second proof that addition is commutative, relying
on rewrites rather than chains of equalities:

\begin{fence}
\begin{code}%
\>[0]\AgdaFunction{+-comm′}\AgdaSpace{}%
\AgdaSymbol{:}\AgdaSpace{}%
\AgdaSymbol{∀}\AgdaSpace{}%
\AgdaSymbol{(}\AgdaBound{m}\AgdaSpace{}%
\AgdaBound{n}\AgdaSpace{}%
\AgdaSymbol{:}\AgdaSpace{}%
\AgdaDatatype{ℕ}\AgdaSymbol{)}\AgdaSpace{}%
\AgdaSymbol{→}\AgdaSpace{}%
\AgdaBound{m}\AgdaSpace{}%
\AgdaOperator{\AgdaFunction{+}}\AgdaSpace{}%
\AgdaBound{n}\AgdaSpace{}%
\AgdaOperator{\AgdaDatatype{≡}}\AgdaSpace{}%
\AgdaBound{n}\AgdaSpace{}%
\AgdaOperator{\AgdaFunction{+}}\AgdaSpace{}%
\AgdaBound{m}\<%
\\
\>[0]\AgdaFunction{+-comm′}\AgdaSpace{}%
\AgdaInductiveConstructor{zero}%
\>[16]\AgdaBound{n}%
\>[19]\AgdaKeyword{rewrite}\AgdaSpace{}%
\AgdaPostulate{+-identity}\AgdaSpace{}%
\AgdaBound{n}%
\>[52]\AgdaSymbol{=}%
\>[55]\AgdaInductiveConstructor{refl}\<%
\\
\>[0]\AgdaFunction{+-comm′}\AgdaSpace{}%
\AgdaSymbol{(}\AgdaInductiveConstructor{suc}\AgdaSpace{}%
\AgdaBound{m}\AgdaSymbol{)}\AgdaSpace{}%
\AgdaBound{n}%
\>[19]\AgdaKeyword{rewrite}\AgdaSpace{}%
\AgdaPostulate{+-suc}\AgdaSpace{}%
\AgdaBound{n}\AgdaSpace{}%
\AgdaBound{m}\AgdaSpace{}%
\AgdaSymbol{|}\AgdaSpace{}%
\AgdaFunction{+-comm′}\AgdaSpace{}%
\AgdaBound{m}\AgdaSpace{}%
\AgdaBound{n}%
\>[52]\AgdaSymbol{=}%
\>[55]\AgdaInductiveConstructor{refl}\<%
\end{code}
\end{fence}

This is far more compact. Among other things, whereas the previous proof
required \texttt{cong\ suc\ (+-comm\ m\ n)} as the justification to
invoke the inductive hypothesis, here it is sufficient to rewrite with
\texttt{+-comm\ m\ n}, as rewriting automatically takes congruence into
account. Although proofs with rewriting are shorter, proofs as chains of
equalities are easier to follow, and we will stick with the latter when
feasible.

\hypertarget{rewriting-expanded}{%
\section{Rewriting expanded}\label{rewriting-expanded}}

The \texttt{rewrite} notation is in fact shorthand for an appropriate
use of \texttt{with} abstraction:

\begin{fence}
\begin{code}%
\>[0]\AgdaFunction{even-comm′}\AgdaSpace{}%
\AgdaSymbol{:}\AgdaSpace{}%
\AgdaSymbol{∀}\AgdaSpace{}%
\AgdaSymbol{(}\AgdaBound{m}\AgdaSpace{}%
\AgdaBound{n}\AgdaSpace{}%
\AgdaSymbol{:}\AgdaSpace{}%
\AgdaDatatype{ℕ}\AgdaSymbol{)}\<%
\\
\>[0][@{}l@{\AgdaIndent{0}}]%
\>[2]\AgdaSymbol{→}%
\>[473I]\AgdaDatatype{even}\AgdaSpace{}%
\AgdaSymbol{(}\AgdaBound{m}\AgdaSpace{}%
\AgdaOperator{\AgdaFunction{+}}\AgdaSpace{}%
\AgdaBound{n}\AgdaSymbol{)}\<%
\\
\>[.][@{}l@{}]\<[473I]%
\>[4]\AgdaComment{------------}\<%
\\
%
\>[2]\AgdaSymbol{→}\AgdaSpace{}%
\AgdaDatatype{even}\AgdaSpace{}%
\AgdaSymbol{(}\AgdaBound{n}\AgdaSpace{}%
\AgdaOperator{\AgdaFunction{+}}\AgdaSpace{}%
\AgdaBound{m}\AgdaSymbol{)}\<%
\\
\>[0]\AgdaFunction{even-comm′}\AgdaSpace{}%
\AgdaBound{m}\AgdaSpace{}%
\AgdaBound{n}\AgdaSpace{}%
\AgdaBound{ev}\AgdaSpace{}%
\AgdaKeyword{with}%
\>[25]\AgdaBound{m}\AgdaSpace{}%
\AgdaOperator{\AgdaFunction{+}}\AgdaSpace{}%
\AgdaBound{n}%
\>[32]\AgdaSymbol{|}\AgdaSpace{}%
\AgdaFunction{+-comm}\AgdaSpace{}%
\AgdaBound{m}\AgdaSpace{}%
\AgdaBound{n}\<%
\\
\>[0]\AgdaSymbol{...}%
\>[21]\AgdaSymbol{|}\AgdaSpace{}%
\AgdaDottedPattern{\AgdaSymbol{.(}}\AgdaDottedPattern{\AgdaBound{n}}\AgdaSpace{}%
\AgdaDottedPattern{\AgdaOperator{\AgdaFunction{+}}}\AgdaSpace{}%
\AgdaDottedPattern{\AgdaBound{m}}\AgdaDottedPattern{\AgdaSymbol{)}}\AgdaSpace{}%
\AgdaSymbol{|}\AgdaSpace{}%
\AgdaInductiveConstructor{refl}%
\>[45]\AgdaSymbol{=}\AgdaSpace{}%
\AgdaBound{ev}\<%
\end{code}
\end{fence}

In general, one can follow \texttt{with} by any number of expressions,
separated by bars, where each following equation has the same number of
patterns. We often write expressions and the corresponding patterns so
they line up in columns, as above. Here the first column asserts that
\texttt{m\ +\ n} and \texttt{n\ +\ m} are identical, and the second
column justifies that assertion with evidence of the appropriate
equality. Note also the use of the \emph{dot pattern},
\texttt{.(n\ +\ m)}. A dot pattern consists of a dot followed by an
expression, and is used when other information forces the value matched
to be equal to the value of the expression in the dot pattern. In this
case, the identification of \texttt{m\ +\ n} and \texttt{n\ +\ m} is
justified by the subsequent matching of \texttt{+-comm\ m\ n} against
\texttt{refl}. One might think that the first clause is redundant as the
information is inherent in the second clause, but in fact Agda is rather
picky on this point: omitting the first clause or reversing the order of
the clauses will cause Agda to report an error. (Try it and see!)

In this case, we can avoid rewrite by simply applying the substitution
function defined earlier:

\begin{fence}
\begin{code}%
\>[0]\AgdaFunction{even-comm″}\AgdaSpace{}%
\AgdaSymbol{:}\AgdaSpace{}%
\AgdaSymbol{∀}\AgdaSpace{}%
\AgdaSymbol{(}\AgdaBound{m}\AgdaSpace{}%
\AgdaBound{n}\AgdaSpace{}%
\AgdaSymbol{:}\AgdaSpace{}%
\AgdaDatatype{ℕ}\AgdaSymbol{)}\<%
\\
\>[0][@{}l@{\AgdaIndent{0}}]%
\>[2]\AgdaSymbol{→}%
\>[502I]\AgdaDatatype{even}\AgdaSpace{}%
\AgdaSymbol{(}\AgdaBound{m}\AgdaSpace{}%
\AgdaOperator{\AgdaFunction{+}}\AgdaSpace{}%
\AgdaBound{n}\AgdaSymbol{)}\<%
\\
\>[.][@{}l@{}]\<[502I]%
\>[4]\AgdaComment{------------}\<%
\\
%
\>[2]\AgdaSymbol{→}\AgdaSpace{}%
\AgdaDatatype{even}\AgdaSpace{}%
\AgdaSymbol{(}\AgdaBound{n}\AgdaSpace{}%
\AgdaOperator{\AgdaFunction{+}}\AgdaSpace{}%
\AgdaBound{m}\AgdaSymbol{)}\<%
\\
\>[0]\AgdaFunction{even-comm″}\AgdaSpace{}%
\AgdaBound{m}\AgdaSpace{}%
\AgdaBound{n}%
\>[16]\AgdaSymbol{=}%
\>[19]\AgdaFunction{subst}\AgdaSpace{}%
\AgdaDatatype{even}\AgdaSpace{}%
\AgdaSymbol{(}\AgdaFunction{+-comm}\AgdaSpace{}%
\AgdaBound{m}\AgdaSpace{}%
\AgdaBound{n}\AgdaSymbol{)}\<%
\end{code}
\end{fence}

Nonetheless, rewrite is a vital part of the Agda toolkit. We will use it
sparingly, but it is occasionally essential.

\hypertarget{leibniz-equality}{%
\section{Leibniz equality}\label{leibniz-equality}}

The form of asserting equality that we have used is due to Martin-Löf,
and was published in 1975. An older form is due to Leibniz, and was
published in 1686. Leibniz asserted the \emph{identity of
indiscernibles}: two objects are equal if and only if they satisfy the
same properties. This principle sometimes goes by the name Leibniz' Law,
and is closely related to Spock's Law, ``A difference that makes no
difference is no difference''. Here we define Leibniz equality, and show
that two terms satisfy Leibniz equality if and only if they satisfy
Martin-Löf equality.

Leibniz equality is usually formalised to state that \texttt{x\ ≐\ y}
holds if every property \texttt{P} that holds of \texttt{x} also holds
of \texttt{y}. Perhaps surprisingly, this definition is sufficient to
also ensure the converse, that every property \texttt{P} that holds of
\texttt{y} also holds of \texttt{x}.

Let \texttt{x} and \texttt{y} be objects of type \texttt{A}. We say that
\texttt{x\ ≐\ y} holds if for every predicate \texttt{P} over type
\texttt{A} we have that \texttt{P\ x} implies \texttt{P\ y}:

\begin{fence}
\begin{code}%
\>[0]\AgdaOperator{\AgdaFunction{\AgdaUnderscore{}≐\AgdaUnderscore{}}}\AgdaSpace{}%
\AgdaSymbol{:}\AgdaSpace{}%
\AgdaSymbol{∀}\AgdaSpace{}%
\AgdaSymbol{\{}\AgdaBound{A}\AgdaSpace{}%
\AgdaSymbol{:}\AgdaSpace{}%
\AgdaPrimitiveType{Set}\AgdaSymbol{\}}\AgdaSpace{}%
\AgdaSymbol{(}\AgdaBound{x}\AgdaSpace{}%
\AgdaBound{y}\AgdaSpace{}%
\AgdaSymbol{:}\AgdaSpace{}%
\AgdaBound{A}\AgdaSymbol{)}\AgdaSpace{}%
\AgdaSymbol{→}\AgdaSpace{}%
\AgdaPrimitiveType{Set₁}\<%
\\
\>[0]\AgdaOperator{\AgdaFunction{\AgdaUnderscore{}≐\AgdaUnderscore{}}}\AgdaSpace{}%
\AgdaSymbol{\{}\AgdaBound{A}\AgdaSymbol{\}}\AgdaSpace{}%
\AgdaBound{x}\AgdaSpace{}%
\AgdaBound{y}\AgdaSpace{}%
\AgdaSymbol{=}\AgdaSpace{}%
\AgdaSymbol{∀}\AgdaSpace{}%
\AgdaSymbol{(}\AgdaBound{P}\AgdaSpace{}%
\AgdaSymbol{:}\AgdaSpace{}%
\AgdaBound{A}\AgdaSpace{}%
\AgdaSymbol{→}\AgdaSpace{}%
\AgdaPrimitiveType{Set}\AgdaSymbol{)}\AgdaSpace{}%
\AgdaSymbol{→}\AgdaSpace{}%
\AgdaBound{P}\AgdaSpace{}%
\AgdaBound{x}\AgdaSpace{}%
\AgdaSymbol{→}\AgdaSpace{}%
\AgdaBound{P}\AgdaSpace{}%
\AgdaBound{y}\<%
\end{code}
\end{fence}

We cannot write the left-hand side of the equation as \texttt{x\ ≐\ y},
and instead we write \texttt{\_≐\_\ \{A\}\ x\ y} to provide access to
the implicit parameter \texttt{A} which appears on the right-hand side.

This is our first use of \emph{levels}. We cannot assign \texttt{Set}
the type \texttt{Set}, since this would lead to contradictions such as
Russell's Paradox and Girard's Paradox. Instead, there is a hierarchy of
types, where \texttt{Set\ :\ Set₁}, \texttt{Set₁\ :\ Set₂}, and so on.
In fact, \texttt{Set} itself is just an abbreviation for \texttt{Set₀}.
Since the equation defining \texttt{\_≐\_} mentions \texttt{Set} on the
right-hand side, the corresponding signature must use \texttt{Set₁}. We
say a bit more about levels below.

Leibniz equality is reflexive and transitive, where the first follows by
a variant of the identity function and the second by a variant of
function composition:

\begin{fence}
\begin{code}%
\>[0]\AgdaFunction{refl-≐}\AgdaSpace{}%
\AgdaSymbol{:}\AgdaSpace{}%
\AgdaSymbol{∀}\AgdaSpace{}%
\AgdaSymbol{\{}\AgdaBound{A}\AgdaSpace{}%
\AgdaSymbol{:}\AgdaSpace{}%
\AgdaPrimitiveType{Set}\AgdaSymbol{\}}\AgdaSpace{}%
\AgdaSymbol{\{}\AgdaBound{x}\AgdaSpace{}%
\AgdaSymbol{:}\AgdaSpace{}%
\AgdaBound{A}\AgdaSymbol{\}}\<%
\\
\>[0][@{}l@{\AgdaIndent{0}}]%
\>[2]\AgdaSymbol{→}\AgdaSpace{}%
\AgdaBound{x}\AgdaSpace{}%
\AgdaOperator{\AgdaFunction{≐}}\AgdaSpace{}%
\AgdaBound{x}\<%
\\
\>[0]\AgdaFunction{refl-≐}\AgdaSpace{}%
\AgdaBound{P}\AgdaSpace{}%
\AgdaBound{Px}%
\>[13]\AgdaSymbol{=}%
\>[16]\AgdaBound{Px}\<%
\\
%
\\[\AgdaEmptyExtraSkip]%
\>[0]\AgdaFunction{trans-≐}\AgdaSpace{}%
\AgdaSymbol{:}\AgdaSpace{}%
\AgdaSymbol{∀}\AgdaSpace{}%
\AgdaSymbol{\{}\AgdaBound{A}\AgdaSpace{}%
\AgdaSymbol{:}\AgdaSpace{}%
\AgdaPrimitiveType{Set}\AgdaSymbol{\}}\AgdaSpace{}%
\AgdaSymbol{\{}\AgdaBound{x}\AgdaSpace{}%
\AgdaBound{y}\AgdaSpace{}%
\AgdaBound{z}\AgdaSpace{}%
\AgdaSymbol{:}\AgdaSpace{}%
\AgdaBound{A}\AgdaSymbol{\}}\<%
\\
\>[0][@{}l@{\AgdaIndent{0}}]%
\>[2]\AgdaSymbol{→}\AgdaSpace{}%
\AgdaBound{x}\AgdaSpace{}%
\AgdaOperator{\AgdaFunction{≐}}\AgdaSpace{}%
\AgdaBound{y}\<%
\\
%
\>[2]\AgdaSymbol{→}%
\>[569I]\AgdaBound{y}\AgdaSpace{}%
\AgdaOperator{\AgdaFunction{≐}}\AgdaSpace{}%
\AgdaBound{z}\<%
\\
\>[.][@{}l@{}]\<[569I]%
\>[4]\AgdaComment{-----}\<%
\\
%
\>[2]\AgdaSymbol{→}\AgdaSpace{}%
\AgdaBound{x}\AgdaSpace{}%
\AgdaOperator{\AgdaFunction{≐}}\AgdaSpace{}%
\AgdaBound{z}\<%
\\
\>[0]\AgdaFunction{trans-≐}\AgdaSpace{}%
\AgdaBound{x≐y}\AgdaSpace{}%
\AgdaBound{y≐z}\AgdaSpace{}%
\AgdaBound{P}\AgdaSpace{}%
\AgdaBound{Px}%
\>[22]\AgdaSymbol{=}%
\>[25]\AgdaBound{y≐z}\AgdaSpace{}%
\AgdaBound{P}\AgdaSpace{}%
\AgdaSymbol{(}\AgdaBound{x≐y}\AgdaSpace{}%
\AgdaBound{P}\AgdaSpace{}%
\AgdaBound{Px}\AgdaSymbol{)}\<%
\end{code}
\end{fence}

Symmetry is less obvious. We have to show that if \texttt{P\ x} implies
\texttt{P\ y} for all predicates \texttt{P}, then the implication holds
the other way round as well:

\begin{fence}
\begin{code}%
\>[0]\AgdaFunction{sym-≐}\AgdaSpace{}%
\AgdaSymbol{:}\AgdaSpace{}%
\AgdaSymbol{∀}\AgdaSpace{}%
\AgdaSymbol{\{}\AgdaBound{A}\AgdaSpace{}%
\AgdaSymbol{:}\AgdaSpace{}%
\AgdaPrimitiveType{Set}\AgdaSymbol{\}}\AgdaSpace{}%
\AgdaSymbol{\{}\AgdaBound{x}\AgdaSpace{}%
\AgdaBound{y}\AgdaSpace{}%
\AgdaSymbol{:}\AgdaSpace{}%
\AgdaBound{A}\AgdaSymbol{\}}\<%
\\
\>[0][@{}l@{\AgdaIndent{0}}]%
\>[2]\AgdaSymbol{→}%
\>[592I]\AgdaBound{x}\AgdaSpace{}%
\AgdaOperator{\AgdaFunction{≐}}\AgdaSpace{}%
\AgdaBound{y}\<%
\\
\>[.][@{}l@{}]\<[592I]%
\>[4]\AgdaComment{-----}\<%
\\
%
\>[2]\AgdaSymbol{→}\AgdaSpace{}%
\AgdaBound{y}\AgdaSpace{}%
\AgdaOperator{\AgdaFunction{≐}}\AgdaSpace{}%
\AgdaBound{x}\<%
\\
\>[0]\AgdaFunction{sym-≐}\AgdaSpace{}%
\AgdaSymbol{\{}\AgdaBound{A}\AgdaSymbol{\}}\AgdaSpace{}%
\AgdaSymbol{\{}\AgdaBound{x}\AgdaSymbol{\}}\AgdaSpace{}%
\AgdaSymbol{\{}\AgdaBound{y}\AgdaSymbol{\}}\AgdaSpace{}%
\AgdaBound{x≐y}\AgdaSpace{}%
\AgdaBound{P}%
\>[25]\AgdaSymbol{=}%
\>[28]\AgdaFunction{Qy}\<%
\\
\>[0][@{}l@{\AgdaIndent{0}}]%
\>[2]\AgdaKeyword{where}\<%
\\
\>[2][@{}l@{\AgdaIndent{0}}]%
\>[4]\AgdaFunction{Q}\AgdaSpace{}%
\AgdaSymbol{:}\AgdaSpace{}%
\AgdaBound{A}\AgdaSpace{}%
\AgdaSymbol{→}\AgdaSpace{}%
\AgdaPrimitiveType{Set}\<%
\\
%
\>[4]\AgdaFunction{Q}\AgdaSpace{}%
\AgdaBound{z}\AgdaSpace{}%
\AgdaSymbol{=}\AgdaSpace{}%
\AgdaBound{P}\AgdaSpace{}%
\AgdaBound{z}\AgdaSpace{}%
\AgdaSymbol{→}\AgdaSpace{}%
\AgdaBound{P}\AgdaSpace{}%
\AgdaBound{x}\<%
\\
%
\>[4]\AgdaFunction{Qx}\AgdaSpace{}%
\AgdaSymbol{:}\AgdaSpace{}%
\AgdaFunction{Q}\AgdaSpace{}%
\AgdaBound{x}\<%
\\
%
\>[4]\AgdaFunction{Qx}\AgdaSpace{}%
\AgdaSymbol{=}\AgdaSpace{}%
\AgdaFunction{refl-≐}\AgdaSpace{}%
\AgdaBound{P}\<%
\\
%
\>[4]\AgdaFunction{Qy}\AgdaSpace{}%
\AgdaSymbol{:}\AgdaSpace{}%
\AgdaFunction{Q}\AgdaSpace{}%
\AgdaBound{y}\<%
\\
%
\>[4]\AgdaFunction{Qy}\AgdaSpace{}%
\AgdaSymbol{=}\AgdaSpace{}%
\AgdaBound{x≐y}\AgdaSpace{}%
\AgdaFunction{Q}\AgdaSpace{}%
\AgdaFunction{Qx}\<%
\end{code}
\end{fence}

Given \texttt{x\ ≐\ y}, a specific \texttt{P}, we have to construct a
proof that \texttt{P\ y} implies \texttt{P\ x}. To do so, we instantiate
the equality with a predicate \texttt{Q} such that \texttt{Q\ z} holds
if \texttt{P\ z} implies \texttt{P\ x}. The property \texttt{Q\ x} is
trivial by reflexivity, and hence \texttt{Q\ y} follows from
\texttt{x\ ≐\ y}. But \texttt{Q\ y} is exactly a proof of what we
require, that \texttt{P\ y} implies \texttt{P\ x}.

We now show that Martin-Löf equality implies Leibniz equality, and vice
versa. In the forward direction, if we know \texttt{x\ ≡\ y} we need for
any \texttt{P} to take evidence of \texttt{P\ x} to evidence of
\texttt{P\ y}, which is easy since equality of \texttt{x} and \texttt{y}
implies that any proof of \texttt{P\ x} is also a proof of
\texttt{P\ y}:

\begin{fence}
\begin{code}%
\>[0]\AgdaFunction{≡-implies-≐}\AgdaSpace{}%
\AgdaSymbol{:}\AgdaSpace{}%
\AgdaSymbol{∀}\AgdaSpace{}%
\AgdaSymbol{\{}\AgdaBound{A}\AgdaSpace{}%
\AgdaSymbol{:}\AgdaSpace{}%
\AgdaPrimitiveType{Set}\AgdaSymbol{\}}\AgdaSpace{}%
\AgdaSymbol{\{}\AgdaBound{x}\AgdaSpace{}%
\AgdaBound{y}\AgdaSpace{}%
\AgdaSymbol{:}\AgdaSpace{}%
\AgdaBound{A}\AgdaSymbol{\}}\<%
\\
\>[0][@{}l@{\AgdaIndent{0}}]%
\>[2]\AgdaSymbol{→}%
\>[636I]\AgdaBound{x}\AgdaSpace{}%
\AgdaOperator{\AgdaDatatype{≡}}\AgdaSpace{}%
\AgdaBound{y}\<%
\\
\>[.][@{}l@{}]\<[636I]%
\>[4]\AgdaComment{-----}\<%
\\
%
\>[2]\AgdaSymbol{→}\AgdaSpace{}%
\AgdaBound{x}\AgdaSpace{}%
\AgdaOperator{\AgdaFunction{≐}}\AgdaSpace{}%
\AgdaBound{y}\<%
\\
\>[0]\AgdaFunction{≡-implies-≐}\AgdaSpace{}%
\AgdaBound{x≡y}\AgdaSpace{}%
\AgdaBound{P}%
\>[19]\AgdaSymbol{=}%
\>[22]\AgdaFunction{subst}\AgdaSpace{}%
\AgdaBound{P}\AgdaSpace{}%
\AgdaBound{x≡y}\<%
\end{code}
\end{fence}

This direction follows from substitution, which we showed earlier.

In the reverse direction, given that for any \texttt{P} we can take a
proof of \texttt{P\ x} to a proof of \texttt{P\ y} we need to show
\texttt{x\ ≡\ y}:

\begin{fence}
\begin{code}%
\>[0]\AgdaFunction{≐-implies-≡}\AgdaSpace{}%
\AgdaSymbol{:}\AgdaSpace{}%
\AgdaSymbol{∀}\AgdaSpace{}%
\AgdaSymbol{\{}\AgdaBound{A}\AgdaSpace{}%
\AgdaSymbol{:}\AgdaSpace{}%
\AgdaPrimitiveType{Set}\AgdaSymbol{\}}\AgdaSpace{}%
\AgdaSymbol{\{}\AgdaBound{x}\AgdaSpace{}%
\AgdaBound{y}\AgdaSpace{}%
\AgdaSymbol{:}\AgdaSpace{}%
\AgdaBound{A}\AgdaSymbol{\}}\<%
\\
\>[0][@{}l@{\AgdaIndent{0}}]%
\>[2]\AgdaSymbol{→}%
\>[655I]\AgdaBound{x}\AgdaSpace{}%
\AgdaOperator{\AgdaFunction{≐}}\AgdaSpace{}%
\AgdaBound{y}\<%
\\
\>[.][@{}l@{}]\<[655I]%
\>[4]\AgdaComment{-----}\<%
\\
%
\>[2]\AgdaSymbol{→}\AgdaSpace{}%
\AgdaBound{x}\AgdaSpace{}%
\AgdaOperator{\AgdaDatatype{≡}}\AgdaSpace{}%
\AgdaBound{y}\<%
\\
\>[0]\AgdaFunction{≐-implies-≡}\AgdaSpace{}%
\AgdaSymbol{\{}\AgdaBound{A}\AgdaSymbol{\}}\AgdaSpace{}%
\AgdaSymbol{\{}\AgdaBound{x}\AgdaSymbol{\}}\AgdaSpace{}%
\AgdaSymbol{\{}\AgdaBound{y}\AgdaSymbol{\}}\AgdaSpace{}%
\AgdaBound{x≐y}%
\>[29]\AgdaSymbol{=}%
\>[32]\AgdaFunction{Qy}\<%
\\
\>[0][@{}l@{\AgdaIndent{0}}]%
\>[2]\AgdaKeyword{where}\<%
\\
\>[2][@{}l@{\AgdaIndent{0}}]%
\>[4]\AgdaFunction{Q}\AgdaSpace{}%
\AgdaSymbol{:}\AgdaSpace{}%
\AgdaBound{A}\AgdaSpace{}%
\AgdaSymbol{→}\AgdaSpace{}%
\AgdaPrimitiveType{Set}\<%
\\
%
\>[4]\AgdaFunction{Q}\AgdaSpace{}%
\AgdaBound{z}\AgdaSpace{}%
\AgdaSymbol{=}\AgdaSpace{}%
\AgdaBound{x}\AgdaSpace{}%
\AgdaOperator{\AgdaDatatype{≡}}\AgdaSpace{}%
\AgdaBound{z}\<%
\\
%
\>[4]\AgdaFunction{Qx}\AgdaSpace{}%
\AgdaSymbol{:}\AgdaSpace{}%
\AgdaFunction{Q}\AgdaSpace{}%
\AgdaBound{x}\<%
\\
%
\>[4]\AgdaFunction{Qx}\AgdaSpace{}%
\AgdaSymbol{=}\AgdaSpace{}%
\AgdaInductiveConstructor{refl}\<%
\\
%
\>[4]\AgdaFunction{Qy}\AgdaSpace{}%
\AgdaSymbol{:}\AgdaSpace{}%
\AgdaFunction{Q}\AgdaSpace{}%
\AgdaBound{y}\<%
\\
%
\>[4]\AgdaFunction{Qy}\AgdaSpace{}%
\AgdaSymbol{=}\AgdaSpace{}%
\AgdaBound{x≐y}\AgdaSpace{}%
\AgdaFunction{Q}\AgdaSpace{}%
\AgdaFunction{Qx}\<%
\end{code}
\end{fence}

The proof is similar to that for symmetry of Leibniz equality. We take
\texttt{Q} to be the predicate that holds of \texttt{z} if
\texttt{x\ ≡\ z}. Then \texttt{Q\ x} is trivial by reflexivity of
Martin-Löf equality, and hence \texttt{Q\ y} follows from
\texttt{x\ ≐\ y}. But \texttt{Q\ y} is exactly a proof of what we
require, that \texttt{x\ ≡\ y}.

(Parts of this section are adapted from \emph{≐≃≡: Leibniz Equality is
Isomorphic to Martin-Löf Identity, Parametrically}, by Andreas Abel,
Jesper Cockx, Dominique Devries, Andreas Nuyts, and Philip Wadler,
draft, 2017.)

\hypertarget{Equality-unipoly}{%
\section{Universe polymorphism}\label{Equality-unipoly}}

As we have seen, not every type belongs to \texttt{Set}, but instead
every type belongs somewhere in the hierarchy \texttt{Set₀},
\texttt{Set₁}, \texttt{Set₂}, and so on, where \texttt{Set} abbreviates
\texttt{Set₀}, and \texttt{Set₀\ :\ Set₁}, \texttt{Set₁\ :\ Set₂}, and
so on. The definition of equality given above is fine if we want to
compare two values of a type that belongs to \texttt{Set}, but what if
we want to compare two values of a type that belongs to \texttt{Set\ ℓ}
for some arbitrary level \texttt{ℓ}?

The answer is \emph{universe polymorphism}, where a definition is made
with respect to an arbitrary level \texttt{ℓ}. To make use of levels, we
first import the following:

\begin{fence}
\begin{code}%
\>[0]\AgdaKeyword{open}\AgdaSpace{}%
\AgdaKeyword{import}\AgdaSpace{}%
\AgdaModule{Level}\AgdaSpace{}%
\AgdaKeyword{using}\AgdaSpace{}%
\AgdaSymbol{(}\AgdaPostulate{Level}\AgdaSymbol{;}\AgdaSpace{}%
\AgdaOperator{\AgdaPrimitive{\AgdaUnderscore{}⊔\AgdaUnderscore{}}}\AgdaSymbol{)}\AgdaSpace{}%
\AgdaKeyword{renaming}\AgdaSpace{}%
\AgdaSymbol{(}\AgdaPrimitive{zero}\AgdaSpace{}%
\AgdaSymbol{to}\AgdaSpace{}%
\AgdaPrimitive{lzero}\AgdaSymbol{;}\AgdaSpace{}%
\AgdaPrimitive{suc}\AgdaSpace{}%
\AgdaSymbol{to}\AgdaSpace{}%
\AgdaPrimitive{lsuc}\AgdaSymbol{)}\<%
\end{code}
\end{fence}

We rename constructors \texttt{zero} and \texttt{suc} to \texttt{lzero}
and \texttt{lsuc} to avoid confusion between levels and naturals.

Levels are isomorphic to natural numbers, and have similar constructors:

\begin{myDisplay}
lzero : Level
lsuc  : Level → Level
\end{myDisplay}

The names \texttt{Set₀}, \texttt{Set₁}, \texttt{Set₂}, and so on, are
abbreviations for

\begin{myDisplay}
Set lzero
Set (lsuc lzero)
Set (lsuc (lsuc lzero))
\end{myDisplay}

and so on. There is also an operator

\begin{myDisplay}
_⊔_ : Level → Level → Level
\end{myDisplay}

that given two levels returns the larger of the two.

Here is the definition of equality, generalised to an arbitrary level:

\begin{fence}
\begin{code}%
\>[0]\AgdaKeyword{data}\AgdaSpace{}%
\AgdaOperator{\AgdaDatatype{\AgdaUnderscore{}≡′\AgdaUnderscore{}}}\AgdaSpace{}%
\AgdaSymbol{\{}\AgdaBound{ℓ}\AgdaSpace{}%
\AgdaSymbol{:}\AgdaSpace{}%
\AgdaPostulate{Level}\AgdaSymbol{\}}\AgdaSpace{}%
\AgdaSymbol{\{}\AgdaBound{A}\AgdaSpace{}%
\AgdaSymbol{:}\AgdaSpace{}%
\AgdaPrimitiveType{Set}\AgdaSpace{}%
\AgdaBound{ℓ}\AgdaSymbol{\}}\AgdaSpace{}%
\AgdaSymbol{(}\AgdaBound{x}\AgdaSpace{}%
\AgdaSymbol{:}\AgdaSpace{}%
\AgdaBound{A}\AgdaSymbol{)}\AgdaSpace{}%
\AgdaSymbol{:}\AgdaSpace{}%
\AgdaBound{A}\AgdaSpace{}%
\AgdaSymbol{→}\AgdaSpace{}%
\AgdaPrimitiveType{Set}\AgdaSpace{}%
\AgdaBound{ℓ}\AgdaSpace{}%
\AgdaKeyword{where}\<%
\\
\>[0][@{}l@{\AgdaIndent{0}}]%
\>[2]\AgdaInductiveConstructor{refl′}\AgdaSpace{}%
\AgdaSymbol{:}\AgdaSpace{}%
\AgdaBound{x}\AgdaSpace{}%
\AgdaOperator{\AgdaDatatype{≡′}}\AgdaSpace{}%
\AgdaBound{x}\<%
\end{code}
\end{fence}

Similarly, here is the generalised definition of symmetry:

\begin{fence}
\begin{code}%
\>[0]\AgdaFunction{sym′}\AgdaSpace{}%
\AgdaSymbol{:}\AgdaSpace{}%
\AgdaSymbol{∀}\AgdaSpace{}%
\AgdaSymbol{\{}\AgdaBound{ℓ}\AgdaSpace{}%
\AgdaSymbol{:}\AgdaSpace{}%
\AgdaPostulate{Level}\AgdaSymbol{\}}\AgdaSpace{}%
\AgdaSymbol{\{}\AgdaBound{A}\AgdaSpace{}%
\AgdaSymbol{:}\AgdaSpace{}%
\AgdaPrimitiveType{Set}\AgdaSpace{}%
\AgdaBound{ℓ}\AgdaSymbol{\}}\AgdaSpace{}%
\AgdaSymbol{\{}\AgdaBound{x}\AgdaSpace{}%
\AgdaBound{y}\AgdaSpace{}%
\AgdaSymbol{:}\AgdaSpace{}%
\AgdaBound{A}\AgdaSymbol{\}}\<%
\\
\>[0][@{}l@{\AgdaIndent{0}}]%
\>[2]\AgdaSymbol{→}%
\>[732I]\AgdaBound{x}\AgdaSpace{}%
\AgdaOperator{\AgdaDatatype{≡′}}\AgdaSpace{}%
\AgdaBound{y}\<%
\\
\>[.][@{}l@{}]\<[732I]%
\>[4]\AgdaComment{------}\<%
\\
%
\>[2]\AgdaSymbol{→}\AgdaSpace{}%
\AgdaBound{y}\AgdaSpace{}%
\AgdaOperator{\AgdaDatatype{≡′}}\AgdaSpace{}%
\AgdaBound{x}\<%
\\
\>[0]\AgdaFunction{sym′}\AgdaSpace{}%
\AgdaInductiveConstructor{refl′}\AgdaSpace{}%
\AgdaSymbol{=}\AgdaSpace{}%
\AgdaInductiveConstructor{refl′}\<%
\end{code}
\end{fence}

For simplicity, we avoid universe polymorphism in the definitions given
in the text, but most definitions in the standard library, including
those for equality, are generalised to arbitrary levels as above.

Here is the generalised definition of Leibniz equality:

\begin{fence}
\begin{code}%
\>[0]\AgdaOperator{\AgdaFunction{\AgdaUnderscore{}≐′\AgdaUnderscore{}}}\AgdaSpace{}%
\AgdaSymbol{:}\AgdaSpace{}%
\AgdaSymbol{∀}\AgdaSpace{}%
\AgdaSymbol{\{}\AgdaBound{ℓ}\AgdaSpace{}%
\AgdaSymbol{:}\AgdaSpace{}%
\AgdaPostulate{Level}\AgdaSymbol{\}}\AgdaSpace{}%
\AgdaSymbol{\{}\AgdaBound{A}\AgdaSpace{}%
\AgdaSymbol{:}\AgdaSpace{}%
\AgdaPrimitiveType{Set}\AgdaSpace{}%
\AgdaBound{ℓ}\AgdaSymbol{\}}\AgdaSpace{}%
\AgdaSymbol{(}\AgdaBound{x}\AgdaSpace{}%
\AgdaBound{y}\AgdaSpace{}%
\AgdaSymbol{:}\AgdaSpace{}%
\AgdaBound{A}\AgdaSymbol{)}\AgdaSpace{}%
\AgdaSymbol{→}\AgdaSpace{}%
\AgdaPrimitiveType{Set}\AgdaSpace{}%
\AgdaSymbol{(}\AgdaPrimitive{lsuc}\AgdaSpace{}%
\AgdaBound{ℓ}\AgdaSymbol{)}\<%
\\
\>[0]\AgdaOperator{\AgdaFunction{\AgdaUnderscore{}≐′\AgdaUnderscore{}}}\AgdaSpace{}%
\AgdaSymbol{\{}\AgdaBound{ℓ}\AgdaSymbol{\}}\AgdaSpace{}%
\AgdaSymbol{\{}\AgdaBound{A}\AgdaSymbol{\}}\AgdaSpace{}%
\AgdaBound{x}\AgdaSpace{}%
\AgdaBound{y}\AgdaSpace{}%
\AgdaSymbol{=}\AgdaSpace{}%
\AgdaSymbol{∀}\AgdaSpace{}%
\AgdaSymbol{(}\AgdaBound{P}\AgdaSpace{}%
\AgdaSymbol{:}\AgdaSpace{}%
\AgdaBound{A}\AgdaSpace{}%
\AgdaSymbol{→}\AgdaSpace{}%
\AgdaPrimitiveType{Set}\AgdaSpace{}%
\AgdaBound{ℓ}\AgdaSymbol{)}\AgdaSpace{}%
\AgdaSymbol{→}\AgdaSpace{}%
\AgdaBound{P}\AgdaSpace{}%
\AgdaBound{x}\AgdaSpace{}%
\AgdaSymbol{→}\AgdaSpace{}%
\AgdaBound{P}\AgdaSpace{}%
\AgdaBound{y}\<%
\end{code}
\end{fence}

Before the signature used \texttt{Set₁} as the type of a term that
includes \texttt{Set}, whereas here the signature uses
\texttt{Set\ (lsuc\ ℓ)} as the type of a term that includes
\texttt{Set\ ℓ}.

Most other functions in the standard library are also generalised to
arbitrary levels. For instance, here is the definition of composition.

\begin{fence}
\begin{code}%
\>[0]\AgdaOperator{\AgdaFunction{\AgdaUnderscore{}∘\AgdaUnderscore{}}}\AgdaSpace{}%
\AgdaSymbol{:}\AgdaSpace{}%
\AgdaSymbol{∀}\AgdaSpace{}%
\AgdaSymbol{\{}\AgdaBound{ℓ₁}\AgdaSpace{}%
\AgdaBound{ℓ₂}\AgdaSpace{}%
\AgdaBound{ℓ₃}\AgdaSpace{}%
\AgdaSymbol{:}\AgdaSpace{}%
\AgdaPostulate{Level}\AgdaSymbol{\}}\AgdaSpace{}%
\AgdaSymbol{\{}\AgdaBound{A}\AgdaSpace{}%
\AgdaSymbol{:}\AgdaSpace{}%
\AgdaPrimitiveType{Set}\AgdaSpace{}%
\AgdaBound{ℓ₁}\AgdaSymbol{\}}\AgdaSpace{}%
\AgdaSymbol{\{}\AgdaBound{B}\AgdaSpace{}%
\AgdaSymbol{:}\AgdaSpace{}%
\AgdaPrimitiveType{Set}\AgdaSpace{}%
\AgdaBound{ℓ₂}\AgdaSymbol{\}}\AgdaSpace{}%
\AgdaSymbol{\{}\AgdaBound{C}\AgdaSpace{}%
\AgdaSymbol{:}\AgdaSpace{}%
\AgdaPrimitiveType{Set}\AgdaSpace{}%
\AgdaBound{ℓ₃}\AgdaSymbol{\}}\<%
\\
\>[0][@{}l@{\AgdaIndent{0}}]%
\>[2]\AgdaSymbol{→}\AgdaSpace{}%
\AgdaSymbol{(}\AgdaBound{B}\AgdaSpace{}%
\AgdaSymbol{→}\AgdaSpace{}%
\AgdaBound{C}\AgdaSymbol{)}\AgdaSpace{}%
\AgdaSymbol{→}\AgdaSpace{}%
\AgdaSymbol{(}\AgdaBound{A}\AgdaSpace{}%
\AgdaSymbol{→}\AgdaSpace{}%
\AgdaBound{B}\AgdaSymbol{)}\AgdaSpace{}%
\AgdaSymbol{→}\AgdaSpace{}%
\AgdaBound{A}\AgdaSpace{}%
\AgdaSymbol{→}\AgdaSpace{}%
\AgdaBound{C}\<%
\\
\>[0]\AgdaSymbol{(}\AgdaBound{g}\AgdaSpace{}%
\AgdaOperator{\AgdaFunction{∘}}\AgdaSpace{}%
\AgdaBound{f}\AgdaSymbol{)}\AgdaSpace{}%
\AgdaBound{x}%
\>[11]\AgdaSymbol{=}%
\>[14]\AgdaBound{g}\AgdaSpace{}%
\AgdaSymbol{(}\AgdaBound{f}\AgdaSpace{}%
\AgdaBound{x}\AgdaSymbol{)}\<%
\end{code}
\end{fence}

Further information on levels can be found in the
\href{https://agda.readthedocs.io/en/v2.6.1/language/universe-levels.html}{Agda
docs}.

\hypertarget{standard-library}{%
\section{Standard library}\label{standard-library}}

Definitions similar to those in this chapter can be found in the
standard library. The Agda standard library defines \texttt{\_≡⟨\_⟩\_}
as \texttt{step-≡},
\href{https://github.com/agda/agda-stdlib/blob/master/CHANGELOG/v1.3.md\#changes-to-how-equational-reasoning-is-implemented}{which
reverses the order of the arguments}. The standard library also defines
a syntax macro, which is automatically imported whenever you import
\texttt{step-≡}, which recovers the original argument order:

\begin{fence}
\begin{code}%
\>[0]\AgdaComment{-- import Relation.Binary.PropositionalEquality as Eq}\<%
\\
\>[0]\AgdaComment{-- open Eq using (\AgdaUnderscore{}≡\AgdaUnderscore{}; refl; trans; sym; cong; cong-app; subst)}\<%
\\
\>[0]\AgdaComment{-- open Eq.≡-Reasoning using (begin\AgdaUnderscore{}; \AgdaUnderscore{}≡⟨⟩\AgdaUnderscore{}; step-≡; \AgdaUnderscore{}∎)}\<%
\end{code}
\end{fence}

Here the imports are shown as comments rather than code to avoid
collisions, as mentioned in the introduction.

\hypertarget{unicode}{%
\section{Unicode}\label{unicode}}

This chapter uses the following unicode:

\begin{myDisplay}
≡  U+2261  IDENTICAL TO (\==, \equiv)
⟨  U+27E8  MATHEMATICAL LEFT ANGLE BRACKET (\<)
⟩  U+27E9  MATHEMATICAL RIGHT ANGLE BRACKET (\>)
∎  U+220E  END OF PROOF (\qed)
≐  U+2250  APPROACHES THE LIMIT (\.=)
ℓ  U+2113  SCRIPT SMALL L (\ell)
⊔  U+2294  SQUARE CUP (\lub)
₀  U+2080  SUBSCRIPT ZERO (\_0)
₁  U+2081  SUBSCRIPT ONE (\_1)
₂  U+2082  SUBSCRIPT TWO (\_2)
\end{myDisplay}

