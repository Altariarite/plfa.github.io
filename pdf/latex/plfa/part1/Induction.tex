\hypertarget{Induction}{%
\chapter{Induction: Proof by Induction}\label{Induction}}

\begin{fence}
\begin{code}%
\>[0]\AgdaKeyword{module}\AgdaSpace{}%
\AgdaModule{plfa.part1.Induction}\AgdaSpace{}%
\AgdaKeyword{where}\<%
\end{code}
\end{fence}

\begin{quote}
Induction makes you feel guilty for getting something out of nothing
\ldots{} but it is one of the greatest ideas of civilization. -- Herbert
Wilf
\end{quote}

Now that we've defined the naturals and operations upon them, our next
step is to learn how to prove properties that they satisfy. As hinted by
their name, properties of \emph{inductive datatypes} are proved by
\emph{induction}.

\hypertarget{imports}{%
\section{Imports}\label{imports}}

We require equality as in the previous chapter, plus the naturals and
some operations upon them. We also import a couple of new operations,
\texttt{cong}, \texttt{sym}, and \texttt{\_≡⟨\_⟩\_}, which are explained
below:

\begin{fence}
\begin{code}%
\>[0]\AgdaKeyword{import}\AgdaSpace{}%
\AgdaModule{Relation.Binary.PropositionalEquality}\AgdaSpace{}%
\AgdaSymbol{as}\AgdaSpace{}%
\AgdaModule{Eq}\<%
\\
\>[0]\AgdaKeyword{open}\AgdaSpace{}%
\AgdaModule{Eq}\AgdaSpace{}%
\AgdaKeyword{using}\AgdaSpace{}%
\AgdaSymbol{(}\AgdaOperator{\AgdaDatatype{\AgdaUnderscore{}≡\AgdaUnderscore{}}}\AgdaSymbol{;}\AgdaSpace{}%
\AgdaInductiveConstructor{refl}\AgdaSymbol{;}\AgdaSpace{}%
\AgdaFunction{cong}\AgdaSymbol{;}\AgdaSpace{}%
\AgdaFunction{sym}\AgdaSymbol{)}\<%
\\
\>[0]\AgdaKeyword{open}\AgdaSpace{}%
\AgdaModule{Eq.≡-Reasoning}\AgdaSpace{}%
\AgdaKeyword{using}\AgdaSpace{}%
\AgdaSymbol{(}\AgdaOperator{\AgdaFunction{begin\AgdaUnderscore{}}}\AgdaSymbol{;}\AgdaSpace{}%
\AgdaOperator{\AgdaFunction{\AgdaUnderscore{}≡⟨⟩\AgdaUnderscore{}}}\AgdaSymbol{;}\AgdaSpace{}%
\AgdaFunction{step-≡}\AgdaSymbol{;}\AgdaSpace{}%
\AgdaOperator{\AgdaFunction{\AgdaUnderscore{}∎}}\AgdaSymbol{)}\<%
\\
\>[0]\AgdaKeyword{open}\AgdaSpace{}%
\AgdaKeyword{import}\AgdaSpace{}%
\AgdaModule{Data.Nat}\AgdaSpace{}%
\AgdaKeyword{using}\AgdaSpace{}%
\AgdaSymbol{(}\AgdaDatatype{ℕ}\AgdaSymbol{;}\AgdaSpace{}%
\AgdaInductiveConstructor{zero}\AgdaSymbol{;}\AgdaSpace{}%
\AgdaInductiveConstructor{suc}\AgdaSymbol{;}\AgdaSpace{}%
\AgdaOperator{\AgdaPrimitive{\AgdaUnderscore{}+\AgdaUnderscore{}}}\AgdaSymbol{;}\AgdaSpace{}%
\AgdaOperator{\AgdaPrimitive{\AgdaUnderscore{}*\AgdaUnderscore{}}}\AgdaSymbol{;}\AgdaSpace{}%
\AgdaOperator{\AgdaPrimitive{\AgdaUnderscore{}∸\AgdaUnderscore{}}}\AgdaSymbol{)}\<%
\end{code}
\end{fence}

\hypertarget{properties-of-operators}{%
\section{Properties of operators}\label{properties-of-operators}}

Operators pop up all the time, and mathematicians have agreed on names
for some of the most common properties.

\begin{itemize}
\item
  \emph{Identity}. Operator \texttt{+} has left identity \texttt{0} if
  \texttt{0\ +\ n\ ≡\ n}, and right identity \texttt{0} if
  \texttt{n\ +\ 0\ ≡\ n}, for all \texttt{n}. A value that is both a
  left and right identity is just called an identity. Identity is also
  sometimes called \emph{unit}.
\item
  \emph{Associativity}. Operator \texttt{+} is associative if the
  location of parentheses does not matter:
  \texttt{(m\ +\ n)\ +\ p\ ≡\ m\ +\ (n\ +\ p)}, for all \texttt{m},
  \texttt{n}, and \texttt{p}.
\item
  \emph{Commutativity}. Operator \texttt{+} is commutative if order of
  arguments does not matter: \texttt{m\ +\ n\ ≡\ n\ +\ m}, for all
  \texttt{m} and \texttt{n}.
\item
  \emph{Distributivity}. Operator \texttt{*} distributes over operator
  \texttt{+} from the left if
  \texttt{(m\ +\ n)\ *\ p\ ≡\ (m\ *\ p)\ +\ (n\ *\ p)}, for all
  \texttt{m}, \texttt{n}, and \texttt{p}, and from the right if
  \texttt{m\ *\ (p\ +\ q)\ ≡\ (m\ *\ p)\ +\ (m\ *\ q)}, for all
  \texttt{m}, \texttt{p}, and \texttt{q}.
\end{itemize}

Addition has identity \texttt{0} and multiplication has identity
\texttt{1}; addition and multiplication are both associative and
commutative; and multiplication distributes over addition.

If you ever bump into an operator at a party, you now know how to make
small talk, by asking whether it has a unit and is associative or
commutative. If you bump into two operators, you might ask them if one
distributes over the other.

Less frivolously, if you ever bump into an operator while reading a
technical paper, this gives you a way to orient yourself, by checking
whether or not it has an identity, is associative or commutative, or
distributes over another operator. A careful author will often call out
these properties---or their lack---for instance by pointing out that a
newly introduced operator is associative but not commutative.

\hypertarget{Induction-operators}{%
\subsubsection{\texorpdfstring{Exercise \texttt{operators}
(practice)}{Exercise operators (practice)}}\label{Induction-operators}}

Give another example of a pair of operators that have an identity and
are associative, commutative, and distribute over one another. (You do
not have to prove these properties.)

Give an example of an operator that has an identity and is associative
but is not commutative. (You do not have to prove these properties.)

\hypertarget{associativity}{%
\section{Associativity}\label{associativity}}

One property of addition is that it is \emph{associative}, that is, that
the location of the parentheses does not matter:

\begin{myDisplay}
(m + n) + p ≡ m + (n + p)
\end{myDisplay}

Here \texttt{m}, \texttt{n}, and \texttt{p} are variables that range
over all natural numbers.

We can test the proposition by choosing specific numbers for the three
variables:

\begin{fence}
\begin{code}%
\>[0]\AgdaFunction{\AgdaUnderscore{}}\AgdaSpace{}%
\AgdaSymbol{:}\AgdaSpace{}%
\AgdaSymbol{(}\AgdaNumber{3}\AgdaSpace{}%
\AgdaOperator{\AgdaPrimitive{+}}\AgdaSpace{}%
\AgdaNumber{4}\AgdaSymbol{)}\AgdaSpace{}%
\AgdaOperator{\AgdaPrimitive{+}}\AgdaSpace{}%
\AgdaNumber{5}\AgdaSpace{}%
\AgdaOperator{\AgdaDatatype{≡}}\AgdaSpace{}%
\AgdaNumber{3}\AgdaSpace{}%
\AgdaOperator{\AgdaPrimitive{+}}\AgdaSpace{}%
\AgdaSymbol{(}\AgdaNumber{4}\AgdaSpace{}%
\AgdaOperator{\AgdaPrimitive{+}}\AgdaSpace{}%
\AgdaNumber{5}\AgdaSymbol{)}\<%
\\
\>[0]\AgdaSymbol{\AgdaUnderscore{}}%
\>[38I]\AgdaSymbol{=}\<%
\\
\>[.][@{}l@{}]\<[38I]%
\>[2]\AgdaOperator{\AgdaFunction{begin}}\<%
\\
\>[2][@{}l@{\AgdaIndent{0}}]%
\>[4]\AgdaSymbol{(}\AgdaNumber{3}\AgdaSpace{}%
\AgdaOperator{\AgdaPrimitive{+}}\AgdaSpace{}%
\AgdaNumber{4}\AgdaSymbol{)}\AgdaSpace{}%
\AgdaOperator{\AgdaPrimitive{+}}\AgdaSpace{}%
\AgdaNumber{5}\<%
\\
%
\>[2]\AgdaOperator{\AgdaFunction{≡⟨⟩}}\<%
\\
\>[2][@{}l@{\AgdaIndent{0}}]%
\>[4]\AgdaNumber{7}\AgdaSpace{}%
\AgdaOperator{\AgdaPrimitive{+}}\AgdaSpace{}%
\AgdaNumber{5}\<%
\\
%
\>[2]\AgdaOperator{\AgdaFunction{≡⟨⟩}}\<%
\\
\>[2][@{}l@{\AgdaIndent{0}}]%
\>[4]\AgdaNumber{12}\<%
\\
%
\>[2]\AgdaOperator{\AgdaFunction{≡⟨⟩}}\<%
\\
\>[2][@{}l@{\AgdaIndent{0}}]%
\>[4]\AgdaNumber{3}\AgdaSpace{}%
\AgdaOperator{\AgdaPrimitive{+}}\AgdaSpace{}%
\AgdaNumber{9}\<%
\\
%
\>[2]\AgdaOperator{\AgdaFunction{≡⟨⟩}}\<%
\\
\>[2][@{}l@{\AgdaIndent{0}}]%
\>[4]\AgdaNumber{3}\AgdaSpace{}%
\AgdaOperator{\AgdaPrimitive{+}}\AgdaSpace{}%
\AgdaSymbol{(}\AgdaNumber{4}\AgdaSpace{}%
\AgdaOperator{\AgdaPrimitive{+}}\AgdaSpace{}%
\AgdaNumber{5}\AgdaSymbol{)}\<%
\\
%
\>[2]\AgdaOperator{\AgdaFunction{∎}}\<%
\end{code}
\end{fence}

Here we have displayed the computation as a chain of equations, one term
to a line. It is often easiest to read such chains from the top down
until one reaches the simplest term (in this case, \texttt{12}), and
then from the bottom up until one reaches the same term.

The test reveals that associativity is perhaps not as obvious as first
it appears. Why should \texttt{7\ +\ 5} be the same as \texttt{3\ +\ 9}?
We might want to gather more evidence, testing the proposition by
choosing other numbers. But---since there are an infinite number of
naturals---testing can never be complete. Is there any way we can be
sure that associativity holds for \emph{all} the natural numbers?

The answer is yes! We can prove a property holds for all naturals using
\emph{proof by induction}.

\hypertarget{proof-by-induction}{%
\section{Proof by induction}\label{proof-by-induction}}

Recall the definition of natural numbers consists of a \emph{base case}
which tells us that \texttt{zero} is a natural, and an \emph{inductive
case} which tells us that if \texttt{m} is a natural then
\texttt{suc\ m} is also a natural.

Proof by induction follows the structure of this definition. To prove a
property of natural numbers by induction, we need to prove two cases.
First is the \emph{base case}, where we show the property holds for
\texttt{zero}. Second is the \emph{inductive case}, where we assume the
property holds for an arbitrary natural \texttt{m} (we call this the
\emph{inductive hypothesis}), and then show that the property must also
hold for \texttt{suc\ m}.

If we write \texttt{P\ m} for a property of \texttt{m}, then what we
need to demonstrate are the following two inference rules:

\begin{myDisplay}
------
P zero

P m
---------
P (suc m)
\end{myDisplay}

Let's unpack these rules. The first rule is the base case, and requires
we show that property \texttt{P} holds for \texttt{zero}. The second
rule is the inductive case, and requires we show that if we assume the
inductive hypothesis---namely that \texttt{P} holds for
\texttt{m}---then it follows that \texttt{P} also holds for
\texttt{suc\ m}.

Why does this work? Again, it can be explained by a creation story. To
start with, we know no properties:

\begin{myDisplay}
-- In the beginning, no properties are known.
\end{myDisplay}

Now, we apply the two rules to all the properties we know about. The
base case tells us that \texttt{P\ zero} holds, so we add it to the set
of known properties. The inductive case tells us that if \texttt{P\ m}
holds (on the day before today) then \texttt{P\ (suc\ m)} also holds
(today). We didn't know about any properties before today, so the
inductive case doesn't apply:

\begin{myDisplay}
-- On the first day, one property is known.
P zero
\end{myDisplay}

Then we repeat the process, so on the next day we know about all the
properties from the day before, plus any properties added by the rules.
The base case tells us that \texttt{P\ zero} holds, but we already knew
that. But now the inductive case tells us that since \texttt{P\ zero}
held yesterday, then \texttt{P\ (suc\ zero)} holds today:

\begin{myDisplay}
-- On the second day, two properties are known.
P zero
P (suc zero)
\end{myDisplay}

And we repeat the process again. Now the inductive case tells us that
since \texttt{P\ zero} and \texttt{P\ (suc\ zero)} both hold, then
\texttt{P\ (suc\ zero)} and \texttt{P\ (suc\ (suc\ zero))} also hold. We
already knew about the first of these, but the second is new:

\begin{myDisplay}
-- On the third day, three properties are known.
P zero
P (suc zero)
P (suc (suc zero))
\end{myDisplay}

You've got the hang of it by now:

\begin{myDisplay}
-- On the fourth day, four properties are known.
P zero
P (suc zero)
P (suc (suc zero))
P (suc (suc (suc zero)))
\end{myDisplay}

The process continues. On the \emph{n}'th day there will be \emph{n}
distinct properties that hold. The property of every natural number will
appear on some given day. In particular, the property \texttt{P\ n}
first appears on day \emph{n+1}.

\hypertarget{our-first-proof-associativity}{%
\section{Our first proof:
associativity}\label{our-first-proof-associativity}}

To prove associativity, we take \texttt{P\ m} to be the property:

\begin{myDisplay}
(m + n) + p ≡ m + (n + p)
\end{myDisplay}

Here \texttt{n} and \texttt{p} are arbitrary natural numbers, so if we
can show the equation holds for all \texttt{m} it will also hold for all
\texttt{n} and \texttt{p}. The appropriate instances of the inference
rules are:

\begin{myDisplay}
-------------------------------
(zero + n) + p ≡ zero + (n + p)

(m + n) + p ≡ m + (n + p)
---------------------------------
(suc m + n) + p ≡ suc m + (n + p)
\end{myDisplay}

If we can demonstrate both of these, then associativity of addition
follows by induction.

Here is the proposition's statement and proof:

\begin{fence}
\begin{code}%
\>[0]\AgdaFunction{+-assoc}\AgdaSpace{}%
\AgdaSymbol{:}\AgdaSpace{}%
\AgdaSymbol{∀}\AgdaSpace{}%
\AgdaSymbol{(}\AgdaBound{m}\AgdaSpace{}%
\AgdaBound{n}\AgdaSpace{}%
\AgdaBound{p}\AgdaSpace{}%
\AgdaSymbol{:}\AgdaSpace{}%
\AgdaDatatype{ℕ}\AgdaSymbol{)}\AgdaSpace{}%
\AgdaSymbol{→}\AgdaSpace{}%
\AgdaSymbol{(}\AgdaBound{m}\AgdaSpace{}%
\AgdaOperator{\AgdaPrimitive{+}}\AgdaSpace{}%
\AgdaBound{n}\AgdaSymbol{)}\AgdaSpace{}%
\AgdaOperator{\AgdaPrimitive{+}}\AgdaSpace{}%
\AgdaBound{p}\AgdaSpace{}%
\AgdaOperator{\AgdaDatatype{≡}}\AgdaSpace{}%
\AgdaBound{m}\AgdaSpace{}%
\AgdaOperator{\AgdaPrimitive{+}}\AgdaSpace{}%
\AgdaSymbol{(}\AgdaBound{n}\AgdaSpace{}%
\AgdaOperator{\AgdaPrimitive{+}}\AgdaSpace{}%
\AgdaBound{p}\AgdaSymbol{)}\<%
\\
\>[0]\AgdaFunction{+-assoc}\AgdaSpace{}%
\AgdaInductiveConstructor{zero}\AgdaSpace{}%
\AgdaBound{n}\AgdaSpace{}%
\AgdaBound{p}\AgdaSpace{}%
\AgdaSymbol{=}\<%
\\
\>[0][@{}l@{\AgdaIndent{0}}]%
\>[2]\AgdaOperator{\AgdaFunction{begin}}\<%
\\
\>[2][@{}l@{\AgdaIndent{0}}]%
\>[4]\AgdaSymbol{(}\AgdaInductiveConstructor{zero}\AgdaSpace{}%
\AgdaOperator{\AgdaPrimitive{+}}\AgdaSpace{}%
\AgdaBound{n}\AgdaSymbol{)}\AgdaSpace{}%
\AgdaOperator{\AgdaPrimitive{+}}\AgdaSpace{}%
\AgdaBound{p}\<%
\\
%
\>[2]\AgdaOperator{\AgdaFunction{≡⟨⟩}}\<%
\\
\>[2][@{}l@{\AgdaIndent{0}}]%
\>[4]\AgdaBound{n}\AgdaSpace{}%
\AgdaOperator{\AgdaPrimitive{+}}\AgdaSpace{}%
\AgdaBound{p}\<%
\\
%
\>[2]\AgdaOperator{\AgdaFunction{≡⟨⟩}}\<%
\\
\>[2][@{}l@{\AgdaIndent{0}}]%
\>[4]\AgdaInductiveConstructor{zero}\AgdaSpace{}%
\AgdaOperator{\AgdaPrimitive{+}}\AgdaSpace{}%
\AgdaSymbol{(}\AgdaBound{n}\AgdaSpace{}%
\AgdaOperator{\AgdaPrimitive{+}}\AgdaSpace{}%
\AgdaBound{p}\AgdaSymbol{)}\<%
\\
%
\>[2]\AgdaOperator{\AgdaFunction{∎}}\<%
\\
\>[0]\AgdaFunction{+-assoc}\AgdaSpace{}%
\AgdaSymbol{(}\AgdaInductiveConstructor{suc}\AgdaSpace{}%
\AgdaBound{m}\AgdaSymbol{)}\AgdaSpace{}%
\AgdaBound{n}\AgdaSpace{}%
\AgdaBound{p}\AgdaSpace{}%
\AgdaSymbol{=}\<%
\\
\>[0][@{}l@{\AgdaIndent{0}}]%
\>[2]\AgdaOperator{\AgdaFunction{begin}}\<%
\\
\>[2][@{}l@{\AgdaIndent{0}}]%
\>[4]\AgdaSymbol{(}\AgdaInductiveConstructor{suc}\AgdaSpace{}%
\AgdaBound{m}\AgdaSpace{}%
\AgdaOperator{\AgdaPrimitive{+}}\AgdaSpace{}%
\AgdaBound{n}\AgdaSymbol{)}\AgdaSpace{}%
\AgdaOperator{\AgdaPrimitive{+}}\AgdaSpace{}%
\AgdaBound{p}\<%
\\
%
\>[2]\AgdaOperator{\AgdaFunction{≡⟨⟩}}\<%
\\
\>[2][@{}l@{\AgdaIndent{0}}]%
\>[4]\AgdaInductiveConstructor{suc}\AgdaSpace{}%
\AgdaSymbol{(}\AgdaBound{m}\AgdaSpace{}%
\AgdaOperator{\AgdaPrimitive{+}}\AgdaSpace{}%
\AgdaBound{n}\AgdaSymbol{)}\AgdaSpace{}%
\AgdaOperator{\AgdaPrimitive{+}}\AgdaSpace{}%
\AgdaBound{p}\<%
\\
%
\>[2]\AgdaOperator{\AgdaFunction{≡⟨⟩}}\<%
\\
\>[2][@{}l@{\AgdaIndent{0}}]%
\>[4]\AgdaInductiveConstructor{suc}\AgdaSpace{}%
\AgdaSymbol{((}\AgdaBound{m}\AgdaSpace{}%
\AgdaOperator{\AgdaPrimitive{+}}\AgdaSpace{}%
\AgdaBound{n}\AgdaSymbol{)}\AgdaSpace{}%
\AgdaOperator{\AgdaPrimitive{+}}\AgdaSpace{}%
\AgdaBound{p}\AgdaSymbol{)}\<%
\\
%
\>[2]\AgdaFunction{≡⟨}\AgdaSpace{}%
\AgdaFunction{cong}\AgdaSpace{}%
\AgdaInductiveConstructor{suc}\AgdaSpace{}%
\AgdaSymbol{(}\AgdaFunction{+-assoc}\AgdaSpace{}%
\AgdaBound{m}\AgdaSpace{}%
\AgdaBound{n}\AgdaSpace{}%
\AgdaBound{p}\AgdaSymbol{)}\AgdaSpace{}%
\AgdaFunction{⟩}\<%
\\
\>[2][@{}l@{\AgdaIndent{0}}]%
\>[4]\AgdaInductiveConstructor{suc}\AgdaSpace{}%
\AgdaSymbol{(}\AgdaBound{m}\AgdaSpace{}%
\AgdaOperator{\AgdaPrimitive{+}}\AgdaSpace{}%
\AgdaSymbol{(}\AgdaBound{n}\AgdaSpace{}%
\AgdaOperator{\AgdaPrimitive{+}}\AgdaSpace{}%
\AgdaBound{p}\AgdaSymbol{))}\<%
\\
%
\>[2]\AgdaOperator{\AgdaFunction{≡⟨⟩}}\<%
\\
\>[2][@{}l@{\AgdaIndent{0}}]%
\>[4]\AgdaInductiveConstructor{suc}\AgdaSpace{}%
\AgdaBound{m}\AgdaSpace{}%
\AgdaOperator{\AgdaPrimitive{+}}\AgdaSpace{}%
\AgdaSymbol{(}\AgdaBound{n}\AgdaSpace{}%
\AgdaOperator{\AgdaPrimitive{+}}\AgdaSpace{}%
\AgdaBound{p}\AgdaSymbol{)}\<%
\\
%
\>[2]\AgdaOperator{\AgdaFunction{∎}}\<%
\end{code}
\end{fence}

We have named the proof \texttt{+-assoc}. In Agda, identifiers can
consist of any sequence of characters not including spaces or the
characters \texttt{@.()\{\};\_}.

Let's unpack this code. The signature states that we are defining the
identifier \texttt{+-assoc} which provides evidence for the proposition:

\begin{myDisplay}
∀ (m n p : ℕ) → (m + n) + p ≡ m + (n + p)
\end{myDisplay}

The upside down A is pronounced ``for all'', and the proposition asserts
that for all natural numbers \texttt{m}, \texttt{n}, and \texttt{p} the
equation \texttt{(m\ +\ n)\ +\ p\ ≡\ m\ +\ (n\ +\ p)} holds. Evidence
for the proposition is a function that accepts three natural numbers,
binds them to \texttt{m}, \texttt{n}, and \texttt{p}, and returns
evidence for the corresponding instance of the equation.

For the base case, we must show:

\begin{myDisplay}
(zero + n) + p ≡ zero + (n + p)
\end{myDisplay}

Simplifying both sides with the base case of addition yields the
equation:

\begin{myDisplay}
n + p ≡ n + p
\end{myDisplay}

This holds trivially. Reading the chain of equations in the base case of
the proof, the top and bottom of the chain match the two sides of the
equation to be shown, and reading down from the top and up from the
bottom takes us to \texttt{n\ +\ p} in the middle. No justification
other than simplification is required.

For the inductive case, we must show:

\begin{myDisplay}
(suc m + n) + p ≡ suc m + (n + p)
\end{myDisplay}

Simplifying both sides with the inductive case of addition yields the
equation:

\begin{myDisplay}
suc ((m + n) + p) ≡ suc (m + (n + p))
\end{myDisplay}

This in turn follows by prefacing \texttt{suc} to both sides of the
induction hypothesis:

\begin{myDisplay}
(m + n) + p ≡ m + (n + p)
\end{myDisplay}

Reading the chain of equations in the inductive case of the proof, the
top and bottom of the chain match the two sides of the equation to be
shown, and reading down from the top and up from the bottom takes us to
the simplified equation above. The remaining equation does not follow
from simplification alone, so we use an additional operator for chain
reasoning, \texttt{\_≡⟨\_⟩\_}, where a justification for the equation
appears within angle brackets. The justification given is:

\begin{myDisplay}
⟨ cong suc (+-assoc m n p) ⟩
\end{myDisplay}

Here, the recursive invocation \texttt{+-assoc\ m\ n\ p} has as its type
the induction hypothesis, and \texttt{cong\ suc} prefaces \texttt{suc}
to each side to yield the needed equation.

A relation is said to be a \emph{congruence} for a given function if it
is preserved by applying that function. If \texttt{e} is evidence that
\texttt{x\ ≡\ y}, then \texttt{cong\ f\ e} is evidence that
\texttt{f\ x\ ≡\ f\ y}, for any function \texttt{f}.

Here the inductive hypothesis is not assumed, but instead proved by a
recursive invocation of the function we are defining,
\texttt{+-assoc\ m\ n\ p}. As with addition, this is well founded
because associativity of larger numbers is proved in terms of
associativity of smaller numbers. In this case,
\texttt{assoc\ (suc\ m)\ n\ p} is proved using \texttt{assoc\ m\ n\ p}.
The correspondence between proof by induction and definition by
recursion is one of the most appealing aspects of Agda.

\hypertarget{induction-as-recursion}{%
\section{Induction as recursion}\label{induction-as-recursion}}

As a concrete example of how induction corresponds to recursion, here is
the computation that occurs when instantiating \texttt{m} to \texttt{2}
in the proof of associativity.

\begin{fence}
\begin{code}%
\>[0]\AgdaFunction{+-assoc-2}\AgdaSpace{}%
\AgdaSymbol{:}\AgdaSpace{}%
\AgdaSymbol{∀}\AgdaSpace{}%
\AgdaSymbol{(}\AgdaBound{n}\AgdaSpace{}%
\AgdaBound{p}\AgdaSpace{}%
\AgdaSymbol{:}\AgdaSpace{}%
\AgdaDatatype{ℕ}\AgdaSymbol{)}\AgdaSpace{}%
\AgdaSymbol{→}\AgdaSpace{}%
\AgdaSymbol{(}\AgdaNumber{2}\AgdaSpace{}%
\AgdaOperator{\AgdaPrimitive{+}}\AgdaSpace{}%
\AgdaBound{n}\AgdaSymbol{)}\AgdaSpace{}%
\AgdaOperator{\AgdaPrimitive{+}}\AgdaSpace{}%
\AgdaBound{p}\AgdaSpace{}%
\AgdaOperator{\AgdaDatatype{≡}}\AgdaSpace{}%
\AgdaNumber{2}\AgdaSpace{}%
\AgdaOperator{\AgdaPrimitive{+}}\AgdaSpace{}%
\AgdaSymbol{(}\AgdaBound{n}\AgdaSpace{}%
\AgdaOperator{\AgdaPrimitive{+}}\AgdaSpace{}%
\AgdaBound{p}\AgdaSymbol{)}\<%
\\
\>[0]\AgdaFunction{+-assoc-2}\AgdaSpace{}%
\AgdaBound{n}\AgdaSpace{}%
\AgdaBound{p}\AgdaSpace{}%
\AgdaSymbol{=}\<%
\\
\>[0][@{}l@{\AgdaIndent{0}}]%
\>[2]\AgdaOperator{\AgdaFunction{begin}}\<%
\\
\>[2][@{}l@{\AgdaIndent{0}}]%
\>[4]\AgdaSymbol{(}\AgdaNumber{2}\AgdaSpace{}%
\AgdaOperator{\AgdaPrimitive{+}}\AgdaSpace{}%
\AgdaBound{n}\AgdaSymbol{)}\AgdaSpace{}%
\AgdaOperator{\AgdaPrimitive{+}}\AgdaSpace{}%
\AgdaBound{p}\<%
\\
%
\>[2]\AgdaOperator{\AgdaFunction{≡⟨⟩}}\<%
\\
\>[2][@{}l@{\AgdaIndent{0}}]%
\>[4]\AgdaInductiveConstructor{suc}\AgdaSpace{}%
\AgdaSymbol{(}\AgdaNumber{1}\AgdaSpace{}%
\AgdaOperator{\AgdaPrimitive{+}}\AgdaSpace{}%
\AgdaBound{n}\AgdaSymbol{)}\AgdaSpace{}%
\AgdaOperator{\AgdaPrimitive{+}}\AgdaSpace{}%
\AgdaBound{p}\<%
\\
%
\>[2]\AgdaOperator{\AgdaFunction{≡⟨⟩}}\<%
\\
\>[2][@{}l@{\AgdaIndent{0}}]%
\>[4]\AgdaInductiveConstructor{suc}\AgdaSpace{}%
\AgdaSymbol{((}\AgdaNumber{1}\AgdaSpace{}%
\AgdaOperator{\AgdaPrimitive{+}}\AgdaSpace{}%
\AgdaBound{n}\AgdaSymbol{)}\AgdaSpace{}%
\AgdaOperator{\AgdaPrimitive{+}}\AgdaSpace{}%
\AgdaBound{p}\AgdaSymbol{)}\<%
\\
%
\>[2]\AgdaFunction{≡⟨}\AgdaSpace{}%
\AgdaFunction{cong}\AgdaSpace{}%
\AgdaInductiveConstructor{suc}\AgdaSpace{}%
\AgdaSymbol{(}\AgdaFunction{+-assoc-1}\AgdaSpace{}%
\AgdaBound{n}\AgdaSpace{}%
\AgdaBound{p}\AgdaSymbol{)}\AgdaSpace{}%
\AgdaFunction{⟩}\<%
\\
\>[2][@{}l@{\AgdaIndent{0}}]%
\>[4]\AgdaInductiveConstructor{suc}\AgdaSpace{}%
\AgdaSymbol{(}\AgdaNumber{1}\AgdaSpace{}%
\AgdaOperator{\AgdaPrimitive{+}}\AgdaSpace{}%
\AgdaSymbol{(}\AgdaBound{n}\AgdaSpace{}%
\AgdaOperator{\AgdaPrimitive{+}}\AgdaSpace{}%
\AgdaBound{p}\AgdaSymbol{))}\<%
\\
%
\>[2]\AgdaOperator{\AgdaFunction{≡⟨⟩}}\<%
\\
\>[2][@{}l@{\AgdaIndent{0}}]%
\>[4]\AgdaNumber{2}\AgdaSpace{}%
\AgdaOperator{\AgdaPrimitive{+}}\AgdaSpace{}%
\AgdaSymbol{(}\AgdaBound{n}\AgdaSpace{}%
\AgdaOperator{\AgdaPrimitive{+}}\AgdaSpace{}%
\AgdaBound{p}\AgdaSymbol{)}\<%
\\
%
\>[2]\AgdaOperator{\AgdaFunction{∎}}\<%
\\
%
\>[2]\AgdaKeyword{where}\<%
\\
%
\>[2]\AgdaFunction{+-assoc-1}\AgdaSpace{}%
\AgdaSymbol{:}\AgdaSpace{}%
\AgdaSymbol{∀}\AgdaSpace{}%
\AgdaSymbol{(}\AgdaBound{n}\AgdaSpace{}%
\AgdaBound{p}\AgdaSpace{}%
\AgdaSymbol{:}\AgdaSpace{}%
\AgdaDatatype{ℕ}\AgdaSymbol{)}\AgdaSpace{}%
\AgdaSymbol{→}\AgdaSpace{}%
\AgdaSymbol{(}\AgdaNumber{1}\AgdaSpace{}%
\AgdaOperator{\AgdaPrimitive{+}}\AgdaSpace{}%
\AgdaBound{n}\AgdaSymbol{)}\AgdaSpace{}%
\AgdaOperator{\AgdaPrimitive{+}}\AgdaSpace{}%
\AgdaBound{p}\AgdaSpace{}%
\AgdaOperator{\AgdaDatatype{≡}}\AgdaSpace{}%
\AgdaNumber{1}\AgdaSpace{}%
\AgdaOperator{\AgdaPrimitive{+}}\AgdaSpace{}%
\AgdaSymbol{(}\AgdaBound{n}\AgdaSpace{}%
\AgdaOperator{\AgdaPrimitive{+}}\AgdaSpace{}%
\AgdaBound{p}\AgdaSymbol{)}\<%
\\
%
\>[2]\AgdaFunction{+-assoc-1}\AgdaSpace{}%
\AgdaBound{n}\AgdaSpace{}%
\AgdaBound{p}\AgdaSpace{}%
\AgdaSymbol{=}\<%
\\
\>[2][@{}l@{\AgdaIndent{0}}]%
\>[4]\AgdaOperator{\AgdaFunction{begin}}\<%
\\
\>[4][@{}l@{\AgdaIndent{0}}]%
\>[6]\AgdaSymbol{(}\AgdaNumber{1}\AgdaSpace{}%
\AgdaOperator{\AgdaPrimitive{+}}\AgdaSpace{}%
\AgdaBound{n}\AgdaSymbol{)}\AgdaSpace{}%
\AgdaOperator{\AgdaPrimitive{+}}\AgdaSpace{}%
\AgdaBound{p}\<%
\\
%
\>[4]\AgdaOperator{\AgdaFunction{≡⟨⟩}}\<%
\\
\>[4][@{}l@{\AgdaIndent{0}}]%
\>[6]\AgdaInductiveConstructor{suc}\AgdaSpace{}%
\AgdaSymbol{(}\AgdaNumber{0}\AgdaSpace{}%
\AgdaOperator{\AgdaPrimitive{+}}\AgdaSpace{}%
\AgdaBound{n}\AgdaSymbol{)}\AgdaSpace{}%
\AgdaOperator{\AgdaPrimitive{+}}\AgdaSpace{}%
\AgdaBound{p}\<%
\\
%
\>[4]\AgdaOperator{\AgdaFunction{≡⟨⟩}}\<%
\\
\>[4][@{}l@{\AgdaIndent{0}}]%
\>[6]\AgdaInductiveConstructor{suc}\AgdaSpace{}%
\AgdaSymbol{((}\AgdaNumber{0}\AgdaSpace{}%
\AgdaOperator{\AgdaPrimitive{+}}\AgdaSpace{}%
\AgdaBound{n}\AgdaSymbol{)}\AgdaSpace{}%
\AgdaOperator{\AgdaPrimitive{+}}\AgdaSpace{}%
\AgdaBound{p}\AgdaSymbol{)}\<%
\\
%
\>[4]\AgdaFunction{≡⟨}\AgdaSpace{}%
\AgdaFunction{cong}\AgdaSpace{}%
\AgdaInductiveConstructor{suc}\AgdaSpace{}%
\AgdaSymbol{(}\AgdaFunction{+-assoc-0}\AgdaSpace{}%
\AgdaBound{n}\AgdaSpace{}%
\AgdaBound{p}\AgdaSymbol{)}\AgdaSpace{}%
\AgdaFunction{⟩}\<%
\\
\>[4][@{}l@{\AgdaIndent{0}}]%
\>[6]\AgdaInductiveConstructor{suc}\AgdaSpace{}%
\AgdaSymbol{(}\AgdaNumber{0}\AgdaSpace{}%
\AgdaOperator{\AgdaPrimitive{+}}\AgdaSpace{}%
\AgdaSymbol{(}\AgdaBound{n}\AgdaSpace{}%
\AgdaOperator{\AgdaPrimitive{+}}\AgdaSpace{}%
\AgdaBound{p}\AgdaSymbol{))}\<%
\\
%
\>[4]\AgdaOperator{\AgdaFunction{≡⟨⟩}}\<%
\\
\>[4][@{}l@{\AgdaIndent{0}}]%
\>[6]\AgdaNumber{1}\AgdaSpace{}%
\AgdaOperator{\AgdaPrimitive{+}}\AgdaSpace{}%
\AgdaSymbol{(}\AgdaBound{n}\AgdaSpace{}%
\AgdaOperator{\AgdaPrimitive{+}}\AgdaSpace{}%
\AgdaBound{p}\AgdaSymbol{)}\<%
\\
%
\>[4]\AgdaOperator{\AgdaFunction{∎}}\<%
\\
%
\>[4]\AgdaKeyword{where}\<%
\\
%
\>[4]\AgdaFunction{+-assoc-0}\AgdaSpace{}%
\AgdaSymbol{:}\AgdaSpace{}%
\AgdaSymbol{∀}\AgdaSpace{}%
\AgdaSymbol{(}\AgdaBound{n}\AgdaSpace{}%
\AgdaBound{p}\AgdaSpace{}%
\AgdaSymbol{:}\AgdaSpace{}%
\AgdaDatatype{ℕ}\AgdaSymbol{)}\AgdaSpace{}%
\AgdaSymbol{→}\AgdaSpace{}%
\AgdaSymbol{(}\AgdaNumber{0}\AgdaSpace{}%
\AgdaOperator{\AgdaPrimitive{+}}\AgdaSpace{}%
\AgdaBound{n}\AgdaSymbol{)}\AgdaSpace{}%
\AgdaOperator{\AgdaPrimitive{+}}\AgdaSpace{}%
\AgdaBound{p}\AgdaSpace{}%
\AgdaOperator{\AgdaDatatype{≡}}\AgdaSpace{}%
\AgdaNumber{0}\AgdaSpace{}%
\AgdaOperator{\AgdaPrimitive{+}}\AgdaSpace{}%
\AgdaSymbol{(}\AgdaBound{n}\AgdaSpace{}%
\AgdaOperator{\AgdaPrimitive{+}}\AgdaSpace{}%
\AgdaBound{p}\AgdaSymbol{)}\<%
\\
%
\>[4]\AgdaFunction{+-assoc-0}\AgdaSpace{}%
\AgdaBound{n}\AgdaSpace{}%
\AgdaBound{p}\AgdaSpace{}%
\AgdaSymbol{=}\<%
\\
\>[4][@{}l@{\AgdaIndent{0}}]%
\>[6]\AgdaOperator{\AgdaFunction{begin}}\<%
\\
\>[6][@{}l@{\AgdaIndent{0}}]%
\>[8]\AgdaSymbol{(}\AgdaNumber{0}\AgdaSpace{}%
\AgdaOperator{\AgdaPrimitive{+}}\AgdaSpace{}%
\AgdaBound{n}\AgdaSymbol{)}\AgdaSpace{}%
\AgdaOperator{\AgdaPrimitive{+}}\AgdaSpace{}%
\AgdaBound{p}\<%
\\
%
\>[6]\AgdaOperator{\AgdaFunction{≡⟨⟩}}\<%
\\
\>[6][@{}l@{\AgdaIndent{0}}]%
\>[8]\AgdaBound{n}\AgdaSpace{}%
\AgdaOperator{\AgdaPrimitive{+}}\AgdaSpace{}%
\AgdaBound{p}\<%
\\
%
\>[6]\AgdaOperator{\AgdaFunction{≡⟨⟩}}\<%
\\
\>[6][@{}l@{\AgdaIndent{0}}]%
\>[8]\AgdaNumber{0}\AgdaSpace{}%
\AgdaOperator{\AgdaPrimitive{+}}\AgdaSpace{}%
\AgdaSymbol{(}\AgdaBound{n}\AgdaSpace{}%
\AgdaOperator{\AgdaPrimitive{+}}\AgdaSpace{}%
\AgdaBound{p}\AgdaSymbol{)}\<%
\\
%
\>[6]\AgdaOperator{\AgdaFunction{∎}}\<%
\end{code}
\end{fence}

\hypertarget{terminology-and-notation}{%
\section{Terminology and notation}\label{terminology-and-notation}}

The symbol \texttt{∀} appears in the statement of associativity to
indicate that it holds for all numbers \texttt{m}, \texttt{n}, and
\texttt{p}. We refer to \texttt{∀} as the \emph{universal quantifier},
and it is discussed further in Chapter
\protect\hyperlink{Quantifiers}{Quantifiers}.

Evidence for a universal quantifier is a function. The notations

\begin{myDisplay}
+-assoc : ∀ (m n p : ℕ) → (m + n) + p ≡ m + (n + p)
\end{myDisplay}

and

\begin{myDisplay}
+-assoc : ∀ (m : ℕ) → ∀ (n : ℕ) → ∀ (p : ℕ) → (m + n) + p ≡ m + (n + p)
\end{myDisplay}

are equivalent. They differ from a function type such as
\texttt{ℕ\ →\ ℕ\ →\ ℕ} in that variables are associated with each
argument type, and the result type may mention (or depend upon) these
variables; hence they are called \emph{dependent functions}.

\hypertarget{our-second-proof-commutativity}{%
\section{Our second proof:
commutativity}\label{our-second-proof-commutativity}}

Another important property of addition is that it is \emph{commutative},
that is, that the order of the operands does not matter:

\begin{myDisplay}
m + n ≡ n + m
\end{myDisplay}

The proof requires that we first demonstrate two lemmas.

\hypertarget{the-first-lemma}{%
\subsection{The first lemma}\label{the-first-lemma}}

The base case of the definition of addition states that zero is a
left-identity:

\begin{myDisplay}
zero + n ≡ n
\end{myDisplay}

Our first lemma states that zero is also a right-identity:

\begin{myDisplay}
m + zero ≡ m
\end{myDisplay}

Here is the lemma's statement and proof:

\begin{fence}
\begin{code}%
\>[0]\AgdaFunction{+-identityʳ}\AgdaSpace{}%
\AgdaSymbol{:}\AgdaSpace{}%
\AgdaSymbol{∀}\AgdaSpace{}%
\AgdaSymbol{(}\AgdaBound{m}\AgdaSpace{}%
\AgdaSymbol{:}\AgdaSpace{}%
\AgdaDatatype{ℕ}\AgdaSymbol{)}\AgdaSpace{}%
\AgdaSymbol{→}\AgdaSpace{}%
\AgdaBound{m}\AgdaSpace{}%
\AgdaOperator{\AgdaPrimitive{+}}\AgdaSpace{}%
\AgdaInductiveConstructor{zero}\AgdaSpace{}%
\AgdaOperator{\AgdaDatatype{≡}}\AgdaSpace{}%
\AgdaBound{m}\<%
\\
\>[0]\AgdaFunction{+-identityʳ}\AgdaSpace{}%
\AgdaInductiveConstructor{zero}\AgdaSpace{}%
\AgdaSymbol{=}\<%
\\
\>[0][@{}l@{\AgdaIndent{0}}]%
\>[2]\AgdaOperator{\AgdaFunction{begin}}\<%
\\
\>[2][@{}l@{\AgdaIndent{0}}]%
\>[4]\AgdaInductiveConstructor{zero}\AgdaSpace{}%
\AgdaOperator{\AgdaPrimitive{+}}\AgdaSpace{}%
\AgdaInductiveConstructor{zero}\<%
\\
%
\>[2]\AgdaOperator{\AgdaFunction{≡⟨⟩}}\<%
\\
\>[2][@{}l@{\AgdaIndent{0}}]%
\>[4]\AgdaInductiveConstructor{zero}\<%
\\
%
\>[2]\AgdaOperator{\AgdaFunction{∎}}\<%
\\
\>[0]\AgdaFunction{+-identityʳ}\AgdaSpace{}%
\AgdaSymbol{(}\AgdaInductiveConstructor{suc}\AgdaSpace{}%
\AgdaBound{m}\AgdaSymbol{)}\AgdaSpace{}%
\AgdaSymbol{=}\<%
\\
\>[0][@{}l@{\AgdaIndent{0}}]%
\>[2]\AgdaOperator{\AgdaFunction{begin}}\<%
\\
\>[2][@{}l@{\AgdaIndent{0}}]%
\>[4]\AgdaInductiveConstructor{suc}\AgdaSpace{}%
\AgdaBound{m}\AgdaSpace{}%
\AgdaOperator{\AgdaPrimitive{+}}\AgdaSpace{}%
\AgdaInductiveConstructor{zero}\<%
\\
%
\>[2]\AgdaOperator{\AgdaFunction{≡⟨⟩}}\<%
\\
\>[2][@{}l@{\AgdaIndent{0}}]%
\>[4]\AgdaInductiveConstructor{suc}\AgdaSpace{}%
\AgdaSymbol{(}\AgdaBound{m}\AgdaSpace{}%
\AgdaOperator{\AgdaPrimitive{+}}\AgdaSpace{}%
\AgdaInductiveConstructor{zero}\AgdaSymbol{)}\<%
\\
%
\>[2]\AgdaFunction{≡⟨}\AgdaSpace{}%
\AgdaFunction{cong}\AgdaSpace{}%
\AgdaInductiveConstructor{suc}\AgdaSpace{}%
\AgdaSymbol{(}\AgdaFunction{+-identityʳ}\AgdaSpace{}%
\AgdaBound{m}\AgdaSymbol{)}\AgdaSpace{}%
\AgdaFunction{⟩}\<%
\\
\>[2][@{}l@{\AgdaIndent{0}}]%
\>[4]\AgdaInductiveConstructor{suc}\AgdaSpace{}%
\AgdaBound{m}\<%
\\
%
\>[2]\AgdaOperator{\AgdaFunction{∎}}\<%
\end{code}
\end{fence}

The signature states that we are defining the identifier
\texttt{+-identityʳ} which provides evidence for the proposition:

\begin{myDisplay}
∀ (m : ℕ) → m + zero ≡ m
\end{myDisplay}

Evidence for the proposition is a function that accepts a natural
number, binds it to \texttt{m}, and returns evidence for the
corresponding instance of the equation. The proof is by induction on
\texttt{m}.

For the base case, we must show:

\begin{myDisplay}
zero + zero ≡ zero
\end{myDisplay}

Simplifying with the base case of addition, this is straightforward.

For the inductive case, we must show:

\begin{myDisplay}
(suc m) + zero = suc m
\end{myDisplay}

Simplifying both sides with the inductive case of addition yields the
equation:

\begin{myDisplay}
suc (m + zero) = suc m
\end{myDisplay}

This in turn follows by prefacing \texttt{suc} to both sides of the
induction hypothesis:

\begin{myDisplay}
m + zero ≡ m
\end{myDisplay}

Reading the chain of equations down from the top and up from the bottom
takes us to the simplified equation above. The remaining equation has
the justification:

\begin{myDisplay}
⟨ cong suc (+-identityʳ m) ⟩
\end{myDisplay}

Here, the recursive invocation \texttt{+-identityʳ\ m} has as its type
the induction hypothesis, and \texttt{cong\ suc} prefaces \texttt{suc}
to each side to yield the needed equation. This completes the first
lemma.

\hypertarget{the-second-lemma}{%
\subsection{The second lemma}\label{the-second-lemma}}

The inductive case of the definition of addition pushes \texttt{suc} on
the first argument to the outside:

\begin{myDisplay}
suc m + n ≡ suc (m + n)
\end{myDisplay}

Our second lemma does the same for \texttt{suc} on the second argument:

\begin{myDisplay}
m + suc n ≡ suc (m + n)
\end{myDisplay}

Here is the lemma's statement and proof:

\begin{fence}
\begin{code}%
\>[0]\AgdaFunction{+-suc}\AgdaSpace{}%
\AgdaSymbol{:}\AgdaSpace{}%
\AgdaSymbol{∀}\AgdaSpace{}%
\AgdaSymbol{(}\AgdaBound{m}\AgdaSpace{}%
\AgdaBound{n}\AgdaSpace{}%
\AgdaSymbol{:}\AgdaSpace{}%
\AgdaDatatype{ℕ}\AgdaSymbol{)}\AgdaSpace{}%
\AgdaSymbol{→}\AgdaSpace{}%
\AgdaBound{m}\AgdaSpace{}%
\AgdaOperator{\AgdaPrimitive{+}}\AgdaSpace{}%
\AgdaInductiveConstructor{suc}\AgdaSpace{}%
\AgdaBound{n}\AgdaSpace{}%
\AgdaOperator{\AgdaDatatype{≡}}\AgdaSpace{}%
\AgdaInductiveConstructor{suc}\AgdaSpace{}%
\AgdaSymbol{(}\AgdaBound{m}\AgdaSpace{}%
\AgdaOperator{\AgdaPrimitive{+}}\AgdaSpace{}%
\AgdaBound{n}\AgdaSymbol{)}\<%
\\
\>[0]\AgdaFunction{+-suc}\AgdaSpace{}%
\AgdaInductiveConstructor{zero}\AgdaSpace{}%
\AgdaBound{n}\AgdaSpace{}%
\AgdaSymbol{=}\<%
\\
\>[0][@{}l@{\AgdaIndent{0}}]%
\>[2]\AgdaOperator{\AgdaFunction{begin}}\<%
\\
\>[2][@{}l@{\AgdaIndent{0}}]%
\>[4]\AgdaInductiveConstructor{zero}\AgdaSpace{}%
\AgdaOperator{\AgdaPrimitive{+}}\AgdaSpace{}%
\AgdaInductiveConstructor{suc}\AgdaSpace{}%
\AgdaBound{n}\<%
\\
%
\>[2]\AgdaOperator{\AgdaFunction{≡⟨⟩}}\<%
\\
\>[2][@{}l@{\AgdaIndent{0}}]%
\>[4]\AgdaInductiveConstructor{suc}\AgdaSpace{}%
\AgdaBound{n}\<%
\\
%
\>[2]\AgdaOperator{\AgdaFunction{≡⟨⟩}}\<%
\\
\>[2][@{}l@{\AgdaIndent{0}}]%
\>[4]\AgdaInductiveConstructor{suc}\AgdaSpace{}%
\AgdaSymbol{(}\AgdaInductiveConstructor{zero}\AgdaSpace{}%
\AgdaOperator{\AgdaPrimitive{+}}\AgdaSpace{}%
\AgdaBound{n}\AgdaSymbol{)}\<%
\\
%
\>[2]\AgdaOperator{\AgdaFunction{∎}}\<%
\\
\>[0]\AgdaFunction{+-suc}\AgdaSpace{}%
\AgdaSymbol{(}\AgdaInductiveConstructor{suc}\AgdaSpace{}%
\AgdaBound{m}\AgdaSymbol{)}\AgdaSpace{}%
\AgdaBound{n}\AgdaSpace{}%
\AgdaSymbol{=}\<%
\\
\>[0][@{}l@{\AgdaIndent{0}}]%
\>[2]\AgdaOperator{\AgdaFunction{begin}}\<%
\\
\>[2][@{}l@{\AgdaIndent{0}}]%
\>[4]\AgdaInductiveConstructor{suc}\AgdaSpace{}%
\AgdaBound{m}\AgdaSpace{}%
\AgdaOperator{\AgdaPrimitive{+}}\AgdaSpace{}%
\AgdaInductiveConstructor{suc}\AgdaSpace{}%
\AgdaBound{n}\<%
\\
%
\>[2]\AgdaOperator{\AgdaFunction{≡⟨⟩}}\<%
\\
\>[2][@{}l@{\AgdaIndent{0}}]%
\>[4]\AgdaInductiveConstructor{suc}\AgdaSpace{}%
\AgdaSymbol{(}\AgdaBound{m}\AgdaSpace{}%
\AgdaOperator{\AgdaPrimitive{+}}\AgdaSpace{}%
\AgdaInductiveConstructor{suc}\AgdaSpace{}%
\AgdaBound{n}\AgdaSymbol{)}\<%
\\
%
\>[2]\AgdaFunction{≡⟨}\AgdaSpace{}%
\AgdaFunction{cong}\AgdaSpace{}%
\AgdaInductiveConstructor{suc}\AgdaSpace{}%
\AgdaSymbol{(}\AgdaFunction{+-suc}\AgdaSpace{}%
\AgdaBound{m}\AgdaSpace{}%
\AgdaBound{n}\AgdaSymbol{)}\AgdaSpace{}%
\AgdaFunction{⟩}\<%
\\
\>[2][@{}l@{\AgdaIndent{0}}]%
\>[4]\AgdaInductiveConstructor{suc}\AgdaSpace{}%
\AgdaSymbol{(}\AgdaInductiveConstructor{suc}\AgdaSpace{}%
\AgdaSymbol{(}\AgdaBound{m}\AgdaSpace{}%
\AgdaOperator{\AgdaPrimitive{+}}\AgdaSpace{}%
\AgdaBound{n}\AgdaSymbol{))}\<%
\\
%
\>[2]\AgdaOperator{\AgdaFunction{≡⟨⟩}}\<%
\\
\>[2][@{}l@{\AgdaIndent{0}}]%
\>[4]\AgdaInductiveConstructor{suc}\AgdaSpace{}%
\AgdaSymbol{(}\AgdaInductiveConstructor{suc}\AgdaSpace{}%
\AgdaBound{m}\AgdaSpace{}%
\AgdaOperator{\AgdaPrimitive{+}}\AgdaSpace{}%
\AgdaBound{n}\AgdaSymbol{)}\<%
\\
%
\>[2]\AgdaOperator{\AgdaFunction{∎}}\<%
\end{code}
\end{fence}

The signature states that we are defining the identifier \texttt{+-suc}
which provides evidence for the proposition:

\begin{myDisplay}
∀ (m n : ℕ) → m + suc n ≡ suc (m + n)
\end{myDisplay}

Evidence for the proposition is a function that accepts two natural
numbers, binds them to \texttt{m} and \texttt{n}, and returns evidence
for the corresponding instance of the equation. The proof is by
induction on \texttt{m}.

For the base case, we must show:

\begin{myDisplay}
zero + suc n ≡ suc (zero + n)
\end{myDisplay}

Simplifying with the base case of addition, this is straightforward.

For the inductive case, we must show:

\begin{myDisplay}
suc m + suc n ≡ suc (suc m + n)
\end{myDisplay}

Simplifying both sides with the inductive case of addition yields the
equation:

\begin{myDisplay}
suc (m + suc n) ≡ suc (suc (m + n))
\end{myDisplay}

This in turn follows by prefacing \texttt{suc} to both sides of the
induction hypothesis:

\begin{myDisplay}
m + suc n ≡ suc (m + n)
\end{myDisplay}

Reading the chain of equations down from the top and up from the bottom
takes us to the simplified equation in the middle. The remaining
equation has the justification:

\begin{myDisplay}
⟨ cong suc (+-suc m n) ⟩
\end{myDisplay}

Here, the recursive invocation \texttt{+-suc\ m\ n} has as its type the
induction hypothesis, and \texttt{cong\ suc} prefaces \texttt{suc} to
each side to yield the needed equation. This completes the second lemma.

\hypertarget{the-proposition}{%
\subsection{The proposition}\label{the-proposition}}

Finally, here is our proposition's statement and proof:

\begin{fence}
\begin{code}%
\>[0]\AgdaFunction{+-comm}\AgdaSpace{}%
\AgdaSymbol{:}\AgdaSpace{}%
\AgdaSymbol{∀}\AgdaSpace{}%
\AgdaSymbol{(}\AgdaBound{m}\AgdaSpace{}%
\AgdaBound{n}\AgdaSpace{}%
\AgdaSymbol{:}\AgdaSpace{}%
\AgdaDatatype{ℕ}\AgdaSymbol{)}\AgdaSpace{}%
\AgdaSymbol{→}\AgdaSpace{}%
\AgdaBound{m}\AgdaSpace{}%
\AgdaOperator{\AgdaPrimitive{+}}\AgdaSpace{}%
\AgdaBound{n}\AgdaSpace{}%
\AgdaOperator{\AgdaDatatype{≡}}\AgdaSpace{}%
\AgdaBound{n}\AgdaSpace{}%
\AgdaOperator{\AgdaPrimitive{+}}\AgdaSpace{}%
\AgdaBound{m}\<%
\\
\>[0]\AgdaFunction{+-comm}\AgdaSpace{}%
\AgdaBound{m}\AgdaSpace{}%
\AgdaInductiveConstructor{zero}\AgdaSpace{}%
\AgdaSymbol{=}\<%
\\
\>[0][@{}l@{\AgdaIndent{0}}]%
\>[2]\AgdaOperator{\AgdaFunction{begin}}\<%
\\
\>[2][@{}l@{\AgdaIndent{0}}]%
\>[4]\AgdaBound{m}\AgdaSpace{}%
\AgdaOperator{\AgdaPrimitive{+}}\AgdaSpace{}%
\AgdaInductiveConstructor{zero}\<%
\\
%
\>[2]\AgdaFunction{≡⟨}\AgdaSpace{}%
\AgdaFunction{+-identityʳ}\AgdaSpace{}%
\AgdaBound{m}\AgdaSpace{}%
\AgdaFunction{⟩}\<%
\\
\>[2][@{}l@{\AgdaIndent{0}}]%
\>[4]\AgdaBound{m}\<%
\\
%
\>[2]\AgdaOperator{\AgdaFunction{≡⟨⟩}}\<%
\\
\>[2][@{}l@{\AgdaIndent{0}}]%
\>[4]\AgdaInductiveConstructor{zero}\AgdaSpace{}%
\AgdaOperator{\AgdaPrimitive{+}}\AgdaSpace{}%
\AgdaBound{m}\<%
\\
%
\>[2]\AgdaOperator{\AgdaFunction{∎}}\<%
\\
\>[0]\AgdaFunction{+-comm}\AgdaSpace{}%
\AgdaBound{m}\AgdaSpace{}%
\AgdaSymbol{(}\AgdaInductiveConstructor{suc}\AgdaSpace{}%
\AgdaBound{n}\AgdaSymbol{)}\AgdaSpace{}%
\AgdaSymbol{=}\<%
\\
\>[0][@{}l@{\AgdaIndent{0}}]%
\>[2]\AgdaOperator{\AgdaFunction{begin}}\<%
\\
\>[2][@{}l@{\AgdaIndent{0}}]%
\>[4]\AgdaBound{m}\AgdaSpace{}%
\AgdaOperator{\AgdaPrimitive{+}}\AgdaSpace{}%
\AgdaInductiveConstructor{suc}\AgdaSpace{}%
\AgdaBound{n}\<%
\\
%
\>[2]\AgdaFunction{≡⟨}\AgdaSpace{}%
\AgdaFunction{+-suc}\AgdaSpace{}%
\AgdaBound{m}\AgdaSpace{}%
\AgdaBound{n}\AgdaSpace{}%
\AgdaFunction{⟩}\<%
\\
\>[2][@{}l@{\AgdaIndent{0}}]%
\>[4]\AgdaInductiveConstructor{suc}\AgdaSpace{}%
\AgdaSymbol{(}\AgdaBound{m}\AgdaSpace{}%
\AgdaOperator{\AgdaPrimitive{+}}\AgdaSpace{}%
\AgdaBound{n}\AgdaSymbol{)}\<%
\\
%
\>[2]\AgdaFunction{≡⟨}\AgdaSpace{}%
\AgdaFunction{cong}\AgdaSpace{}%
\AgdaInductiveConstructor{suc}\AgdaSpace{}%
\AgdaSymbol{(}\AgdaFunction{+-comm}\AgdaSpace{}%
\AgdaBound{m}\AgdaSpace{}%
\AgdaBound{n}\AgdaSymbol{)}\AgdaSpace{}%
\AgdaFunction{⟩}\<%
\\
\>[2][@{}l@{\AgdaIndent{0}}]%
\>[4]\AgdaInductiveConstructor{suc}\AgdaSpace{}%
\AgdaSymbol{(}\AgdaBound{n}\AgdaSpace{}%
\AgdaOperator{\AgdaPrimitive{+}}\AgdaSpace{}%
\AgdaBound{m}\AgdaSymbol{)}\<%
\\
%
\>[2]\AgdaOperator{\AgdaFunction{≡⟨⟩}}\<%
\\
\>[2][@{}l@{\AgdaIndent{0}}]%
\>[4]\AgdaInductiveConstructor{suc}\AgdaSpace{}%
\AgdaBound{n}\AgdaSpace{}%
\AgdaOperator{\AgdaPrimitive{+}}\AgdaSpace{}%
\AgdaBound{m}\<%
\\
%
\>[2]\AgdaOperator{\AgdaFunction{∎}}\<%
\end{code}
\end{fence}

The first line states that we are defining the identifier
\texttt{+-comm} which provides evidence for the proposition:

\begin{myDisplay}
∀ (m n : ℕ) → m + n ≡ n + m
\end{myDisplay}

Evidence for the proposition is a function that accepts two natural
numbers, binds them to \texttt{m} and \texttt{n}, and returns evidence
for the corresponding instance of the equation. The proof is by
induction on \texttt{n}. (Not on \texttt{m} this time!)

For the base case, we must show:

\begin{myDisplay}
m + zero ≡ zero + m
\end{myDisplay}

Simplifying both sides with the base case of addition yields the
equation:

\begin{myDisplay}
m + zero ≡ m
\end{myDisplay}

The remaining equation has the justification
\texttt{⟨\ +-identityʳ\ m\ ⟩}, which invokes the first lemma.

For the inductive case, we must show:

\begin{myDisplay}
m + suc n ≡ suc n + m
\end{myDisplay}

Simplifying both sides with the inductive case of addition yields the
equation:

\begin{myDisplay}
m + suc n ≡ suc (n + m)
\end{myDisplay}

We show this in two steps. First, we have:

\begin{myDisplay}
m + suc n ≡ suc (m + n)
\end{myDisplay}

which is justified by the second lemma, \texttt{⟨\ +-suc\ m\ n\ ⟩}. Then
we have

\begin{myDisplay}
suc (m + n) ≡ suc (n + m)
\end{myDisplay}

which is justified by congruence and the induction hypothesis,
\texttt{⟨\ cong\ suc\ (+-comm\ m\ n)\ ⟩}. This completes the proof.

Agda requires that identifiers are defined before they are used, so we
must present the lemmas before the main proposition, as we have done
above. In practice, one will often attempt to prove the main proposition
first, and the equations required to do so will suggest what lemmas to
prove.

\hypertarget{Induction-sections}{%
\section{Our first corollary: rearranging}\label{Induction-sections}}

We can apply associativity to rearrange parentheses however we like.
Here is an example:

\begin{fence}
\begin{code}%
\>[0]\AgdaFunction{+-rearrange}\AgdaSpace{}%
\AgdaSymbol{:}\AgdaSpace{}%
\AgdaSymbol{∀}\AgdaSpace{}%
\AgdaSymbol{(}\AgdaBound{m}\AgdaSpace{}%
\AgdaBound{n}\AgdaSpace{}%
\AgdaBound{p}\AgdaSpace{}%
\AgdaBound{q}\AgdaSpace{}%
\AgdaSymbol{:}\AgdaSpace{}%
\AgdaDatatype{ℕ}\AgdaSymbol{)}\AgdaSpace{}%
\AgdaSymbol{→}\AgdaSpace{}%
\AgdaSymbol{(}\AgdaBound{m}\AgdaSpace{}%
\AgdaOperator{\AgdaPrimitive{+}}\AgdaSpace{}%
\AgdaBound{n}\AgdaSymbol{)}\AgdaSpace{}%
\AgdaOperator{\AgdaPrimitive{+}}\AgdaSpace{}%
\AgdaSymbol{(}\AgdaBound{p}\AgdaSpace{}%
\AgdaOperator{\AgdaPrimitive{+}}\AgdaSpace{}%
\AgdaBound{q}\AgdaSymbol{)}\AgdaSpace{}%
\AgdaOperator{\AgdaDatatype{≡}}\AgdaSpace{}%
\AgdaBound{m}\AgdaSpace{}%
\AgdaOperator{\AgdaPrimitive{+}}\AgdaSpace{}%
\AgdaSymbol{(}\AgdaBound{n}\AgdaSpace{}%
\AgdaOperator{\AgdaPrimitive{+}}\AgdaSpace{}%
\AgdaBound{p}\AgdaSymbol{)}\AgdaSpace{}%
\AgdaOperator{\AgdaPrimitive{+}}\AgdaSpace{}%
\AgdaBound{q}\<%
\\
\>[0]\AgdaFunction{+-rearrange}\AgdaSpace{}%
\AgdaBound{m}\AgdaSpace{}%
\AgdaBound{n}\AgdaSpace{}%
\AgdaBound{p}\AgdaSpace{}%
\AgdaBound{q}\AgdaSpace{}%
\AgdaSymbol{=}\<%
\\
\>[0][@{}l@{\AgdaIndent{0}}]%
\>[2]\AgdaOperator{\AgdaFunction{begin}}\<%
\\
\>[2][@{}l@{\AgdaIndent{0}}]%
\>[4]\AgdaSymbol{(}\AgdaBound{m}\AgdaSpace{}%
\AgdaOperator{\AgdaPrimitive{+}}\AgdaSpace{}%
\AgdaBound{n}\AgdaSymbol{)}\AgdaSpace{}%
\AgdaOperator{\AgdaPrimitive{+}}\AgdaSpace{}%
\AgdaSymbol{(}\AgdaBound{p}\AgdaSpace{}%
\AgdaOperator{\AgdaPrimitive{+}}\AgdaSpace{}%
\AgdaBound{q}\AgdaSymbol{)}\<%
\\
%
\>[2]\AgdaFunction{≡⟨}\AgdaSpace{}%
\AgdaFunction{+-assoc}\AgdaSpace{}%
\AgdaBound{m}\AgdaSpace{}%
\AgdaBound{n}\AgdaSpace{}%
\AgdaSymbol{(}\AgdaBound{p}\AgdaSpace{}%
\AgdaOperator{\AgdaPrimitive{+}}\AgdaSpace{}%
\AgdaBound{q}\AgdaSymbol{)}\AgdaSpace{}%
\AgdaFunction{⟩}\<%
\\
\>[2][@{}l@{\AgdaIndent{0}}]%
\>[4]\AgdaBound{m}\AgdaSpace{}%
\AgdaOperator{\AgdaPrimitive{+}}\AgdaSpace{}%
\AgdaSymbol{(}\AgdaBound{n}\AgdaSpace{}%
\AgdaOperator{\AgdaPrimitive{+}}\AgdaSpace{}%
\AgdaSymbol{(}\AgdaBound{p}\AgdaSpace{}%
\AgdaOperator{\AgdaPrimitive{+}}\AgdaSpace{}%
\AgdaBound{q}\AgdaSymbol{))}\<%
\\
%
\>[2]\AgdaFunction{≡⟨}\AgdaSpace{}%
\AgdaFunction{cong}\AgdaSpace{}%
\AgdaSymbol{(}\AgdaBound{m}\AgdaSpace{}%
\AgdaOperator{\AgdaPrimitive{+\AgdaUnderscore{}}}\AgdaSymbol{)}\AgdaSpace{}%
\AgdaSymbol{(}\AgdaFunction{sym}\AgdaSpace{}%
\AgdaSymbol{(}\AgdaFunction{+-assoc}\AgdaSpace{}%
\AgdaBound{n}\AgdaSpace{}%
\AgdaBound{p}\AgdaSpace{}%
\AgdaBound{q}\AgdaSymbol{))}\AgdaSpace{}%
\AgdaFunction{⟩}\<%
\\
\>[2][@{}l@{\AgdaIndent{0}}]%
\>[4]\AgdaBound{m}\AgdaSpace{}%
\AgdaOperator{\AgdaPrimitive{+}}\AgdaSpace{}%
\AgdaSymbol{((}\AgdaBound{n}\AgdaSpace{}%
\AgdaOperator{\AgdaPrimitive{+}}\AgdaSpace{}%
\AgdaBound{p}\AgdaSymbol{)}\AgdaSpace{}%
\AgdaOperator{\AgdaPrimitive{+}}\AgdaSpace{}%
\AgdaBound{q}\AgdaSymbol{)}\<%
\\
%
\>[2]\AgdaFunction{≡⟨}\AgdaSpace{}%
\AgdaFunction{sym}\AgdaSpace{}%
\AgdaSymbol{(}\AgdaFunction{+-assoc}\AgdaSpace{}%
\AgdaBound{m}\AgdaSpace{}%
\AgdaSymbol{(}\AgdaBound{n}\AgdaSpace{}%
\AgdaOperator{\AgdaPrimitive{+}}\AgdaSpace{}%
\AgdaBound{p}\AgdaSymbol{)}\AgdaSpace{}%
\AgdaBound{q}\AgdaSymbol{)}\AgdaSpace{}%
\AgdaFunction{⟩}\<%
\\
\>[2][@{}l@{\AgdaIndent{0}}]%
\>[4]\AgdaSymbol{(}\AgdaBound{m}\AgdaSpace{}%
\AgdaOperator{\AgdaPrimitive{+}}\AgdaSpace{}%
\AgdaSymbol{(}\AgdaBound{n}\AgdaSpace{}%
\AgdaOperator{\AgdaPrimitive{+}}\AgdaSpace{}%
\AgdaBound{p}\AgdaSymbol{))}\AgdaSpace{}%
\AgdaOperator{\AgdaPrimitive{+}}\AgdaSpace{}%
\AgdaBound{q}\<%
\\
%
\>[2]\AgdaOperator{\AgdaFunction{∎}}\<%
\end{code}
\end{fence}

No induction is required, we simply apply associativity twice. A few
points are worthy of note.

First, addition associates to the left, so
\texttt{m\ +\ (n\ +\ p)\ +\ q} stands for
\texttt{(m\ +\ (n\ +\ p))\ +\ q}.

Second, we use \texttt{sym} to interchange the sides of an equation.
Proposition \texttt{+-assoc\ n\ p\ q} shifts parentheses from right to
left:

\begin{myDisplay}
(n + p) + q ≡ n + (p + q)
\end{myDisplay}

To shift them the other way, we use \texttt{sym\ (+-assoc\ n\ p\ q)}:

\begin{myDisplay}
n + (p + q) ≡ (n + p) + q
\end{myDisplay}

In general, if \texttt{e} provides evidence for \texttt{x\ ≡\ y} then
\texttt{sym\ e} provides evidence for \texttt{y\ ≡\ x}.

Third, Agda supports a variant of the \emph{section} notation introduced
by Richard Bird. We write \texttt{(x\ +\_)} for the function that
applied to \texttt{y} returns \texttt{x\ +\ y}. Thus, applying the
congruence \texttt{cong\ (m\ +\_)} takes the above equation into:

\begin{myDisplay}
m + (n + (p + q)) ≡ m + ((n + p) + q)
\end{myDisplay}

Similarly, we write \texttt{(\_+\ x)} for the function that applied to
\texttt{y} returns \texttt{y\ +\ x}; the same works for any infix
operator.

\hypertarget{creation-one-last-time}{%
\section{Creation, one last time}\label{creation-one-last-time}}

Returning to the proof of associativity, it may be helpful to view the
inductive proof (or, equivalently, the recursive definition) as a
creation story. This time we are concerned with judgments asserting
associativity:

\begin{myDisplay}
 -- In the beginning, we know nothing about associativity.
\end{myDisplay}

Now, we apply the rules to all the judgments we know about. The base
case tells us that \texttt{(zero\ +\ n)\ +\ p\ ≡\ zero\ +\ (n\ +\ p)}
for every natural \texttt{n} and \texttt{p}. The inductive case tells us
that if \texttt{(m\ +\ n)\ +\ p\ ≡\ m\ +\ (n\ +\ p)} (on the day before
today) then \texttt{(suc\ m\ +\ n)\ +\ p\ ≡\ suc\ m\ +\ (n\ +\ p)}
(today). We didn't know any judgments about associativity before today,
so that rule doesn't give us any new judgments:

\begin{myDisplay}
-- On the first day, we know about associativity of 0.
(0 + 0) + 0 ≡ 0 + (0 + 0)   ...   (0 + 4) + 5 ≡ 0 + (4 + 5)   ...
\end{myDisplay}

Then we repeat the process, so on the next day we know about all the
judgments from the day before, plus any judgments added by the rules.
The base case tells us nothing new, but now the inductive case adds more
judgments:

\begin{myDisplay}
-- On the second day, we know about associativity of 0 and 1.
(0 + 0) + 0 ≡ 0 + (0 + 0)   ...   (0 + 4) + 5 ≡ 0 + (4 + 5)   ...
(1 + 0) + 0 ≡ 1 + (0 + 0)   ...   (1 + 4) + 5 ≡ 1 + (4 + 5)   ...
\end{myDisplay}

And we repeat the process again:

\begin{myDisplay}
-- On the third day, we know about associativity of 0, 1, and 2.
(0 + 0) + 0 ≡ 0 + (0 + 0)   ...   (0 + 4) + 5 ≡ 0 + (4 + 5)   ...
(1 + 0) + 0 ≡ 1 + (0 + 0)   ...   (1 + 4) + 5 ≡ 1 + (4 + 5)   ...
(2 + 0) + 0 ≡ 2 + (0 + 0)   ...   (2 + 4) + 5 ≡ 2 + (4 + 5)   ...
\end{myDisplay}

You've got the hang of it by now:

\begin{myDisplay}
-- On the fourth day, we know about associativity of 0, 1, 2, and 3.
(0 + 0) + 0 ≡ 0 + (0 + 0)   ...   (0 + 4) + 5 ≡ 0 + (4 + 5)   ...
(1 + 0) + 0 ≡ 1 + (0 + 0)   ...   (1 + 4) + 5 ≡ 1 + (4 + 5)   ...
(2 + 0) + 0 ≡ 2 + (0 + 0)   ...   (2 + 4) + 5 ≡ 2 + (4 + 5)   ...
(3 + 0) + 0 ≡ 3 + (0 + 0)   ...   (3 + 4) + 5 ≡ 3 + (4 + 5)   ...
\end{myDisplay}

The process continues. On the \emph{m}'th day we will know all the
judgments where the first number is less than \emph{m}.

There is also a completely finite approach to generating the same
equations, which is left as an exercise for the reader.

\hypertarget{Induction-finite-plus-assoc}{%
\subsubsection{\texorpdfstring{Exercise \texttt{finite-\textbar{}-assoc}
(stretch)}{Exercise finite-\textbar-assoc (stretch)}}\label{Induction-finite-plus-assoc}}

Write out what is known about associativity of addition on each of the
first four days using a finite story of creation, as
\protect\hyperlink{Naturals-finite-creation}{earlier}.

\begin{fence}
\begin{code}%
\>[0]\AgdaComment{-- Your code goes here}\<%
\end{code}
\end{fence}

\hypertarget{associativity-with-rewrite}{%
\section{Associativity with rewrite}\label{associativity-with-rewrite}}

There is more than one way to skin a cat. Here is a second proof of
associativity of addition in Agda, using \texttt{rewrite} rather than
chains of equations:

\begin{fence}
\begin{code}%
\>[0]\AgdaFunction{+-assoc′}\AgdaSpace{}%
\AgdaSymbol{:}\AgdaSpace{}%
\AgdaSymbol{∀}\AgdaSpace{}%
\AgdaSymbol{(}\AgdaBound{m}\AgdaSpace{}%
\AgdaBound{n}\AgdaSpace{}%
\AgdaBound{p}\AgdaSpace{}%
\AgdaSymbol{:}\AgdaSpace{}%
\AgdaDatatype{ℕ}\AgdaSymbol{)}\AgdaSpace{}%
\AgdaSymbol{→}\AgdaSpace{}%
\AgdaSymbol{(}\AgdaBound{m}\AgdaSpace{}%
\AgdaOperator{\AgdaPrimitive{+}}\AgdaSpace{}%
\AgdaBound{n}\AgdaSymbol{)}\AgdaSpace{}%
\AgdaOperator{\AgdaPrimitive{+}}\AgdaSpace{}%
\AgdaBound{p}\AgdaSpace{}%
\AgdaOperator{\AgdaDatatype{≡}}\AgdaSpace{}%
\AgdaBound{m}\AgdaSpace{}%
\AgdaOperator{\AgdaPrimitive{+}}\AgdaSpace{}%
\AgdaSymbol{(}\AgdaBound{n}\AgdaSpace{}%
\AgdaOperator{\AgdaPrimitive{+}}\AgdaSpace{}%
\AgdaBound{p}\AgdaSymbol{)}\<%
\\
\>[0]\AgdaFunction{+-assoc′}\AgdaSpace{}%
\AgdaInductiveConstructor{zero}%
\>[17]\AgdaBound{n}\AgdaSpace{}%
\AgdaBound{p}%
\>[46]\AgdaSymbol{=}%
\>[49]\AgdaInductiveConstructor{refl}\<%
\\
\>[0]\AgdaFunction{+-assoc′}\AgdaSpace{}%
\AgdaSymbol{(}\AgdaInductiveConstructor{suc}\AgdaSpace{}%
\AgdaBound{m}\AgdaSymbol{)}\AgdaSpace{}%
\AgdaBound{n}\AgdaSpace{}%
\AgdaBound{p}%
\>[22]\AgdaKeyword{rewrite}\AgdaSpace{}%
\AgdaFunction{+-assoc′}\AgdaSpace{}%
\AgdaBound{m}\AgdaSpace{}%
\AgdaBound{n}\AgdaSpace{}%
\AgdaBound{p}%
\>[46]\AgdaSymbol{=}%
\>[49]\AgdaInductiveConstructor{refl}\<%
\end{code}
\end{fence}

For the base case, we must show:

\begin{myDisplay}
(zero + n) + p ≡ zero + (n + p)
\end{myDisplay}

Simplifying both sides with the base case of addition yields the
equation:

\begin{myDisplay}
n + p ≡ n + p
\end{myDisplay}

This holds trivially. The proof that a term is equal to itself is
written \texttt{refl}.

For the inductive case, we must show:

\begin{myDisplay}
(suc m + n) + p ≡ suc m + (n + p)
\end{myDisplay}

Simplifying both sides with the inductive case of addition yields the
equation:

\begin{myDisplay}
suc ((m + n) + p) ≡ suc (m + (n + p))
\end{myDisplay}

After rewriting with the inductive hypothesis these two terms are equal,
and the proof is again given by \texttt{refl}. Rewriting by a given
equation is indicated by the keyword \texttt{rewrite} followed by a
proof of that equation. Rewriting avoids not only chains of equations
but also the need to invoke \texttt{cong}.

\hypertarget{commutativity-with-rewrite}{%
\section{Commutativity with rewrite}\label{commutativity-with-rewrite}}

Here is a second proof of commutativity of addition, using
\texttt{rewrite} rather than chains of equations:

\begin{fence}
\begin{code}%
\>[0]\AgdaFunction{+-identity′}\AgdaSpace{}%
\AgdaSymbol{:}\AgdaSpace{}%
\AgdaSymbol{∀}\AgdaSpace{}%
\AgdaSymbol{(}\AgdaBound{n}\AgdaSpace{}%
\AgdaSymbol{:}\AgdaSpace{}%
\AgdaDatatype{ℕ}\AgdaSymbol{)}\AgdaSpace{}%
\AgdaSymbol{→}\AgdaSpace{}%
\AgdaBound{n}\AgdaSpace{}%
\AgdaOperator{\AgdaPrimitive{+}}\AgdaSpace{}%
\AgdaInductiveConstructor{zero}\AgdaSpace{}%
\AgdaOperator{\AgdaDatatype{≡}}\AgdaSpace{}%
\AgdaBound{n}\<%
\\
\>[0]\AgdaFunction{+-identity′}\AgdaSpace{}%
\AgdaInductiveConstructor{zero}\AgdaSpace{}%
\AgdaSymbol{=}\AgdaSpace{}%
\AgdaInductiveConstructor{refl}\<%
\\
\>[0]\AgdaFunction{+-identity′}\AgdaSpace{}%
\AgdaSymbol{(}\AgdaInductiveConstructor{suc}\AgdaSpace{}%
\AgdaBound{n}\AgdaSymbol{)}\AgdaSpace{}%
\AgdaKeyword{rewrite}\AgdaSpace{}%
\AgdaFunction{+-identity′}\AgdaSpace{}%
\AgdaBound{n}\AgdaSpace{}%
\AgdaSymbol{=}\AgdaSpace{}%
\AgdaInductiveConstructor{refl}\<%
\\
%
\\[\AgdaEmptyExtraSkip]%
\>[0]\AgdaFunction{+-suc′}\AgdaSpace{}%
\AgdaSymbol{:}\AgdaSpace{}%
\AgdaSymbol{∀}\AgdaSpace{}%
\AgdaSymbol{(}\AgdaBound{m}\AgdaSpace{}%
\AgdaBound{n}\AgdaSpace{}%
\AgdaSymbol{:}\AgdaSpace{}%
\AgdaDatatype{ℕ}\AgdaSymbol{)}\AgdaSpace{}%
\AgdaSymbol{→}\AgdaSpace{}%
\AgdaBound{m}\AgdaSpace{}%
\AgdaOperator{\AgdaPrimitive{+}}\AgdaSpace{}%
\AgdaInductiveConstructor{suc}\AgdaSpace{}%
\AgdaBound{n}\AgdaSpace{}%
\AgdaOperator{\AgdaDatatype{≡}}\AgdaSpace{}%
\AgdaInductiveConstructor{suc}\AgdaSpace{}%
\AgdaSymbol{(}\AgdaBound{m}\AgdaSpace{}%
\AgdaOperator{\AgdaPrimitive{+}}\AgdaSpace{}%
\AgdaBound{n}\AgdaSymbol{)}\<%
\\
\>[0]\AgdaFunction{+-suc′}\AgdaSpace{}%
\AgdaInductiveConstructor{zero}\AgdaSpace{}%
\AgdaBound{n}\AgdaSpace{}%
\AgdaSymbol{=}\AgdaSpace{}%
\AgdaInductiveConstructor{refl}\<%
\\
\>[0]\AgdaFunction{+-suc′}\AgdaSpace{}%
\AgdaSymbol{(}\AgdaInductiveConstructor{suc}\AgdaSpace{}%
\AgdaBound{m}\AgdaSymbol{)}\AgdaSpace{}%
\AgdaBound{n}\AgdaSpace{}%
\AgdaKeyword{rewrite}\AgdaSpace{}%
\AgdaFunction{+-suc′}\AgdaSpace{}%
\AgdaBound{m}\AgdaSpace{}%
\AgdaBound{n}\AgdaSpace{}%
\AgdaSymbol{=}\AgdaSpace{}%
\AgdaInductiveConstructor{refl}\<%
\\
%
\\[\AgdaEmptyExtraSkip]%
\>[0]\AgdaFunction{+-comm′}\AgdaSpace{}%
\AgdaSymbol{:}\AgdaSpace{}%
\AgdaSymbol{∀}\AgdaSpace{}%
\AgdaSymbol{(}\AgdaBound{m}\AgdaSpace{}%
\AgdaBound{n}\AgdaSpace{}%
\AgdaSymbol{:}\AgdaSpace{}%
\AgdaDatatype{ℕ}\AgdaSymbol{)}\AgdaSpace{}%
\AgdaSymbol{→}\AgdaSpace{}%
\AgdaBound{m}\AgdaSpace{}%
\AgdaOperator{\AgdaPrimitive{+}}\AgdaSpace{}%
\AgdaBound{n}\AgdaSpace{}%
\AgdaOperator{\AgdaDatatype{≡}}\AgdaSpace{}%
\AgdaBound{n}\AgdaSpace{}%
\AgdaOperator{\AgdaPrimitive{+}}\AgdaSpace{}%
\AgdaBound{m}\<%
\\
\>[0]\AgdaFunction{+-comm′}\AgdaSpace{}%
\AgdaBound{m}\AgdaSpace{}%
\AgdaInductiveConstructor{zero}\AgdaSpace{}%
\AgdaKeyword{rewrite}\AgdaSpace{}%
\AgdaFunction{+-identity′}\AgdaSpace{}%
\AgdaBound{m}\AgdaSpace{}%
\AgdaSymbol{=}\AgdaSpace{}%
\AgdaInductiveConstructor{refl}\<%
\\
\>[0]\AgdaFunction{+-comm′}\AgdaSpace{}%
\AgdaBound{m}\AgdaSpace{}%
\AgdaSymbol{(}\AgdaInductiveConstructor{suc}\AgdaSpace{}%
\AgdaBound{n}\AgdaSymbol{)}\AgdaSpace{}%
\AgdaKeyword{rewrite}\AgdaSpace{}%
\AgdaFunction{+-suc′}\AgdaSpace{}%
\AgdaBound{m}\AgdaSpace{}%
\AgdaBound{n}\AgdaSpace{}%
\AgdaSymbol{|}\AgdaSpace{}%
\AgdaFunction{+-comm′}\AgdaSpace{}%
\AgdaBound{m}\AgdaSpace{}%
\AgdaBound{n}\AgdaSpace{}%
\AgdaSymbol{=}\AgdaSpace{}%
\AgdaInductiveConstructor{refl}\<%
\end{code}
\end{fence}

In the final line, rewriting with two equations is indicated by
separating the two proofs of the relevant equations by a vertical bar;
the rewrite on the left is performed before that on the right.

\hypertarget{building-proofs-interactively}{%
\section{Building proofs
interactively}\label{building-proofs-interactively}}

It is instructive to see how to build the alternative proof of
associativity using the interactive features of Agda in Emacs. Begin by
typing:

\begin{myDisplay}
+-assoc′ : ∀ (m n p : ℕ) → (m + n) + p ≡ m + (n + p)
+-assoc′ m n p = ?
\end{myDisplay}

The question mark indicates that you would like Agda to help with
filling in that part of the code. If you type \texttt{C-c\ C-l}
(control-c followed by control-l), the question mark will be replaced:

\begin{myDisplay}
+-assoc′ : ∀ (m n p : ℕ) → (m + n) + p ≡ m + (n + p)
+-assoc′ m n p = { }0
\end{myDisplay}

The empty braces are called a \emph{hole}, and 0 is a number used for
referring to the hole. The hole may display highlighted in green. Emacs
will also create a new window at the bottom of the screen displaying the
text:

\begin{myDisplay}
?0 : ((m + n) + p) ≡ (m + (n + p))
\end{myDisplay}

This indicates that hole 0 is to be filled in with a proof of the stated
judgment.

We wish to prove the proposition by induction on \texttt{m}. Move the
cursor into the hole and type \texttt{C-c\ C-c}. You will be given the
prompt:

\begin{myDisplay}
pattern variables to case (empty for split on result):
\end{myDisplay}

Typing \texttt{m} will cause a split on that variable, resulting in an
update to the code:

\begin{myDisplay}
+-assoc′ : ∀ (m n p : ℕ) → (m + n) + p ≡ m + (n + p)
+-assoc′ zero n p = { }0
+-assoc′ (suc m) n p = { }1
\end{myDisplay}

There are now two holes, and the window at the bottom tells you what
each is required to prove:

\begin{myDisplay}
?0 : ((zero + n) + p) ≡ (zero + (n + p))
?1 : ((suc m + n) + p) ≡ (suc m + (n + p))
\end{myDisplay}

Going into hole 0 and typing \texttt{C-c\ C-,} will display the text:

\begin{myDisplay}
Goal: (n + p) ≡ (n + p)
————————————————————————————————————————————————————————————
p : ℕ
n : ℕ
\end{myDisplay}

This indicates that after simplification the goal for hole 0 is as
stated, and that variables \texttt{p} and \texttt{n} of the stated types
are available to use in the proof. The proof of the given goal is
trivial, and going into the goal and typing \texttt{C-c\ C-r} will fill
it in. Typing \texttt{C-c\ C-l} renumbers the remaining hole to 0:

\begin{myDisplay}
+-assoc′ : ∀ (m n p : ℕ) → (m + n) + p ≡ m + (n + p)
+-assoc′ zero n p = refl
+-assoc′ (suc m) n p = { }0
\end{myDisplay}

Going into the new hole 0 and typing \texttt{C-c\ C-,} will display the
text:

\begin{myDisplay}
Goal: suc ((m + n) + p) ≡ suc (m + (n + p))
————————————————————————————————————————————————————————————
p : ℕ
n : ℕ
m : ℕ
\end{myDisplay}

Again, this gives the simplified goal and the available variables. In
this case, we need to rewrite by the induction hypothesis, so let's edit
the text accordingly:

\begin{myDisplay}
+-assoc′ : ∀ (m n p : ℕ) → (m + n) + p ≡ m + (n + p)
+-assoc′ zero n p = refl
+-assoc′ (suc m) n p rewrite +-assoc′ m n p = { }0
\end{myDisplay}

Going into the remaining hole and typing \texttt{C-c\ C-,} will display
the text:

\begin{myDisplay}
Goal: suc (m + (n + p)) ≡ suc (m + (n + p))
————————————————————————————————————————————————————————————
p : ℕ
n : ℕ
m : ℕ
\end{myDisplay}

The proof of the given goal is trivial, and going into the goal and
typing \texttt{C-c\ C-r} will fill it in, completing the proof:

\begin{myDisplay}
+-assoc′ : ∀ (m n p : ℕ) → (m + n) + p ≡ m + (n + p)
+-assoc′ zero n p = refl
+-assoc′ (suc m) n p rewrite +-assoc′ m n p = refl
\end{myDisplay}

\hypertarget{Induction-plus-swap}{%
\subsubsection{\texorpdfstring{Exercise \texttt{+-swap}
(recommended)}{Exercise +-swap (recommended)}}\label{Induction-plus-swap}}

Show

\begin{myDisplay}
m + (n + p) ≡ n + (m + p)
\end{myDisplay}

for all naturals \texttt{m}, \texttt{n}, and \texttt{p}. No induction is
needed, just apply the previous results which show addition is
associative and commutative.

\begin{fence}
\begin{code}%
\>[0]\AgdaComment{-- Your code goes here}\<%
\end{code}
\end{fence}

\hypertarget{Induction-times-distrib-plus}{%
\subsubsection{\texorpdfstring{Exercise \texttt{*-distrib-+}
(recommended)}{Exercise *-distrib-+ (recommended)}}\label{Induction-times-distrib-plus}}

Show multiplication distributes over addition, that is,

\begin{myDisplay}
(m + n) * p ≡ m * p + n * p
\end{myDisplay}

for all naturals \texttt{m}, \texttt{n}, and \texttt{p}.

\begin{fence}
\begin{code}%
\>[0]\AgdaComment{-- Your code goes here}\<%
\end{code}
\end{fence}

\hypertarget{Induction-times-assoc}{%
\subsubsection{\texorpdfstring{Exercise \texttt{*-assoc}
(recommended)}{Exercise *-assoc (recommended)}}\label{Induction-times-assoc}}

Show multiplication is associative, that is,

\begin{myDisplay}
(m * n) * p ≡ m * (n * p)
\end{myDisplay}

for all naturals \texttt{m}, \texttt{n}, and \texttt{p}.

\begin{fence}
\begin{code}%
\>[0]\AgdaComment{-- Your code goes here}\<%
\end{code}
\end{fence}

\hypertarget{Induction-times-comm}{%
\subsubsection{\texorpdfstring{Exercise \texttt{*-comm}
(practice)}{Exercise *-comm (practice)}}\label{Induction-times-comm}}

Show multiplication is commutative, that is,

\begin{myDisplay}
m * n ≡ n * m
\end{myDisplay}

for all naturals \texttt{m} and \texttt{n}. As with commutativity of
addition, you will need to formulate and prove suitable lemmas.

\begin{fence}
\begin{code}%
\>[0]\AgdaComment{-- Your code goes here}\<%
\end{code}
\end{fence}

\hypertarget{Induction-zero-monus}{%
\subsubsection{\texorpdfstring{Exercise \texttt{0∸n≡0}
(practice)}{Exercise 0∸n≡0 (practice)}}\label{Induction-zero-monus}}

Show

\begin{myDisplay}
zero ∸ n ≡ zero
\end{myDisplay}

for all naturals \texttt{n}. Did your proof require induction?

\begin{fence}
\begin{code}%
\>[0]\AgdaComment{-- Your code goes here}\<%
\end{code}
\end{fence}

\hypertarget{Induction-monus-plus-assoc}{%
\subsubsection{\texorpdfstring{Exercise \texttt{∸-\textbar{}-assoc}
(practice)}{Exercise ∸-\textbar-assoc (practice)}}\label{Induction-monus-plus-assoc}}

Show that monus associates with addition, that is,

\begin{myDisplay}
m ∸ n ∸ p ≡ m ∸ (n + p)
\end{myDisplay}

for all naturals \texttt{m}, \texttt{n}, and \texttt{p}.

\begin{fence}
\begin{code}%
\>[0]\AgdaComment{-- Your code goes here}\<%
\end{code}
\end{fence}

\hypertarget{exercise-stretch}{%
\subsubsection{\texorpdfstring{Exercise \texttt{+*\^{}}
(stretch)}{Exercise +*\^{} (stretch)}}\label{exercise-stretch}}

Show the following three laws

\begin{myDisplay}
 m ^ (n + p) ≡ (m ^ n) * (m ^ p)  (^-distribˡ-|-*)
 (m * n) ^ p ≡ (m ^ p) * (n ^ p)  (^-distribʳ-*)
 (m ^ n) ^ p ≡ m ^ (n * p)        (^-*-assoc)
\end{myDisplay}

for all \texttt{m}, \texttt{n}, and \texttt{p}.

\hypertarget{Induction-Bin-laws}{%
\subsubsection{\texorpdfstring{Exercise \texttt{Bin-laws}
(stretch)}{Exercise Bin-laws (stretch)}}\label{Induction-Bin-laws}}

Recall that Exercise \protect\hyperlink{Naturals-Bin}{Bin} defines a
datatype \texttt{Bin} of bitstrings representing natural numbers, and
asks you to define functions

\begin{myDisplay}
inc   : Bin → Bin
to    : ℕ → Bin
from  : Bin → ℕ
\end{myDisplay}

Consider the following laws, where \texttt{n} ranges over naturals and
\texttt{b} over bitstrings:

\begin{myDisplay}
from (inc b) ≡ suc (from b)
to (from b) ≡ b
from (to n) ≡ n
\end{myDisplay}

For each law: if it holds, prove; if not, give a counterexample.

\begin{fence}
\begin{code}%
\>[0]\AgdaComment{-- Your code goes here}\<%
\end{code}
\end{fence}

\hypertarget{standard-library}{%
\section{Standard library}\label{standard-library}}

Definitions similar to those in this chapter can be found in the
standard library:

\begin{fence}
\begin{code}%
\>[0]\AgdaKeyword{import}\AgdaSpace{}%
\AgdaModule{Data.Nat.Properties}\AgdaSpace{}%
\AgdaKeyword{using}\AgdaSpace{}%
\AgdaSymbol{(}\AgdaFunction{+-assoc}\AgdaSymbol{;}\AgdaSpace{}%
\AgdaFunction{+-identityʳ}\AgdaSymbol{;}\AgdaSpace{}%
\AgdaFunction{+-suc}\AgdaSymbol{;}\AgdaSpace{}%
\AgdaFunction{+-comm}\AgdaSymbol{)}\<%
\end{code}
\end{fence}

\hypertarget{unicode}{%
\section{Unicode}\label{unicode}}

This chapter uses the following unicode:

\begin{myDisplay}
∀  U+2200  FOR ALL (\forall, \all)
ʳ  U+02B3  MODIFIER LETTER SMALL R (\^r)
′  U+2032  PRIME (\')
″  U+2033  DOUBLE PRIME (\')
‴  U+2034  TRIPLE PRIME (\')
⁗  U+2057  QUADRUPLE PRIME (\')
\end{myDisplay}

Similar to \texttt{\textbackslash{}r}, the command
\texttt{\textbackslash{}\^{}r} gives access to a variety of superscript
rightward arrows, and also a superscript letter \texttt{r}. The command
\texttt{\textbackslash{}\textquotesingle{}} gives access to a range of
primes (\texttt{′\ ″\ ‴\ ⁗}).

