\hypertarget{Bisimulation}{%
\chapter{Bisimulation: Relating reduction systems}\label{Bisimulation}}

\begin{fence}
\begin{code}%
\>[0]\AgdaKeyword{module}\AgdaSpace{}%
\AgdaModule{plfa.part2.Bisimulation}\AgdaSpace{}%
\AgdaKeyword{where}\<%
\end{code}
\end{fence}

Some constructs can be defined in terms of other constructs. In the
previous chapter, we saw how \emph{let} terms can be rewritten as an
application of an abstraction, and how two alternative formulations of
products --- one with projections and one with case --- can be
formulated in terms of each other. In this chapter, we look at how to
formalise such claims.

Given two different systems, with different terms and reduction rules,
we define what it means to claim that one \emph{simulates} the other.
Let's call our two systems \emph{source} and \emph{target}. Let
\texttt{M}, \texttt{N} range over terms of the source, and \texttt{M†},
\texttt{N†} range over terms of the target. We define a relation

\begin{myDisplay}
M ~ M†
\end{myDisplay}

between corresponding terms of the two systems. We have a
\emph{simulation} of the source by the target if every reduction in the
source has a corresponding reduction sequence in the target:

\emph{Simulation}: For every \texttt{M}, \texttt{M†}, and \texttt{N}: If
\texttt{M\ \textasciitilde{}\ M†} and \texttt{M\ —→\ N} then
\texttt{M†\ —↠\ N†} and \texttt{N\ \textasciitilde{}\ N†} for some
\texttt{N†}.

Or, in a diagram:

\begin{myDisplay}
M  --- —→ --- N
|             |
|             |
~             ~
|             |
|             |
M† --- —↠ --- N†
\end{myDisplay}

Sometimes we will have a stronger condition, where each reduction in the
source corresponds to a reduction (rather than a reduction sequence) in
the target:

\begin{myDisplay}
M  --- —→ --- N
|             |
|             |
~             ~
|             |
|             |
M† --- —→ --- N†
\end{myDisplay}

This stronger condition is known as \emph{lock-step} or \emph{on the
nose} simulation.

We are particularly interested in the situation where there is also a
simulation from the target to the source: every reduction in the target
has a corresponding reduction sequence in the source. This situation is
called a \emph{bisimulation}.

Simulation is established by case analysis over all possible reductions
and all possible terms to which they are related. For each reduction
step in the source we must show a corresponding reduction sequence in
the target.

For instance, the source might be lambda calculus with \emph{let} added,
and the target the same system with \texttt{let} translated out. The key
rule defining our relation will be:

\begin{myDisplay}
M ~ M†
N ~ N†
--------------------------------
let x = M in N ~ (ƛ x ⇒ N†) · M†
\end{myDisplay}

All the other rules are congruences: variables relate to themselves, and
abstractions and applications relate if their components relate:

\begin{myDisplay}
-----
x ~ x

N ~ N†
------------------
ƛ x ⇒ N ~ ƛ x ⇒ N†

L ~ L†
M ~ M†
---------------
L · M ~ L† · M†
\end{myDisplay}

Covering the other constructs of our language --- naturals, fixpoints,
products, and so on --- would add little save length.

In this case, our relation can be specified by a function from source to
target:

\begin{myDisplay}
(x) †               =  x
(ƛ x ⇒ N) †         =  ƛ x ⇒ (N †)
(L · M) †           =  (L †) · (M †)
(let x = M in N) †  =  (ƛ x ⇒ (N †)) · (M †)
\end{myDisplay}

And we have

\begin{myDisplay}
M † ≡ N
-------
M ~ N
\end{myDisplay}

and conversely. But in general we may have a relation without any
corresponding function.

This chapter formalises establishing that \texttt{\textasciitilde{}} as
defined above is a simulation from source to target. We leave
establishing it in the reverse direction as an exercise. Another
exercise is to show the alternative formulations of products in Chapter
\protect\hyperlink{More}{More} are in bisimulation.

\hypertarget{imports}{%
\section{Imports}\label{imports}}

We import our source language from Chapter
\protect\hyperlink{More}{More}:

\begin{fence}
\begin{code}%
\>[0]\AgdaKeyword{open}\AgdaSpace{}%
\AgdaKeyword{import}\AgdaSpace{}%
\AgdaModule{plfa.part2.More}\<%
\end{code}
\end{fence}

\hypertarget{simulation}{%
\section{Simulation}\label{simulation}}

The simulation is a straightforward formalisation of the rules in the
introduction:

\begin{fence}
\begin{code}%
\>[0]\AgdaKeyword{infix}%
\>[7]\AgdaNumber{4}\AgdaSpace{}%
\AgdaOperator{\AgdaDatatype{\AgdaUnderscore{}\textasciitilde{}\AgdaUnderscore{}}}\<%
\\
\>[0]\AgdaKeyword{infix}%
\>[7]\AgdaNumber{5}\AgdaSpace{}%
\AgdaOperator{\AgdaInductiveConstructor{\textasciitilde{}ƛ\AgdaUnderscore{}}}\<%
\\
\>[0]\AgdaKeyword{infix}%
\>[7]\AgdaNumber{7}\AgdaSpace{}%
\AgdaOperator{\AgdaInductiveConstructor{\AgdaUnderscore{}\textasciitilde{}·\AgdaUnderscore{}}}\<%
\\
%
\\[\AgdaEmptyExtraSkip]%
\>[0]\AgdaKeyword{data}\AgdaSpace{}%
\AgdaOperator{\AgdaDatatype{\AgdaUnderscore{}\textasciitilde{}\AgdaUnderscore{}}}\AgdaSpace{}%
\AgdaSymbol{:}\AgdaSpace{}%
\AgdaSymbol{∀}\AgdaSpace{}%
\AgdaSymbol{\{}\AgdaBound{Γ}\AgdaSpace{}%
\AgdaBound{A}\AgdaSymbol{\}}\AgdaSpace{}%
\AgdaSymbol{→}\AgdaSpace{}%
\AgdaSymbol{(}\AgdaBound{Γ}\AgdaSpace{}%
\AgdaOperator{\AgdaDatatype{⊢}}\AgdaSpace{}%
\AgdaBound{A}\AgdaSymbol{)}\AgdaSpace{}%
\AgdaSymbol{→}\AgdaSpace{}%
\AgdaSymbol{(}\AgdaBound{Γ}\AgdaSpace{}%
\AgdaOperator{\AgdaDatatype{⊢}}\AgdaSpace{}%
\AgdaBound{A}\AgdaSymbol{)}\AgdaSpace{}%
\AgdaSymbol{→}\AgdaSpace{}%
\AgdaPrimitiveType{Set}\AgdaSpace{}%
\AgdaKeyword{where}\<%
\\
%
\\[\AgdaEmptyExtraSkip]%
\>[0][@{}l@{\AgdaIndent{0}}]%
\>[2]\AgdaInductiveConstructor{\textasciitilde{}`}%
\>[23I]\AgdaSymbol{:}\AgdaSpace{}%
\AgdaSymbol{∀}\AgdaSpace{}%
\AgdaSymbol{\{}\AgdaBound{Γ}\AgdaSpace{}%
\AgdaBound{A}\AgdaSymbol{\}}\AgdaSpace{}%
\AgdaSymbol{\{}\AgdaBound{x}\AgdaSpace{}%
\AgdaSymbol{:}\AgdaSpace{}%
\AgdaBound{Γ}\AgdaSpace{}%
\AgdaOperator{\AgdaDatatype{∋}}\AgdaSpace{}%
\AgdaBound{A}\AgdaSymbol{\}}\<%
\\
\>[.][@{}l@{}]\<[23I]%
\>[5]\AgdaComment{---------}\<%
\\
\>[2][@{}l@{\AgdaIndent{0}}]%
\>[3]\AgdaSymbol{→}\AgdaSpace{}%
\AgdaOperator{\AgdaInductiveConstructor{`}}\AgdaSpace{}%
\AgdaBound{x}\AgdaSpace{}%
\AgdaOperator{\AgdaDatatype{\textasciitilde{}}}\AgdaSpace{}%
\AgdaOperator{\AgdaInductiveConstructor{`}}\AgdaSpace{}%
\AgdaBound{x}\<%
\\
%
\\[\AgdaEmptyExtraSkip]%
%
\>[2]\AgdaOperator{\AgdaInductiveConstructor{\textasciitilde{}ƛ\AgdaUnderscore{}}}\AgdaSpace{}%
\AgdaSymbol{:}\AgdaSpace{}%
\AgdaSymbol{∀}\AgdaSpace{}%
\AgdaSymbol{\{}\AgdaBound{Γ}\AgdaSpace{}%
\AgdaBound{A}\AgdaSpace{}%
\AgdaBound{B}\AgdaSymbol{\}}\AgdaSpace{}%
\AgdaSymbol{\{}\AgdaBound{N}\AgdaSpace{}%
\AgdaBound{N†}\AgdaSpace{}%
\AgdaSymbol{:}\AgdaSpace{}%
\AgdaBound{Γ}\AgdaSpace{}%
\AgdaOperator{\AgdaInductiveConstructor{,}}\AgdaSpace{}%
\AgdaBound{A}\AgdaSpace{}%
\AgdaOperator{\AgdaDatatype{⊢}}\AgdaSpace{}%
\AgdaBound{B}\AgdaSymbol{\}}\<%
\\
\>[2][@{}l@{\AgdaIndent{0}}]%
\>[4]\AgdaSymbol{→}%
\>[50I]\AgdaBound{N}\AgdaSpace{}%
\AgdaOperator{\AgdaDatatype{\textasciitilde{}}}\AgdaSpace{}%
\AgdaBound{N†}\<%
\\
\>[.][@{}l@{}]\<[50I]%
\>[6]\AgdaComment{----------}\<%
\\
%
\>[4]\AgdaSymbol{→}\AgdaSpace{}%
\AgdaOperator{\AgdaInductiveConstructor{ƛ}}\AgdaSpace{}%
\AgdaBound{N}\AgdaSpace{}%
\AgdaOperator{\AgdaDatatype{\textasciitilde{}}}\AgdaSpace{}%
\AgdaOperator{\AgdaInductiveConstructor{ƛ}}\AgdaSpace{}%
\AgdaBound{N†}\<%
\\
%
\\[\AgdaEmptyExtraSkip]%
%
\>[2]\AgdaOperator{\AgdaInductiveConstructor{\AgdaUnderscore{}\textasciitilde{}·\AgdaUnderscore{}}}\AgdaSpace{}%
\AgdaSymbol{:}\AgdaSpace{}%
\AgdaSymbol{∀}\AgdaSpace{}%
\AgdaSymbol{\{}\AgdaBound{Γ}\AgdaSpace{}%
\AgdaBound{A}\AgdaSpace{}%
\AgdaBound{B}\AgdaSymbol{\}}\AgdaSpace{}%
\AgdaSymbol{\{}\AgdaBound{L}\AgdaSpace{}%
\AgdaBound{L†}\AgdaSpace{}%
\AgdaSymbol{:}\AgdaSpace{}%
\AgdaBound{Γ}\AgdaSpace{}%
\AgdaOperator{\AgdaDatatype{⊢}}\AgdaSpace{}%
\AgdaBound{A}\AgdaSpace{}%
\AgdaOperator{\AgdaInductiveConstructor{⇒}}\AgdaSpace{}%
\AgdaBound{B}\AgdaSymbol{\}}\AgdaSpace{}%
\AgdaSymbol{\{}\AgdaBound{M}\AgdaSpace{}%
\AgdaBound{M†}\AgdaSpace{}%
\AgdaSymbol{:}\AgdaSpace{}%
\AgdaBound{Γ}\AgdaSpace{}%
\AgdaOperator{\AgdaDatatype{⊢}}\AgdaSpace{}%
\AgdaBound{A}\AgdaSymbol{\}}\<%
\\
\>[2][@{}l@{\AgdaIndent{0}}]%
\>[4]\AgdaSymbol{→}\AgdaSpace{}%
\AgdaBound{L}\AgdaSpace{}%
\AgdaOperator{\AgdaDatatype{\textasciitilde{}}}\AgdaSpace{}%
\AgdaBound{L†}\<%
\\
%
\>[4]\AgdaSymbol{→}%
\>[80I]\AgdaBound{M}\AgdaSpace{}%
\AgdaOperator{\AgdaDatatype{\textasciitilde{}}}\AgdaSpace{}%
\AgdaBound{M†}\<%
\\
\>[.][@{}l@{}]\<[80I]%
\>[6]\AgdaComment{---------------}\<%
\\
%
\>[4]\AgdaSymbol{→}\AgdaSpace{}%
\AgdaBound{L}\AgdaSpace{}%
\AgdaOperator{\AgdaInductiveConstructor{·}}\AgdaSpace{}%
\AgdaBound{M}\AgdaSpace{}%
\AgdaOperator{\AgdaDatatype{\textasciitilde{}}}\AgdaSpace{}%
\AgdaBound{L†}\AgdaSpace{}%
\AgdaOperator{\AgdaInductiveConstructor{·}}\AgdaSpace{}%
\AgdaBound{M†}\<%
\\
%
\\[\AgdaEmptyExtraSkip]%
%
\>[2]\AgdaInductiveConstructor{\textasciitilde{}let}\AgdaSpace{}%
\AgdaSymbol{:}\AgdaSpace{}%
\AgdaSymbol{∀}\AgdaSpace{}%
\AgdaSymbol{\{}\AgdaBound{Γ}\AgdaSpace{}%
\AgdaBound{A}\AgdaSpace{}%
\AgdaBound{B}\AgdaSymbol{\}}\AgdaSpace{}%
\AgdaSymbol{\{}\AgdaBound{M}\AgdaSpace{}%
\AgdaBound{M†}\AgdaSpace{}%
\AgdaSymbol{:}\AgdaSpace{}%
\AgdaBound{Γ}\AgdaSpace{}%
\AgdaOperator{\AgdaDatatype{⊢}}\AgdaSpace{}%
\AgdaBound{A}\AgdaSymbol{\}}\AgdaSpace{}%
\AgdaSymbol{\{}\AgdaBound{N}\AgdaSpace{}%
\AgdaBound{N†}\AgdaSpace{}%
\AgdaSymbol{:}\AgdaSpace{}%
\AgdaBound{Γ}\AgdaSpace{}%
\AgdaOperator{\AgdaInductiveConstructor{,}}\AgdaSpace{}%
\AgdaBound{A}\AgdaSpace{}%
\AgdaOperator{\AgdaDatatype{⊢}}\AgdaSpace{}%
\AgdaBound{B}\AgdaSymbol{\}}\<%
\\
\>[2][@{}l@{\AgdaIndent{0}}]%
\>[4]\AgdaSymbol{→}\AgdaSpace{}%
\AgdaBound{M}\AgdaSpace{}%
\AgdaOperator{\AgdaDatatype{\textasciitilde{}}}\AgdaSpace{}%
\AgdaBound{M†}\<%
\\
%
\>[4]\AgdaSymbol{→}%
\>[112I]\AgdaBound{N}\AgdaSpace{}%
\AgdaOperator{\AgdaDatatype{\textasciitilde{}}}\AgdaSpace{}%
\AgdaBound{N†}\<%
\\
\>[.][@{}l@{}]\<[112I]%
\>[6]\AgdaComment{----------------------}\<%
\\
%
\>[4]\AgdaSymbol{→}\AgdaSpace{}%
\AgdaInductiveConstructor{`let}\AgdaSpace{}%
\AgdaBound{M}\AgdaSpace{}%
\AgdaBound{N}\AgdaSpace{}%
\AgdaOperator{\AgdaDatatype{\textasciitilde{}}}\AgdaSpace{}%
\AgdaSymbol{(}\AgdaOperator{\AgdaInductiveConstructor{ƛ}}\AgdaSpace{}%
\AgdaBound{N†}\AgdaSymbol{)}\AgdaSpace{}%
\AgdaOperator{\AgdaInductiveConstructor{·}}\AgdaSpace{}%
\AgdaBound{M†}\<%
\end{code}
\end{fence}

The language in Chapter \protect\hyperlink{More}{More} has more
constructs, which we could easily add. However, leaving the simulation
small lets us focus on the essence. It's a handy technical trick that we
can have a large source language, but only bother to include in the
simulation the terms of interest.

\hypertarget{exercise-_-practice}{%
\subsubsection{\texorpdfstring{Exercise \texttt{\_†}
(practice)}{Exercise \_† (practice)}}\label{exercise-_-practice}}

Formalise the translation from source to target given in the
introduction. Show that \texttt{M\ †\ ≡\ N} implies
\texttt{M\ \textasciitilde{}\ N}, and conversely.

\textbf{Hint:} For simplicity, we focus on only a few constructs of the
language, so \texttt{\_†} should be defined only on relevant terms. One
way to do this is to use a decidable predicate to pick out terms in the
domain of \texttt{\_†}, using
\protect\hyperlink{Decidable-proof-by-reflection}{proof by reflection}.

\begin{fence}
\begin{code}%
\>[0]\AgdaComment{-- Your code goes here}\<%
\end{code}
\end{fence}

\hypertarget{simulation-commutes-with-values}{%
\section{Simulation commutes with
values}\label{simulation-commutes-with-values}}

We need a number of technical results. The first is that simulation
commutes with values. That is, if \texttt{M\ \textasciitilde{}\ M†} and
\texttt{M} is a value then \texttt{M†} is also a value:

\begin{fence}
\begin{code}%
\>[0]\AgdaFunction{\textasciitilde{}val}\AgdaSpace{}%
\AgdaSymbol{:}\AgdaSpace{}%
\AgdaSymbol{∀}\AgdaSpace{}%
\AgdaSymbol{\{}\AgdaBound{Γ}\AgdaSpace{}%
\AgdaBound{A}\AgdaSymbol{\}}\AgdaSpace{}%
\AgdaSymbol{\{}\AgdaBound{M}\AgdaSpace{}%
\AgdaBound{M†}\AgdaSpace{}%
\AgdaSymbol{:}\AgdaSpace{}%
\AgdaBound{Γ}\AgdaSpace{}%
\AgdaOperator{\AgdaDatatype{⊢}}\AgdaSpace{}%
\AgdaBound{A}\AgdaSymbol{\}}\<%
\\
\>[0][@{}l@{\AgdaIndent{0}}]%
\>[2]\AgdaSymbol{→}\AgdaSpace{}%
\AgdaBound{M}\AgdaSpace{}%
\AgdaOperator{\AgdaDatatype{\textasciitilde{}}}\AgdaSpace{}%
\AgdaBound{M†}\<%
\\
%
\>[2]\AgdaSymbol{→}%
\>[136I]\AgdaDatatype{Value}\AgdaSpace{}%
\AgdaBound{M}\<%
\\
\>[.][@{}l@{}]\<[136I]%
\>[4]\AgdaComment{--------}\<%
\\
%
\>[2]\AgdaSymbol{→}\AgdaSpace{}%
\AgdaDatatype{Value}\AgdaSpace{}%
\AgdaBound{M†}\<%
\\
\>[0]\AgdaFunction{\textasciitilde{}val}\AgdaSpace{}%
\AgdaInductiveConstructor{\textasciitilde{}`}%
\>[18]\AgdaSymbol{()}\<%
\\
\>[0]\AgdaFunction{\textasciitilde{}val}\AgdaSpace{}%
\AgdaSymbol{(}\AgdaOperator{\AgdaInductiveConstructor{\textasciitilde{}ƛ}}\AgdaSpace{}%
\AgdaBound{\textasciitilde{}N}\AgdaSymbol{)}%
\>[18]\AgdaInductiveConstructor{V-ƛ}%
\>[23]\AgdaSymbol{=}%
\>[26]\AgdaInductiveConstructor{V-ƛ}\<%
\\
\>[0]\AgdaFunction{\textasciitilde{}val}\AgdaSpace{}%
\AgdaSymbol{(}\AgdaBound{\textasciitilde{}L}\AgdaSpace{}%
\AgdaOperator{\AgdaInductiveConstructor{\textasciitilde{}·}}\AgdaSpace{}%
\AgdaBound{\textasciitilde{}M}\AgdaSymbol{)}%
\>[18]\AgdaSymbol{()}\<%
\\
\>[0]\AgdaFunction{\textasciitilde{}val}\AgdaSpace{}%
\AgdaSymbol{(}\AgdaInductiveConstructor{\textasciitilde{}let}\AgdaSpace{}%
\AgdaBound{\textasciitilde{}M}\AgdaSpace{}%
\AgdaBound{\textasciitilde{}N}\AgdaSymbol{)}\AgdaSpace{}%
\AgdaSymbol{()}\<%
\end{code}
\end{fence}

It is a straightforward case analysis, where here the only value of
interest is a lambda abstraction.

\hypertarget{exercise-valuxb9-practice}{%
\subsubsection{\texorpdfstring{Exercise \texttt{\textasciitilde{}val⁻¹}
(practice)}{Exercise \textasciitilde val⁻¹ (practice)}}\label{exercise-valuxb9-practice}}

Show that this also holds in the reverse direction: if
\texttt{M\ \textasciitilde{}\ M†} and \texttt{Value\ M†} then
\texttt{Value\ M}.

\begin{fence}
\begin{code}%
\>[0]\AgdaComment{-- Your code goes here}\<%
\end{code}
\end{fence}

\hypertarget{simulation-commutes-with-renaming}{%
\section{Simulation commutes with
renaming}\label{simulation-commutes-with-renaming}}

The next technical result is that simulation commutes with renaming.
That is, if \texttt{ρ} maps any judgment \texttt{Γ\ ∋\ A} to a judgment
\texttt{Δ\ ∋\ A}, and if \texttt{M\ \textasciitilde{}\ M†} then
\texttt{rename\ ρ\ M\ \textasciitilde{}\ rename\ ρ\ M†}:

\begin{fence}
\begin{code}%
\>[0]\AgdaFunction{\textasciitilde{}rename}\AgdaSpace{}%
\AgdaSymbol{:}\AgdaSpace{}%
\AgdaSymbol{∀}\AgdaSpace{}%
\AgdaSymbol{\{}\AgdaBound{Γ}\AgdaSpace{}%
\AgdaBound{Δ}\AgdaSymbol{\}}\<%
\\
\>[0][@{}l@{\AgdaIndent{0}}]%
\>[2]\AgdaSymbol{→}%
\>[154I]\AgdaSymbol{(}\AgdaBound{ρ}\AgdaSpace{}%
\AgdaSymbol{:}\AgdaSpace{}%
\AgdaSymbol{∀}\AgdaSpace{}%
\AgdaSymbol{\{}\AgdaBound{A}\AgdaSymbol{\}}\AgdaSpace{}%
\AgdaSymbol{→}\AgdaSpace{}%
\AgdaBound{Γ}\AgdaSpace{}%
\AgdaOperator{\AgdaDatatype{∋}}\AgdaSpace{}%
\AgdaBound{A}\AgdaSpace{}%
\AgdaSymbol{→}\AgdaSpace{}%
\AgdaBound{Δ}\AgdaSpace{}%
\AgdaOperator{\AgdaDatatype{∋}}\AgdaSpace{}%
\AgdaBound{A}\AgdaSymbol{)}\<%
\\
\>[.][@{}l@{}]\<[154I]%
\>[4]\AgdaComment{----------------------------------------------------------}\<%
\\
%
\>[2]\AgdaSymbol{→}\AgdaSpace{}%
\AgdaSymbol{(∀}\AgdaSpace{}%
\AgdaSymbol{\{}\AgdaBound{A}\AgdaSymbol{\}}\AgdaSpace{}%
\AgdaSymbol{\{}\AgdaBound{M}\AgdaSpace{}%
\AgdaBound{M†}\AgdaSpace{}%
\AgdaSymbol{:}\AgdaSpace{}%
\AgdaBound{Γ}\AgdaSpace{}%
\AgdaOperator{\AgdaDatatype{⊢}}\AgdaSpace{}%
\AgdaBound{A}\AgdaSymbol{\}}\AgdaSpace{}%
\AgdaSymbol{→}\AgdaSpace{}%
\AgdaBound{M}\AgdaSpace{}%
\AgdaOperator{\AgdaDatatype{\textasciitilde{}}}\AgdaSpace{}%
\AgdaBound{M†}\AgdaSpace{}%
\AgdaSymbol{→}\AgdaSpace{}%
\AgdaFunction{rename}\AgdaSpace{}%
\AgdaBound{ρ}\AgdaSpace{}%
\AgdaBound{M}\AgdaSpace{}%
\AgdaOperator{\AgdaDatatype{\textasciitilde{}}}\AgdaSpace{}%
\AgdaFunction{rename}\AgdaSpace{}%
\AgdaBound{ρ}\AgdaSpace{}%
\AgdaBound{M†}\AgdaSymbol{)}\<%
\\
\>[0]\AgdaFunction{\textasciitilde{}rename}\AgdaSpace{}%
\AgdaBound{ρ}\AgdaSpace{}%
\AgdaSymbol{(}\AgdaInductiveConstructor{\textasciitilde{}`}\AgdaSymbol{)}%
\>[24]\AgdaSymbol{=}%
\>[27]\AgdaInductiveConstructor{\textasciitilde{}`}\<%
\\
\>[0]\AgdaFunction{\textasciitilde{}rename}\AgdaSpace{}%
\AgdaBound{ρ}\AgdaSpace{}%
\AgdaSymbol{(}\AgdaOperator{\AgdaInductiveConstructor{\textasciitilde{}ƛ}}\AgdaSpace{}%
\AgdaBound{\textasciitilde{}N}\AgdaSymbol{)}%
\>[24]\AgdaSymbol{=}%
\>[27]\AgdaOperator{\AgdaInductiveConstructor{\textasciitilde{}ƛ}}\AgdaSpace{}%
\AgdaSymbol{(}\AgdaFunction{\textasciitilde{}rename}\AgdaSpace{}%
\AgdaSymbol{(}\AgdaFunction{ext}\AgdaSpace{}%
\AgdaBound{ρ}\AgdaSymbol{)}\AgdaSpace{}%
\AgdaBound{\textasciitilde{}N}\AgdaSymbol{)}\<%
\\
\>[0]\AgdaFunction{\textasciitilde{}rename}\AgdaSpace{}%
\AgdaBound{ρ}\AgdaSpace{}%
\AgdaSymbol{(}\AgdaBound{\textasciitilde{}L}\AgdaSpace{}%
\AgdaOperator{\AgdaInductiveConstructor{\textasciitilde{}·}}\AgdaSpace{}%
\AgdaBound{\textasciitilde{}M}\AgdaSymbol{)}%
\>[24]\AgdaSymbol{=}%
\>[27]\AgdaSymbol{(}\AgdaFunction{\textasciitilde{}rename}\AgdaSpace{}%
\AgdaBound{ρ}\AgdaSpace{}%
\AgdaBound{\textasciitilde{}L}\AgdaSymbol{)}\AgdaSpace{}%
\AgdaOperator{\AgdaInductiveConstructor{\textasciitilde{}·}}\AgdaSpace{}%
\AgdaSymbol{(}\AgdaFunction{\textasciitilde{}rename}\AgdaSpace{}%
\AgdaBound{ρ}\AgdaSpace{}%
\AgdaBound{\textasciitilde{}M}\AgdaSymbol{)}\<%
\\
\>[0]\AgdaFunction{\textasciitilde{}rename}\AgdaSpace{}%
\AgdaBound{ρ}\AgdaSpace{}%
\AgdaSymbol{(}\AgdaInductiveConstructor{\textasciitilde{}let}\AgdaSpace{}%
\AgdaBound{\textasciitilde{}M}\AgdaSpace{}%
\AgdaBound{\textasciitilde{}N}\AgdaSymbol{)}%
\>[24]\AgdaSymbol{=}%
\>[27]\AgdaInductiveConstructor{\textasciitilde{}let}\AgdaSpace{}%
\AgdaSymbol{(}\AgdaFunction{\textasciitilde{}rename}\AgdaSpace{}%
\AgdaBound{ρ}\AgdaSpace{}%
\AgdaBound{\textasciitilde{}M}\AgdaSymbol{)}\AgdaSpace{}%
\AgdaSymbol{(}\AgdaFunction{\textasciitilde{}rename}\AgdaSpace{}%
\AgdaSymbol{(}\AgdaFunction{ext}\AgdaSpace{}%
\AgdaBound{ρ}\AgdaSymbol{)}\AgdaSpace{}%
\AgdaBound{\textasciitilde{}N}\AgdaSymbol{)}\<%
\end{code}
\end{fence}

The structure of the proof is similar to the structure of renaming
itself: reconstruct each term with recursive invocation, extending the
environment where appropriate (in this case, only for the body of an
abstraction).

\hypertarget{simulation-commutes-with-substitution}{%
\section{Simulation commutes with
substitution}\label{simulation-commutes-with-substitution}}

The third technical result is that simulation commutes with
substitution. It is more complex than renaming, because where we had one
renaming map \texttt{ρ} here we need two substitution maps, \texttt{σ}
and \texttt{σ†}.

The proof first requires we establish an analogue of extension. If
\texttt{σ} and \texttt{σ†} both map any judgment \texttt{Γ\ ∋\ A} to a
judgment \texttt{Δ\ ⊢\ A}, such that for every \texttt{x} in
\texttt{Γ\ ∋\ A} we have \texttt{σ\ x\ \textasciitilde{}\ σ†\ x}, then
for any \texttt{x} in \texttt{Γ\ ,\ B\ ∋\ A} we have
\texttt{exts\ σ\ x\ \textasciitilde{}\ exts\ σ†\ x}:

\begin{fence}
\begin{code}%
\>[0]\AgdaFunction{\textasciitilde{}exts}\AgdaSpace{}%
\AgdaSymbol{:}\AgdaSpace{}%
\AgdaSymbol{∀}\AgdaSpace{}%
\AgdaSymbol{\{}\AgdaBound{Γ}\AgdaSpace{}%
\AgdaBound{Δ}\AgdaSymbol{\}}\<%
\\
\>[0][@{}l@{\AgdaIndent{0}}]%
\>[2]\AgdaSymbol{→}\AgdaSpace{}%
\AgdaSymbol{\{}\AgdaBound{σ}%
\>[8]\AgdaSymbol{:}\AgdaSpace{}%
\AgdaSymbol{∀}\AgdaSpace{}%
\AgdaSymbol{\{}\AgdaBound{A}\AgdaSymbol{\}}\AgdaSpace{}%
\AgdaSymbol{→}\AgdaSpace{}%
\AgdaBound{Γ}\AgdaSpace{}%
\AgdaOperator{\AgdaDatatype{∋}}\AgdaSpace{}%
\AgdaBound{A}\AgdaSpace{}%
\AgdaSymbol{→}\AgdaSpace{}%
\AgdaBound{Δ}\AgdaSpace{}%
\AgdaOperator{\AgdaDatatype{⊢}}\AgdaSpace{}%
\AgdaBound{A}\AgdaSymbol{\}}\<%
\\
%
\>[2]\AgdaSymbol{→}\AgdaSpace{}%
\AgdaSymbol{\{}\AgdaBound{σ†}\AgdaSpace{}%
\AgdaSymbol{:}\AgdaSpace{}%
\AgdaSymbol{∀}\AgdaSpace{}%
\AgdaSymbol{\{}\AgdaBound{A}\AgdaSymbol{\}}\AgdaSpace{}%
\AgdaSymbol{→}\AgdaSpace{}%
\AgdaBound{Γ}\AgdaSpace{}%
\AgdaOperator{\AgdaDatatype{∋}}\AgdaSpace{}%
\AgdaBound{A}\AgdaSpace{}%
\AgdaSymbol{→}\AgdaSpace{}%
\AgdaBound{Δ}\AgdaSpace{}%
\AgdaOperator{\AgdaDatatype{⊢}}\AgdaSpace{}%
\AgdaBound{A}\AgdaSymbol{\}}\<%
\\
%
\>[2]\AgdaSymbol{→}%
\>[243I]\AgdaSymbol{(∀}\AgdaSpace{}%
\AgdaSymbol{\{}\AgdaBound{A}\AgdaSymbol{\}}\AgdaSpace{}%
\AgdaSymbol{→}\AgdaSpace{}%
\AgdaSymbol{(}\AgdaBound{x}\AgdaSpace{}%
\AgdaSymbol{:}\AgdaSpace{}%
\AgdaBound{Γ}\AgdaSpace{}%
\AgdaOperator{\AgdaDatatype{∋}}\AgdaSpace{}%
\AgdaBound{A}\AgdaSymbol{)}\AgdaSpace{}%
\AgdaSymbol{→}\AgdaSpace{}%
\AgdaBound{σ}\AgdaSpace{}%
\AgdaBound{x}\AgdaSpace{}%
\AgdaOperator{\AgdaDatatype{\textasciitilde{}}}\AgdaSpace{}%
\AgdaBound{σ†}\AgdaSpace{}%
\AgdaBound{x}\AgdaSymbol{)}\<%
\\
\>[.][@{}l@{}]\<[243I]%
\>[4]\AgdaComment{--------------------------------------------------}\<%
\\
%
\>[2]\AgdaSymbol{→}\AgdaSpace{}%
\AgdaSymbol{(∀}\AgdaSpace{}%
\AgdaSymbol{\{}\AgdaBound{A}\AgdaSpace{}%
\AgdaBound{B}\AgdaSymbol{\}}\AgdaSpace{}%
\AgdaSymbol{→}\AgdaSpace{}%
\AgdaSymbol{(}\AgdaBound{x}\AgdaSpace{}%
\AgdaSymbol{:}\AgdaSpace{}%
\AgdaBound{Γ}\AgdaSpace{}%
\AgdaOperator{\AgdaInductiveConstructor{,}}\AgdaSpace{}%
\AgdaBound{B}\AgdaSpace{}%
\AgdaOperator{\AgdaDatatype{∋}}\AgdaSpace{}%
\AgdaBound{A}\AgdaSymbol{)}\AgdaSpace{}%
\AgdaSymbol{→}\AgdaSpace{}%
\AgdaFunction{exts}\AgdaSpace{}%
\AgdaBound{σ}\AgdaSpace{}%
\AgdaBound{x}\AgdaSpace{}%
\AgdaOperator{\AgdaDatatype{\textasciitilde{}}}\AgdaSpace{}%
\AgdaFunction{exts}\AgdaSpace{}%
\AgdaBound{σ†}\AgdaSpace{}%
\AgdaBound{x}\AgdaSymbol{)}\<%
\\
\>[0]\AgdaFunction{\textasciitilde{}exts}\AgdaSpace{}%
\AgdaBound{\textasciitilde{}σ}\AgdaSpace{}%
\AgdaInductiveConstructor{Z}%
\>[16]\AgdaSymbol{=}%
\>[19]\AgdaInductiveConstructor{\textasciitilde{}`}\<%
\\
\>[0]\AgdaFunction{\textasciitilde{}exts}\AgdaSpace{}%
\AgdaBound{\textasciitilde{}σ}\AgdaSpace{}%
\AgdaSymbol{(}\AgdaOperator{\AgdaInductiveConstructor{S}}\AgdaSpace{}%
\AgdaBound{x}\AgdaSymbol{)}%
\>[16]\AgdaSymbol{=}%
\>[19]\AgdaFunction{\textasciitilde{}rename}\AgdaSpace{}%
\AgdaOperator{\AgdaInductiveConstructor{S\AgdaUnderscore{}}}\AgdaSpace{}%
\AgdaSymbol{(}\AgdaBound{\textasciitilde{}σ}\AgdaSpace{}%
\AgdaBound{x}\AgdaSymbol{)}\<%
\end{code}
\end{fence}

The structure of the proof is similar to the structure of extension
itself. The newly introduced variable trivially relates to itself, and
otherwise we apply renaming to the hypothesis.

With extension under our belts, it is straightforward to show
substitution commutes. If \texttt{σ} and \texttt{σ†} both map any
judgment \texttt{Γ\ ∋\ A} to a judgment \texttt{Δ\ ⊢\ A}, such that for
every \texttt{x} in \texttt{Γ\ ∋\ A} we have
\texttt{σ\ x\ \textasciitilde{}\ σ†\ x}, and if
\texttt{M\ \textasciitilde{}\ M†}, then
\texttt{subst\ σ\ M\ \textasciitilde{}\ subst\ σ†\ M†}:

\begin{fence}
\begin{code}%
\>[0]\AgdaFunction{\textasciitilde{}subst}\AgdaSpace{}%
\AgdaSymbol{:}\AgdaSpace{}%
\AgdaSymbol{∀}\AgdaSpace{}%
\AgdaSymbol{\{}\AgdaBound{Γ}\AgdaSpace{}%
\AgdaBound{Δ}\AgdaSymbol{\}}\<%
\\
\>[0][@{}l@{\AgdaIndent{0}}]%
\>[2]\AgdaSymbol{→}\AgdaSpace{}%
\AgdaSymbol{\{}\AgdaBound{σ}%
\>[8]\AgdaSymbol{:}\AgdaSpace{}%
\AgdaSymbol{∀}\AgdaSpace{}%
\AgdaSymbol{\{}\AgdaBound{A}\AgdaSymbol{\}}\AgdaSpace{}%
\AgdaSymbol{→}\AgdaSpace{}%
\AgdaBound{Γ}\AgdaSpace{}%
\AgdaOperator{\AgdaDatatype{∋}}\AgdaSpace{}%
\AgdaBound{A}\AgdaSpace{}%
\AgdaSymbol{→}\AgdaSpace{}%
\AgdaBound{Δ}\AgdaSpace{}%
\AgdaOperator{\AgdaDatatype{⊢}}\AgdaSpace{}%
\AgdaBound{A}\AgdaSymbol{\}}\<%
\\
%
\>[2]\AgdaSymbol{→}\AgdaSpace{}%
\AgdaSymbol{\{}\AgdaBound{σ†}\AgdaSpace{}%
\AgdaSymbol{:}\AgdaSpace{}%
\AgdaSymbol{∀}\AgdaSpace{}%
\AgdaSymbol{\{}\AgdaBound{A}\AgdaSymbol{\}}\AgdaSpace{}%
\AgdaSymbol{→}\AgdaSpace{}%
\AgdaBound{Γ}\AgdaSpace{}%
\AgdaOperator{\AgdaDatatype{∋}}\AgdaSpace{}%
\AgdaBound{A}\AgdaSpace{}%
\AgdaSymbol{→}\AgdaSpace{}%
\AgdaBound{Δ}\AgdaSpace{}%
\AgdaOperator{\AgdaDatatype{⊢}}\AgdaSpace{}%
\AgdaBound{A}\AgdaSymbol{\}}\<%
\\
%
\>[2]\AgdaSymbol{→}%
\>[311I]\AgdaSymbol{(∀}\AgdaSpace{}%
\AgdaSymbol{\{}\AgdaBound{A}\AgdaSymbol{\}}\AgdaSpace{}%
\AgdaSymbol{→}\AgdaSpace{}%
\AgdaSymbol{(}\AgdaBound{x}\AgdaSpace{}%
\AgdaSymbol{:}\AgdaSpace{}%
\AgdaBound{Γ}\AgdaSpace{}%
\AgdaOperator{\AgdaDatatype{∋}}\AgdaSpace{}%
\AgdaBound{A}\AgdaSymbol{)}\AgdaSpace{}%
\AgdaSymbol{→}\AgdaSpace{}%
\AgdaBound{σ}\AgdaSpace{}%
\AgdaBound{x}\AgdaSpace{}%
\AgdaOperator{\AgdaDatatype{\textasciitilde{}}}\AgdaSpace{}%
\AgdaBound{σ†}\AgdaSpace{}%
\AgdaBound{x}\AgdaSymbol{)}\<%
\\
\>[.][@{}l@{}]\<[311I]%
\>[4]\AgdaComment{---------------------------------------------------------}\<%
\\
%
\>[2]\AgdaSymbol{→}\AgdaSpace{}%
\AgdaSymbol{(∀}\AgdaSpace{}%
\AgdaSymbol{\{}\AgdaBound{A}\AgdaSymbol{\}}\AgdaSpace{}%
\AgdaSymbol{\{}\AgdaBound{M}\AgdaSpace{}%
\AgdaBound{M†}\AgdaSpace{}%
\AgdaSymbol{:}\AgdaSpace{}%
\AgdaBound{Γ}\AgdaSpace{}%
\AgdaOperator{\AgdaDatatype{⊢}}\AgdaSpace{}%
\AgdaBound{A}\AgdaSymbol{\}}\AgdaSpace{}%
\AgdaSymbol{→}\AgdaSpace{}%
\AgdaBound{M}\AgdaSpace{}%
\AgdaOperator{\AgdaDatatype{\textasciitilde{}}}\AgdaSpace{}%
\AgdaBound{M†}\AgdaSpace{}%
\AgdaSymbol{→}\AgdaSpace{}%
\AgdaFunction{subst}\AgdaSpace{}%
\AgdaBound{σ}\AgdaSpace{}%
\AgdaBound{M}\AgdaSpace{}%
\AgdaOperator{\AgdaDatatype{\textasciitilde{}}}\AgdaSpace{}%
\AgdaFunction{subst}\AgdaSpace{}%
\AgdaBound{σ†}\AgdaSpace{}%
\AgdaBound{M†}\AgdaSymbol{)}\<%
\\
\>[0]\AgdaFunction{\textasciitilde{}subst}\AgdaSpace{}%
\AgdaBound{\textasciitilde{}σ}\AgdaSpace{}%
\AgdaSymbol{(}\AgdaInductiveConstructor{\textasciitilde{}`}\AgdaSpace{}%
\AgdaSymbol{\{}\AgdaArgument{x}\AgdaSpace{}%
\AgdaSymbol{=}\AgdaSpace{}%
\AgdaBound{x}\AgdaSymbol{\})}%
\>[24]\AgdaSymbol{=}%
\>[27]\AgdaBound{\textasciitilde{}σ}\AgdaSpace{}%
\AgdaBound{x}\<%
\\
\>[0]\AgdaFunction{\textasciitilde{}subst}\AgdaSpace{}%
\AgdaBound{\textasciitilde{}σ}\AgdaSpace{}%
\AgdaSymbol{(}\AgdaOperator{\AgdaInductiveConstructor{\textasciitilde{}ƛ}}\AgdaSpace{}%
\AgdaBound{\textasciitilde{}N}\AgdaSymbol{)}%
\>[24]\AgdaSymbol{=}%
\>[27]\AgdaOperator{\AgdaInductiveConstructor{\textasciitilde{}ƛ}}\AgdaSpace{}%
\AgdaSymbol{(}\AgdaFunction{\textasciitilde{}subst}\AgdaSpace{}%
\AgdaSymbol{(}\AgdaFunction{\textasciitilde{}exts}\AgdaSpace{}%
\AgdaBound{\textasciitilde{}σ}\AgdaSymbol{)}\AgdaSpace{}%
\AgdaBound{\textasciitilde{}N}\AgdaSymbol{)}\<%
\\
\>[0]\AgdaFunction{\textasciitilde{}subst}\AgdaSpace{}%
\AgdaBound{\textasciitilde{}σ}\AgdaSpace{}%
\AgdaSymbol{(}\AgdaBound{\textasciitilde{}L}\AgdaSpace{}%
\AgdaOperator{\AgdaInductiveConstructor{\textasciitilde{}·}}\AgdaSpace{}%
\AgdaBound{\textasciitilde{}M}\AgdaSymbol{)}%
\>[24]\AgdaSymbol{=}%
\>[27]\AgdaSymbol{(}\AgdaFunction{\textasciitilde{}subst}\AgdaSpace{}%
\AgdaBound{\textasciitilde{}σ}\AgdaSpace{}%
\AgdaBound{\textasciitilde{}L}\AgdaSymbol{)}\AgdaSpace{}%
\AgdaOperator{\AgdaInductiveConstructor{\textasciitilde{}·}}\AgdaSpace{}%
\AgdaSymbol{(}\AgdaFunction{\textasciitilde{}subst}\AgdaSpace{}%
\AgdaBound{\textasciitilde{}σ}\AgdaSpace{}%
\AgdaBound{\textasciitilde{}M}\AgdaSymbol{)}\<%
\\
\>[0]\AgdaFunction{\textasciitilde{}subst}\AgdaSpace{}%
\AgdaBound{\textasciitilde{}σ}\AgdaSpace{}%
\AgdaSymbol{(}\AgdaInductiveConstructor{\textasciitilde{}let}\AgdaSpace{}%
\AgdaBound{\textasciitilde{}M}\AgdaSpace{}%
\AgdaBound{\textasciitilde{}N}\AgdaSymbol{)}%
\>[24]\AgdaSymbol{=}%
\>[27]\AgdaInductiveConstructor{\textasciitilde{}let}\AgdaSpace{}%
\AgdaSymbol{(}\AgdaFunction{\textasciitilde{}subst}\AgdaSpace{}%
\AgdaBound{\textasciitilde{}σ}\AgdaSpace{}%
\AgdaBound{\textasciitilde{}M}\AgdaSymbol{)}\AgdaSpace{}%
\AgdaSymbol{(}\AgdaFunction{\textasciitilde{}subst}\AgdaSpace{}%
\AgdaSymbol{(}\AgdaFunction{\textasciitilde{}exts}\AgdaSpace{}%
\AgdaBound{\textasciitilde{}σ}\AgdaSymbol{)}\AgdaSpace{}%
\AgdaBound{\textasciitilde{}N}\AgdaSymbol{)}\<%
\end{code}
\end{fence}

Again, the structure of the proof is similar to the structure of
substitution itself: reconstruct each term with recursive invocation,
extending the environment where appropriate (in this case, only for the
body of an abstraction).

From the general case of substitution, it is also easy to derive the
required special case. If \texttt{N\ \textasciitilde{}\ N†} and
\texttt{M\ \textasciitilde{}\ M†}, then
\texttt{N\ {[}\ M\ {]}\ \textasciitilde{}\ N†\ {[}\ M†\ {]}}:

\begin{fence}
\begin{code}%
\>[0]\AgdaFunction{\textasciitilde{}sub}\AgdaSpace{}%
\AgdaSymbol{:}\AgdaSpace{}%
\AgdaSymbol{∀}\AgdaSpace{}%
\AgdaSymbol{\{}\AgdaBound{Γ}\AgdaSpace{}%
\AgdaBound{A}\AgdaSpace{}%
\AgdaBound{B}\AgdaSymbol{\}}\AgdaSpace{}%
\AgdaSymbol{\{}\AgdaBound{N}\AgdaSpace{}%
\AgdaBound{N†}\AgdaSpace{}%
\AgdaSymbol{:}\AgdaSpace{}%
\AgdaBound{Γ}\AgdaSpace{}%
\AgdaOperator{\AgdaInductiveConstructor{,}}\AgdaSpace{}%
\AgdaBound{B}\AgdaSpace{}%
\AgdaOperator{\AgdaDatatype{⊢}}\AgdaSpace{}%
\AgdaBound{A}\AgdaSymbol{\}}\AgdaSpace{}%
\AgdaSymbol{\{}\AgdaBound{M}\AgdaSpace{}%
\AgdaBound{M†}\AgdaSpace{}%
\AgdaSymbol{:}\AgdaSpace{}%
\AgdaBound{Γ}\AgdaSpace{}%
\AgdaOperator{\AgdaDatatype{⊢}}\AgdaSpace{}%
\AgdaBound{B}\AgdaSymbol{\}}\<%
\\
\>[0][@{}l@{\AgdaIndent{0}}]%
\>[2]\AgdaSymbol{→}\AgdaSpace{}%
\AgdaBound{N}\AgdaSpace{}%
\AgdaOperator{\AgdaDatatype{\textasciitilde{}}}\AgdaSpace{}%
\AgdaBound{N†}\<%
\\
%
\>[2]\AgdaSymbol{→}%
\>[401I]\AgdaBound{M}\AgdaSpace{}%
\AgdaOperator{\AgdaDatatype{\textasciitilde{}}}\AgdaSpace{}%
\AgdaBound{M†}\<%
\\
\>[.][@{}l@{}]\<[401I]%
\>[4]\AgdaComment{-----------------------}\<%
\\
%
\>[2]\AgdaSymbol{→}\AgdaSpace{}%
\AgdaSymbol{(}\AgdaBound{N}\AgdaSpace{}%
\AgdaOperator{\AgdaFunction{[}}\AgdaSpace{}%
\AgdaBound{M}\AgdaSpace{}%
\AgdaOperator{\AgdaFunction{]}}\AgdaSymbol{)}\AgdaSpace{}%
\AgdaOperator{\AgdaDatatype{\textasciitilde{}}}\AgdaSpace{}%
\AgdaSymbol{(}\AgdaBound{N†}\AgdaSpace{}%
\AgdaOperator{\AgdaFunction{[}}\AgdaSpace{}%
\AgdaBound{M†}\AgdaSpace{}%
\AgdaOperator{\AgdaFunction{]}}\AgdaSymbol{)}\<%
\\
\>[0]\AgdaFunction{\textasciitilde{}sub}\AgdaSpace{}%
\AgdaSymbol{\{}\AgdaBound{Γ}\AgdaSymbol{\}}\AgdaSpace{}%
\AgdaSymbol{\{}\AgdaBound{A}\AgdaSymbol{\}}\AgdaSpace{}%
\AgdaSymbol{\{}\AgdaBound{B}\AgdaSymbol{\}}\AgdaSpace{}%
\AgdaBound{\textasciitilde{}N}\AgdaSpace{}%
\AgdaBound{\textasciitilde{}M}\AgdaSpace{}%
\AgdaSymbol{=}\AgdaSpace{}%
\AgdaFunction{\textasciitilde{}subst}\AgdaSpace{}%
\AgdaSymbol{\{}\AgdaBound{Γ}\AgdaSpace{}%
\AgdaOperator{\AgdaInductiveConstructor{,}}\AgdaSpace{}%
\AgdaBound{B}\AgdaSymbol{\}}\AgdaSpace{}%
\AgdaSymbol{\{}\AgdaBound{Γ}\AgdaSymbol{\}}\AgdaSpace{}%
\AgdaFunction{\textasciitilde{}σ}\AgdaSpace{}%
\AgdaSymbol{\{}\AgdaBound{A}\AgdaSymbol{\}}\AgdaSpace{}%
\AgdaBound{\textasciitilde{}N}\<%
\\
\>[0][@{}l@{\AgdaIndent{0}}]%
\>[2]\AgdaKeyword{where}\<%
\\
%
\>[2]\AgdaFunction{\textasciitilde{}σ}\AgdaSpace{}%
\AgdaSymbol{:}\AgdaSpace{}%
\AgdaSymbol{∀}\AgdaSpace{}%
\AgdaSymbol{\{}\AgdaBound{A}\AgdaSymbol{\}}\AgdaSpace{}%
\AgdaSymbol{→}\AgdaSpace{}%
\AgdaSymbol{(}\AgdaBound{x}\AgdaSpace{}%
\AgdaSymbol{:}\AgdaSpace{}%
\AgdaBound{Γ}\AgdaSpace{}%
\AgdaOperator{\AgdaInductiveConstructor{,}}\AgdaSpace{}%
\AgdaBound{B}\AgdaSpace{}%
\AgdaOperator{\AgdaDatatype{∋}}\AgdaSpace{}%
\AgdaBound{A}\AgdaSymbol{)}\AgdaSpace{}%
\AgdaSymbol{→}\AgdaSpace{}%
\AgdaSymbol{\AgdaUnderscore{}}\AgdaSpace{}%
\AgdaOperator{\AgdaDatatype{\textasciitilde{}}}\AgdaSpace{}%
\AgdaSymbol{\AgdaUnderscore{}}\<%
\\
%
\>[2]\AgdaFunction{\textasciitilde{}σ}\AgdaSpace{}%
\AgdaInductiveConstructor{Z}%
\>[12]\AgdaSymbol{=}%
\>[15]\AgdaBound{\textasciitilde{}M}\<%
\\
%
\>[2]\AgdaFunction{\textasciitilde{}σ}\AgdaSpace{}%
\AgdaSymbol{(}\AgdaOperator{\AgdaInductiveConstructor{S}}\AgdaSpace{}%
\AgdaBound{x}\AgdaSymbol{)}%
\>[12]\AgdaSymbol{=}%
\>[15]\AgdaInductiveConstructor{\textasciitilde{}`}\<%
\end{code}
\end{fence}

Once more, the structure of the proof resembles the original.

\hypertarget{the-relation-is-a-simulation}{%
\section{The relation is a
simulation}\label{the-relation-is-a-simulation}}

Finally, we can show that the relation actually is a simulation. In
fact, we will show the stronger condition of a lock-step simulation.
What we wish to show is:

\emph{Lock-step simulation}: For every \texttt{M}, \texttt{M†}, and
\texttt{N}: If \texttt{M\ \textasciitilde{}\ M†} and \texttt{M\ —→\ N}
then \texttt{M†\ —→\ N†} and \texttt{N\ \textasciitilde{}\ N†} for some
\texttt{N†}.

Or, in a diagram:

\begin{myDisplay}
M  --- —→ --- N
|             |
|             |
~             ~
|             |
|             |
M† --- —→ --- N†
\end{myDisplay}

We first formulate a concept corresponding to the lower leg of the
diagram, that is, its right and bottom edges:

\begin{fence}
\begin{code}%
\>[0]\AgdaKeyword{data}\AgdaSpace{}%
\AgdaDatatype{Leg}\AgdaSpace{}%
\AgdaSymbol{\{}\AgdaBound{Γ}\AgdaSpace{}%
\AgdaBound{A}\AgdaSymbol{\}}\AgdaSpace{}%
\AgdaSymbol{(}\AgdaBound{M†}\AgdaSpace{}%
\AgdaBound{N}\AgdaSpace{}%
\AgdaSymbol{:}\AgdaSpace{}%
\AgdaBound{Γ}\AgdaSpace{}%
\AgdaOperator{\AgdaDatatype{⊢}}\AgdaSpace{}%
\AgdaBound{A}\AgdaSymbol{)}\AgdaSpace{}%
\AgdaSymbol{:}\AgdaSpace{}%
\AgdaPrimitiveType{Set}\AgdaSpace{}%
\AgdaKeyword{where}\<%
\\
%
\\[\AgdaEmptyExtraSkip]%
\>[0][@{}l@{\AgdaIndent{0}}]%
\>[2]\AgdaInductiveConstructor{leg}\AgdaSpace{}%
\AgdaSymbol{:}\AgdaSpace{}%
\AgdaSymbol{∀}\AgdaSpace{}%
\AgdaSymbol{\{}\AgdaBound{N†}\AgdaSpace{}%
\AgdaSymbol{:}\AgdaSpace{}%
\AgdaBound{Γ}\AgdaSpace{}%
\AgdaOperator{\AgdaDatatype{⊢}}\AgdaSpace{}%
\AgdaBound{A}\AgdaSymbol{\}}\<%
\\
\>[2][@{}l@{\AgdaIndent{0}}]%
\>[4]\AgdaSymbol{→}\AgdaSpace{}%
\AgdaBound{N}\AgdaSpace{}%
\AgdaOperator{\AgdaDatatype{\textasciitilde{}}}\AgdaSpace{}%
\AgdaBound{N†}\<%
\\
%
\>[4]\AgdaSymbol{→}%
\>[467I]\AgdaBound{M†}\AgdaSpace{}%
\AgdaOperator{\AgdaDatatype{—→}}\AgdaSpace{}%
\AgdaBound{N†}\<%
\\
\>[.][@{}l@{}]\<[467I]%
\>[6]\AgdaComment{--------}\<%
\\
%
\>[4]\AgdaSymbol{→}\AgdaSpace{}%
\AgdaDatatype{Leg}\AgdaSpace{}%
\AgdaBound{M†}\AgdaSpace{}%
\AgdaBound{N}\<%
\end{code}
\end{fence}

For our formalisation, in this case, we can use a stronger relation than
\texttt{—↠}, replacing it by \texttt{—→}.

We can now state and prove that the relation is a simulation. Again, in
this case, we can use a stronger relation than \texttt{—↠}, replacing it
by \texttt{—→}:

\begin{fence}
\begin{code}%
\>[0]\AgdaFunction{sim}\AgdaSpace{}%
\AgdaSymbol{:}\AgdaSpace{}%
\AgdaSymbol{∀}\AgdaSpace{}%
\AgdaSymbol{\{}\AgdaBound{Γ}\AgdaSpace{}%
\AgdaBound{A}\AgdaSymbol{\}}\AgdaSpace{}%
\AgdaSymbol{\{}\AgdaBound{M}\AgdaSpace{}%
\AgdaBound{M†}\AgdaSpace{}%
\AgdaBound{N}\AgdaSpace{}%
\AgdaSymbol{:}\AgdaSpace{}%
\AgdaBound{Γ}\AgdaSpace{}%
\AgdaOperator{\AgdaDatatype{⊢}}\AgdaSpace{}%
\AgdaBound{A}\AgdaSymbol{\}}\<%
\\
\>[0][@{}l@{\AgdaIndent{0}}]%
\>[2]\AgdaSymbol{→}\AgdaSpace{}%
\AgdaBound{M}\AgdaSpace{}%
\AgdaOperator{\AgdaDatatype{\textasciitilde{}}}\AgdaSpace{}%
\AgdaBound{M†}\<%
\\
%
\>[2]\AgdaSymbol{→}%
\>[487I]\AgdaBound{M}\AgdaSpace{}%
\AgdaOperator{\AgdaDatatype{—→}}\AgdaSpace{}%
\AgdaBound{N}\<%
\\
\>[.][@{}l@{}]\<[487I]%
\>[4]\AgdaComment{---------}\<%
\\
%
\>[2]\AgdaSymbol{→}\AgdaSpace{}%
\AgdaDatatype{Leg}%
\>[9]\AgdaBound{M†}\AgdaSpace{}%
\AgdaBound{N}\<%
\\
\>[0]\AgdaFunction{sim}\AgdaSpace{}%
\AgdaInductiveConstructor{\textasciitilde{}`}%
\>[20]\AgdaSymbol{()}\<%
\\
\>[0]\AgdaFunction{sim}\AgdaSpace{}%
\AgdaSymbol{(}\AgdaOperator{\AgdaInductiveConstructor{\textasciitilde{}ƛ}}\AgdaSpace{}%
\AgdaBound{\textasciitilde{}N}\AgdaSymbol{)}%
\>[20]\AgdaSymbol{()}\<%
\\
\>[0]\AgdaFunction{sim}\AgdaSpace{}%
\AgdaSymbol{(}\AgdaBound{\textasciitilde{}L}\AgdaSpace{}%
\AgdaOperator{\AgdaInductiveConstructor{\textasciitilde{}·}}\AgdaSpace{}%
\AgdaBound{\textasciitilde{}M}\AgdaSymbol{)}%
\>[20]\AgdaSymbol{(}\AgdaInductiveConstructor{ξ-·₁}\AgdaSpace{}%
\AgdaBound{L—→}\AgdaSymbol{)}\<%
\\
\>[0][@{}l@{\AgdaIndent{0}}]%
\>[2]\AgdaKeyword{with}\AgdaSpace{}%
\AgdaFunction{sim}\AgdaSpace{}%
\AgdaBound{\textasciitilde{}L}\AgdaSpace{}%
\AgdaBound{L—→}\<%
\\
\>[0]\AgdaSymbol{...}%
\>[5]\AgdaSymbol{|}\AgdaSpace{}%
\AgdaInductiveConstructor{leg}\AgdaSpace{}%
\AgdaBound{\textasciitilde{}L′}\AgdaSpace{}%
\AgdaBound{L†—→}%
\>[36]\AgdaSymbol{=}%
\>[39]\AgdaInductiveConstructor{leg}\AgdaSpace{}%
\AgdaSymbol{(}\AgdaBound{\textasciitilde{}L′}\AgdaSpace{}%
\AgdaOperator{\AgdaInductiveConstructor{\textasciitilde{}·}}\AgdaSpace{}%
\AgdaBound{\textasciitilde{}M}\AgdaSymbol{)}%
\>[57]\AgdaSymbol{(}\AgdaInductiveConstructor{ξ-·₁}\AgdaSpace{}%
\AgdaBound{L†—→}\AgdaSymbol{)}\<%
\\
\>[0]\AgdaFunction{sim}\AgdaSpace{}%
\AgdaSymbol{(}\AgdaBound{\textasciitilde{}V}\AgdaSpace{}%
\AgdaOperator{\AgdaInductiveConstructor{\textasciitilde{}·}}\AgdaSpace{}%
\AgdaBound{\textasciitilde{}M}\AgdaSymbol{)}%
\>[20]\AgdaSymbol{(}\AgdaInductiveConstructor{ξ-·₂}\AgdaSpace{}%
\AgdaBound{VV}\AgdaSpace{}%
\AgdaBound{M—→}\AgdaSymbol{)}\<%
\\
\>[0][@{}l@{\AgdaIndent{0}}]%
\>[2]\AgdaKeyword{with}\AgdaSpace{}%
\AgdaFunction{sim}\AgdaSpace{}%
\AgdaBound{\textasciitilde{}M}\AgdaSpace{}%
\AgdaBound{M—→}\<%
\\
\>[0]\AgdaSymbol{...}%
\>[5]\AgdaSymbol{|}\AgdaSpace{}%
\AgdaInductiveConstructor{leg}\AgdaSpace{}%
\AgdaBound{\textasciitilde{}M′}\AgdaSpace{}%
\AgdaBound{M†—→}%
\>[36]\AgdaSymbol{=}%
\>[39]\AgdaInductiveConstructor{leg}\AgdaSpace{}%
\AgdaSymbol{(}\AgdaBound{\textasciitilde{}V}\AgdaSpace{}%
\AgdaOperator{\AgdaInductiveConstructor{\textasciitilde{}·}}\AgdaSpace{}%
\AgdaBound{\textasciitilde{}M′}\AgdaSymbol{)}%
\>[57]\AgdaSymbol{(}\AgdaInductiveConstructor{ξ-·₂}\AgdaSpace{}%
\AgdaSymbol{(}\AgdaFunction{\textasciitilde{}val}\AgdaSpace{}%
\AgdaBound{\textasciitilde{}V}\AgdaSpace{}%
\AgdaBound{VV}\AgdaSymbol{)}\AgdaSpace{}%
\AgdaBound{M†—→}\AgdaSymbol{)}\<%
\\
\>[0]\AgdaFunction{sim}\AgdaSpace{}%
\AgdaSymbol{((}\AgdaOperator{\AgdaInductiveConstructor{\textasciitilde{}ƛ}}\AgdaSpace{}%
\AgdaBound{\textasciitilde{}N}\AgdaSymbol{)}\AgdaSpace{}%
\AgdaOperator{\AgdaInductiveConstructor{\textasciitilde{}·}}\AgdaSpace{}%
\AgdaBound{\textasciitilde{}V}\AgdaSymbol{)}\AgdaSpace{}%
\AgdaSymbol{(}\AgdaInductiveConstructor{β-ƛ}\AgdaSpace{}%
\AgdaBound{VV}\AgdaSymbol{)}%
\>[36]\AgdaSymbol{=}%
\>[39]\AgdaInductiveConstructor{leg}\AgdaSpace{}%
\AgdaSymbol{(}\AgdaFunction{\textasciitilde{}sub}\AgdaSpace{}%
\AgdaBound{\textasciitilde{}N}\AgdaSpace{}%
\AgdaBound{\textasciitilde{}V}\AgdaSymbol{)}%
\>[57]\AgdaSymbol{(}\AgdaInductiveConstructor{β-ƛ}\AgdaSpace{}%
\AgdaSymbol{(}\AgdaFunction{\textasciitilde{}val}\AgdaSpace{}%
\AgdaBound{\textasciitilde{}V}\AgdaSpace{}%
\AgdaBound{VV}\AgdaSymbol{))}\<%
\\
\>[0]\AgdaFunction{sim}\AgdaSpace{}%
\AgdaSymbol{(}\AgdaInductiveConstructor{\textasciitilde{}let}\AgdaSpace{}%
\AgdaBound{\textasciitilde{}M}\AgdaSpace{}%
\AgdaBound{\textasciitilde{}N}\AgdaSymbol{)}%
\>[20]\AgdaSymbol{(}\AgdaInductiveConstructor{ξ-let}\AgdaSpace{}%
\AgdaBound{M—→}\AgdaSymbol{)}\<%
\\
\>[0][@{}l@{\AgdaIndent{0}}]%
\>[2]\AgdaKeyword{with}\AgdaSpace{}%
\AgdaFunction{sim}\AgdaSpace{}%
\AgdaBound{\textasciitilde{}M}\AgdaSpace{}%
\AgdaBound{M—→}\<%
\\
\>[0]\AgdaSymbol{...}%
\>[5]\AgdaSymbol{|}\AgdaSpace{}%
\AgdaInductiveConstructor{leg}\AgdaSpace{}%
\AgdaBound{\textasciitilde{}M′}\AgdaSpace{}%
\AgdaBound{M†—→}%
\>[36]\AgdaSymbol{=}%
\>[39]\AgdaInductiveConstructor{leg}\AgdaSpace{}%
\AgdaSymbol{(}\AgdaInductiveConstructor{\textasciitilde{}let}\AgdaSpace{}%
\AgdaBound{\textasciitilde{}M′}\AgdaSpace{}%
\AgdaBound{\textasciitilde{}N}\AgdaSymbol{)}\AgdaSpace{}%
\AgdaSymbol{(}\AgdaInductiveConstructor{ξ-·₂}\AgdaSpace{}%
\AgdaInductiveConstructor{V-ƛ}\AgdaSpace{}%
\AgdaBound{M†—→}\AgdaSymbol{)}\<%
\\
\>[0]\AgdaFunction{sim}\AgdaSpace{}%
\AgdaSymbol{(}\AgdaInductiveConstructor{\textasciitilde{}let}\AgdaSpace{}%
\AgdaBound{\textasciitilde{}V}\AgdaSpace{}%
\AgdaBound{\textasciitilde{}N}\AgdaSymbol{)}%
\>[20]\AgdaSymbol{(}\AgdaInductiveConstructor{β-let}\AgdaSpace{}%
\AgdaBound{VV}\AgdaSymbol{)}%
\>[36]\AgdaSymbol{=}%
\>[39]\AgdaInductiveConstructor{leg}\AgdaSpace{}%
\AgdaSymbol{(}\AgdaFunction{\textasciitilde{}sub}\AgdaSpace{}%
\AgdaBound{\textasciitilde{}N}\AgdaSpace{}%
\AgdaBound{\textasciitilde{}V}\AgdaSymbol{)}%
\>[57]\AgdaSymbol{(}\AgdaInductiveConstructor{β-ƛ}\AgdaSpace{}%
\AgdaSymbol{(}\AgdaFunction{\textasciitilde{}val}\AgdaSpace{}%
\AgdaBound{\textasciitilde{}V}\AgdaSpace{}%
\AgdaBound{VV}\AgdaSymbol{))}\<%
\end{code}
\end{fence}

The proof is by case analysis, examining each possible instance of
\texttt{M\ \textasciitilde{}\ M†} and each possible instance of
\texttt{M\ —→\ M†}, using recursive invocation whenever the reduction is
by a \texttt{ξ} rule, and hence contains another reduction. In its
structure, it looks a little bit like a proof of progress:

\begin{itemize}
\item
  If the related terms are variables, no reduction applies.
\item
  If the related terms are abstractions, no reduction applies.
\item
  If the related terms are applications, there are three subcases:

  \begin{itemize}
  \item
    The source term reduces via \texttt{ξ-·₁}, in which case the target
    term does as well. Recursive invocation gives us

    \begin{myDisplay}
    L  --- —→ ---  L′
    |              |
    |              |
    ~              ~
    |              |
    |              |
    L† --- —→ --- L′†
    \end{myDisplay}

    from which follows:

    \begin{myDisplay}
     L · M  --- —→ ---  L′ · M
       |                   |
       |                   |
       ~                   ~
       |                   |
       |                   |
    L† · M† --- —→ --- L′† · M†
    \end{myDisplay}
  \item
    The source term reduces via \texttt{ξ-·₂}, in which case the target
    term does as well. Recursive invocation gives us

    \begin{myDisplay}
    M  --- —→ ---  M′
    |              |
    |              |
    ~              ~
    |              |
    |              |
    M† --- —→ --- M′†
    \end{myDisplay}

    from which follows:

    \begin{myDisplay}
     V · M  --- —→ ---  V · M′
       |                  |
       |                  |
       ~                  ~
       |                  |
       |                  |
    V† · M† --- —→ --- V† · M′†
    \end{myDisplay}

    Since simulation commutes with values and \texttt{V} is a value,
    \texttt{V†} is also a value.
  \item
    The source term reduces via \texttt{β-ƛ}, in which case the target
    term does as well:

    \begin{myDisplay}
     (ƛ x ⇒ N) · V  --- —→ ---  N [ x := V ]
          |                           |
          |                           |
          ~                           ~
          |                           |
          |                           |
    (ƛ x ⇒ N†) · V† --- —→ --- N† [ x :=  V† ]
    \end{myDisplay}

    Since simulation commutes with values and \texttt{V} is a value,
    \texttt{V†} is also a value. Since simulation commutes with
    substitution and \texttt{N\ \textasciitilde{}\ N†} and
    \texttt{V\ \textasciitilde{}\ V†}, we have
    \texttt{N\ {[}\ x\ :=\ V\ {]}\ \textasciitilde{}\ N†\ {[}\ x\ :=\ V†\ {]}}.
  \end{itemize}
\item
  If the related terms are a let and an application of an abstraction,
  there are two subcases:

  \begin{itemize}
  \item
    The source term reduces via \texttt{ξ-let}, in which case the target
    term reduces via \texttt{ξ-·₂}. Recursive invocation gives us

    \begin{myDisplay}
    M  --- —→ ---  M′
    |              |
    |              |
    ~              ~
    |              |
    |              |
    M† --- —→ --- M′†
    \end{myDisplay}

    from which follows:

    \begin{myDisplay}
    let x = M in N --- —→ --- let x = M′ in N
          |                         |
          |                         |
          ~                         ~
          |                         |
          |                         |
    (ƛ x ⇒ N) · M  --- —→ --- (ƛ x ⇒ N) · M′
    \end{myDisplay}
  \item
    The source term reduces via \texttt{β-let}, in which case the target
    term reduces via \texttt{β-ƛ}:

    \begin{myDisplay}
    let x = V in N  --- —→ ---  N [ x := V ]
          |                         |
          |                         |
          ~                         ~
          |                         |
          |                         |
    (ƛ x ⇒ N†) · V† --- —→ --- N† [ x := V† ]
    \end{myDisplay}

    Since simulation commutes with values and \texttt{V} is a value,
    \texttt{V†} is also a value. Since simulation commutes with
    substitution and \texttt{N\ \textasciitilde{}\ N†} and
    \texttt{V\ \textasciitilde{}\ V†}, we have
    \texttt{N\ {[}\ x\ :=\ V\ {]}\ \textasciitilde{}\ N†\ {[}\ x\ :=\ V†\ {]}}.
  \end{itemize}
\end{itemize}

\hypertarget{exercise-simuxb9-practice}{%
\subsubsection{\texorpdfstring{Exercise \texttt{sim⁻¹}
(practice)}{Exercise sim⁻¹ (practice)}}\label{exercise-simuxb9-practice}}

Show that we also have a simulation in the other direction, and hence
that we have a bisimulation.

\begin{fence}
\begin{code}%
\>[0]\AgdaComment{-- Your code goes here}\<%
\end{code}
\end{fence}

\hypertarget{exercise-products-practice}{%
\subsubsection{\texorpdfstring{Exercise \texttt{products}
(practice)}{Exercise products (practice)}}\label{exercise-products-practice}}

Show that the two formulations of products in Chapter
\protect\hyperlink{More}{More} are in bisimulation. The only constructs
you need to include are variables, and those connected to functions and
products. In this case, the simulation is \emph{not} lock-step.

\begin{fence}
\begin{code}%
\>[0]\AgdaComment{-- Your code goes here}\<%
\end{code}
\end{fence}

\hypertarget{unicode}{%
\section{Unicode}\label{unicode}}

This chapter uses the following unicode:

\begin{myDisplay}
†  U+2020  DAGGER (\dag)
⁻  U+207B  SUPERSCRIPT MINUS (\^-)
¹  U+00B9  SUPERSCRIPT ONE (\^1)
\end{myDisplay}

